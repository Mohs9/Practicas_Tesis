\chapter{Conclusiones }

En este trabajo se estableció el marco teórico dentro del cual se puntualizan
algunas de las herramientas importantes para describir completamente las
mediciones difusas para sistemas de dos partículas. A continuación se
desglosarán los temas que se abordaron en este proyecto. 

El operador de densidad facilita trabajar en la descripción de sistemas cuánticos cuyo estado es una mezcla estadística de estados puros y en la descripción de subsistemas de un sistema cuántico compuesto. Además, el formalismo y las propiedades de la matriz de densidad, generalizan la representación de estados cuánticos. Asimismo, los postulados de la mecánica cuántica se pueden reformular cómodamente en el lenguaje de la matriz de densidad.

%Se investigó el formalismo y las propiedades de la matriz de densidad, puesto que generaliza la representación de estados cuánticos. Los postulados de la mecánica cuántica pueden ser reformulados cómodamente en el lenguaje de la matriz de densidad y este operador tiene la ventaja que facilita trabajar en la descripción de sistemas cuánticos cuyo estado es una mezcla estadística de estados puros y en la descripción de subsistemas de un sistema cuántico compuesto. 
%Se estudió la teoría de 
Las operaciones cuánticas %debido a que 
son una herramienta 
capaz de describir los efectos de una medición en un sistema cuántico y los cambios
de estados discretos. 
%Las operaciones cuánticas se definen como operaciones completamente positivas que preservan la traza y son linealmente convexas en el conjunto de matrices de densidad. 
Las operaciones cuánticas que preservan la traza pueden escribirse de una forma conveniente como la suma de operadores, para un conjunto de operadores $\{K_i\}$, conocidos como operadores de Kraus, que juegan un papel importante en la descripción de mediciones en sistemas no ideales. 

Este trabajo se enfocó en dar inicio a la exploración de operadores de Kraus
que describan completamente mediciones difusas para sistemas de dos partículas.
Dichas mediciones difusas son aquellas en las se realiza una medición pero con
cierta probabilidad se produce un intercambio de partículas.  
%El problema que se estableció en este trabajo fue el de investigar operadores
%de Kraus que describan completamente las mediciones difusas, concretamente
%para sistemas de dos partículas. 
Los operadores de Kraus para mediciones difusas se exploraron con medidas POVM
debido a que proporcionan un mapeo de salidas, el cual da una una probabilidad
de cada posible salida de medición para cualquier estado de entrada. Por otra
parte, las medidas POVM pueden descomponerse en operadores de Kraus de manera
no única. Estos operadores son especialmente útiles puesto que pueden describir
lo que le sucede al estado inmediatamente después de la medición. Sin embargo,
cuando se realiza una medición lo que le sucede al estado inicial depende en
como los POVM son implementados en el laboratorio. La descomposición más
sencilla de los operadores POVM para el caso de dos partículas fue aplicarle la
raíz cuadrada a un conjunto de operadores POVM, tales que produzcan un mapeo de
las probabilidades. 

Adicionalmente, se realizó un programa en el cual se pueden obtener varios ejemplos numéricos en los cuales se mapea un operador de estado inicial a un estado al que se le aplica la operación cuántica representada como suma de operadores de Kraus obtenidos con la descomposición de una medida POVM\@.

Por otro lado, se exploró otra forma de aproximarse al problema utilizando instrumentos cuánticos. Para ello se planteó una alternativa de instrumento cuántico basado en si la medición difusa se debió al intercambio de partículas. %, al intercambio en los resultados que el aparato de medición registró o bien si con cierta probabilidad no es posible distinguir en que estado de salida se encuentra. 
 El instrumento cuántico utilizado contienen la información de las posibles salidas de la medición  y del efecto de la misma en el estado inicial. Además, se utiliza la traza parcial para obtener el mapeo de la posibilidades o el estado posterior a la medición descrito por un conjunto de operadores de Kraus.

El estudio de la descripción de la mediciones difusas para sistemas de dos partículas con los operadores de Kraus sentaron las bases para poder explorar operadores que puedan describir completamente la medición difusa para sistemas de más de dos partículas que se pretende continuar en el trabajo de graduación de licenciatura.
