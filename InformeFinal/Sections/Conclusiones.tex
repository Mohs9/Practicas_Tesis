\chapter{Conclusiones }

En este trabajo se estableció el marco teórico dentro del cual se puntualizan algunas de las herramientas importantes para describir completamente las mediciones difusas para sistemas de dos partículas. Se investigó el formalismo y las propiedades de la matriz de densidad, puesto que es una herramienta que generaliza la representación de estados cuánticos. Los postulados de la mecánica cuántica pueden ser reformulados cómodamente en el lenguaje de la matriz de densidad y este operador tiene la ventaja que facilita trabajar en la descripción de sistemas cuánticos cuyo estado es una mezcla estadística de estados puros y la descripción de subsistemas de un sistema cuántico compuesto. 

Se estudió la teoría de las operaciones cuánticas debido a que son un medio capaz de describir la evolución de sistemas cuánticos abiertos y cambios de estados discretos. Las operaciones cuánticas se definen como operaciones completamente positivas que preservan la traza y son linealmente convexas en el conjunto de matrices de densidad. Las operaciones cuánticas que preservan la traza pueden escribirse de una forma conveniente como la suma de operadores, para un conjunto de operadores $\{K_i\}$, conocidos como operadores de Kraus, que juegan un papel importante en la descripción de mediciones para sistemas no ideales. Con estas herramientas teóricas se pueden abordar varios problemas, sin embargo este trabajo se enfocó en dar inicio en la exploración de operadores que describan completamente mediciones difusas para sistemas de varias partículas. 


El problema que se estableció en este trabajo fue el de explorar operadores de Kraus que describan completamente las mediciones difusas, concretamente para sistemas de dos partículas. Una medición difusa para un sistema de dos partículas se obtiene cuando se desea medir el observable $A\otimes B$ pero con cierta probabilidad $(1-p)$ se realiza la medición del observable $B\otimes A $. Una de las herramientas importantes que se abordó en el trabajo fueron las medidas POVM debido a que proporcionan un mapeo de salidas, el cual para cualquier estado de entrada da una una probabilidad de cada posible salida de medición. Además las medidas POVM pueden descomponerse en operadores de Kraus de manera no única. Los operadores de Kraus son especialmente útiles puesto que pueden describir lo que lo sucede al estado inmediatamente después de la medición. Sin embargo, cuando se realiza una medición lo que le sucede al estado inicial depende en como los POVM son implementados en el laboratorio. La descomposición más sencilla de los operadores POVM para el caso de dos partículas fue aplicarle la raíz cuadrada a un conjunto de operadores POVM, tales que produzcan un mapeo de las probabilidades. Además se realizó un programa en el cual se pueden obtener varios ejemplos numéricos en los cuales se mapea un operador de estado inicial a un estado final después de aplicarle la operación cuántica representada como suma de operadores de Kraus obtenidos con la descomposición de una medida POVM\@.


Por otro lado, se exploró otras formas de aproximarse al problema utilizando otras herramientas tales como los instrumentos cuánticos y la traza parcial. Para ello se plantearon tres alternativas de instrumentos cuánticos basados en la interpretación de la medición. Si la medición difusa se debió al intercambio de partículas, al intercambio de resultados o bien si el resultado que obtiene no es posible distinguirlo entre dos de los posibles resultados. Los instrumentos cuánticos utilizado contienen la información de las posibles salidas de la medición  y del efecto de la misma en el estado inicial. Se utiliza la traza parcial para obtener el mapeo de la posibilidades o el estado posterior a la medición descrito por un conjunto de operadores de Kraus.

El estudio de la descripción de la mediciones difusas para sistemas de dos partículas con los operadores de Kraus sentaron las bases para poder explorar operadores que puedan describir completamente la medición difusa para sistemas de más de dos partículas que se pretende continuar en el trabajo de graduación de licenciatura.
