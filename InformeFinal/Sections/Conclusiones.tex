\chapter{Conclusiones }

En este trabajo se estableció el marco teórico dentro del cual se puntualizan
algunas de las herramientas importantes para describir completamente las
mediciones difusas para sistemas de dos partículas. \cpnote{HAce 
falta un puente entre estas dos frases. es mas, mejor 
deja esto como un solo parrafo, y dices que vas a desglosar lo que 
se einvestigo. de ahi arrancas parrafo con matriz de dencidad y le sigues
parrafo a parrafo} Se investigó el formalismo
y las propiedades de la matriz de densidad, puesto que es una herramienta que
generaliza la representación de estados cuánticos. Los postulados de la
mecánica cuántica pueden ser reformulados cómodamente en el lenguaje de la
matriz de densidad y este operador tiene la ventaja que facilita trabajar en la
descripción de sistemas cuánticos cuyo estado es una mezcla estadística de
estados puros y en la descripción de subsistemas de un sistema cuántico
compuesto. 

Se estudió la teoría de las operaciones cuánticas debido a que son un medio
\cpnote{Como que un medio?}
capaz de describir los efectos de una medición en un sistema cuántico y cambios
de estados discretos. 
\cpnote{para que dices lo que sigue de este parrafo? porque es importante?}
Las operaciones cuánticas se definen como operaciones
completamente positivas que preservan la traza y son linealmente convexas en el
conjunto de matrices de densidad. Las operaciones cuánticas que preservan la
traza pueden escribirse de una forma conveniente como la suma de operadores,
para un conjunto de operadores $\{K_i\}$, conocidos como operadores de Kraus,
que juegan un papel importante en la descripción de mediciones para sistemas no
ideales. 

Con estas herramientas teóricas se pueden abordar varios problemas, no obstante
este trabajo se enfocó en dar inicio en la exploración de operadores de Kraus
que describan completamente mediciones difusas para sistemas de dos partículas
\cpnote{Son conclusiones. La primera parte de la frase no va. }.
Dichas mediciones difusas en sistemas de dos partículas son aquellas en las se
desea medir el observable $A\otimes B$ pero con cierta probabilidad $(1-p)$ el
aparato de medición confunde las partículas y en su lugar se realiza la
medición del observable $B\otimes A $.  \cpnote{Croe que lo fundamental 
aca es que hay un intercambio de particulas. Asi evitas usar formulas
y queda mas claro conceptualmente}

%El problema que se estableció en este trabajo fue el de investigar operadores
%de Kraus que describan completamente las mediciones difusas, concretamente
%para sistemas de dos partículas. 


\cpnote{Creo qeu hace falta resaltar al principio de los parrafos lo realmente importante
y no dar una receta. Reformula las conclusiones. Quizá es cosa de poner una frase
al inicio o moverle algo. Paro de revisar las conclusiones.}
Una de las herramientas importantes con las que se exploraron lo operadores de Kraus para mediciones difusas son las medidas POVM debido a que proporcionan un mapeo de salidas, el cual para cualquier estado de entrada da una una probabilidad de cada posible salida de medición. Además las medidas POVM pueden descomponerse en operadores de Kraus de manera no única. Los operadores de Kraus son especialmente útiles puesto que pueden describir lo que lo sucede al estado inmediatamente después de la medición. Sin embargo, cuando se realiza una medición lo que le sucede al estado inicial depende en como los POVM son implementados en el laboratorio. La descomposición más sencilla de los operadores POVM para el caso de dos partículas fue aplicarle la raíz cuadrada a un conjunto de operadores POVM, tales que produzcan un mapeo de las probabilidades. Además se realizó un programa en el cual se pueden obtener varios ejemplos numéricos en los cuales se mapea un operador de estado inicial a un estado al que se le aplica la operación cuántica representada como suma de operadores de Kraus obtenidos con la descomposición de una medida POVM\@.

Por otro lado, se exploró otra forma de aproximarse al problema utilizando instrumentos cuánticos. Para ello se planteó una alternativa de instrumento cuántico basados en si la medición difusa se debió al intercambio de partículas. %, al intercambio en los resultados que el aparato de medición registró o bien si con cierta probabilidad no es posible distinguir en que estado de salida se encuentra. 
 El instrumento cuántico utilizado contienen la información de las posibles salidas de la medición  y del efecto de la misma en el estado inicial. Adicionalmente, se utiliza la traza parcial para obtener el mapeo de la posibilidades o el estado posterior a la medición descrito por un conjunto de operadores de Kraus.

El estudio de la descripción de la mediciones difusas para sistemas de dos partículas con los operadores de Kraus sentaron las bases para poder explorar operadores que puedan describir completamente la medición difusa para sistemas de más de dos partículas que se pretende continuar en el trabajo de graduación de licenciatura.
