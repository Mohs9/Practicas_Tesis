\chapter{Conclusiones y trabajo futuro}

\section{Conclusiones}


En este trabajo se estableció el marco teórico dentro del cual se puntualizan algunas de las herramientas importantes para describir completamente las mediciones difusas para sistemas de dos partículas. Se investigó el formalismo y las propiedades de la matriz de densidad, puesto que es una herramienta que generaliza la representación de estados cuánticos. Los postulados de la mecánica cuántica pueden ser reformulados cómodamente en el lenguaje de la matriz de densidad y este operador tiene la ventaja que facilita trabajar en la descripción de sistemas cuánticos cuyo estado no se conoce  y la descripción de subsistemas de un sistema cuántico compuesto. 

Se estudió la teoría de las operaciones cuánticas debido a que son un medio capaz de describir la evolución de sistemas cuánticos abiertos y cambios de estados discretos. Las operaciones cuánticas se definen como operaciones completamente positivas que preservan la traza y son linealmente convexas en el conjunto de matrices de densidad. Las operaciones cuánticas que preservan la traza pueden escribirse de una forma conveniente como la suma de operadores, para un conjunto de operadores $\{K_i\}$, conocidos como operadores de Kraus, que juegan un papel importante en la descripción de mediciones para sistemas no ideales. Con estas herramientas teóricas se pueden abordar varios problemas, sin embargo este trabajo se enfocó en dar inicio en la exploración de operadores que describan completamente mediciones difusas para sistemas de varias partículas. %Estos operadores 


\begin{comment}



El problema que se estableció en este trabajo fue el de estudiar las operaciones PCE en sistemas de qubits que cumplen con las condiciones para ser canales cuánticos. Específicamente, se estudió el caso de 1 qubit. Una operación PCE (Pauli component erasing) es una operación lineal que borra las
componentes del vector de Bloch (generalizado, en el caso de n qubits). El problema en el contexto de los sistemas cerrados es trivial, todas las operaciones PCE representan dinámicas físicas que podrían atravesar los estados cuánticos. Sin embargo, para el caso de los sistemas abiertos la condición de
completa positividad debe ser cumplida por una operación PCE para que describa una evolución física. Se diseñó un método numérico para evaluar la condición de CP de las operaciones PCE y se encontró que 5 de las 8 operaciones PCE de 1 qubit son operaciones cuánticas: la identidad, el  canal totalmente despolarizante y tres canales que mapean la esfera de Bloch a una línea sobre cada uno de los ejes. El estudio de las operaciones PCE de 1 qubit sentaron las bases para el estudio  de una caracterización general de las operaciones PCE que se pretende continuar en el trabajo de graduación de licenciatura.
\end{comment}

\section{Trabajo futuro}