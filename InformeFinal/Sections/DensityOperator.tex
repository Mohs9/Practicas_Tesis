\chapter{Operador de densidad}\label{OpDensidad}
% Intro {{{
El operador de densidad es una herramienta en mecánica cuántica que describe el
estado cuántico de un sistema físico. Este operador, también conocido como
<<matriz de densidad>>, contiene todas las propiedades estadísticas de un sistema cuántico, incluso cuando no es posible describir el sistema mediante un estado puro. El operador de densidad es una generalización de los vectores de estado, puesto que estas matrices pueden representar estados mixtos. \cpnote{Tambien el vector de estado.}\rrnote{Arreglé la parte que se refería al vector de estado. }  En la primera sección de este capítulo, se presenta esta herramienta, comenzando por la definición y motivación de este operador. Luego, en la segunda sección  se presenta las propiedades y reformulación de los postulados de la mecánica cuántica utilizando este operador. Por último, en la tercera sección se expone el operador de densidad reducido, una de las aplicaciones más importantes del operador de densidad.
\cpnote{Para qe dejar esperando al lector? En ciencia somos muy directos. Mejor puedes poner aca la aplicacion, lo de traza parcial pues}. \rrnote{Ya he agregado la aplicación}

\cpnote{Usas corrector de ortografía? Veo algunos errores como medio obvios. Dale una pasada y luego me avisas}
\rrnote{Ya he pasado el corrector de ortografía}
% }}}
\section{Definición y motivación  del operador de densidad} % {{{
El lenguaje de los operadores de densidad proporciona un  medio conveniente
para describir los sistemas cuánticos, tales que su estado no es completamente
conocido. Al considerar un sistema cuántico que se encuentra en alguno de los
estados $|\psi_j \rangle $ con probabilidad $p_j$. El conjunto de todos los
posibles estados conforman un ensamble de estados puros del sistema, con lo
cual se puede enunciar la siguiente definición. 

\begin{definition}[\textbf{Operador de densidad}] El operador de densidad
$\rho$ correspondiente al ensamble de estados puros $\{p_j,|\psi_j \rangle \}$
es definido como {\cite{wilde2011classical}}
  	\begin{equation}{\label{defdensityoperator}}
  		\rho=\sum_{j}p_j|\psi_j\rangle \langle \psi_j|.
  	\end{equation}
\end{definition}
A menudo se utiliza el término matriz de densidad para referirse al operador de
densidad, en la práctica estos dos términos se utilizan indistintamente. Esta
formulación, es equivalente a la aproximación del vector de estado, sin embargo
a veces es mucho más fácil acercarse a problemas desde este nuevo punto de
vista.
 
Considérese una medición en un ensamble, de algún observable $\mathcal{O}$. Una
pregunta importante sería, ¿cuál es el valor promedio de $\mathcal{O}$ cuando
se reproduce un número muy grande de mediciones? La respuesta está dada por el
promedio $\langle \mathcal{O} \rangle$ del ensamble, el cual se define por
{\cite{sakurai2017modern}} \begin{equation}
 	\label{expectationvalue}
 	 \begin{split}
 		\langle \mathcal{O} \rangle &= \sum_{i}p_i \langle\psi_i|\mathcal{O}|\psi_i\rangle\\
 		&=\sum_i\sum_{j}\sum_k \langle\psi_i||\phi_j\rangle \langle \phi_j|\mathcal{O}|\phi_k\rangle \langle \phi_k||\psi_i\rangle\\
 		&=\sum_{j}\sum_k \left(\sum_{i} p_i \langle \phi_k|\psi_i\rangle  \langle\psi_i|\phi_j\rangle\right) \langle \phi_j|\mathcal{O}|\phi_k\rangle, 
 	\end{split}
\end{equation} donde $\{|\phi_j\rangle \}$ es una base más general del espacio de Hilbert del sistema. La última línea en la ecuación {\ref{expectationvalue}} es una
motivación para  definir el operador de densidad  en la forma de la ecuación
{\ref{defdensityoperator}}. El operador de densidad contiene toda la
información físicamente importante que se puede obtener acerca del ensamble
{\cite{sakurai2017modern}}\begin{equation}
 	\label{expectationvalue_traza}
 	\begin{split}
 		\langle \mathcal{O} \rangle &=\sum_{j}\sum_k \langle \phi_k|\rho|\phi_j\rangle \langle \phi_j|\mathcal{O}|\phi_k\rangle\\
 		&= \rm{tr}(\rho \mathcal{O}).
 	\end{split}
\end{equation}

Además las mediciones se pueden describir fácilmente utilizando matrices de densidad. Si se realiza una medición descrita por lo operadores $M_m$ y con un estado inicial $|\psi_i\rangle$, luego la probabilidad de obtener el resultado $m$, dado el estado $i$ es $p(m|i)=\langle \psi_i|M_m^{\dagger}M_m|\psi_i\rangle = \tr(M_m^{\dagger} M_m |\psi_i\rangle \langle\psi_i|)$. Por la ley de probabilidad total\footnote{Si  $X$ y $Y$ son dos variables aleatorias, entonces las probabilidades para $Y $ pueden expresarse en términos de las probabilidades para $X$, y las probabilidades condicionales para $ Y$ dada $X$,
		\[p(y) =\sum_x p(y|x)p(x),\] donde la suma sobre los valores de $x$ que $X$ puede tomar {\cite{nielsen_chuang_2010}}.}, la probabilidad de obtener un resultado $m$ es {\cite{nielsen_chuang_2010}}

\cpnote{Hay varias ecuaciones que no tiene puntuacion. Por ejemplo la 1.1 y la
1.3. Porfa revisa en todo el documento que todas tienen la puntuacion
apropiada. Puede ser que no requieran nada, o una coma o un punto. Tambien quiero que notes que cuando dejas un espacio entre la ecuación y el texto, lo toma como un parrafo nuevo. Porfa arregla eso (por ejemplo en la 1.2). Tambien quiero que me contestes los comentarios, por ejemplo diciendo que ya lo atendiste, o que estas en desacuerdo. Eso aplica para este comentario y todos los anteriores. }
\rrnote{Ya he corregido todos los espacios que estaban demás en las ecuaciones y sus respectivas puntuaciones.}\begin{equation}
	\label{probabilidad_m}
	\begin{split}
	p(m) &= \sum_{i}p(m|i)p_i\\
	&=\sum_i p_i\tr(M_m^{\dagger} M_m |\psi_i\rangle \langle\psi_i|)\\
	&=\sum_{i}\tr(M_m^{\dagger} M_m \rho)
	\end{split}
\end{equation} y el estado después de la medición será $|\psi_i^m\rangle=\frac{M_m|\psi_i\rangle}{\sqrt{\langle\psi_j|M_m^{\dagger}M_m|\psi_i\rangle}}$, por lo que el operador de densidad correspondiente al ensamble de estados $|\psi_i^m\rangle$ con probabilidades $p(i|m)$ estará dado por \[\rho_m=\sum_i p(i|m)\frac{M_m|\psi_i\rangle \langle \psi_i|M^{\dagger}}{\langle\psi_j|M_m^{\dagger}M_m|\psi_i\rangle}.\] Utilizando el teorema de Bayes, para $p(i|m)=\frac{p(m|i)p_i}{p_m}$ se obtiene finalmente que el operador de densidad $\rho_m$ es{\cite{nielsen_chuang_2010}} \begin{equation}
	\rho_m=\sum_i p_i \dfrac{M_m|\psi_i\rangle \langle \psi_i|M^{\dagger}}{\tr(M^{\dagger}_m M_m \rho )}=\dfrac{M_m \rho M_m^{\dagger}}{\tr (M^{\dagger}_m M_m\rho )}.
\end{equation}


Ahora si, por alguna razón, el registro de la medición se pierde. Se tendría un sistema cuántico en el estado $\rho_m$ con probabilidad $p(m)$, pero ya no se conocería el valor real de $m$. El estado del sistema cuántico estaría descrito por el operador densidad siguiente \begin{equation}
	\rho'=\sum_m p(m)\rho_m=\tr(M^{\dagger}_m M_m\rho )\dfrac{M_m\rho M_m^{\dagger}}{\tr (M^{\dagger}_m M_m \rho )}=M_m\rho M_m^{\dagger}.
\end{equation}


 
 Ahora, supóngase que la evolución de un sistema cerrado es descrito por el operador $U$. Si el sistema inicial se encuentra en el estado $|\psi_i\rangle$ con probabilidad $p_i$ luego de la evolución, el sistema se encontrará en el estado $U|\psi_i\rangle$ con probabilidad $p_i$. Luego, la evolución del operador de densidad está descrita por {\cite{nielsen_chuang_2010}}\[\rho=\sum_{i}p_i|\psi_i\rangle \langle \psi_i|\xrightarrow{U}\rho'=\sum_{i}Up_i|\psi_i\rangle \langle \psi_i|U^{\dagger}=U\rho U^{\dagger}.\]


Hasta el momento se puede ver con estas motivaciones que los postulados básicos de la mecánica cuántica relacionados con la medición y la evolución unitaria pueden ser reformulados en el lenguaje de los operadores de densidad. En la siguiente sección se profundiza esta reformulación de los postulados.


% }}}
\section{Propiedades del operador de densidad}\label{postulates} % {{{

En esta sección se desarrolla algunas de las características y propiedades de
los operadores de densidad. Asimismo se presenta la formulación de los
postulados de la mecánica cuántica. \cpnote{Creo qeu ya habiamos hablado de iniciar 
con ``Así como''. Creo qeu no se usa en español formal al iniciar un frase. Corrige
este y por ahí busca en lo que sigue a ver si lo vuelves a usar al inicio de una frase.}\rrnote{Ya he corregido la frase ``Así como'', en este párrafo y en el documento restante.}

Para iniciar se enuncia la siguiente proposición y el siguiente teorema. 

\begin{proposition}El operador de densidad es un operador hermítico.
	
\end{proposition}


\begin{proof}
	\begin{equation}
		\begin{split}
			\rho^\dagger&={\left(\sum_{i} p_i|\psi_i\rangle \langle \psi_i|\right)}^{\dagger}\\
			&=\sum_{i} p_i {\left(|\psi_i\rangle \langle\psi_i|\right)}^{\dagger}\\
			&=\sum_{i} p_i |\psi_i\rangle \langle\psi_i|\\
			&=\rho.
		\end{split}
	\end{equation}
	
\end{proof}

De acuerdo a Nielsen y Chuang {\cite{nielsen_chuang_2010}}:\begin{theorem}[\textbf{Caracterización del operador de densidad}] Un operador $\rho$ es un operador de densidad, asociado a algún ensamble $\{p_i, |\psi_i\rangle\}$ si y solo si este satisface las siguientes condiciones:
\begin{enumerate}
	\item $\rho$ tiene una traza igual a uno.
	\item $\rho $ es un operador \textit{positivo semidefinido}.
\end{enumerate}	
\end{theorem}


Ahora se procede a demostrar formalmente este teorema.


\begin{proof}
	Primero se supondrá que $\rho $ es un operador de densidad por lo que debe cumplir con la definición {\ref{defdensityoperator}}, al computar la traza se obtiene que \begin{equation*}
		\begin{split}
			{\rm tr(\rho)}&=\tr\left(\sum_{i} p_i|\psi_i\rangle \langle \psi_i|\right) \\
			&=	\sum_{i}p_i \tr(|\psi_i\rangle \langle \psi_i|)\\
			&=\sum_{i}p_i \sum_j\langle \psi_j|\psi_i\rangle \langle \psi_i|\psi_j\rangle\\
			&=\sum_{i}\sum_j p_i \delta_{ij} \delta_{ij}\\
			&=\sum_i p_i=1.\\
		\end{split}
	\end{equation*}

Ahora para la segunda condición, se toma un vector de estado arbitrario $|\varphi \rangle$ \begin{equation*}
	\begin{split}
	\langle \varphi |	\rho|\varphi \rangle&=\sum_{i}p_i\langle \varphi |\psi_i\rangle \langle \psi_i|\varphi \rangle \\
	&=\sum_{i}p_i| \langle \psi_i|\varphi \rangle|^2 \ge 0.\\
	\end{split}
\end{equation*}
La desigualdad se sigue porque cada $p_i$ es una probabilidad y por lo tanto no es negativa. Por la definición de operadores positivos semidefinidos, $\rho$ cumple la segunda condición.


Ahora se debe mostrar el converso del enunciado anterior. Suponiendo $\rho$ es un operador que cumple las dos condiciones, entonces se debe demostrar que es un operador de densidad.

Dado que $\rho$ es un operador positivo semidefinido entonces tiene una descomposición espectral dada por la siguiente ecuación {\cite{nielsen_chuang_2010}} \[\rho=\sum_i \lambda_i |i\rangle \langle i|,\] con $|i\rangle$ vectores ortogonales y $\lambda_i$ son valores propios reales no negativos, de la matriz $\rho$. Además satisface que la traza es uno 
\[\tr(\rho)=1=\sum_i\lambda_i.\]

Luego, un sistema estado $|i\rangle$ con una probabilidad $\lambda_i$ tendrá un operador asociado $\rho$. Esto significa que, el ensamble $\{\lambda_i, |i\rangle\}$  corresponde al operador $\rho$ que cumple con {\ref{defdensityoperator}}.


\end{proof}



Este teorema  permite obtener una definición equivalente del operador de densidad y con ello es posible reformular los postulados de la mecánica cuántica utilizando este operador. Nielsen y Chuang {\cite{nielsen_chuang_2010}} presentan la siguiente reformulación.


\setlength{\leftskip}{1cm}

 \textbf{Postulado 1:} Cualquier sistema físico tiene asociado un espacio de Hilbert $\mathcal{H}$, conocido como el \textit{espacio de estado} del sistema. El sistema está completamente descrito por su \textit{operador de densidad}, el cual es un operador $\rho$ positivo semidefinido con traza uno, actuando en el espacio de estado del sistema. Si un sistema cuántico está en el estado $\rho_i$ con una probabilidad $p_i$, entonces el operador de densidad para el sistema es $\sum_{i}p_i\rho_i$.


\textbf{Postulado 2:} La evolución de un sistema cuántico cerrado está descrita por una \textit{transformación unitaria}. Esto es, el estado $\rho$ del sistema en el tiempo $t_1$ está relacionado con el estado $\rho'$ del sistema en el tiempo $t_2$ por un operador unitario $U$ que depende solamente de $t_1$ y $t_2$, \begin{equation}\label{postulado 2}
\rho'=U\rho U^{\dagger}.
\end{equation}


\textbf{Postulado 3:} Las mediciones cuánticas están descritas por una colección $\{M_m\}$ de \textit{operadores de medición}. Estos operadores actúan en el espacio de estado del sistema que se está midiendo. El índice $m$ se refiere a las salidas de la medición que pueden ocurrir en el experimento. Si el estado del sistema cuántico es $\rho$ inmediatamente antes de la medición, luego la probabilidad de obtener el resultado $m$ está dada por \begin{equation}\label{probaility3postulate}
	p(m)=\tr(M^\dagger M \rho)
\end{equation} y el estado del sistema después de la medición es  \begin{equation}\label{state3postulate}
	\rho_m'=\dfrac{M_m\rho M_m^\dagger}{\tr(M_m^\dagger M_m \rho)}.
\end{equation}

Los operadores de medición satisfacen la relación de completitud \begin{equation}\label{completitud3postulate}
	  	\sum_m M_m^\dagger M_m=\mathds{1}.
\end{equation}

\textbf{Postulado 4:} El espacio de Hilbert de un sistema físico compuesto es el producto tensorial de los espacios de Hilbert individuales de cada uno de los sistemas que componen al sistema total. Es decir, si el sistema total se compone de $N$ subsistemas, entonces \begin{equation}\label{Htotal4postulado}
	\mathcal{H}_{\text{total}}=\mathcal{H}_1\otimes \mathcal{H}_2\otimes \cdots \otimes \mathcal{H}_N.
\end{equation}
  Más aún, si el sistema número $i$, con $i=1,2,\ldots,N$, está preparado en el estado $\rho_i$, luego el estado del sistema total será  \begin{equation}\label{rhototal4postulado}
	\rho_{\text{total}}=\rho_1\otimes \rho_2 \otimes \cdots \rho_N.
\end{equation}


\setlength{\leftskip}{0pt}

Con esta reformulación de los postulados de la mecánica cuántica se tiene la
ventaja que es más fácil trabajar en la descripción de sistemas cuánticos cuyos estados son mixtos y la descripción de subsistemas de un sistema cuántico compuesto. Ahora, es necesario discutir estos conceptos, nuevas definiciones y hechos sobre operadores de densidad. \cpnote{Acá hay como una contradiccion. Dice que el estado no se conoce, pero en la linea 
anterior hablas del estado del sistema. Ese es un punto delicado. Si uieres intenta iterar y 
si no, hablamos un poco al respecto. Ahora ya lei el parrafo siguiente y tienes una solucion 
simple. Decir que puedes formular la cuantiva para estados mixtos, pero luego de discutir los 
conceptos. }\rrnote{Ya arregle la contradicción que había.}

Un sistema cuántico cuyo estado es conocido exactamente, se dice que
se encuentra en \textit{estado puro}. En este caso, el operador de densidad es
simplemente $\rho=|\psi \rangle \langle \psi|$. De lo contrario, $\rho$ está en
un \textit{estado mixto}\footnote{En algunas referencias, se  usa el término
<<estado mixto>> para incluir ambos estados cuánticos puros y mixtos. El origen
de este uso puede ser que no se asume necesariamente que un estado es puro.Además, el término <<estado puro>> se usa a menudo en referencia a un vector de
estado $|\psi\rangle $, para distinguirlo de un operador de densidad	$\rho$.
}; se dice que es una mezcla estadística de los diferentes estado puros en el ensamble de
$\rho$ {\cite{nielsen_chuang_2010}}. 

Estos conceptos dan pie a incluir las siguientes definiciones {\cite{wilde2011classical}}: 
\begin{definition}[\textbf{Estado máximamente mezclado} \cpnote{Normalmente se le dice 
estado máximamente mezclado. O donde lo viste asi?}\rrnote{ya corregí la traducción.}] El estado de máxima mezcla
$\sigma$ es el operador de densidad correspondiente a un conjunto uniforme de
estados ortogonales $\{\frac{1}{d},|x\ra\}$, donde $d$ es la dimensión del
espacio de Hilbert. El estado de máxima mezcla $\sigma$ es entonces igual a
\begin{equation}\label{maximallyMixedState}
		\sigma \equiv \dfrac{1}{d}\sum_x|x\rala x|=\dfrac{\mathds{1}}{d}.
	\end{equation}
\end{definition}


\begin{definition}[\textbf{Pureza}] La pureza $P(\rho)$ de un operador de densidad $\rho$ se define como sigue {\cite{wilde2011classical}} \begin{equation}
	P(\rho)\equiv\tr(\rho^\dagger\rho)=\tr(\rho^2).
\end{equation}

	\end{definition}

 %\cpnote{Esta frase está muy rara. De donde la sacas? Que es el ruido de un estado?}

\begin{proposition}
	La pureza de un operador de densidad es igual a uno si y solo si es un estado puro, y la pureza de un operador de densidad es estrictamente menor que uno si y solo si es un estado mixto.
	\end{proposition}


\begin{proof}
	Sea un operador de densidad dado por a siguiente ecuación, \[\rho=\sum_i p_i |\psi_i\rangle \langle \psi_i|\] con $p_i\ge 0$ y $\sum_{i}p_i=1$, se tiene que:
	
	\begin{equation*}
		\begin{split}
			\rho^2&=\sum_{i,j} p_i p_j|\psi_i\rala \psi_i|\psi_j\rala \psi_j|\\
			&=\sum_{i,j} p_i p_j|\psi_i \rala \psi_j|\delta_{ij}\\
			&=\sum_i p_i^2|\psi_i \rala \psi_i|.
			\end{split}
	\end{equation*}
	Luego se obtiene la pureza, \[\tr(\rho^2)=\tr{\left(\sum_i p_i^2|\psi_i \rala \psi_i|\right)}=\sum_i p_i^2 \tr(|\psi_i \rala\psi_i|)=\sum_i p_i^2.\]
	
	Ahora, dado que $p_i^2 \le p_i $ puesto que $0 \le p_i \le 1$. Luego, se tiene \[\sum_i p_i^2\le\sum_i p_i=1.\]
	
	Suponiendo que $\tr(\rho^2)=1$, entonces $\sum_i p_i^{2}=1$. Para $0<p_i<1$ se tiene que  $p_i^2<p_i$ y solamente una $p_i$ debe ser 1. Por lo tanto, $\rho=|\psi_i\rala\psi_i|$ es un estado puro.
	 \cpnote{Creo que sobra el ``Y''}\cpnote{para $p_i=0$ 
se cumple la igualdad\ldots} \cpnote{Preferira quitar la ecuacion. Es confuso, 
porque es solo para este caso}. \cpnote{También esta raro 
comenzar la frase con Y en este caso} 
\cpnote{Porfa arregla este parrafo. Creo qe lo escribiste cansada. }\rrnote{He quitado los ``Y'' del inicio y simplificado el párrafo }  
    
     
 \cpnote{Esta palabra no existe en español, segun la RAE} \rrnote{Ya he cambiado esta frase} Por otro lado, suponiendo que $\rho
$ es puro, $\rho=|\psi\rala \psi|$
\[\tr(\rho^2)=\tr(|\psi\rala \psi|\psi\rala \psi|)=\tr(|\psi\rala
\psi|)=\la\psi|\psi\ra=1.\]
 \cpnote{Trata de quitar todas las fraces que comiences con ``Y'' mas bien ver si 
lo upuedes uitar o substituir.}\rrnote{Ya he revisado del resto y quitado los ``Y'' del inicio.} Por otro lado, si $\rho $ es mixto, luego $\rho=\sum_i
p_i|\psi_i\rangle \langle \psi_i|$, con todo $p_i<1$ \[\tr(\rho^2)=\sum_i p_i^2
\tr(|\psi_i \rala\psi_i|)=\sum_i p_i^2<\sum_i p_i=1.\]
	\end{proof}



Antes de pasar a la siguiente sección, es indispensable discutir qué clase de
ensambles pueden dar una matriz de densidad particular y cuándo dos conjuntos
de vectores $\{|\psi_i\ra\}$ y$ \{|\phi_j\ra \}$ generan el mismo operador de
densidad. La respuesta a estas interrogantes tiene muchas aplicaciones en
información y computación cuántica. Para ello se presenta el siguiente
teorema y su demostración que enuncian Nielsen y Chuang
{\cite{nielsen_chuang_2010}}. 
\begin{theorem}[\textbf{Libertad unitaria en el ensamble para matrices de densidad}]
Los conjuntos $\{|\psi_i\ra\}$ y $ \{|\phi_j\ra \}$, no necesariamente
normalizados, generan la misma matriz de densidad si y solo si
\begin{equation}\label{teorema2.4}
|\psi_i\ra= \sum_{j}u_{ij}|\phi_j\ra,
\end{equation} donde $u_{ij}$ es una matriz unitaria compleja, con índices $i$ y $j$. Se completa con vectores 0 adicionales en caso uno de los conjuntos tenga menos elementos.
\end{theorem}

\begin{proof}
	
	Primero, se supone que $|\psi_i\ra= \sum_{j}u_{ij}|\phi_j\ra$ para alguna matriz $u_{ij}$. Luego \begin{equation*}
		\begin{split}
			\sum_{i}|\psi_i\ra \la \psi_i|&=\sum_{ijk} u_{ij}u_{ik}^*|\phi_j\ra \la \phi_k|\\
			&=\sum_{jk}\left(\sum_i  u_{ki}^\dagger u_{ij}\right) |\phi_j\ra \la \phi_k|\\
			&=\sum_{jk}\delta_{jk}|\phi_j\ra \la \phi_k|\\
			&=\sum_{j} |\phi_j\ra \la \phi_j|,\\
	\end{split}
	\end{equation*}	
	con lo que se prueba que $\{|\psi_i\ra\}$ y$ \{|\phi_j\ra \}$ generan el mismo operador. Notar que para estado normalizados $\{|\tilde{\psi}_i\ra\}$ y$ \{|\tilde{\phi}_j\ra \}$ y distribuciones de probabilidad $p_i$ y $q_j$, si $\sqrt{p_i}|\tilde{\psi}_i\ra=\sum_{j} u_{ij} \sqrt{q_j}|\tilde{\phi}_j\ra$ se obtendrá el mismo operador de densidad $\rho=\sum_i p_i |\tilde{\phi}_i\rala\tilde{\phi}_i|= \sum_j q_j|\tilde{\phi}_j \rala\tilde{\phi}_j| $.
\cpnote{Creo que es importante acá mencionar que pasa con las probabilidades. Pirque no las
tienes ahi, pero si estan ahi de manera implicita.}\rrnote{Mencioné lo de las probabilidades.}
	
Por otro lado, \cpnote{revisa en todo el documento donde usas esa palabra}\rrnote{Ya quité la palabra conversamente del documento} se supone que ambos conjuntos generan el mismo operador
\[Q=\sum_{i}|\psi_i\ra \la \psi_i|=\sum_{j} |\phi_j\ra \la \phi_j|.\]
	
Luego la  descomposición espectral del operador es $Q = \sum_k \lambda_k |k \rala k|$ tal que los vectores $ |k\ra$ son ortonormales, y $\lambda_k$ son estrictamente  positivos. Sea $|\psi\ra$  un vector ortonormal al espacio generado por $|\tilde{k}\ra=\sqrt{\lambda_k}|k\ra$, de ello $\la \psi|\tilde{k}\ra \la \tilde{k}|\psi\ra=0$ para todo $ k$, y luego se ve que 	\[0=\la \psi |Q|\psi\ra=\sum_{i}\la\psi|\psi_i\rala \psi_i|\psi \ra =\sum_i |\la\psi|\psi_i\ra|^2 \] y $\la\psi|\psi_i\ra=0$ para todo $i$ y todo $|\psi \ra$ ortonormal al espacio generado  por el $|\tilde{k}\ra$. Esto se sigue que cada $|\psi_i\ra $ puede expresarse como una combinación lineal de los vectores  $|\tilde{k}\ra$, $|\psi_i\ra=\sum_k c_{ik} |\tilde{k}\ra$. 
	
Como  $Q=\sum_k  |\tilde{k}\rala \tilde{k}|=\sum_{i}|\psi_i\rala\psi_i|$, se ve
que \[\sum_k |\tilde{k}\rala \tilde{k}|=\sum_{kl} \left(\sum_i
c_{ik}c_{il}^{*}\right)|\tilde{k}\rala \tilde{l}|,\] los operadores $|\tilde{k}
\rala \tilde{l}|$ es fácil ver  que son linealmente independientes y debe ser
que $\sum_i c_{ik}c_{il}^{*}=\delta_{kl}$. Esto asegura que se puedan agregar
columnas extra a $c$ para obtener una matriz $v$ tal que
$|\psi_i\ra=\sum_{k}v_{ik}|\tilde{k}\ra  $, donde se han agregado  vectores cero a la
lista de $|\tilde{k}\ra$. Similarmente, se puede encontrar una matriz unitaria
$w$ tal que $|\phi_j\ra=\sum_{k}w_{jk}|\tilde{k}\ra$. Luego $|\psi_i\ra=\sum_j
u_{ij}|\phi_j\ra$, donde  $u=vw^\dagger$ es unitaria. De manera similar se concluye que para ensambles que generan el mismo operador de densidad $\rho=\sum_i p_i |\tilde{\phi}_i\rala\tilde{\phi}_i|= \sum_j q_j|\tilde{\phi}_j \rala\tilde{\phi}_j| $ entonces se tiene una matriz unitaria $u$ tal que  $\sqrt{p_i}|\tilde{\psi}_i\ra=\sum_{j} u_{ij} \sqrt{q_j}|\tilde{\phi}_j\ra$.
\end{proof}
\cpnote{Veo muy complicada esta prueba. De donde la sacaste? Porque así de larga?}\rrnote{De Nielsen y Chuang en la página 104$-$105.}

% }}}
\section{El operador de densidad reducido} % {{{

Una de las aplicaciones más importantes del operador de densidad es el operador
de densidad reducido, que es una herramienta descriptiva para subsistemas de un
sistema cuántico. La matriz de densidad reducida es útil es
para el análisis de sistemas cuánticos compuestos y será discutida
en esta sección.


Supóngase que se tiene sistemas $A$ y $B$, cuyos estados están descritos por el
operador de densidad $\rho_{AB}$. El operador de densidad reducido o local para
el sistema $A$ es definido por  
\begin{equation}
	\rho_A=\tr_B(\rho_{AB}),
\end{equation} 
donde $\tr_B$ es un mapeo de operadores conocidos como la traza parcial sobre
el sistema $B$.\cpnote{``Y''}\rrnote{ya lo corregí.}La traza parcial tiene la siguiente definición, presentada
por Wilde {\cite{wilde2011classical}}.

\begin{definition}[\textbf{Traza parcial}]
Sea  $ \rho_{AB}$ un operador cuadrado actuando en un producto tensorial del
espacio de Hilbert $\mathcal{H}_A \otimes \mathcal{ H}_B$ y sea $\{|l\ra_b\}$
una base ortonormal para el espacio $\mathcal{H}_B$. Luego la traza parcial
sobre el sistema $\mathcal{H}_B$ está definida como sigue: 
\begin{equation}
		\tr_B(\rho_{AB})\equiv\sum_{l} (\mathds{1}\otimes\la l|_B)\rho_{AB}(\mathds{1}\otimes|l\ra_B).	
\end{equation} 
Por simplicidad, no se suele escribir los operadores identidad y se escribe de la siguiente forma:
\begin{equation}
	\tr_B(\rho_{AB})\equiv\sum_{l} \la l|_B\rho_{AB}|l\ra_B.
\end{equation}
\end{definition}

Por la misma razón que la definición de la traza es invariante bajo la elección
de una base ortonormal, lo mismo es cierto para la operación de traza parcial.
También se observa, a partir de la definición anterior, que la traza parcial es
una operación lineal.

En conclusión, dado un operador $\rho_{AB}$ que describe un estado conjunto de
los sistemas $A$ y $B$, es posible calcular un operador de densidad local
$\rho_A$, que describe el estado local de $A$ si el sistema $B$ es inaccesible
para $A$.


Según Wilde {\cite{wilde2011classical}}, existe una forma alternativa de
describir la traza parcial, la cual es útil estar consciente. Para un estado
simple de la forma  
\begin{equation}
	|x_1\rala x_2|_A\otimes|y_1\rala y_2|_B,
\end{equation} 
la acción de la traza partial es como sigue: 
\begin{equation}
	\begin{split}
	\tr_B(	|x_1\rala x_2|_A\otimes|y_1\rala y_2|_B)&=|x_1\rala x_2|_A\tr(|y_1\rala y_2|_B)\\
	&=|x_1\rala x_2|_A \la y_2|y_1\ra,
	\end{split}
\end{equation}
donde se calcula la traza del segundo sistema para obtener el operador de
densidad local del primero.

Esto se puede generalizar para un operador de densidad arbitrario $\rho_{AB}$,
con una base ortonormal $\{|i \ra_A \otimes |j\ra_B \} _{i,j}$ para un estado
conformado por dos sistemas:
\begin{equation}
	\begin{split}
	\rho_{AB}&=\sum_{i,j,k,l}\lambda_{i,j;k,l}(|i\ra_A \otimes|j\ra_B)(\la k|_A\otimes \la l|_B)\\
	&=\sum_{i,j,k,l}\lambda_{i,j;k,l} |i\rala k|_A\otimes|j\rala l|_B,
	\end{split}
\end{equation}
los coeficientes $\lambda_{i,j;k,l}$  son los elementos de la matriz $\rho_{AB}$ en la base respectiva.


Luego al aplicar la traza parcial y usando la linealidad de la traza, se obtiene 
\begin{equation}
	\begin{split}
		\rho_A&=\tr_B\left(\sum_{i,j,k,l}\lambda_{i,j;k,l} |i\rala k|_A\otimes|j\rala l|_B\right)\\
		&=\sum_{i,j,k,l}\lambda_{i,j;k,l} \tr_B(|i\rala k|_A\otimes|j\rala l|_B)\\
		&=\sum_{i,j,k,l}\lambda_{i,j;k,l}|i\rala k|_A \tr(|j\rala l|_B)\\
		&=\sum_{i,j,k,l}\lambda_{i,j;k,l}|i\rala k|_A \la l|j\ra \\
		&=\sum_{i,j,k}\lambda_{i,j;k,j}|i\rala k|_A\\
		&=\sum_{i,k}\left(\sum_j\lambda_{i,j;k,j}\right)|i\rala k|_A.
	\end{split}
\end{equation}

La razón principal para estudiar el operador de densidad reducido es que es la única operación que proporciona la descripción correcta de observables para subsistemas de un sistema compuesto. 

Según Nielsen y Chuang {\cite{nielsen_chuang_2010}}, si  $\mathcal{M}$ es un observable en el sistema descrito en el sistema de Hilbert $\mathcal{H}_A$, y se tiene algún dispositivo de medición, el cual es capaz de realizar la medición de $\mathcal{M}$. Sea $\tilde{\mathcal{M}}$ un observable para la misma medición, realizada en el sistema compuesto $\mathcal{H}_A\otimes \mathcal{H}_B$. Entonces, es necesario mostrar que $\tilde{\mathcal{M}}$ debe ser igual a $\mathcal{M}\otimes \mathds{1}_B$. Notar que si el sistema $\mathcal{H}_A\otimes \mathcal{H}_B$ está preparado en el estado $|m \ra \otimes|\psi\ra $, donde $|m\ra $ es un vector propio de $\mathcal{M}$ con valor propio $m$ y $|\psi\ra$ es un estado de $\mathcal{H}_B$, luego el dispositivo de medición debe proporcionar  el resultado $m$ para la medición, con probabilidad uno. En consecuencia, si $P_m$ es el operador de proyección al espacio propio de $m$ del observable $\mathcal{M}$, entonces el operador de proyector correspondiente para $\tilde{\mathcal{M}}$ es $P_m\otimes \mathds{1}_B$. Luego, se tiene que \[\mathcal{\tilde{M}}=\sum_m mP_m\otimes \mathds{1}_B=\mathcal{M}\otimes \mathds{1}_B.\]

La traza parcial da como resultado las estadísticas de medición correctas para las observaciones en una parte del sistema. Para verlo, suponga que se realiza una medición el sistema $\mathcal{H}_A$ descrito por el observable $\mathcal{M}$. La consistencia física requiere que  para asociar un \textit{estado}, $\rho_A$, al sistema $\mathcal{H}_A$, debe tener la propiedad que los promedios de medición sean los mismos ya sea que se calculen a través de $\rho_A$ o $\rho_{AB}$, {\cite{nielsen_chuang_2010}} \begin{equation}\label{promedioDeMedicion}\tr(\mathcal{M}\rho_A)=\tr(\mathcal{\tilde{M}}\rho_{AB})=\tr((\mathcal{\tilde{M}}\otimes \mathds{1}_B)\rho_{AB}).\end{equation} 

De acuerdo a Nielsen y Chuang {\cite{nielsen_chuang_2010}} la ecuación {\ref{promedioDeMedicion}} es satisfecha si se escoge $\rho_A\equiv \tr_B(\rho _{AB})$. De hecho, la traza parcial debe ser la única función que cumpla esta propiedad. Para comprobar la propiedad de unicidad, se toma  $f(\cdot)$ como algún mapeo de operadores de densidad en $\mathcal{H}_A\otimes \mathcal{H}_B$ a los operadores de densidad en $\mathcal{H}_A$ tal que \[\tr(\mathcal{M} f(\rho_{AB} )) = \tr((\mathcal{M}\otimes \mathds{1}_B )\rho_{AB} ),\] para todos los observables $\mathcal{M}$. 

Sea $\mathcal{M}_i$ una base ortonormal de operadores para el espacio de operadores Hermíticos con respecto al producto internos $(X,Y) \equiv \tr(XY )$. Luego la expansión $f (\rho_{AB} )$ en esta base resulta que\[\begin{split}
	f(\rho_{AB})=&\sum_{i}\mathcal{M}_i\tr(\mathcal{M}_i f(\rho_{AB}))\\
	=&\sum_{i}\mathcal{M}_i \tr((\mathcal{M}_i\otimes \mathds{1}_B)\rho_{AB}),\\
\end{split}\] de ello se sigue que $f$ está únicamente determinada por la ecuación {\ref{promedioDeMedicion}}. Más aún, la traza parcial satisface {\ref{promedioDeMedicion}}, entonces es la única función que tiene esta propiedad {\cite{nielsen_chuang_2010}}.

Con el operador de densidad reducido se finaliza el marco teórico del lenguaje
del operador de densidad necesario para el objetivo de este proyecto. En la
siguiente sección, se estará hablando sobre operadores POVM que son de interés
en este trabajo.



\cpnote{Aca falta algo importante. Porque es razonable usar justo esta deifinción de traza parcial? 
Hay un ``box'' en el chuang donde discuten eso. Porfa incluye, en tu lenguaje y citando apropiadamente, esa discusión. }\rrnote{Y he agregado la discusión}







% }}}

