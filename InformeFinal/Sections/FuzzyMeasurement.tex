\chapter{Mediciones difusas}


En el capítulo {\ref{MedidaPOVM}} y en la sección {\ref{Medicion_RepresentacionDeKraus}} se habló sobre medidas POVM y sobre mediciones. En este capítulo se presentan las mediciones difusas de sistemas cuánticos de dos partículas (que puede ser generalizado para más de dos partículas) en las que se puden identificar partículas individuales, sin embargo, siempre hay una probabilidad de identificarlas erróneamente. Estas detecciones imperfectas se exponen en la referencia {\cite{Pineda_2021}}. También se utiliza el lenguaje de las operaciones cuánticas por medio de los operadores de Kraus para poder describir los sistemas cuánticos de dos partículas.

La estructura de este capítulo es la siguiente. En la sección 4.1 se establece el problema de las mediciones difusas para un sistema de dos partículas. Luego, en la sección 4.2, se discutirán algunos conjuntos de operadores de Kraus que describan los efectos de las mediciones en el estado de entrada del sistema para dos partículas, asimismo, que proporcionen la probabilidad de cada resultado de medición posible para cualquier estado inicial. Finalmente, en la sección 4.3 se presentan algunos ejemplos numéricos de las mediciones y su descripción utilizando los operadores en la sección 4.2.



\section{Mediciones difusas para dos partículas}

Pineda, Davalos, Viviescas y Rosado {\cite{Pineda_2021}}, proponen el siguiente ejemplo. <<Una cadena de iones se hace brillar  y se obtiene una señal fluorescente. Sin embargo, debido a la imperfecciones del detector, no es posible determinar
determinar el origen exacto de la señal fluorescente. La información obtenida en este caso se vuelve borrosa, pero su cuantificación aún es posible>>.

Si se  realiza una  medición en un sistema de muchas partículas, pero no se está seguro en cual partícula esta medición fue aplicada, se obtiene una medición difusa. En el caso de un sistema de dos partículas, en el cual se desea realizar una medición del observable $A\otimes B$, el dispositivo de medición confunde las partículas con una probabilidad $(1-p)$, y a veces en su lugar, realiza la medición de $B\otimes A$. Luego, si $\rho$ es el estado inicial del sistema, el valor esperado para la salida de la medición difusa (FM por sus siglas en inglés), será 
\begin{equation}
    \la A\otimes B\ra_{\mathcal{F_{\text{2p}}(\rho)}}=\tr(\mathcal{F_{\text{2p}}(\rho)}A\otimes B)=p\tr(\rho A\otimes B)+(1-p)\tr(\rho B\otimes A),
\end{equation}



donde el operador \textit{difuso } $\mathcal{F}_{\text{2p}}(\rho)$ se define como sigue 

\begin{equation}\label{operador_difuso}
    \mathcal{F}_{\text{2p}}(\rho)\equiv p\rho+(1-p)S_{01}\rho S_{01}^\dagger,
\end{equation}

 y  $S_{ij}$ es la operación de intercambio \textit{SWAP} con respecto a las partículas $i$ y $j$ del estado del sistema {\cite{Pineda_2021}}. Esto puede generalizarse para un sistema de $n$ partículas, con un operador de permutación $P$.%Si el aparato de medición equivoca las partículas con una probabilidad $p_P$, de acuerdo a la permutación $P$, la salida de la medición difusa FM, por sus siglas en inglés, del operador $M $ es $\tr(M\mathcal{F}[\rho])$ . 

 \section{Operadores de Kraus para mediciones difusas}

 Primero se considera un sistema conjunto $\mathcal{H}=\mathcal{H}_1\otimes \mathcal{H}_2$, en el que se desea medir el observable $A\otimes B$. Pero el aparato de medición confunde las partículas y con una probabilidad $(1-p) $ realiza la medición del observable $B\otimes A$ (notar que en general $[A\otimes B,B\otimes A]\ne 0$). 

   
\subsection{Aproximaciones utilizando instrumento cuánticos}

Una forma útil de aproximarse a describir completamente las mediciones difusas es considerar una herramienta conocida como \textit{instrumento cuántico}, el cual es un mapeo que toma un operador de entrada y tiene como salida un ensamble que correlaciona un sistema clásico que contiene la salida de la medición y un sistema cuántico que contiene el estado posterior a la medición. 


Por lo que si las salidas de la medición se registran en un sistema clásico $\{|\alpha\rala \alpha |\}$ modelados como un conjunto ortonormal de operadores de proyección. Y $\{\E_\alpha(\rho)\}$ es un conjunto de operaciones cuánticas. Para describir el instrumento cuántico $\mathcal{I}$, los mapeos $\E_\alpha$ serán usados para describir  el estado posterior a la medición condicionados por la salida clásica de la medición $\alpha$. El instrumneto cuántico se define entonces:

\begin{equation}
    \mathcal{I}(\rho)\equiv\sum_\alpha |\alpha\rala\alpha|\otimes \E_\alpha(\rho),
\end{equation}


donde el sistema clásico servirá como un apuntador que indica que operadores aplicar al estado inicial. Esto es muy útil debido a que es una forma resumida de describir completamente la medición difusa y no es necesario considerar mediciones selectivas.


Para obtener la salida de la medición y su repectiva probabilidad se puede realizar la traza parcial sobre el sitema cuántico y para obtener el mapeo completamente positivo y que preserva la traza para una medida no selectiva basta con trazar parcialmente sobre el sistema clásico o apuntador.



Utilizando instrumentos cuánticos las mediciones difusas se pueden entender como tres alternativas diferentes. 

\subsubsection{Primera alternativa}
La primera de ellas es considerar aplicar el operador difuso definido en la ecuación {\ref{operador_difuso}} al estado inicial y luego aplicarle una medida proyectiva, para modelar el sistema cuántico. El sistema clásico simplemente serán los operadores de proyección de las salidas de medición.

\begin{equation}
    \begin{split}
        \mathcal{I}_1(\rho)&=\sum_{a_j,b_k}P_{a_j,b_k}\otimes P_{a_j,b_k} \mathcal{F}_{2p}(\rho) P_{a_j,b_k}\\
        &=\sum_{a_j,b_k}P_{a_j,b_k}\otimes[p P_{a_j,b_k}\rho P_{a_j,b_k}+(1-p)P_{a_j,b_k}S\rho S^\dagger P_{a_j,b_k}],
\end{split}
\end{equation}

donde $j,k=0,1$ y $P_{a_j,b_k}=|a_j b_k\rala a_j b_k|$ con $a_j$ y $b_k$ son los valores propios de los operadores $A$ y $B$ respectivamente.


Notar que si el sistema apuntador indica que la salida de le medición es $a_j,b_k$, el estado de salida estará en el subespacio que indica el sistema clásico.

\subsubsection{Segunda alternativa}


En la segunda alternativa se aplica el operador difuso al sistema clásico y el sistema cuántico es simplemente aplicar una medición proyectiva.

\begin{equation}
    \begin{split}
        \mathcal{I}_2(\rho)&=\sum_{a_j,b_k}\mathcal{F}_{2p}(P_{a_j,b_k})\otimes P_{a_j,b_k} \rho P_{a_j,b_k}\\
        &=\sum_{a_j,b_k}[pP_{a_j,b_k}+(1-p)SP_{a_j,b_k}S^\dagger]\otimes P_{a_j,b_k} \rho P_{a_j,b_k}\\
        &=\sum_{a_j,b_k} pP_{a_j,b_k} \otimes P_{a_j,b_k} \rho P_{a_j,b_k}+(1-p)SP_{a_j,b_k}S^\dagger\otimes P_{a_j,b_k} \rho P_{a_j,b_k},\\
    \end{split}
\end{equation} 


En esta alternativa al aplicarle el operador difuso al sistema clásico se puede notar que aunque el apuntador indique que el resultado es $|b_k a_j\ra$ es posible que con una probabilidad $(1-p)$, el resultado real sea $|a_j b_k\ra$.



\subsubsection{Tercera alternativa}
Para la tercera alternativa se considera un tipo medición difusa diferente a las dos anteriores. En esta alternativa se asume que con una probabilidad $q$ se realiza una medida proyectiva ideal, pero con una probalidad $(1-q)$ se realiza una medición en la que no es posible distinguir entre los posibles estados.

Para esta alternativa se introduce un nuevo operador de proyección $P'_{a_j,b_k}=|a_j b_k\rala a_j b_k|+|b_k a_j\rala b_k a_j|$, donde los estados $|a_j b_k\rala a_j b_k|$ son ortogonales con $|b_k a_j\rala b_k a_j|$. El operador $P'_{a_j,b_k}$ se interpreta como la falta de conocimiento si el estado se encuentra en $|a_j b_k\ra$ o en $|b_k a_j\ra$. 



Con una probabilidad $q$ el instrumento cuántico es \[\sum_{a_j,b_k}P_{a_j,b_k}\otimes P_{a_j,b_k}\rho P_{a_j,b_k}\] y con una probailidad $(1-q)$ es \[\sum_{a_j,b_k} P_{a_j,b_k}\otimes \frac{1}{2} P'_{a_j,b_k}\rho P'_{a_j,b_k}.\]

El factor $(1/2)$ es la probabilidad con la que el sistema apuntador indica  que el estado está en $|a_j b_k\ra $ o en $|b_k a_j\ra$ si el estado inicial se encuentra en el subespacio de $P'_{a_j,b_k}$. Por lo que el instrumento es la suma convexa de estos dos posibilidades.

\begin{equation}
    \mathcal{I}_3(\rho)=\sum_{a_j,b_k} q [P_{a_j,b_k}\otimes P_{a_j,b_k}]+ (1-q)[P_{a_j,b_k}\otimes \frac{1}{2} P'_{a_j,b_k}].
\end{equation}  

\subsection{POVM para mediciones difusas}

En el capítulo {\ref{MedidaPOVM}} se hablo de las medidas POVM, estas pueden entenderse como un mapeo que toma dos argumentos, una salida de medición $\alpha \in S $, donde $S$ es el conjunto de todos los posibles resultados de la medición, y un estado cuántico.
\begin{equation}\begin{split}
    \{E\}:S\times \mathcal{H}\longrightarrow [0,1]\\
    p(\alpha,\rho)=\tr(E_\alpha\rho),
\end{split}\end{equation}

donde $\mathcal{H}$ es el conjunto de los operadores de densidad. Este mapeo que proporcionan las medidas POVM es muy útil puesto que para describir completamente una medición es necesario tener un mapeo de salidas, el cual provee para cualquier estado de entrada una probabilidad de cada posible salida de medición. 


Sin embargo, el problema es que también se necesita la descripción del estado posterior a la medición y este no puede ser especificado por una medida POVM\@. Si se desea el estado posterior a la medición se debe tener más información acerca de los operadores $E_\alpha$. Es por ello que para describir el efecto de la medición difusa en un estado de entrada $\rho$, es conveniente establecer un conjunto de operadores de Kraus $\{K_\alpha\}$ tales que 

\begin{equation}E_\alpha=K_\alpha^\dagger K_\alpha.\end{equation}

Esto se conoce como descomposición del operador de medición. Debido a que $E_\alpha$ es un operador positivo se puede asumir que $K_\alpha=\sqrt{E_\alpha}$. Y luego de introducir los operadores de Kraus el estado de la medición es posterior a la medición POVM será 
\begin{equation}\rho'_\alpha=\dfrac{K_\alpha \rho K_\alpha^\dagger}{\tr(K_\alpha^\dagger K_\alpha \rho)}.\end{equation}

Las medidas POVM dan información sobre el estado de entrada, para mediciones difusas debido a que se tiene una probabilidad $p$ de medir el observable $A\otimes B$ y con una probabilidad $(1-p)$ se puede medir el operador $B\otimes A$ se puede proponer intuitivamente el siguiente conjunto de operadores

\begin{equation}\{E_{a_j, b_k}\}= \{p P_{a_j,b_k}+(1-p)SP_{a_j,b_k}S^\dagger\},\end{equation}

donde $P_{a_j,b_k} = |a_j, b_k\rala a_j,b_k|$ es el operador de proyección dado por los vectores propios del observable $A\otimes B$. Para esta medida POVM los operadores de Kraus en los que se puede descomponer de forma sencilla, son los siguientes 
\begin{equation}
K_{a_j, b_k}= \sqrt{p P_{a_j,b_k}+(1-p)SP_{a_j,b_k}S^\dagger}.
\end{equation}

Claramente la descomposición en operadores de Kraus no es única debido a que al multiplicar por la derecha alguna matriz unitaria $U$, $UK_{a_j,b_k}$ también será un nuevo operador de Kraus que cumpla con definir la misma medida POVM\@.






%  Los operadores de Kraus definen una medida POVM $\{E_\alpha\}$, con $E_\alpha=$ y con esta medida es posible obtener la probabilidad de obtener una salida en específico. Por tanto este conjunto de operadores describen completamente y de forma única la medición.%TAMBIEN SE TIENE QUE TENER PRESENTE QUE A VARIOS CONJUNTOS DE OPERADORES DE KRAUS LES PUEDE CORRESPONDER EL MISMO RESULTADO (teorema {\ref{libertad_unitaria}}).}

\subsection{Otra aproximación}
    Ahora, considerando una medida proyectiva, supongase que el observable es no degenerado y la salida de la medición sobre el estado inicial $\rho $ fue el valor propio $a_j b_k$ con $j,k=0,1$ y el operador de proyección será $P_{a_j,b_k}$. Por lo que el estado después de la medición será 
    
    \begin{equation}
       \begin{split}
           \rho'&= p\dfrac{ P_{a_j,b_k}\rho P_{a_j,b_k}}{\tr(\rho P_{a_j,b_k})}+ (1-p)\dfrac{S P_{a_j,b_k}\rho P_{a_j,b_k}S^\dagger}{\tr(\rho P_{a_j,b_k} S^\dagger SP_{a_j,b_k})}\\
           &=\dfrac{p P_{a_j,b_k}\rho P_{a_j,b_k}+(1-p) S P_{a_j,b_k}\rho P_{a_j,b_k}S^\dagger}{\tr(\rho P_{a_j,b_k})},
       \end{split}
    \end{equation}
    donde $S$ es el operador SWAP\@. Por lo que los operadores de Kraus que describen el efecto de la medición para una medida selectiva serán $\{\sqrt{p}P_{a_j,b_k}, (\sqrt{1-p})S P_{a_j,b_k}\}$, con $j,k=0,1$. Este mapeo contempla la forma de la operaciones cuánticas normalizadas $\E(\rho)/\tr(\E(\rho))$. %Para esta aproximación se realizó un  programa que mapea un operador de estado inicial a un estado final después de una medición difusa, que se describe con esta aproximación con la que se pueden obtner varios ejemplos. EL programa se puee obtener en \href{}{este repositorio}

    



\section{Ejemplos sobre los efectos de una medición difusa}

En esta sección se explorarán algunos ejemplos numéricos, con operadores particulares y utilizando instrumentos cuánticos con las tres alternativas presentadas en a sección anterior.

Primero se considera el operador $\sigma_z \otimes \sigma_z$, con vectores propios: $\{|00\ra, |01\ra, |10\ra, |11\ra\}$. Y sea $\rho $ el estado inicial
\[\rho={\begin{pmatrix}6/8&0&0 &1/8\\0&2/8&0 &0\\0 &0 &1/8 &0 \\1/8&0&0&1/8\\\end{pmatrix}}\]


Por lo que  para la primera alternativa 

\begin{equation}\label{primer_alternativa_ejemplo}
    \begin{split}
        \mathcal{I}_1(\rho)&=\sum_{a_j,b_k}P_{a_j,b_k}\otimes[p P_{a_j,b_k}\rho P_{a_j,b_k}+(1-p)P_{a_j,b_k}S\rho S^\dagger P_{a_j,b_k}]\\
        &=P_{00}\otimes P_{00} \la 00|\rho|00\ra + P_{11}\otimes P_{11} \la 11|\rho|11\ra\\
        &+P_{01} \otimes P_{01}[p \la 01|\rho|01\ra+(1-p)\la 10|\rho|10\ra]\\
        &+P_{10} \otimes P_{10}[p \la 10|\rho|10\ra+(1-p)\la 01|\rho|01\ra]\\
        &=P_{00}\otimes (6/8)P_{00} + P_{11}\otimes (1/8)P_{11}+P_{01} \otimes P_{01}[2p/8 +(1-p)/8]\\
        &+P_{10} \otimes P_{10}[p/8+2(1-p)/8]\\
    \end{split}
\end{equation}


Al realizar la traza parcial sobre le sistema cuántico se obtendrá la probabilidad de cada una de las salidas posibles.
\begin{equation}\label{traza_S1_ejemplo1}
    \begin{split}
        \tr_S(\mathcal{I}_1(\rho)) &= P_{00} \la 00|\rho|00\ra + P_{11} \la 11|\rho|11\ra\\
        &+P_{01}[p \la 01|\rho|01\ra+(1-p)\la 10|\rho|10\ra]\\
        &+P_{10}[p \la 10|\rho|10\ra+(1-p)\la 01|\rho|01\ra]\\
        &=(6/8)P_{00} + (1/8)P_{11}\\
        &+[2p/8 +(1-p)/8]P_{01}+ [p/8+2(1-p)/8]P_{10}\\
    \end{split}
\end{equation}

Y al realizar la traza parcial sobre el sistema apuntador se obtiene el efecto de la medición.

\begin{equation}\label{traza_A1_ejemplo1}
    \begin{split}
        \tr_A(\mathcal{I}_1(\rho)) &= P_{00} \rho P_{00} + P_{11}\rho P_{11}\\
        &+p P_{01}\rho P_{01}+(1-p) |01\rala 10|\rho|10\rala01|\\
        &+p P_{10}\rho P_{10}+(1-p) |10\rala 01|\rho|01\rala10|\\
        &=(6/8)P_{00} + (1/8)P_{11}\\
        &+[2p/8 +(1-p)/8]P_{01}+ [p/8+2(1-p)/8]P_{10}\\
    \end{split}
\end{equation}


Para la segunda alternativa se procede de forma parecida sin embargo el efecto de la medición es diferente.


\begin{equation}\label{segunda_alternativa_ejemplo1}
    \begin{split}
        \mathcal{I}_2(\rho)&=\sum_{a_j,b_k}\mathcal{F}_{2p}(P_{a_j,b_k})\otimes P_{a_j,b_k} \rho P_{a_j,b_k}\\
        &=P_{00}\otimes P_{00} \la 00|\rho|00\ra + P_{11}\otimes P_{11} \la 11|\rho|11\ra\\
        &+P_{01} \otimes [pP_{01} \la 01|\rho|01\ra+(1-p)P_{10}\la 10|\rho|10\ra]\\
        &+P_{10} \otimes [p P_{10}\la 10|\rho|10\ra+(1-p)P_{01}\la 01|\rho|01\ra]\\
        &=P_{00}\otimes (6/8)P_{00} + P_{11}\otimes (1/8)P_{11}\\
        &+   P_{01} \otimes [2p/8P_{01} +(1-p)/8P_{10}]+P_{10} \otimes[p/8 P_{10}+2(1-p)/8 P_{01}].\\
    \end{split}
\end{equation}


De nuevo, al realizar la traza parcial sobre le sistema cuántico se obtendrá la probabilidad de cada una de las salidas posibles.
\begin{equation}\label{traza_S2_ejemplo1}
    \begin{split}
        \tr_S(\mathcal{I}_2(\rho)) &= P_{00} \la 00|\rho|00\ra + P_{11} \la 11|\rho|11\ra\\
        &+P_{01}[p \la 01|\rho|01\ra+(1-p)\la 10|\rho|10\ra]\\
        &+P_{10}[p \la 10|\rho|10\ra+(1-p)\la 01|\rho|01\ra]\\
        &=(6/8)P_{00} + (1/8)P_{11}\\
        &+[2p/8 +(1-p)/8]P_{01}+ [p/8+2(1-p)/8]P_{10}.\\
    \end{split}
\end{equation}

Y luego al realizar la traza parcial sobre el sistema apuntador se obtiene el efecto de la medición sobre el estado de entrada.

\begin{equation}\label{traza_A2_ejemplo1}
    \begin{split}
        \tr_A(\mathcal{I}_2(\rho)) &= P_{00} \rho P_{00} + P_{11}\rho P_{11}+ P_{01}\rho P_{01}+ P_{10}\rho P_{10}\\
        &=(6/8)P_{00} + (1/8)P_{11}+(2/8)P_{01} + (1/8)P_{10}.\\
    \end{split}
\end{equation}

Como se puede ver en la ecuación {\ref{traza_S1_ejemplo1}} y en la ecuación {\ref{traza_S2_ejemplo1}} los resultados son los mismos, las probabilidades de obtener cada una de las posibles salidas son iguales en las dos primeras alternativas. Sin embargo, el mapeo que se obtiene al realizar la traza parcial sobre el apuntador es distinto. 


Para la tercera alternativa tanto el efecto de la medición como el mapeo de las probabilidades es distinito al de las otras dos alternativas. Primero hay que notar que para este caso $P'_{00}=2P_{00}$,$P'_{11}= 2P_{11}$ y $P'_{10}=P'_{01}$


\begin{equation}\label{tercera_alternativa_ejemplo1}
    \begin{split}
        \mathcal{I}_3(\rho)&=\sum_{a_j,b_k} q [P_{a_j,b_k}\otimes P_{a_j,b_k}]+ (1-q)[P_{a_j,b_k}\otimes \frac{1}{2} P'_{a_j,b_k}]\\
        &=P_{00}\otimes P_{00}\rho P_{00} + P_{11}\otimes P_{11} \rho P_{11}\\
        &+P_{01} \otimes qP_{01}\rho P_{01} +P_{10}\otimes q P_{10}\rho P_{10} \\
        &+P_{01} \otimes \frac{1-q}{2}P'_{01}\rho P'_{01}+P_{10}\otimes \frac{1-q}{2}P'_{10}\rho P'_{10}\\
        &=P_{00}\otimes P_{00} \la 00|\rho|00\ra + P_{11}\otimes P_{11} \la 11|\rho|11\ra\\
        &+P_{01} \otimes \left[qP_{01}\rho P_{01}+ \frac{(1-q)}{2} P'_{01}\rho P'_{01}\right] \\
        &+P_{10}\otimes \left[qP_{10}\rho P_{10}+ \frac{(1-q)}{2} P'_{01}\rho P'_{01}\right]\\
        %&=P_{00}\otimes (6/8)P_{00} + P_{11}\otimes (1/8)P_{11}\\
        %&+   P_{01} \otimes [2p/8P_{01} +(1-p)/8P_{10}]+P_{10} \otimes[p/8 P_{10}+2(1-p)/8 P_{01}].\\
    \end{split}
\end{equation}

Ahora, al realizar la traza parcial sobre le sistema cuántico se obtiene lo siguiente. 
\begin{equation}\label{traza_S3_ejemplo1}
    \begin{split}
        \tr_S(\mathcal{I}_3(\rho)) 
        &=P_{00} \la 00|\rho|00\ra + P_{11} \la 11|\rho|11\ra\\
        &+P_{01}  \left[q \la 01|\rho| 01\ra+ \frac{(1-q)}{2} (\la 01|\rho |01\ra +\la 10 |\rho |10\ra)\right] \\
        &+P_{10}  \left[q \la 10|\rho| 10\ra+ \frac{(1-q)}{2} (\la 01|\rho| 01\ra +\la 10 |\rho |10\ra)\right] \\
        &=P_{00} \la 00|\rho|00\ra + P_{11} \la 11|\rho|11\ra\\
        &+P_{01}  \left[\frac{1+q}{2}\la 01|\rho| 01\ra+ \frac{(1-q)}{2} \la 10 |\rho |10\ra\right] \\
        &+P_{10}  \left[\frac{1+q}{2}\la 10|\rho| 10\ra+ \frac{(1-q)}{2} \la 01 |\rho |01\ra\right] \\
        &=(6/8)P_{00} + (1/8)P_{11}\\
        &+  \left[(1+q)/8+ (1-q)/16\right] P_{01}+  \left[(1+q)/(16)+ (1-q)/8 \right] P_{10}\\
    \end{split}
\end{equation}

Esta traza parcial será igual a la ecuación {\ref{traza_S1_ejemplo1}} y {\ref{traza_S2_ejemplo1}}, solo cuando $q=2p-1$ y $1/2<p<1$. En caso contrario las ecuaciones no son equivalentes. 

Y finalmente el efecto que realiza esta medición en el estado entrada se calcula con la traza parcial sobre el sistema clásico.


\begin{equation}\label{traza_A3_ejemplo1}
    \begin{split}
        \tr_A(\mathcal{I}_3(\rho)) &=P_{00} \rho P_{00} + P_{11} \rho P_{01}\\
        &+\left[qP_{01}\rho P_{01}+ \frac{(1-q)}{2} P'_{01}\rho P'_{01}\right] \\
        &+\left[qP_{10}\rho P_{10}+ \frac{(1-q)}{2} P'_{01}\rho P'_{01}\right]\\
        &=P_{00} \rho P_{00} + P_{11} \rho P_{01}+q\left[P_{01}\rho P_{01}+ P_{10}\rho P_{10}\right]+{(1-q)} P'_{01}\rho P'_{01}.\\
    \end{split}
\end{equation}



Ahora comparando estos resultados con una medida proyectiva ideal se puede obtener la diferencia entre realizar medida difusa y una ideal. Para una media proyectiva ideal, el instrumento cuántico para este ejemplo en específico estará dado por:


\begin{equation}\label{proyectiva_ideal_ejemplo}
    \begin{split}
        \mathcal{I}_P(\rho)&=\sum_{a_j,b_k}P_{a_j,b_k}\otimes[P_{a_j,b_k}\rho P_{a_j,b_k}]\\
        &=P_{00}\otimes P_{00} \la 00|\rho|00\ra + P_{11}\otimes P_{11} \la 11|\rho|11\ra\\
        &+P_{01} \otimes P_{01}\la 01|\rho|01\ra +P_{10} \otimes P_{10} \la 10|\rho|10\ra\\
        &=P_{00}\otimes (6/8)P_{00} + P_{11}\otimes (1/8)P_{11}+P_{01} \otimes (2/8)P_{01} +P_{10} \otimes (1/8)P_{10}.\\
    \end{split}
\end{equation}


Al aplicarle la traza parcial al sistema cuántico se obtiene, el mapeo de probabilidades.

\begin{equation}\label{traza_SP_ejemplo1}
    \begin{split}
        \tr_S(\mathcal{I}_P(\rho)) &= P_{00} \la 00|\rho|00\ra + P_{11} \la 11|\rho|11\ra+P_{01}\la 01|\rho|01\ra +P_{10} \la 10|\rho|10\ra\\
        &=(6/8)P_{00} + (1/8)P_{11}+(2/8)P_{01} + (1/8)P_{10}.\\
    \end{split}
\end{equation}

Claramente difiere de la ecuación {\ref{traza_S1_ejemplo1}} y por ende de {\ref{traza_S2_ejemplo1}}, incluso en este ejemplo muy simple, donde los operadores $A=\sigma _z=B$ se puede notar que al realizar una medición difusa las probabilidades de las posibles salidas varían con una medición ideal.

Continuando con el ejemplo del observable $\sigma_z \otimes \sigma_z$, la medida POVM es simplemente 

\[\{E_{a_j, b_k}\}= \{P_{00},p P_{01}+(1-p)P_{10},p P_{10}+(1-p)P_{01},P_{11}\}.\] 


La descomposición de estos operadores por tanto son las siguientes 

\begin{equation}
    \begin{split}
        K_{a_0,b_0}&=\sqrt{P_{00}}=P_{00}\\
        K_{a_0,b_1}&=\sqrt{p P_{01}+(1-p)P_{10}}=\sqrt{p}P_{10}+\sqrt{1-p}P_{01}\\
        K_{a_1,b_0}&=\sqrt{p P_{10}+(1-p)P_{01}}=\sqrt{p}P_{01}+\sqrt{1-p}P_{10}\\
        K_{a_1,b_1}&=\sqrt{P_{11}}=P_{11},\\
\end{split}
\end{equation} 

%por lo que para una medida no selectiva, la operación cuántica estará descrita por estos operadores de Kraus

\[
\begin{split}
    \mathcal{E}(\rho)&=  P_{00}\rho P_{00}+(\sqrt{p}P_{10}+\sqrt{1-p}P_{01})\rho(\sqrt{p}P_{10}+\sqrt{1-p}P_{01})\\
    &+ (\sqrt{p}P_{01}+\sqrt{1-p}P_{10})\rho (\sqrt{p P_{01}+(1-p)P_{10}})+P_{11}\rho P_{11}.\\
\end{split}
\]

%físicamente este canal descibe la medición difusa.