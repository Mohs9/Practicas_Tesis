\addcontentsline{toc}{chapter}{Introducción}
\chapter*{Introducción}

El problema de la medición ha jugado un papel muy importante en la física cuántica desde su fundación. Ciertamente, se ha logrado avanzar en este problema, a través de los años gracias al desarrollo de las técnicas de medición y al entendimiento de los procesos de medición en mecánica cuántica {\cite{Pineda_2021}}. 

Debido a que en la naturaleza es posible obtener  mediciones imperfectas, en este trabajo se trata particularmente con \textit{mediciones difusas}, en las que en un sistema de varias partículas existe una posibilidad finita de identificar erróneamente las partículas. Durante el proyecto de prácticas se trabajó en la descripción completa de las mediciones difusas para el caso en el que el sistema solo cuenta con dos partículas. Para poder abordar el problema es necesario estudiar un marco conceptual previo, iniciando por la reformulación de los postulados de la mecánica cuántica en el lenguaje de la matriz de densidad.

Una descripción completa de una medición debe proporcionar las probabilidades respectivas de los diferentes resultados de la medición, posibles. Y describir el estado posterior a la medición del sistema. Por lo que, en este contexto es necesario estudiar las medidas cuyos valores son operadores positivos (POVM por su siglas en inglés) las cuales son una herramienta que generaliza las medidas proyectivas y que permiten un efecto más suave en el sistema medido.  

Asismismo se recurre al lenguaje de las operaciones cuánticas y su representación como  suma de operadores. Debido a que la teoría de las operaciones cuánticas propone un formalismo para describir la evolución de los sistemas abiertos de manera discreta {\cite{nielsen_chuang_2010}}. Las operaciones cuánticas son una herramienta que permiten describir la evolución de los sitemas más allá de los ideales. Más aún permite representarla de una manera valiosa como suma de operadores tales que están relacionados con las medidas POVM \@.

El informe tiene la siguiente estructura.  En el capítulo 1 se presenta la revisión y estudio bibliográfico que se realizó sobre el formalismo de la matriz de densidad en mecánica cuántica. La matriz de densidad es una herramienta utilizada para describir al estado de un sistema cuántico que será útil en el desarrollo de todo el trabajo. En el capítulo 2 se expone el estudio del formalismo de las medidas POVM\@. Se introduce primero las medidas proyectivas y luego se generalizan con los operadores POVM\@. En el capítulo 3 se estudia el formalismo de las operaciones cuánticas y su representación como suma de operadores. Además de presentar el proceso de tomografía cuántica. En el último capítulo se formula el enunciado del problema de las mediciones difusas en sistemas de dos partículas. Luego,