\addcontentsline{toc}{chapter}{Introducción}
\chapter*{Introducción}

El problema de la medición ha jugado un papel muy importante en la física
cuántica desde su fundación. Ciertamente, se ha logrado avanzar en este
problema, a través de los años gracias al desarrollo de las técnicas de
medición y al entendimiento de los procesos de medición en mecánica cuántica
{\cite{Pineda_2021}}. 

%Debido a que en la naturaleza es posible obtener mediciones imperfectas
En este trabajo se trata particularmente con \textit{mediciones difusas}, en las
que en un sistema de varias partículas existe una posibilidad finita de
identificar erróneamente las partículas. Durante el proyecto de prácticas se
trabajó en la descripción completa de las mediciones difusas para el caso en el
que el sistema solo cuenta con dos partículas.

Para poder abordar el problema es necesario estudiar un marco conceptual previo, iniciando por la reformulación de los postulados de la mecánica cuántica en el lenguaje de la matriz de densidad. La matriz de densidad es una herramienta utilizada para describir al estado de un sistema cuántico que será útil en el desarrollo de todo el trabajo. En el primer capítulo  se presenta la revisión y estudio bibliográfico que se realizó sobre el formalismo de la matriz de densidad en mecánica cuántica. 

Una descripción completa de una medición debe proporcionar las probabilidades
respectivas de los diferentes resultados posibles de la medición y el estado
posterior a la medición del sistema. 
En este contexto es necesario estudiar las medidas cuyos valores son operadores
positivos (POVM por sus siglas en inglés) las cuales son una herramienta que
generaliza las medidas proyectivas y que permiten un efecto más suave en el
sistema medido. En el segundo capítulo se expone el estudio del formalismo de
las medidas POVM\@. Se introduce primero las medidas proyectivas y luego se
generalizan con los operadores POVM\@. 

Asimismo se recurre a lenguaje de las operaciones cuánticas porque permiten
representar la evolución de los sistemas más allá de los ideales, de una manera
valiosa como suma de operadores, los cuales están relacionados con las medidas
POVM\@. Además, la teoría de las operaciones cuánticas propone un formalismo
para describir la evolución de los sistemas abiertos de manera discreta
{\cite{nielsen_chuang_2010}}. En el tercer capítulo se estudia el formalismo de
las operaciones cuánticas y su representación como suma de operadores.
%Adicionalmente se presenta el proceso de tomografía cuántica. 
% \cpnote{Ver si esta 
% ultima frase va o no va, ya cuando tengamos mas listo el tercer capitulo}\rrnote{Sí, quité la última frase}


En suma, incluso en sistemas de dos partículas en los que se realiza una
medición de un observable, en la cual existe una posibilidad de identificar las
partículas equivocadamente, es posible describir completamente dicha medición
utilizando las herramientas mencionadas anteriormente. En especial, la
representación como suma de operadores de las operaciones cuánticas y los
operadores POVM serán de utilidad para la descripción de las mediciones
difusas. En el último capítulo se formula el enunciado del problema de las
mediciones difusas en sistemas de dos partículas. Y finalmente, se exploran
algunos conjuntos de operadores de Kraus que describen las mediciones difusas y
ejemplos numéricos de la aplicación de los operadores. 

%Asimismo se recurre al lenguaje de las operaciones cuánticas y su representación como  suma de operadores. Debido a que la teoría de las operaciones cuánticas propone un formalismo para describir la evolución de los sistemas abiertos de manera discreta {\cite{nielsen_chuang_2010}}. Las operaciones cuánticas son una herramienta que permiten describir la evolución de los sistemas más allá de los ideales. Más aún permite representarla de una manera valiosa como suma de operadores tales que están relacionados con las medidas POVM\@. 
% \cpnote{Estas pasando como por la estructura general del trabajo. Sin embargo no hablas del capitulo 4 sino dos frases en la parte de la estructura. Te propongo que armes un parrafo hablando del 4, y que quizá elimines el siguiente parrafo, indicando en los parrafos anteriores donde va, por ejemplo, lo de los POVMs.} \rrnote{Agregué un último párrafo y la parte donde indicaba la estructura del documento la fui colocando en cada uno de los párrafos. }



