\chapter{Medidas POVM}\label{MedidaPOVM}
Hasta ahora, se ha descrito el espacio de estados cuánticos. Ahora, se abordará el tema del proceso de medición cuántica. Aunque el proceso de medición  sigue siendo algo enigmático, aquí simplemente, se toma como cierto, sin prueba, el postulado relativo al colapso de la función de onda, mencionado en el capítulo anterior (sección {\ref{postulates}}). Según Nielsen y Chuang {\cite{nielsen_chuang_2010}}, el postulado de la medici\'on cuántica, involucra dos elementos. El primero, proporciona una regla que describe las estadísticas de medición, es decir, las probabilidades respectivas de los diferentes resultados de medición posibles. En segundo lugar, da una regla que describe el estado posterior a la medición del sistema.  En la primera sección, se hablará de un caso particular de las mediciones, las mediciones proyectivas. En la segunda sección de este capítulo, se discutirá acerca de los operadores POVM\@. Y finalmente, en la tercera sección se presenta el teorema de dilatación de Naimark. 

%El postulado de la medición tiene la virtud de la generalidad, mas no de la precisión, sin embargo presenta algunas ventajas como una estructura matemáticamente más simple y además que muchos problemas en información cuántica implican medidas generales










\section{Medidas proyectivas PVM}
La medición cuántica realizada en un sistema inicial $\rho$ produce  la m-ésima salida con una probabilidad $p(m)$ dada por la ecuación {\ref{probaility3postulate}} y transforma a $\rho $ en $\rho_m$ descrito por la ecuación {\ref{state3postulate}}. En tales mediciones, se registran resultados concretos etiquetados por el subíndice $m$ y son llamadas \textit{selectivas}. Si no se realiza una selección basada en la salida de la medición, el estado inicial es transformado en una combinación convexa de todas las salidas posibles, dado por 
\begin{equation}\label{non-selective-measure}
	\rho'=\sum_m M_m\rho M_m^\dagger,
\end{equation}
esto describe una medición cuántica \textit{no selectiva}, esto es un estado mixto, incluso si el estado inicial $\rho$ es puro. En general, no se puede recibir ninguna información sobre el destino de un sistema cuántico sin realizar una medida que perturbe su evolución temporal unitaria {\cite{2007geometry}}.

En una medida proyectiva PVM (por sus siglas en inglés, projection-valued measure) los operadores de medición son operadores proyectores ortogonales  tales que $M_{m}=M_{m}^{\dagger}=P_{m}$, y $P_{m}P_{n}=\delta_{mn}P_{m}$. Una medición proyectiva es descrita por un observable $\mathcal{O}$ y las salidas de la medición están etiquetadas por los valores propios de $\mathcal{O}$. Dado que el $\mathcal{O}$ es operador hermítico, se puede utilizar la descomposición espectral para representarlo como 

\begin{equation}
	\mathcal{O}=\sum_m \lambda_m P_m ,
\end{equation}

  puede obtenerse un conjunto ortogonal de operadores de medición $P_m=\sum_i |e_i^{(m)}\rala e_i^{(m)}|$, que satisfacen la relación de completitud {\cite{2007geometry}}. 



En una medición proyectiva selectiva la salida etiquetada por $\lambda_m$ ocurre con probabilidad $p(m)=\tr(P_m\rho P_m)$, y por la propiedad cíclica de la traza,  $p(m)=\tr(P_m\rho)$; el estado inicial es transformado como 

\begin{equation}\label{stateProjective}
	\rho\to	\rho_m=\dfrac{P_m\rho P_m}{\tr(P_m \rho)}.
\end{equation}

 Y en una medición proyectiva no selectiva, el estado inicial es transformado en una mezcla 

\begin{equation}
	\rho \to \rho'=\sum_m P_m \rho P_m
\end{equation} 


El estado conmuta con el observable $\mathcal{O}$. El valor esperado del observavle estará dado por

\begin{equation}
	\la \mathcal{O} \ra=\sum_m p_m\lambda_m=\sum_m \lambda_m \tr(P_m\rho)=\tr(\mathcal{O}\rho).
\end{equation}

Una característica clave de las mediciones proyectivas consiste en la repetibilidad. Esto significa que si la medición es repetida, el estado en {\ref{stateProjective}} sigue siendo el mismo, y da el mismo resultado {\cite{2007geometry}}.

\section{Operadores POVM}

Muchas de las mediciones no son repetibles, por ejemplo, en un experimento que se mide el sistema solo una vez al finalizar éste. Por lo que para algunas aplicaciones, el estado posterior a la medición del sistema es de poco interés, siendo el principal elemento de interés las probabilidades de los respectivos resultados de medición. En tales casos, existe una herramienta matemática conocida como formalismo POVM que es especialmente bien adaptado al análisis de las medidas (el acrónimo POVM significa <<medida valorada por el operador positivo>>). Este formalismo es una simple consecuencia de la descripción general de las medidas introducidas en el Postulado 3, pero la teoría de es bastante elegante y ampliamente utilizada que se discutirá en esta sección {\cite{nielsen_chuang_2010}}. Por lo que se procede a dar una definición de las medidas POVM\@.

\begin{definition}[\textbf{POVM}] Una medida POVM (positive operator-valued measure) es un conjunto $\{E_{m}\} _{m}$ de operadores llamados <<efectos>> que satisfacen las siguientes condiciones:
	\begin{enumerate}
		\item Positividad. $\la \psi |E_m|\psi \ra \ge 0 $ para cualquier vector $|\psi\ra$.
		\item Hermiticidad. $E_m=E_{m}^\dagger$.
		\item  Completitud. $\sum_m E_m =\mathds{1}$.
	\end{enumerate}
\end{definition}

Una medida POVM aplicada al estado $\rho$ produce la m-ésima salida, con una probabilidad $p(m)=\tr(E_m \rho)$. Notar que los elementos $E_m$ del conjunto de operadores hermíticos no necesariamente conmutan. El nombre POVM sugiere correctamente que la suma discreta puede ser reeplazada por una integral sobre un conjunto de índices continuos. Los POVM se ajustan en el marco general del postulado de la medición cuántica, ya que se puede elegir $E_m=M_m M_m^{\dagger}$. Sin embargo es importante notar que los POVM no determinan los operadores de medición $M_m$ de manera única, a excepoión del caso particular de las mediciones proyectivas. Lo que sucede con el estado exactamente, cuando una medición es realizada depende en como los POVM son implementados en el laboratorio {\cite{2007geometry}}.   

Según Bengtsson y Zyczkowski {\cite{2007geometry}}, un POVM es llamado informacionalmente completo si las estadísticas del POVM determinan de forma única la matriz de densidad. Esto requiee que el POVM tenga $d^2$ elementos, donde $d$ es la dimensión del espacio de Hilbert, una medida proyectiva no es suficiente. Un POVM es llamado \textit{puro} si cada operadores $E_m$ es de rango uno, tales que existe un estado puro $|\phi_m\ra$ tal que $E_m$ es proporcional a $|\phi_m\rala\phi_m|$. Un POVM que es impuro puede siempre volverse en un estado puro, reemplazando cada operador $E_m$ por su descomposición espectral. 

Un conjunto de $k$ estados puros $\phi_m$ definen un POVM puro si y solo si el estado de máxima  mezcla (\ref{maximallyMixedState}) puede ser descompuesto como  $\sigma=\sum_{m=1}^{k}p_m|\phi_m\rala\phi_m|$, donde $\{p_m\}$ forma un conjunto adecuado de coeficientes positivos. De hecho cualquier ensamble de estados puros o mixtos representando a $\sigma$ definen un POVM\@. Para cualquier conjunto de operadores $E_m$ definiendo un POVM se toman estados cuánticos $\rho_m=E_m/\tr(E_m)$ y una probabilidad $p_m=\tr E_m/d$ para obtener el estado de mezcla máxima:

\begin{equation}
	\sum_{m=1}^{k} p_m\rho _{m}=\sum_{m=1}^k \dfrac{1}{d}E_m=\dfrac{\mathds{1}}{d}=\sigma.
\end{equation}

Conversamente, cualquier ensamble de matrices de densidad definen un POVM {\cite{2007geometry}}.


Para mostrar la utilidad del formalismo POVM como una forma intuitiva de obtener información sobre las mediciones cuánticas en problemas en las que solo importan las estadísticas de medición, Nielsen y Chuang {\cite{nielsen_chuang_2010}}, proponen un ejemplo sencillo e intuitivo.

Antes de presentar el ejemplo, vale la pena demostrar la siguiente proposición.

\begin{proposition}Los estados no ortogonales no pueden ser distinguibles de forma fiable por ninguna medida {\cite{nielsen_chuang_2010}}.
\end{proposition}

\begin{proof}
Supongase que existe una medición que pueda distinguir estados ortogonales.


Si el estado $|\psi_1\ra $ ($|\psi_2\ra $) se prepara, entonces la
probabilidad de medir $j$, $f(j) =1$ $(f(j)=2)$ debe ser 1, donde $f(\cdot)$ representa
la regla que se usa para adivinar cuál es el índice $i$ de la salida, $f(j)=i$. 
Definiendo $E_i\equiv \sum_{j:f(j)=i}M_j^\dagger M_j$, estas observaciones pueden
ser escritas como


\begin{equation}\label{E1psi1}
	\la \psi_1|E_1|\psi_1\ra=1
\end{equation}

\begin{equation}\label{E2psi2}
	\la \psi_2|E_2|\psi_2\ra=1
\end{equation}

Dado que $\sum_i E_i=\mathds{1}$, se sigue que $\sum_i \la \psi_1|E_i|\psi_1\ra=1$,
y como $\la \psi_1|E_1|\psi_1\ra=1$ y se debe tener que $\la \psi_1|E_2|\psi_1\ra=0$,
y luego $\sqrt{E_2}|\psi_1\ra=0$. Además se puede descomponer  
$|\psi_2\ra=\alpha|\psi_1\ra +\beta|\phi\ra$ donde $|\phi\ra$ sí es ortonormal
a $|\psi_1\ra $, $|\alpha|^2+|\beta|^2=1$, y $|\beta|<1$ debido a que $|\psi_1\ra$
y $|\psi_2\ra$ no son ortogonales. Luego $\sqrt{E_2}|\psi_2\ra=\beta \sqrt{E_2}|\phi\ra$, Lo
cual implica una contradicción, con{\ref{E2psi2}}, puesto que

\begin{equation}
	\la \psi_2|E_2|\psi_2\ra=|\beta|^2 \la\phi|E_2|\phi\ra \le |\beta|^2<1,
\end{equation}

donde la segunda inecuación se obtiene de la observación de que 
\[ \la \phi|E_2|\phi\ra \le \sum_i\la \phi|E_i|\phi\ra= \la\phi|\phi\ra. \]


\end{proof}


Ahora se puede proceder con el ejemplo sobre POVM\@. Supóngase que se tienen dos intermediarios \textit{A} y \textit{B}, el intermediario \textit{A} prepara un conjunto de estados $\{|\psi_1\ra=|0\ra\}$ y $|\psi_2\ra=(|0\ra +|1\ra)/\sqrt{2}$. El intermidiario \textit{A} le al \textit{B} uno de los dos qubits. Debido a que estos dos qubit no son ortogonales, es imposible que \textit{B} pueda determinar cual de los dos le han dado con perfecta fiabilidad. Sin embargo, es posible que $B$ realice una medición la cual distinga los estados algunas veces, pero nunca cometa un error de identificación errónea. Considérese un POVM conteniendo los siguientes tres elementos 

	\begin{align}
		E_1 & \equiv \dfrac{\sqrt{2}}{1+\sqrt{2}} |1\rala1|, \label{eq11} \\
		E_2 & \equiv \dfrac{\sqrt{2}}{1+\sqrt{2}} \dfrac{(|0\ra -|1\ra)(\la0|-\la1|)}{2}, \label{eq12} \\
		E_3 & \equiv \mathds{1}-E_1-E_2\label{eq13} 
	\end{align}



Es fácil ver que los tres operadores son hermíticos, positivos y debido a como se construye el tercer operador se satisface las relaciones de completitud. Por lo que cumplen con ser POVM\@. Ahora, supongase que $B$ recibió el estado $|\psi_1\ra=|0\ra$, y realiza una medición descrita por el POVM $\{E_1,E_2,E_3\}$. Hay una probabilidad cero de que observe el resultado $E_1$, debido a que $E_1$ ha sido construido sabiamente para asegurar que $p(1)=\la\psi_1|E_1|\psi_1\ra=0$. Luego, si el resultado de su medición es $E_1$ luego $B$ puede concluir con seguridad que el estado que recibió, fue $|\psi_2\ra $. Un pensamiento similar conduce a mostrar que si la salida de la medición $E_2$ ocurre, el estado que $B$ recibió sería $|\psi_1\ra$. Algunas veces, sin embargo, $B$ obtendrá el resultado de la medición $E_3$ y no podrá inferir nada acerca de la identidad del estado que se le dio. Mas el punto clave, es que $B$ nunca comete un error al identificar el estado que se le ha dado. Esta infalibilidad tiene el precio de que, a veces, $B$ no obtiene información sobre la identidad del estado.



\section{El teorema de Naimark}

En esta tercera sección se habla de que las medidas POVM no solo son más generales que las medidas proyectivas, sino que también pueden verse como un caso especial de estas últimas. Dado cualquier POVM puro con $k$ elementos y un estado $\rho$ en un espacio de Hilbert de dimensión $d$, se puede encontrar un estado $\rho \otimes \rho_0$ en un espacio de Hilbert  $\mathcal{H}\otimes\mathcal{H'}$ tal que las estadísticas de la medición POVM original se reproducen exactamente mediante una medida proyectiva de $\rho \otimes \rho_0$. Esta afirmación es consecuencia del teorema del Naimark {\cite{2007geometry}}.




\begin{theorem}[\textbf{Teorema de Naimark}] Cualquier POVM $\{E_i\}$ en el espacio de Hilbert $\mathcal{H}$, existe una medida proyectiva PVM $\{P_i\}$ en un espacio de Hilbert más grande $\mathcal{K}$, de tal manera que $E_i=\Pi^\dagger P_i\Pi$, donde $\Pi$ proyecta desde $\mathcal{K}$ hacia $\mathcal{H}$.
\end{theorem}
	

Una forma de construir la operadores $\{P_i\}$ tomar  $\mathcal{K}=\mathcal{H}\otimes\mathcal{H'}$, $P _{i}=\mathds{1} _{H}\otimes |i\rangle \langle i|_{H'}$, y
$\Pi=\sum _{i=1}^{n}{\sqrt {E_{i}}}_{H}\otimes {|i\rangle }_{H'}$. La probabilidad de obtener la salida $i$ con estos operadores PVM, es la misma a la probabilidad de obtener esta con el POVM original {\cite{2007geometry}}.


\begin{equation}
 \begin{split}
 p_i&=\tr \left(\Pi\rho\Pi^{\dagger }P _{i}\right)\\
 &=\tr \left(\Pi\rho \Pi^{\dagger }\left[\mathds{1}_{H}\otimes |i\rangle \langle i|_{H'}\right]\right)\\
 &=\tr \left(\rho\left(\sum _{j=1}^{n}{\sqrt {E{j}}}_{H}^{\dagger }\otimes {\langle j|}_{H'}\right)\mathds{1} _{H}\otimes |i\rangle \langle i|_{B}\left(\sum _{k=1}^{n}{\sqrt {E_{k}}}_{H}\otimes {|k\rangle }_{H'}\right)\right)\\
 &=\tr \left(\rho {\sqrt {E_{i}}}_{H}\mathds{1} _{H}{\sqrt {E_{i}}}_{H}\right)\\
 &=\tr (\rho E_{i})
 \end{split}
\end{equation}
































