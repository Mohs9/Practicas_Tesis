\chapter{Medidas POVM}\label{MedidaPOVM}
% Intro {{{
Hasta ahora, se ha descrito el espacio de estados cuánticos. Ahora, se abordará
el tema del proceso de medición cuántica. Aunque el proceso de medición  sigue
siendo algo enigmático, aquí simplemente se toma como cierto el
postulado relacionado al colapso de la función de onda, mencionado en el capítulo
anterior (sección {\ref{postulates}}) \cpnote{Esta frase está innecesariamente compilcada larga
y con muchas comas.}\rrnote{Ya simplifiqué la oración.}. Según Nielsen y Chuang
{\cite{nielsen_chuang_2010}}, el postulado de la medición cuántica, involucra
dos elementos. El primero, proporciona una regla que describe las estadísticas
de medición, es decir, las probabilidades respectivas de los diferentes
resultados de medición posibles. En segundo lugar, da una regla que describe el
estado posterior a la medición del sistema.  En la primera sección, se hablará
de un caso particular de las mediciones, las mediciones proyectivas. En la
segunda sección de este capítulo, se discutirá acerca de los operadores POVM\@.
Finalmente, se expone el teorema de dilatación de
Naimark. 

%El postulado de la medición tiene la virtud de la generalidad, mas no de la
%precisión, sin embargo presenta algunas ventajas como una estructura
%matemáticamente más simple y además que muchos problemas en información
%cuántica implican medidas generales
% }}}
\section{Medidas proyectivas PVM} % {{{
La medición cuántica realizada en un estado $\rho$ \cpnote{seria estado y no sistema. 
creo qeu ya te has equivocado un par de veces con esos conceptos. Ojo con eso. También no 
hay necesidad de decir inicial acá. } produce  la $m$-ésima \cpnote{En este caso, como nos 
referimos a la variable ``m'' se escribe $m$-ésima y no m-ésima. }\rrnote{Ya he corregido lo de las variables $m$}
salida con una probabilidad $p(m)$ dada por la ecuación
(\ref{probaility3postulate}) y transforma a $\rho $ en $\rho_m$ descrito por la
ecuación (\ref{state3postulate}). En tales mediciones, se registran resultados
concretos etiquetados por el subíndice $m$ y son llamadas \textit{selectivas}.
Si no se realiza una selección basada en la salida de la medición, el estado
inicial es transformado en una combinación convexa de todas las salidas
posibles, dado por 
\begin{equation}\label{non-selective-measure}
	\rho'=\sum_m M_m\rho M_m^\dagger.
\end{equation}
Esto es un estado mixto que  describe una medición cuántica \textit{no selectiva}, incluso si el estado inicial $\rho$ es puro \cpnote{Esta frase está muy
larga}.\rrnote{acorté la frase}% En general, no se puede recibir ninguna información sobre el destino de un sistema cuántico sin realizar una medida que perturbe su evolución temporal unitaria \cpnote{porque hablanmos aca de evolucion temporal unitaria?}
{\cite{2007geometry}}.

En una medida proyectiva PVM (por sus siglas en inglés, projection-valued
measure) los operadores de medición son operadores proyectores ortogonales
tales que $M_{m}=M_{m}^{\dagger}=P_{m}$, y $P_{m}P_{n}=\delta_{mn}P_{m}$. Una
medición proyectiva es descrita por un observable $\mathcal{O}$ y las salidas
de la medición están etiquetadas por los valores propios de $\mathcal{O}$. Dado
que el $\mathcal{O}$ es operador hermítico, se puede utilizar la descomposición
espectral para representarlo como 
\begin{equation}
	\mathcal{O}=\sum_m \lambda_m P_m ,
\end{equation}
puede obtenerse un conjunto ortogonal de operadores de medición $P_m=\sum_i
|e_i^{(m)}\rala e_i^{(m)}|$, que satisfacen la relación de completitud
{\cite{2007geometry}}. 

En una medición proyectiva selectiva la salida etiquetada por $\lambda_m$
ocurre con probabilidad $p(m)=\tr(P_m\rho P_m)$, y por la propiedad cíclica de
la traza,  $p(m)=\tr(P_m\rho)$; el estado inicial es transformado como 
\begin{equation}\label{stateProjective}
	\rho\to	\rho_m=\dfrac{P_m\rho P_m}{\tr(P_m \rho)}.
\end{equation}

 En una medición proyectiva no selectiva, el estado inicial es transformado en
una mezcla  
\begin{equation}
	\rho \to \rho'=\sum_m P_m \rho P_m.
\end{equation} 


El estado conmuta con el observable $\mathcal{O}$. El valor esperado del
observable estará dado por 
\begin{equation}
	\la \mathcal{O} \ra=\sum_m p_m\lambda_m=\sum_m \lambda_m
\tr(P_m\rho)=\tr(\mathcal{O}\rho).
\end{equation}

Una característica clave de las mediciones proyectivas consiste en la
repetibilidad. Esto significa que si la medición es repetida, el estado en
{\ref{stateProjective}} sigue siendo el mismo, y da el mismo resultado
{\cite{2007geometry}}.
% }}}
\section{Operadores POVM}\label{operadoresPOVM} % {{{
\cpnote{no entiendo a que te refieres o quizá no entiendes lo es uqe una medicion no sea repetible.}\rrnote{Quité la frase anterior porque releyendola, pienso que es innecesaria y estaba mal planteada.}
Para algunas aplicaciones, el estado posterior a la medición del sistema es de poco interés, siendo el
principal elemento de interés las probabilidades de los respectivos resultados
de medición. En tales casos, existe una herramienta matemática conocida como
formalismo POVM que es especialmente bien adaptado al análisis de las medidas
(el acrónimo POVM significa <<medida valorada por el operador positivo>>). Este
formalismo es una simple consecuencia de la descripción general de las medidas
introducidas en el Postulado 3, pero la teoría de es bastante elegante y
ampliamente utilizada que se discutirá en esta sección
{\cite{nielsen_chuang_2010}}. Se procede a dar una definición de las medidas
POVM\@. \cpnote{Siento que acá bajo la calidad del escrito. Dale una mirada a esta
seccion (o mas bien de aca al final) y luego ya la veo}\rrnote{He agregado algunos párrafos más y quitado otros, sin embargo no he encontrado más información relevante.}

\begin{definition}[\textbf{POVM}] Una medida POVM (positive operator-valued measure) es un conjunto $\{E_{m}\} _{m}$ de operadores llamados <<efectos>> que satisfacen las siguientes condiciones {\cite{2007geometry}}:
	\begin{enumerate}
		\item Positividad. $\la \psi |E_m|\psi \ra \ge 0 $ para cualquier vector $|\psi\ra$.
		\item Hermiticidad. $E_m=E_{m}^\dagger$.
		\item  Completitud. $\sum_m E_m =\mathds{1}$.
	\end{enumerate}
\end{definition}

Una medida POVM aplicada al estado $\rho$ produce la $m$-ésima salida, con una
probabilidad $p(m)=\tr(E_m \rho)$. Un ejemplo de medida POVM son las ya discutidas medidas proyectivas descritas por los operadores de proyección, tales que $P_m P_n=\delta_{mn}P_{m}$ y $\sum_m Pm = \mathds{1}$. Solo en este caso todos los elementos POVM son los mismos que los propios operadores de medición,$E_m=P_m P_m^\dagger=P_m$ {\cite{nielsen_chuang_2010}}.


Notar que los elementos $E_m$ del conjunto de operadores hermíticos no necesariamente conmutan. Los POVM se ajustan en el marco general del
postulado de la medición cuántica, ya que se puede elegir $E_m=M_m
M_m^{\dagger}$. Sin embargo es importante notar que los POVM no determinan los
operadores de medición $M_m$ de manera única, a excepción del caso particular
de las mediciones proyectivas. Lo que sucede con el estado exactamente, cuando
una medición es realizada depende en como los POVM son implementados en el
laboratorio {\cite{2007geometry}}.   

Según Bengtsson y Zyczkowski {\cite{2007geometry}}, un POVM es llamado \textit{puro} si cada operadores $E_m$ es de rango uno, tales que existe un estado puro $|\phi_m\ra$ tal que $E_m$ es proporcional a $|\phi_m\rala\phi_m|$. Un POVM que es impuro puede siempre volverse en un estado puro, reemplazando cada operador $E_m$ por su descomposición espectral. 

Un conjunto de $k$ estados puros $|\phi_m \ra$ definen un POVM puro si y solo si el estado máximamente mezclado (\ref{maximallyMixedState}) puede ser descompuesto como  $\sigma=\sum_{m=1}^{k}p_m|\phi_m\rala\phi_m|$, donde $\{p_m\}$ forma un conjunto adecuado de coeficientes positivos. De hecho cualquier ensamble de estados puros o mixtos representando a $\sigma$ definen un POVM\@. Para cualquier conjunto de operadores $E_m$ definiendo un POVM se toman estados
cuánticos $\rho_m=E_m/\tr(E_m)$ y una probabilidad $p_m=\tr E_m/d$ para obtener
el estado máximamente mezclado: 
\begin{equation}
	\sum_{m=1}^{k} p_m\rho _{m}=\sum_{m=1}^k \dfrac{1}{d}E_m=\dfrac{\mathds{1}}{d}=\sigma.
\end{equation}

Por otro lado, cualquier ensamble de matrices de densidad definen un POVM
{\cite{2007geometry}}.


Según Nielsen y Chuang {\cite{nielsen_chuang_2010}}, por lo general los textos introductorios a la mecánica cuántica describen solamente la medidas proyectivas y en consecuencia el formalismo de las medidas POVM y la descripción del tercer postulado mencionado en la sección {\ref{postulates}}, no es tan conocido. En computación e información cuántica, el objetivo es lograr un buen nivel de control sobre las mediciones que se pueden realizar, por consiguiente utilizar un formalismo más completo para la descripción de las mediciones es de gran ayuda. Por supuesto, las medidas proyectivas son completamente equivalentes a las mediciones más generales cuando se toman en cuenta los demás axiomas.


Existen varias razones para utilizar el formalismo general de las mediciones. La primera es que matemáticamente las mediciones generales son menos restrictivas y simples que las medidas proyectivas. Por ejemplo, estas mediciones generales no necesariamente exigen la condición de idempotencia $P^2=P$ como si lo hacen las medidas proyectivas. Otra razón para utilizarlas es que existen muchos problemas en información y computación cuántica tales como la forma óptima de distinguir un conjunto de estados cuánticos cuya respuesta implica una medición general, en lugar de una medición proyectiva {\cite{nielsen_chuang_2010}}.


Una tercera razón para utilizar las mediciones generales esta relacionada con la propiedad de repetibilidad de las mediciones proyectivas. Las medidas proyectivas son repetibles en el sentido que si se realiza una medición proyectiva una vez y se obtiene un resultado $m$, al repetir la medición el resultado será nuevamente $m$ y el estado también será el mismo. Este hecho indica que muchas medidas importantes en mecánica cuántica no son medidas proyectivas.  Para tales mediciones, debe emplearse el postulado general de medición. Y en este contexto las medidas POVM son un caso especial del formalismo de la medición general, que proporciona el medio más simple en el que se pueden estudiar las estadísticas generales de medición, sin necesidad de conocer el estado posterior a la medición. Son una conveniencia matemática que a veces brinda información adicional sobre las mediciones cuánticas {\cite{nielsen_chuang_2010}}.


Para mostrar la utilidad del formalismo POVM como una forma intuitiva de
obtener información sobre las mediciones cuánticas en problemas en las que solo
importan las estadísticas de medición, Nielsen y Chuang
{\cite{nielsen_chuang_2010}}, proponen un ejemplo sencillo e intuitivo.

Antes de presentar el ejemplo, vale la pena mostrar la siguiente proposición
propuesta y demostrada por Nielsen y Chuang {\cite{nielsen_chuang_2010}}.

\begin{proposition}
Los estados no ortogonales no pueden ser distinguibles de forma fiable por alguna medida.
\end{proposition}
\begin{proof}
Se demostrará por reducción al absurdo. Supóngase que existe una medición que
pueda distinguir estados ortogonales.

Si el estado $|\psi_1\ra $ ($|\psi_2\ra $) se prepara, entonces la
probabilidad de medir $j$ debe ser 1, tal que  $f(j) =1$ $(f(j)=2)$, donde $f(\cdot)$ representa
la regla que se usa para adivinar cuál es el índice $i$ de la salida, $f(j)=i$. 
Definiendo $E_i\equiv \sum_{j:f(j)=i}M_j^\dagger M_j$, estas observaciones pueden ser escritas como 
\begin{align}
	\la \psi_1|E_1|\psi_1\ra=1, \label{E1psi1} \\
	\la \psi_2|E_2|\psi_2\ra=1\label{E2psi2}.
\end{align}

Dado que $\sum_i E_i=\mathds{1}$, se sigue que $\sum_i \la \psi_1|E_i|\psi_1\ra=1$,
y como $\la \psi_1|E_1|\psi_1\ra=1$ y se debe tener que $\la \psi_1|E_2|\psi_1\ra=0$,
y luego $\sqrt{E_2}|\psi_1\ra=0$. 


Además se puede descomponer  
$|\psi_2\ra=\alpha|\psi_1\ra +\beta|\phi\ra$, donde $|\phi\ra$ sí es ortonormal
a $|\psi_1\ra $, $|\alpha|^2+|\beta|^2=1$, y $|\beta|<1$ debido a que $|\psi_1\ra$
y $|\psi_2\ra$ no son ortogonales. Luego $\sqrt{E_2}|\psi_2\ra=\beta \sqrt{E_2}|\phi\ra$, Lo
cual implica una contradicción, con {\ref{E2psi2}}, puesto que \begin{equation}
	\la \psi_2|E_2|\psi_2\ra=|\beta|^2 \la\phi|E_2|\phi\ra \le |\beta|^2<1,
\end{equation} 
donde la segunda inecuación se obtiene de la observación \[ \la \phi|E_2|\phi\ra \le \sum_i\la \phi|E_i|\phi\ra= \la\phi|\phi\ra=1. \]
\end{proof}
Ahora se puede proceder con el ejemplo sobre POVM\@. Supóngase que se tienen
dos intermediarios \textit{A} y \textit{B}, el intermediario \textit{A} prepara
un conjunto de estados $\{|\psi_1\ra=|0\ra\}$ y $|\psi_2\ra=(|0\ra
+|1\ra)/\sqrt{2}$. El intermediario \textit{A} le da a \textit{B} uno de los
dos qubits. Debido a que estos dos qubit no son ortogonales, es imposible que
\textit{B} pueda determinar cual de los dos le han dado con perfecta
fiabilidad. Sin embargo, es posible que $B$ realice una medición la cual
distinga los estados algunas veces, pero nunca cometa un error de
identificación. Considérese un POVM conteniendo los siguientes tres elementos 
	\begin{align}
		E_1 & \equiv \dfrac{\sqrt{2}}{1+\sqrt{2}} |1\rala1|, \label{eq11} \\
		E_2 & \equiv \dfrac{\sqrt{2}}{1+\sqrt{2}} \dfrac{(|0\ra -|1\ra)(\la0|-\la1|)}{2}, \label{eq12} \\
		E_3 & \equiv \mathds{1}-E_1-E_2\label{eq13}. 
	\end{align}

Es fácil ver que los tres operadores son hermíticos, positivos y debido a como
se construye el tercer operador se satisface las relaciones de completitud. Los
operadores cumplen con ser POVM\@. Ahora, supóngase que $B$ recibió el estado
$|\psi_1\ra=|0\ra$, y realiza una medición descrita por el POVM
$\{E_1,E_2,E_3\}$. Hay una probabilidad cero de que observe el resultado $E_1$,
debido a que $E_1$ ha sido construido sabiamente para asegurar que
$p(1)=\la\psi_1|E_1|\psi_1\ra=0$. Luego, si el resultado de su medición es
$E_1$, $B$ puede concluir con seguridad que el estado que recibió, fue
$|\psi_2\ra $. Un pensamiento similar conduce a mostrar que si la salida de la
medición $E_2$ ocurre, el estado que $B$ recibió sería $|\psi_1\ra$. Algunas
veces, sin embargo, $B$ obtendrá el resultado de la medición $E_3$ y no podrá
inferir nada acerca de la identidad del estado que se le dio. Mas el punto
clave, es que $B$ nunca comete un error al identificar el estado que se le ha
dado. Esta infalibilidad tiene el precio que, a veces, $B$ no obtiene
información sobre la identidad del estado.

% }}}
\subsection{El teorema de Naimark} % {{{

En esta sección se habla de que las medidas POVM no solo son más
generales que las medidas proyectivas, sino que también pueden verse como un
caso especial de estas últimas. Dado cualquier POVM puro con $k$ elementos y un
estado $\rho$ en un espacio de Hilbert de dimensión $d$, se puede encontrar un
estado $\rho \otimes \rho_0$ en un espacio de Hilbert
$\mathcal{H}\otimes\mathcal{H'}$ tal que las estadísticas de la medición POVM
original se reproducen exactamente mediante una medida proyectiva de $\rho
\otimes \rho_0$. Esta afirmación es consecuencia del teorema del Naimark
{\cite{2007geometry}}.

\begin{theorem}[\textbf{Teorema de Naimark}] 
Cualquier POVM $\{E_i\}$ en el espacio de Hilbert $\mathcal{H}$, existe una
medida proyectiva PVM $\{P_i\}$ en un espacio de Hilbert más grande
$\mathcal{K}$, de tal manera que $E_i=\Pi^\dagger P_i\Pi$, donde $\Pi$ proyecta
desde $\mathcal{K}$ hacia $\mathcal{H}$.
\end{theorem}
	

Una forma de construir la operadores $\{P_i\}$ tomar
$\mathcal{K}=\mathcal{H}\otimes\mathcal{H'}$, $P _{i}=\mathds{1} _{H}\otimes
|i\rangle \langle i|_{H'}$, y
$\Pi=\sum _{i=1}^{n}{\sqrt {E_{i}} }_{H}\otimes {|i\rangle }_{H'}$. La
probabilidad de obtener la salida $i$ con estos operadores PVM, es la misma a
la probabilidad de obtener esta con el POVM original {\cite{2007geometry}}
\begin{equation}
 \begin{split}
 p_i&=\tr \left(\Pi\rho\Pi^{\dagger }P _{i}\right)\\
 &=\tr \left(\Pi\rho \Pi^{\dagger }\left[\mathds{1}_{H}\otimes |i\rangle \langle i|_{H'}\right]\right)\\
 &=\tr \left(\rho\left(\sum _{j=1}^{n}{\sqrt {E{j} }}_{H}^{\dagger }\otimes {\langle j|}_{H'}\right)\mathds{1} _{H}\otimes |i\rangle \langle i|_{B}\left(\sum _{k=1}^{n}{\sqrt {E_{k} }}_{H}\otimes {|k\rangle }_{H'}\right)\right)\\
 &=\tr \left(\rho {\sqrt {E_{i} }}_{H}\mathds{1} _{H}{\sqrt {E_{i} }}_{H}\right)\\
 &=\tr (\rho E_{i}).
 \end{split}
\end{equation}

Con el teorema de Naimark  se ultima el formalismo de las medidas POVM necesario para el objetivo de este trabajo. En el siguiente capítulo, se discutirá sobre las operaciones cuánticas y su importancia en sistemas abiertos.
% }}}


