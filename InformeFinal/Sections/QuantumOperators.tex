\chapter{Operaciones cuánticas}\label{OpCuanticas}
% Intro {{{


Los sistemas reales sufren interacciones indeseadas con el mundo exterior, éstas aparecen como ruido en información cuántica. \cpnote{No es buen estilo repetir palabras sin un buen motivo. Puedes partir esta frase en dos y decir ``\ldots exterior. Éstas se presentan\ldots'' es mas, no diría ``presentan'' pues creo que no es lo que quieres decir}\rrnote{Ya lo he corregido.}  
 %Para construir sistemas de procesamiento en información cuántica útiles es necesario entender y controlar dicho ruido.
 Las operaciones cuánticas son un herramienta capaz de describir dicho ruido cuántico y el comportamiento de sistemas cuánticos abiertos. %\cpnote{nop, ese no es el motivo\ldots}  
 Adicionalmente, el formalismo de las operaciones cuánticas describe los efectos de la medición en un sistema cuántico y representa la descripción de los cambios
 %Otra de las aplicaciones de las operaciones cuánticas es que se adaptan a la descripción de
 de estados discretos, esto es transformaciones entre el estado inicial $\rho$ y el estado final $\rho'$, sin necesidad de referenciar al paso del tiempo {\cite{nielsen_chuang_2010}}.\cpnote{Este ultimo mensaje me parece valioso, pero trata de hacer mas lógico y sincero este parrafo} \rrnote{He tratado de hacer que el párrafo lleve una mejor secuencia.}


Este capítulo tiene la siguiente estructura. En la primer sección, se discute
el formalismo de las operaciones cuánticas desde una perspectiva axiomática. En
la segunda sección se aborda las operaciones cuánticas desde un punto de vista
diferente, que permitirá familiarizarse con la teoría básica de las operaciones
cuánticas e involucra a los operadores conocidos como operadores de Kraus. Además se discute sobre las mediciones y su representación como suma de operadores. Y finalmente se presenta un teorema importante sobre la libertad unitaria en la representación de suma de operadores para una operación cuántica.
 %En la última sección se habla sobre el procedimiento de tomografía cuántica con el que se puede determinar experimentalmente la dinámica a la que se somete un sistema cuántico. 
 \cpnote{De como viene el escrito, parece qeu esta tercera sección está fuera de contexto. Trata de hilar mejor esta idea. De hecho no se porque 
la vas a discutir\ldots }\rrnote{La iba a discutir porque pensé que podría servir en algún momento pero creo que no será necesario para este proyecto.}
%En la tercera sección se ilustran algunos ejemplos de canales cuánticos que algunos como la despolarización, la amortiguación de amplitud y la amortiguación de fase. 
% }}}
\section{Aproximación axiomática de las operaciones cuánticas} % {{{
En esta sección se abordan las operaciones cuánticas desde un punto de vista
axiomático el cual será motivado por requerimientos físicos. \cpnote{porque ``por lo tanto''? no veo porque\ldots quizá escribiste esta seccion un poco cansada
porfa revisa.}\rrnote{Ya he revisado esta y la siguiente sección y ya he hecho algunas correcciones.}Nielsen y
Chuang {\cite{nielsen_chuang_2010}}
presentan la siguiente definición de operación cuántica \begin{definition}[\textbf{Operación cuántica}]\label{DefE(rho)} Una operación cuántica $\mathcal{E}$ es un mapeo de un conjunto de operadores en un espacio de Hilbert $\mathcal{H}_A$ de entrada a otro conjunto de operadores en un espacio de Hilbert $\mathcal{H}_B$ de salida, $\E: \mathcal{H}_A \rightarrow \mathcal{H}_B$, con las siguientes propiedades axiomáticas:

    \begin{itemize}
        \item \textit{Axioma 1:} La traza $\tr (\mathcal{E}(\rho))$ es la probabilidad de que el proceso representado por $\mathcal{E}$ ocurra, cuando $\rho$ es el estado inicial. Consecuentemente, $0 \le \tr (\mathcal{E}(\rho)) \le 1$ para cualquier estado $\rho$.
        \item \textit{Axioma 2:} El mapeo $\mathcal{E}$ es linealmente convexo en el conjunto de matrices de densidad, esto es, para probabilidades $\{p_i\}$, \[\mathcal{E}\left(\sum _i p_i \rho _i\right)=\sum_i p_i \mathcal{E}(\rho_i).\]
        \item\textit{Axioma 3:} El mapeo $\E$ es completamente positivo. Esto significa que $\E$ mapea a operadores en el espacio de Hilbert $\mathcal{H}_{A}$  a operadores en el espacio de Hilbert $\mathcal{H}_B$ y si se extiende el espacio de Hilbert de entrada a $\mathcal{H}_A\otimes\mathcal{H}_{A'}$ y se considera el mapeo extendido $\E \otimes \mathds{1}$ que mapea de $\mathcal{H}_A \otimes \mathcal{H}_{A'} $ a $\mathcal{H}_B \otimes \mathcal{H}_{A'}$, entonces tal  extensión es positiva también.
    \end{itemize}
\end{definition}

En el caso de las mediciones resulta conveniente el primer axioma, se hace la convención de que $\E$ no necesariamente preserve la traza de los operadores de densidad, $\tr(\rho)=1$. Para verlo mejor, suponga que se realiza una medición proyectiva en la base computacional de un solo qubit. Entonces la operación cuántica $\E$ describirá este proceso si se define el mapeo como $\E_0\equiv|0\rala0|\rho |0\rala 0|$ y $\E_1\equiv|1\rala1|\rho |1\rala 1|$. Las probabilidad de las salidas serán entonces $\tr (\E_0(\rho))$ y $\tr (\E_1(\rho))$ respectivamente.  Con esta convención la normalización correcta para el estado final será \[\dfrac{\E(\rho)}{\tr (\E (\rho))}.\]

En el caso que no se realice ninguna medición, esto se reduce al requisito de que $\tr[\E(\rho)] = 1 = \tr(\rho)$, para todo $\rho$. En este caso la operación cuántica $\E$ conserva la traza, ya que por sí sola proporciona una descripción completa del proceso cuántico. Por otro lado, si existe una $\rho$ tal que $\tr(\E(\rho)) < 1$, entonces $ \E$ por si sola no provee una descripción completa del proceso que ocurrió en el sistema. Una operación cuántica física es aquella que satisface que la probabilidad nunca es mayor a uno {\cite{nielsen_chuang_2010}}.

\begin{comment}
Asimismo, una razón física para proponer el segundo axioma, es que se espera que la evolución de un estado cuántico sea lineal debido a que de esa forma es compatible con la interpretación del operador de densidad como un ensamble de posibles estados. Supóngase que $\E$ mapea al estado inicial $\rho$  en el tiempo $t=0$ al estado final al tiempo $t=T$, el estado $\rho_i$ es preparado con una probabilidad $p_i$. Luego el estado de evolución temporal en $t = T$ será $\E(\rho_i )$ con probabilidad $p_i$, por lo tanto el estado final $\rho'$ evoluciona como \begin{equation}
\rho'= \sum_i p_i \E (\rho_i).
\end{equation}

Por otro lado, el estado inicial es descrito por $\sum_i p_i \rho$, que evoluciona así 
\begin{equation}
    \rho'= \E\left(\sum_i p_i \rho_i\right).
\end{equation} Igualando las dos ecuaciones anteriores, se tiene  que $\E$ debe actuar linealmente, en combinaciones convexas de estados {\cite{preskill2020quantum}}.
\end{comment}

Asimismo, para ver una razón física para proponer el segundo axioma, supóngase que un estado inicial $\rho_i$ está preparado con un probabilidad $p_i$ y luego se realiza la medición. Si el estado es $\rho_i $ luego la salida de la medición  $\alpha$ ocurre con la probabilidad condicional $p(\alpha|i)$, y el estado de la medición  posterior es $\E_\alpha(\rho_i)/p(\alpha|i)$; luego el ensamble de estado después de la medición está descrita por el operador de densidad 
\begin{equation}\label{lineal_y_convexo}
    \rho'=\sum_i p(i|\alpha)\dfrac{\E_\alpha(\rho_i)}{p(\alpha|i)},
\end{equation}

donde $p(i|\alpha)$ es la probabilidad a posteriori que el estado $\rho_i$ fuera preparado, tomando en cuenta la información obtenida haciendo la medición {\cite{preskill2020quantum}}. 

Por otro lado, aplicando la operación $\E_\alpha$  a la combinación convexa del estado inicial $\{\rho_i\}$ \begin{equation}
    \rho'=\dfrac{\E_\alpha\left(\sum_i p_i \rho_i\right)}{p_\alpha},
\end{equation} tomando en cuenta la regla de Bayes  $p(i|\alpha)=\dfrac{p_i p(\alpha|i)}{ p_\alpha} $ a la ecuación {\ref{lineal_y_convexo}},  se ve que $\E_\alpha$ debe ser lineal y convexo {\cite{preskill2020quantum}} \begin{equation}
    \E_\alpha\left(\sum_i p_i \rho_i\right)=\sum_i p_i\E_\alpha(\rho_i). 
\end{equation}

La tercera propiedad también se origina por un requerimiento físico. Es razonable exigir que un mapeo sea completamente positivo si se va a describir la evolución temporal de un sistema cuántico. Aunque el mapeo actúa solo en una parte del sistema, debe representar un estado inicial de todo el sistema hacia un estado final del sistema entero. Si $\rho_{AB}$ es una matriz de densidad de un sistema conjunto de $\mathcal{H}_A$ y $\mathcal{H}_{B}$  y $\E$ actúa solamente sobre $\mathcal{H}_A$, entonces $\E(\rho_{AB})$  debe ser un operador de densidad también. Formalmente, supongamos que introducimos un segundo sistema $B$ (de dimensión finita). Sea $\mathds{1}_{B}$ el mapa identidad del sistema $B$. Entonces el mapa $\E\otimes \mathds{1}_{B}$ debe llevar operadores positivos a operadores positivos {\cite{nielsen_chuang_2010}}, {\cite{preskill2020quantum}}. 

Es importante tener en cuenta que no todos los mapas positivos son completamente positivos; la positividad completa es una condición más fuerte. Un ejemplo de un operador positivo pero no completamente positivo es la transpuesta, $T:\rho \mapsto {\rho}^T$, dado que \[\la \psi |{\rho}^T| \psi \ra=\sum_{i,j} \psi _j^* {(\rho)}^T _{ji} \psi _i=\sum _{i,j} \psi _i{(\rho)} _{ij} \psi_j^{*}=\la {\psi}^*|\rho|\psi ^*\ra,\] cumple con ser positivo para cualquier estado $| \psi \ra$. Sin embargo $ {T} $ no es completamente positivo. Tomando un estado en el espacio $AB$ \[\Psi_{AB} \equiv \sum_i |i,i\ra, \] la extensión de $T$ actuando en este estado es \[T \otimes \mathds{1}:|\Psi\rala \Psi|=\sum_{i,j}|i,i\rala j,j| \mapsto \sum_{i,j}|j,i\rala i,j|.\]

Esto es aplicar el operador SWAP, el cual intercambia los sistemas $A$ y $B$:
\begin{equation}
    \text{SWAP}:|\psi\ra_A\otimes|\phi\ra_B=\sum_{i,j}\psi_i\phi_j |i,j\ra\mapsto\sum_{i,j}\phi_j\psi_i|j,i\ra=|\phi\ra_A\otimes|\psi\ra_B.
\end{equation} Al aplicar dos veces el operador SWAP se obtiene el mismo estado inicial, por tanto el cuadrado de SWAP es la identidad y sus valores propios serán $\pm 1$. El operador SWAP tiene valores propios negativos significa que $T \otimes \mathds{1}$ no es positivo, en consecuencia, no cumple con ser completamente positivo  {\cite{preskill2020quantum}}.




% }}}
\section{Representación como suma de operadores y los operadores de Kraus} % {{{

Una operación cuántica también puede representarse en una forma elegante como la \textit{representación como suma de operadores}. La representación de suma de operadores de un mapeo $\E$, se puede escribir como $\E(\rho)=\sum_i K_i \rho K_i^\dagger$, para algún conjunto de operadores $\{K_i\}$ que satisfacen la condición $\sum_k K_i^\dagger K_i\le \mathds{1}$.

Nielsen y Chuang {\cite{nielsen_chuang_2010}} presentan un teorema y una prueba de que la representación de suma de operadores es equivalente a la definición de la sección anterior.

\begin{theorem}
    El mapeo $\E$ satisface los axiomas de la definición {\ref{DefE(rho)}} si y solo si 
    \begin{equation}
        \E(\rho)=\sum_i K_i \rho K_i^\dagger,
    \end{equation}
    para algún conjunto de operadores $\{K_i\}$ el cual mapea el espacio de Hilbert de entrada al espacio de Hilbert de salida, y $\sum_i K_i^\dagger K_i\le \mathds{1}$
\end{theorem}


\begin{proof}
Suponiendo que $\E$ satisface {\ref{DefE(rho)}}. Se introduce un sistema $\mathcal{H}_B$, con las mismas dimensiones del sistema cuántico original $\mathcal{H}_A$. Sea $|k_A\ra$ y $|k_B\ra$  bases ortonormales para $A$ y $B$ respectivamente. Se define un estado conjunto del sistema $\mathcal{H}_A\otimes \mathcal{H}_B$ como \[|\alpha\ra \equiv \sum_k |k_A\ra |k_B\ra.\] El estado $|\alpha\ra $ es un estado máximamente entrelazado de los sistemas $\mathcal{H}_A$ y $\mathcal{H}_B$.También se define un operador $\sigma$ en espacio de estado de $\mathcal{H}_A\otimes \mathcal{H}_B$ dado por 
\begin{equation}\label{sigma1}
    \sigma \equiv (\E\otimes\mathds{1}_B)(|\alpha\rala \alpha|).
\end{equation}


Se puede pensar en esto como un resultado de aplicar la operación cuántica $\E$ a la mitad de un estado de máxima entrelazamiento del sistema $\mathcal{H}_A\otimes \mathcal{H}_B$. Para saber como actúa $\E$ en un estado arbitrario de $\mathcal{H}_A$, es suficiente con saber cómo este actúa en un solo estado máximamente entrelazado de $\mathcal{H}_A$ con otro sistema.

El truco que permite recuperar a $\E$ de $\sigma$ es el siguiente. Sea $|\psi\ra=\sum_j \psi_j |j_A\ra$ algún estado del sistema $\mathcal{H}_A$. Definiendo un estado correspondiente $|\tilde{\psi}\ra$ del sistema $\mathcal{H}_B$ por $|\tilde{\psi}\ra \equiv \sum_j \psi_j^*|j_B\ra$. Notar que
\begin{equation}
    \begin{split}
        \la \tilde \psi |\sigma |\tilde{\psi}\ra&= \la \tilde \psi |\left(\sum_{kj} \E (|k_A \rala j_A|)\otimes |k_B\rala j_B|\right) |\tilde{\psi}\ra\\
        &=\sum_{kj} \psi_k \psi_j^* \E (|k_A \rala j_A|)\\
        &=\E(|\psi\rala\psi|).
    \end{split}
\end{equation}


Sea $\sigma=\sum_i|s_i\rala s_i|$ una descomposición de $\sigma$, donde los vectores $|s_i\ra$ no necesitan ser normalizados. Definiendo el mapeo \[K_i(|\psi\ra) \equiv \la\tilde{\psi}|s_i\ra.\] Este mapeo es lineal, tal que $K_i$ es un operador lineal en el espacio de estado $\mathcal{H}_A$. Más aún, se tiene que 
\begin{equation}
    \begin{split}
       \sum_i K_i |\psi \rala \psi |K_i^\dagger&=\sum_i  \la\tilde{\psi}|s_i\ra \la s_i|\tilde{\psi}\ra\\
        &=\la \tilde \psi |\sigma |\tilde{\psi}\ra\\
        &=\E(|\psi \rala\psi|).\\
    \end{split}
\end{equation}



De ello se tiene que  \[\E(|\psi\rala \psi|)=\sum_i K_i |\psi \rala \psi |K_i^\dagger,\] para todos los estado $|\psi\ra$, de $\mathcal{H}_A$. Por el segundo axioma, en general se sigue que \[\E(\rho)=\sum_i K_i \rho K_i^\dagger.\] La condición $\sum_i K_i^\dagger K_i\le \mathds{1}$ se sigue del primer axioma, identificando la traza de $\E(\rho) $ con una probabilidad.


Por otro lado, supóngase que $\E(\rho)=\sum_i K_i \rho K_i^\dagger$. Es fácil ver que es lineal, pero falta chequear la completa positividad. Sea $\mathcal{O}$ sea un operador positivo actuando en el espacio de estado del sistema extendido, $\mathcal{H}_A\otimes \mathcal{H}_B$, y sea $|\psi \ra$ algún estado del sistema $\mathcal{H}_A\otimes \mathcal{H}_B$. Definiendo $|\varphi_i\ra \equiv (K_i^\dagger \otimes \mathds{1}_B)|\psi\ra$, se tiene que \[\la \psi |(K_i \otimes \mathds{1}_B)\mathcal{O}(K_i^\dagger \otimes \mathds{1}_B)|\psi\ra=\la \varphi_i |\mathcal{O}|\varphi_i\ra \ge 0,\] por la positividad de $\mathcal{O}$. Se sigue que $\la \psi| (\E \otimes \mathds{1}_B)(\mathcal{O})|\psi\ra=\sum_i \la \varphi_i|\mathcal{O}|\varphi_i\ra \ge 0$, y esto para cualquier operador positivo $\mathcal{O}$, el operador $(\E\otimes \mathds{1}_B)(\mathcal{O})$ es positivo como se requiere.
\end{proof}

La representación de suma de operadores describe la dinámica del sistema principal sin tener que considerar explícitamente las propiedades del entorno; todo lo que se necesita saber está agrupado en los operadores $\{K_i\}$, que actúan solo en el sistema principal {\cite{nielsen_chuang_2010}}.



 La \textit{representación de Kraus} es un mapeo lineal $\E$ que es completamente positivo de la forma $\rho \mapsto \rho'=\sum_i K_i\rho K_i^\dagger$. Los operadores $\{K_i\}$ son conocidos como los \textit{operadores de Kraus} y cumplen con la relación de completitud $\sum_i K_i^\dagger K_i=\mathds{1}_N$. La representación de Kraus fue introducida por el físico alemán Karl Kraus en 1971, basada en el resultado del teorema de Stinespring. Los mapeos completamente positivos que preservan la traza se conocen por varios nombres: \textit{operaciones cuánticas deterministas o propias, canales cuánticos o mapas estocásticos} {\cite{2007geometry}}.


\subsection{Mediciones y representación de Kraus}\label{Medicion_RepresentacionDeKraus}

Considere un sistema primario $\mathcal{H}_Q$, inicialmente en el estado $\rho$,  el cual está en contacto con un sistema auxiliar $\mathcal{H}_A$ (este también es llamado <<ancilla>> y puede ser pensado como el ambiente o un aparato de medición), inicialmente en el estado $\sigma=\sum_k \lambda_k |e_k\rala e_k|$, donde esta representación es la descomposición espectral de $\sigma$. El sistema auxiliar está sujeto a las mediciones de von Neumann, descritas por los operadores de proyección $P_\alpha=\sum_j |f_{\alpha j}\rala f_{\alpha j}|$,  donde $|f_{\alpha j}\ra$ forman una base ortonormal de $\mathcal{H}_A$ y satisfacen la relación de completitud. Si el resultado de la medición es $\alpha$, el sistema auxiliar es observado estando en el subespacio $S_\alpha$. Los dos sistemas interactúan durante un tiempo, interacción descrita por un operador unitario $U$ que actúa sobre el sistema conjunto $\mathcal{H}_Q\otimes \mathcal{H}_A$. El estado no normalizado del sistema después de la medición es obtenido proyectando el estado conjunto, en el subespacio $S_\alpha$ y luego aplicando la traza parcial sobre $\mathcal{H}_A$,

\begin{equation}\label{measurement_model}
    \tr_A(P_\alpha U \rho \otimes \sigma U^\dagger P_\alpha)= \tr_A(P_\alpha U \rho \otimes \sigma U^\dagger)\equiv \E_\alpha(\rho),
\end{equation}


donde $\E_\alpha$ es un mapeo lineal en el sistema de los operadores de densidad. Este método de definir un conjunto de operaciones cuánticas en términos de interacción con un ancilla inicialmente no correlacionado, seguido de la medición en el ancilla, se denomina modelo de medición {\cite{unm2014},\cite{nielsen_chuang_2010}}.

Reescribiendo, la ecuación {\ref{measurement_model}}, en términos de la representación de $P_\alpha$ y de $\sigma$ se obtiene que,
\begin{equation}
    \E_\alpha(\rho)=\sum_{j,k}\sqrt{\lambda_k}\la f_{\alpha j}|U|e_k\ra \rho \la e_k|U|f_{\alpha j}\ra \sqrt{\lambda_k}=\sum_{j,k}M_{\alpha j k}\rho M_ {\alpha j k}^\dagger,
\end{equation}
donde \begin{equation} \label{KrausOp1}
    M_{\alpha j k}\equiv \sqrt{\lambda_k}\la f_{\alpha j}|U|e_k\ra, 
\end{equation}

y cumple la relación de completitud \[\sum_{\alpha, j,k} M_{\alpha,j,k}^\dagger M_{\alpha,j,k}=\sum_{\alpha, j,k} \lambda_k \la e_k|U|f_{\alpha j}\ra \la f_{\alpha j}|U|e_k\ra= \tr_A(U^\dagger U \sigma)=\tr_A(\mathds{1}_Q\otimes\sigma)=\mathds{1}_Q,\] con lo cual se obtiene la representación de Kraus de la operación $\E_\alpha$, y se definen los operadores de Kraus con la ecuación {\ref{KrausOp1}}. La probabilidad de obtener el resultado $\alpha$ en la medición del ancilla es {\cite{unm2014}}
\begin{equation}
    p_\alpha=\tr(P_\alpha U\rho\otimes\sigma U^\dagger)=\tr(\E_\alpha(\rho))=\tr\left(\rho \sum _{j,k} M_{\alpha, j,k}^\dagger M_{\alpha,j,k}\right)=\tr(\rho M_\alpha).
\end{equation}


Los operadores $E_\alpha \equiv \sum _{j,k} M_{\alpha, j,k}^\dagger M_{\alpha,j,k}=\sum_{j,k} \lambda_k \la e_k|U|f_{\alpha j}\ra \la f_{\alpha j}|U|e_k\ra=\tr_A(U^\dagger P_\alpha U\sigma) $ son claramente positivos y satisfacen la relación de completitud. Cualquier modelo de medición da lugar a una medida POVM que describe las estadísticas de medición. El estado normalizado del sistema posterior a la medición, condicionado al resultado $\alpha$, es {\cite{unm2014}}
\begin{equation}
    \rho_\alpha =\dfrac{\E_\alpha (\rho)}{\tr(\E_\alpha(\rho))}=\dfrac{\E_\alpha(\rho)}{p_\alpha}=\dfrac{1}{p_\alpha}\sum_{j,k}M_{\alpha,j,k}\rho M_{\alpha,j,k}^\dagger.
\end{equation}

Si no se sabe el resultado de la medición en el sistema auxiliar, el estado después de la medición es obtenido promediando sobre las posibles resultados de las mediciones {\cite{unm2014}}
\begin{equation}
    \rho'=\sum_\alpha p_\alpha \rho_\alpha=\sum_\alpha \E_\alpha(\rho)=\sum_{\alpha,j,k} M_{\alpha,j,k}\rho M_{\alpha,j,k}^\dagger\equiv\E(\rho).
\end{equation}


\subsection{No unicidad en la representación de Kraus}
 La representación de Kraus provee una descripción bastante general de la dinámica de los sistemas cuánticos abiertos sin embargo esta representación no es única. Es decir que distintos conjuntos de operadores $\{K_i\}$ y $\{M_j\}$ pueden generar la misma operación cuántica. Esto es importante puesto que, diferentes procesos físicos pueden dar lugar a la misma dinámica del sistema y entender la libertad de la representación es crucial para una buena comprensión de la corrección de errores cuánticos. Nielsen y Chuang {\cite{nielsen_chuang_2010}} proponen formalmente el teorema y su demostración.


 \begin{theorem}[Libertad unitaria en la representación de Kraus]\label{Libertad_unitaria}
    
    Sean \[\{K_1,\ldots,K_n\} \text{ y } \{M_1,\ldots , M_m\}\] conjuntos de operadores que generan operaciones cuánticas $\E$ y $\mathcal{F}$, respectivamente. Agregando operadores cero a la lista más corta de operadores, para asegurar que $m=n$. Luego $\E=\mathcal{F}$ si y solo si existe números complejos $u_{ij}$ tales que $K_i=\sum_j u_{ij}M_j$, y $u_{ij}$ es una matriz unitaria de $m$ por $m$
    
 \end{theorem}

\begin{proof}
Para probar este teorema se hace uso del teorema {\ref{teorema2.4}}. Este resultado permite caracterizar la libertad en la representación de Kraus. Suponiendo que $\{K_i\}$ y $\{M_j\}$ son dos conjuntos de operadores para la misma operación cuántica, $\sum_i K_i \rho K_i^\dagger= \sum_j M_j \rho M_j^\dagger$ para todo $\rho$. Se definen los siguientes estados
\begin{equation}
    |e_i\ra\equiv \sum_k (K_i|k_A\ra)|k_B\ra
\end{equation}
\begin{equation}
    |f_j\ra\equiv \sum_k (M_j|k_A\ra)|k_B\ra.
\end{equation}


Además se toma el operador $\sigma$ definido en la ecuación {\ref{sigma1}}, de la cual se obtiene que $\sigma=\sum_i|e_i \rala e_i|=\sum_j |f_j\rala f_j|$, y luego existe una matriz unitaria tal que $  |e_i\ra=\sum_{j}u_{ij}|f_j\ra$.

Para un estado arbitrario $|\psi\ra$ 
\begin{equation}
    \begin{split}
        K_i|\psi\ra&=\sum_k \psi_k K_i|k_A\ra\\
                   &= \sum_{lk}\psi_l(K_i|k_A\ra)\la l_B |k_B\ra\\
                   &=\sum_{l}\psi_l\la l_B| e_i\ra=\sum_{jl}\psi_l u_{ij}\la l_B|f_j\ra\\
                   &=\sum_{jkl}\psi_l u_{ij}(M_j|k_A\ra)\la l_B|k_B\ra\\
                   &=\sum_{jl}\psi_l u_{ij}M_j|l_A\ra=\sum_j u_{ij} M_j |\psi\ra,\\
    \end{split}
\end{equation} luego \[K_i=\sum_j u_{ij}M_j.\]Por otra parte, se supone que $K_i$ y $M_j$ están relacionados por la matriz unitaria $K_i=\sum_j u_{ij}M_j$. Entonces, se tiene que
\begin{equation}
    \begin{split}
        \E(\rho)&=\sum_i  K_i\rho K_i^\dagger\\
            &=\sum_{ij} u_{ij}u_{ij}^*M_j\rho M_j^\dagger\\
            &=\sum_j M_j\rho M_j^\dagger\\
            &=\mathcal{F}(\rho).
    \end{split}
\end{equation}

Con ello se muestra que el conjunto de operadores $\{K_i\}$ es la misma operación cuántica que con el conjunto $\{M_j\}$.
\end{proof}

Para ejemplificar el teorema anterior Nielsen y Chuang {\cite{nielsen_chuang_2010}} proponen las siguientes operaciones $\E(\rho)=\sum_i K_i\rho K_i^\dagger$ y $\mathcal{F}(\rho)=\sum_i M_i\rho M_i^\dagger$, donde los conjuntos $\{K_n\}$ y $\{M_m\}$ tienen elementos definidos por 
\begin{equation}
    \begin{array}{ccc}
        K_1=\dfrac{1}{\sqrt{2}}\begin{bmatrix}
            1&0\\
            0&1\\
        \end{bmatrix},&&K_2=\dfrac{1}{\sqrt{2}}\begin{bmatrix}
            1&0\\
            0&-1\\
        \end{bmatrix}.\\
    \end{array}
\end{equation}

Esta operación $\E$ representa $1/2$ de probabilidad de aplicar el operador unitario y $1/2$ de probabilidad de aplicarle $Z$ al sistema cuántico y
\begin{equation}
    \begin{array}{ccc}
        M_1=\begin{bmatrix}
            1&0\\
            0&0\\
        \end{bmatrix},&&M_2=\begin{bmatrix}
            0&0\\
            0&1\\
        \end{bmatrix},\\
    \end{array}
\end{equation}

la segunda operación $\mathcal{F}$ corresponde a realizar una medición proyectiva en la base $\{|0\ra , |1\ra\}$, con el resultado de la medida desconocida. 



Las operaciones $\E$ y $\mathcal{F}$ son, matemáticamente la misma operación cuántica. Para verlo, notar que existe la matriz unitaria \[U=\begin{bmatrix}u_{11}&u_{12} \\u_{21}&u_{22}\end{bmatrix}=\dfrac{1}{\sqrt{2}}\begin{bmatrix}1&1 \\1&-1\end{bmatrix}\]tal que \[M_1=\sum_{j} u_{1j}K_j=(K_1+K_2)/\sqrt{2}\] y \[M_2=\sum_j u_{2j}K_j=(K_1-K_2)/\sqrt{2}\] luego,
\begin{equation} 
    \begin{split}
        \mathcal{F}(\rho)&=\dfrac{(K_1+K_2)\rho(K_1^\dagger+K_2^\dagger)+(K_1-K_2)\rho(K_1^\dagger-K_2^\dagger)}{2}\\
        &=K_1\rho K_1^\dagger+K_2\rho K_2^\dagger\\
        &=\E(\rho).
    \end{split}
\end{equation}
% }}}

\begin{comment}

\section{Proceso de tomografía cuántica} % {{{

Antes de describir el proceso de tomografía cuántica, vale la pena presentar la tomografía de estado cuántico. La tomografía de estado cuántico es el procedimiento experimental para determinar un estado cuántico desconocido. Es posible estimar $\rho$ si se tiene un gran número de copias de $\rho$. Por ejemplo, si $\rho$ es el estado cuántico producido por algún experimento, se puede repetir el experimento muchas veces para producir muchas copias del estado $\rho$ {\cite{nielsen_chuang_2010}}. 

Suponiendo que se tiene varias copias de un operador de densidad de un qubit $\rho$. El conjunto $\mathds{1}/\sqrt{2}$, $X/\sqrt{2}$, $Y/\sqrt{2}$, $Z/\sqrt{2}$ forma una base ortonormal de matrices, tal que $\rho$  puede escribirse con \[\rho=\dfrac{\tr(\rho)\mathds{1}+\tr(X\rho)X+\tr(Y\rho)Y+\tr(Z\rho)Z}{2}.\]

Por ejemplo, para estimar $\tr(Z\rho)$ se mide el observable $Z$ muchas veces, $m$, obteniendo salidas como $z_1,z_2,\ldots,z_m$, todos iguales a $+1$ o $-1$. El promedio empírico de esas cantidades, $\sum_i z_i/m$, es un estimado para el valor real de $\tr(Z\rho)$. Usando el teorema del límite central para determinar qué tan bien se comporta esta estimación para $m$ grande, donde se vuelve aproximadamente gaussiana con media igual a $\tr(Z\rho)$ y con desviación estándar  $\Delta(Z)/ m$, donde $\Delta(Z)$ es la desviación estándar para una sola medición de $Z$, que tiene un límite superior de 1, por ende, la desviación estándar en la estimación $\sum_i z_i/m$ es como máximo $1/\sqrt{m}$ {\cite{nielsen_chuang_2010}}. 

Al generalizar el procedimiento para el caso de más de un qubit es similar,
\[\rho=\sum_{\vec{v}}\dfrac{\tr(\sigma_{v_1}\otimes\sigma_{v_2}\otimes \ldots \otimes \sigma_{v_n}\rho)\sigma_{v_1}\otimes\sigma_{v_2}\otimes \ldots \otimes \sigma_{v_n}}{2^n},\] donde la suma es sobre los vectores $\vec{v} = (v_1 ,\ldots, v_n )$ con entradas vi elegidas del conjunto $\{0, 1, 2, 3\}$. Al realizar mediciones de observables que son productos de matrices de Pauli, se puede estimar cada término en esta suma, y así obtener una estimación de $\rho$ {\cite{nielsen_chuang_2010}}.

Para el proceso de tomografía cuántica, el objetivo es tener una forma de determinar una representación útil de $\E$  de los datos experimentales disponibles. El objetivo es encontrar un conjunto de operadores de Kraus $\{K_i\}$ para $\E$.


Nielsen y Chuang {\cite{nielsen_chuang_2010}} presentan una forma de determinar los $K_i$  a partir de los parámetros medibles.

Primero, se toman en cuenta una base fija de operadores $\tilde{K_i}$, tales que  \begin{equation}\label{Basefija}
    K_i=\sum_m e_{im}\tilde{K}_m
\end{equation}
para algún conjunto de números complejos $e_{im}$. Luego, 
\begin{equation}\label{operacionEnTerminosDeChi}
    \E(\rho)=\sum_{mn}\tilde{K}_m\rho \tilde{K}_n^\dagger \chi_{mn},
\end{equation}
donde $\chi_{mn}\equiv \sum_i e_{im}e_{in}^*$ son las entradas de una matriz que es Hermítica por definición. Esta expresión, conocida como la \textit{representación de la matriz chi}. Esto muestra que la operación $\E$ puede ser completamente descrita por la matriz $\chi$.

En general, $\chi$ tendrá $d^4-d^2$ parámetros reales independientes, puesto que un mapeo lineal general de matrices de $d$ por $d$ a matrices de $d$ por $d$ está descrito por $d^4$ parámetros independientes, pero debido a que $\rho$ es Hermítica de traza uno, se tienen $d^2$ parámetros fijos.

Sea $\rho_j$, $1 \le j\le d^2$ una base fija para el espacio de matrices de $d\times d$; cualquier matriz de $d\times d$ puede ser escrita como una única combinación de $\rho_j$. Luego, es posible determinar $\E(\rho_j)$ por la tomografía de estado cuántico, para cada $\rho_j$.

Luego, cada $\E(\rho_j)$ puede ser expresada como una combinación lineal de estados de la base 
\begin{equation}\label{E_en_terminos_de_la_base}
    \E(\rho_j)=\sum_k \lambda _{jk}\rho_k.
\end{equation}
Ahora, $\lambda_{jk}$ puede ser determinado por algoritmos de álgebra lineal. Se puede escribir lo siguiente
\begin{equation}
    \tilde{K}_m\rho_j \tilde{K}_n^\dagger=\sum_k\beta_{jk}^{mn}\rho_k,
\end{equation}
donde $\beta_{jk}^{mn}$ son número complejos los cuales también pueden ser determinados por los algoritmos de álgebra lineal, dados los operadores $\tilde{K}_m$ y $\rho_j$.  Combinando las dos últimas expresiones con la ecuación {\ref{operacionEnTerminosDeChi}}
\begin{equation}
\sum_k \sum_{mn}\chi_{mn}\beta_{jk}^{mn}\rho_k=\sum_k\lambda_{jk}\rho_k.
\end{equation}

Debido a la independencia lineal de $\rho_k$ para cada $k$, 
\begin{equation}
    \sum_{mn}\beta_{jk}^{mn}\chi_{mn}=\lambda_{jk}.
\end{equation}

Esta relación es una condición necesaria y suficiente para la matriz $\chi$ para dar la operación cuántica correcta $\E$. Luego $\chi$ y $\lambda$ pueden verse como vectores, y $\beta$ como una matriz de $d^4\times d^4$, con las columnas nombradas por ${mn}$ y las columnas por ${jk}$. Se necesita computar la matriz inversa de $\beta_{jk}^{mn}$. Sea $\kappa$ la matriz inversa generalizada de la matriz $\beta$, que satisface \[\sum_{jk}\kappa_{jk}^{pq}\beta_{jk}^{mn}=\delta_{pm}\delta{qn},\] los elementos de $\chi$ se leen como 
\begin{equation}
    \chi_{mn}=\sum_{jk}\kappa_{jk}^{mn}\lambda_{jk}.
\end{equation}

Ahora que se determinó $\chi$ se puede obtener inmediatamente la representación de Kraus para $\E$ de la siguiente manera. Sea $U^\dagger$ una matriz unitaria que diagonaliza $\chi$,
\begin{equation}
    \chi_{mn}=\sum_{xy}U_{mx}d_x\delta_{xy}U_{ny}^*
\end{equation}

y de ello se puede verificar que  \begin{equation}
    K_i=\sqrt{d_i} \sum_j U_{ji}\tilde{K}_j
\end{equation} son elementos para $\E$. El proceso se puede resumir en determinar $\lambda$ experimentalmente usando el proceso de tomografía de estado, y luego determinar $\chi$ con $\vec{\chi}=\kappa\lambda$, la cual da una descripción completa de $\E$, incluyendo un conjunto de operadores $\{K_i\}$.
\end{comment}



% }}}



