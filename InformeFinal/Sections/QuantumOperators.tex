\chapter{Operaciones cuánticas}


Los sistemas reales tienen interacciones indeseadas con el mundo exterior, estas indeseadasinteracciones se presentan como ruido en información cuántica. Para construir sistemas de procesamiento en información cuántica útiles es necesario entender y controlar dicho ruido. Es por esto que esta sección se centra en el formalismo de las operaciones cuánticas, que son un herramienta capaz de describir ruido cuántico y el comportamiento de sistemas cuánticos abiertos. Otra de las aplicaciones de las operaciones cuánticas en información y computación cuántica  es que se adaptan a al descripción de cambios de estados discretos, esto es transformaciones entre el estado inicial $\rho$ y el estado final $\rho'$, sin necesidad de referenciar al paso del tiempo {\cite{nielsen_chuang_2010}}.


Este capítulo tiene la siguiente estructura. En la primer sección, se discute el formalismo de las operaciones cuánticas desde una perspectiva axiomática. En la segunda sección se aborda las operaciones cuánticas desde un punto de vista diferente, que permitirá familiziarse con la teoría básica de las operaciones cuánticas e involucra a los operadores conocidos como operadores de Kraus. En la tercera sección se ilustran algunos ejemplos de canales cuánticos que algunos como la despolarización, la amortiguación de amplitud y la amortiguación de fase. En la última sección se habla sobre el procedimiento de tomografía cuántica con el que se puede determinar experimentalmente la dinámica a la que se somete un sistema cuántico.

\section{Aproximación axiomática de las operaciones cúanticas}
En esta sección se abordan las operaciones cuánticas desde un punto de vista axiomático el cual será motivado físicamente, por lo que se espera que las operaciones cuánticas obedezcan. Por lo tanto se dará la siguiente definición de operación cuántica que presenta Nielsen y Chuang {\cite{nielsen_chuang_2010}}.



\begin{definition}[\textbf{Operación cuántica}] Una operación cuántica $\mathcal{E}$ es un mapeo de un conjunto de operadores en un espacio de Hilbert $\mathcal{H}_A$ de entrada a otro conjunto de operadores en un espacio de Hilbert $\mathcal{H}_B$ de salida, $\E: \mathcal{H}_A \rightarrow \mathcal{H}_B$, con las siguientes propiedades axiomáticas:

    \begin{itemize}
        \item \textit{Axioma 1:} La traza $\tr (\mathcal{E}(\rho))$ es la probabilidad de que el proceso representado por $\mathcal{E}$ ocurra, cuando $\rho$ es el estado inicial. Consecuentemente, $0 \le \tr (\mathcal{E}(\rho)) \le 1$ para cualquier estado $\rho$.
        \item \textit{Axioma 2:} El mapeo $\mathcal{E}$ es linealmente convexo en el conjunto de matrices de densidad, esto es, para probabilidades $\{p_i\}$,\[\mathcal{E}\left(\sum _i p_i \rho _i\right)=\sum_i p_i \mathcal{E}(\rho_i)\]
        \item\textit{Axioma 3:} El mapeo $\E$ es completamente positivo. Esto significa que $\E$ mapea a operadores en el espacio de Hilbert $\mathcal{H}_{A}$  a operadores en el espacio de Hilbert $\mathcal{H}_B$ y se extiende el espacio de Hilbert de entrada a $\mathcal{H}_A\otimes\mathcal{H}_{A'}$ y se considera el mapeo extendido $\E \otimes \mathds{1}$ que mapea de $\mathcal{H}_A \otimes \mathcal{H}_{A'} $ a $\mathcal{H}_B \otimes \mathcal{H}_{A'}$. Entonces  $\E$ será completamente positivo si la extensión es positiva también.
    \end{itemize}
\end{definition}

En el caso de las mediciones resulta conveniente el primer axioma, se hace la convención de que $\E$ no necesariamente preserve la traza de los operades de densidad, $\tr(\rho)=1$. Para verlo mejor, suponga que se realiza una medición proyectiva en la base computacional de un solo qubit. Entonces la operación cuántica $\E$ describirá este proceso si se define el mapeo como $\E_0\equiv|0\rala0|\rho |0\rala 0|$ y $\E_1\equiv|1\rala1|\rho |1\rala 1|$. Las probabilidad de las salidas serán entonces $\tr (\E_0(\rho))$ y $\tr (\E_1(\rho))$ respectivamente.  Con esta convención la normalización corre para el estado final será \[\dfrac{\E(\rho)}{\tr (\E (\rho))}.\]

En el caso que no se realice ninguna medición, esto se reduce al requisito de que $\tr[\E(\rho)] = 1 = \tr(\rho)$, para todo $\rho$. La operación cuántica es una operación cuántica que conserva la traza, ya que por sí sola $\E$ proporciona una descripción completa del proceso cuántico. Una operación cuántica física es aquella que satisface que la probabilidad nunca es mayor a uno {\cite{nielsen_chuang_2010}}.


Asimismo, una razón para proponer el segundo axioma, es que se espera que la evolución de un estado cuántico sea lineal debido a que de esa forma es compatible con la interpretación del operador de densidad como un ensamble de posibles estados. Supongase que $\E$ mapea al estado inicial $\rho$  en el tiempo $t=0$ al estado final al tiempo $t=T$, el estado $\rho_i$ es preparado con una probabilidad $p_i$. Luego el estado de evolución temporal en $t = T$ será $\E(\rho_i )$ con probabilidad $p_i$, por lo que el estado final $\rho'$ evoluciona como 
\begin{equation}
\rho'= \sum_i p_i \E (\rho_i).
\end{equation}

Por otro lado, el estado inicial es descrito por $\sum_i p_i \rho$, que evoluciona así 
\begin{equation}
    \rho'= \E\left(\sum_i p_i \rho_i\right).
\end{equation}
Igualando las dos ecuaciones anteriores, se tiene  que $\E$ debe actuar linealmente,  en combinaciones convexas de estados {\cite{preskill2020quantum}}.


Además, supongase de nuevo que el estado inicial $\rho_i$ está preparado con un probabilidad $p_i$ y luego se realiza la medición. Si el estado es $\rho_i $ luego la salida de la medición  $a$ ocurre con la probabilidad condicional $p(a|i)$, y el estado de la medición  posterior es $\E_a(\rho_i)/p(a|i)$; luego el ensamble de estado después de la medición está descrita por el operador de densidad 
\begin{equation}
    \rho'=\sum_i p(i|a)\dfrac{\E_a(\rho_i)}{p(a|i)}
\end{equation}

donde $p(i|a)$ es la probabilidad a posteriori de que el estado $\rho_i$ fuera preparado, tomando en cuenta la información obtenida haciendo la medición {\cite{preskill2020quantum}}
. 

Por otro lado aplicando la operación $\E_a$  a la combinación convexa del estado inicial $\{\rho_i\}$ dejando 

\begin{equation}
    \rho'=\dfrac{\E_a\left(\sum_i p_i \rho_i\right)}{p_a}
\end{equation}

tomando en cuenta la regla de Bayes, se ve que $\E_a$ debe ser lineal y convexo {\cite{preskill2020quantum}}

\begin{equation}
    \E_a\left(\sum_i p_i \rho_i\right)=\sum_i p_i\E_a(\rho_i). 
\end{equation}

La tercer propiedad se origina por un requerimienot físico. No solo $\E(\rho)$ debe ser una matriz de densidad válida siempre que $\rho$ sea válida. Si $\rho=\rho_{AA'}$ es una matriz de densidad de un sistema conjunto de $\mathcal{H}_A$ y $\mathcal{H}_{A'}$  y $\E$ actúa solamente sobre $\mathcal{H}_A$, luego $\E(\rho_{AA'})$  debe ser un operador de densidad también. No todos los mapeos positivos son completamente positivos; la completa positividad es una condición más fuerte. Un ejemplo de un operador positivo


pero no completamente positivo es la transpuesta, $T:\rho \mapsto {\rho}^T$, dado que \[\la \psi |{\rho}^T| \psi \ra=\sum_{i,j} \psi _j^* {(\rho)}^T _{ji} \psi _i=\sum _{i,j} \psi _i{(\rho)} _{ij} \psi_j^{*}=\la {\psi}^*|\rho|\psi ^*\ra,\] cumple con ser positivo para cualquier estado $| \psi \ra$. Sin embargo $ {T} $ no es completamente positivo. Tomando un estado en el espacio $AB$ \[\Phi_{AB} \equiv \sum_i |i,i\ra, \] la extensión de $T$ actuando en este estado es \[T \otimes \mathds{1}:|\Psi\rala \Phi|=\sum_{i,j}|i,i\rala j,j| \mapsto \sum_{i,j}|j,i\rala i,j|. \]

esto es aplicar el operador SWAP, el cual intercambia los sistemas $A$ y $B$:

\begin{equation}
    \text{SWAP}:|\psi\ra_A\otimes|\phi\ra_B=\sum_{i,j}\psi_i\phi_j |i,j\ra=\mapsto\sum_{i,j}\phi_j\psi_i|j,i\ra=|\phi\ra_A\otimes|\psi\ra_B
\end{equation}

Y dado que al aplicar dos veces el operador SWAP se obtiene el mismo estado inicial, por tanto el cuadrado de SWAP es la identidad y sus valores propios serán $\pm 1$. Y como SWAP tiene valores propios negativos significa que $T \otimes \mathds{1}$ no es positivo, por lo que no cumple con ser completamente positivo  {\cite{preskill2020quantum}}.





\section{Operadores de Kraus}



\section{Canales cuánticos}



\section{Proceso de tomografía cuántica}