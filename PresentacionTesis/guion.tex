\documentclass[12pt,letterpaper]{book}\raggedbottom{}


% Importing document settings from our file "packages.sty"
\usepackage{../Tesis/packages}


% Beginning of document
\begin{document}
\chapter{GUIÓN}
\section{Introducción}
\subsection{Motivación/Objetivo}
En el transcurso de último siglo la medición se ha destacada como uno de los campos de studio más notables dentro de la mecánica cuántica. Sin embargo, la mayoría de literatura introductoria suele enfocarse en mediciones proyectivas ideales. No obstante, en la práctica  las mediciones que se realizan so imperfectas es por ello que es necesario emplear un formalismo más extenso para poder proporcionar una descripción completa de las mediciones no ideales. En particular en ete trabajo se estudia  la descripción completa de las mediciones difusas. Para ello es necesario un mapeo de probabilidad de obtener las posibles salidas y el estado posterior a la medición.

\subsection{Introducción}
A continuación se abordan algunas de la herramientas generales utilizadas para analizar el proceso de mediciones no ideales, en particular la mediciones difusas. Se inicia en un marco conceptual que contiene las herramientas más básicas  en la descripción de mediciones en sistemas cuánticos. Luego se presenta wl problema de las mediciones difusas de manera concreta. Y finalmente se tratan los resultados obtenidos para sistemas de dos partículas para posteriormente generalizarlos.


\section{Revisión de literatura}


\subsection{Operador difuso}
Los operadores de densidad proporcionan un medio conveniente par describir los sistemas cuánticos generalizando los vectores dde estado. Incluso, en sistemas en los que no se tiene certeza estadística de su preparación. El operador de densidad que corresponde al ensamble de estados puros $ \{p_j,|\psi_j\rala\psi_j |\}$ es tal que \[\rho=\sum p_j |\psi_j\rala\psi_j|\]



\subsection{Medidas POVM y operadores de Kraus}
\begin{definition}[\textbf{POVM}] Una medida POVM (positive operator-valued measure) 
Las medidas POVM son  un conjunto $\{E_{m}\}$ de operadores llamados <<efectos>> que satisfacen las siguientes condiciones:
\begin{enumerate}
    \item Positividad: $\la \psi |E_m|\psi \ra \ge 0 $ para cualquier vector $|\psi\ra$.
    \item Hermiticidad: $E_m=E_{m}^\dagger$.
    \item  Completitud: $\sum_m E_m =\mathds{1}$.
\end{enumerate}
\end{definition}

las cuales se ajustan en el marco general del postulado de la medición cuántica y brindan el mapeo de probabilidades \begin{equation}\begin{split}
    E:S\times \mathcal{B(H)}\longrightarrow [0,1]\\
    E(\alpha,\rho)=\tr(E_\alpha\rho).
\end{split}\end{equation}Por otra parte los operadores de Kraus son un conjunto de operadores $\{K_i\} $ que representan una operación cuántica en forma de suma \begin{equation}
    \E(\rho)=\sum_i K_i\rho K_i^\dagger 
\end{equation}


\section{Mediciones difusas}
\subsection{Definición}
\begin{definition}\label{def:medicion-difusa}Una medición difusa es un proceso no ideal en el cual, debido a ruido del entorno o a fallos en el dispositivo de detección, se presenta la probabilidad de una identificación errónea de las partículas del sistema.
\end{definition}

\textbf{*Agregar una imagen* }


\subsection{Valor esperado y operador difuso}
El valor esperado es una medida de tendencia central que puede caracterizar a la distribución de probabilidad que representa la información capturada por un estado cuántico.

Para un sistema de dos partículas el valor esperado de un observable factorizable es 
\begin{equation}\label{eq:Expected-Value-FM-2p}
    \begin{split}
         p\tr(\rho A\tensor B)+(1-p)\tr(\rho B\otimes A)=&\la {A\otimes B}\ra_{\mathcal{F}_{2\text{p}}(\rho)}.\\
    \end{split}
\end{equation}Lo que permite introducir al operador difuso que se define como \begin{equation}\label{eq:op_F2p}
    \mathcal{F}_{2\text{p}}(\rho):=p\rho + (1-p)S_{12}\rho S_{12}^{\dagger}.
\end{equation}Además, una vez se introdujo este operador, la generalización del valor esperado, para sistemas más grandes es 
\begin{equation}\label{eq:expected-value-fm-general}
    \la \mathcal{O}\ra_{\fuzzy{\rho}}=\sum_{\Pi_i\in S}p_i\tr(\permut{i}{\rho}\mathcal{O}) =\tr(\fuzzy{\rho}\mathcal{O}).
\end{equation} donde \begin{equation}\label{eq:fuzzy-op-nparticles}
    \fuzzy{\rho}=\sum_{\Pi_i\in S}p_{i}\permut{i}{\rho}
 \end{equation} con $\Pi_i$ son los operadores de permutación.


\subsection{Instrumentos cuánticos}
SOn un ensamble que correlaciona un sistema clásico y que contiene la salida de la medición y un sistema cuántico que contiene el estado posterior a la medición \begin{equation}
    \begin{split}
        \mathcal{I}: \mathcal{B(H)}\rightarrow\mathcal{B(H)}_{\text{cl}}\otimes \mathcal{B(H)}_{\text{qu}},\\
    \mathcal{I}(\rho)=\sum_\alpha |\alpha\rala\alpha|\otimes \E_\alpha(\rho).
    \end{split}
\end{equation}sta herramienta es útil para aproximarse
a describir completamente las mediciones difusas de forma resumida y sin necesidad de
considerar mediciones selectivas. Además permiten analizarlas desde diferentes perspectivas.




\section{Resultados}

\subsection{POVM y operadores de Kraus para dos partículas}
Los efectos para una medición difusa son \[\{\mathcal{F}_{2\text{p}}(P_{a_j,b_k})\}\]los cuales proporciona el mapeo de probabilidades tal que \[   E(a_j b_k, \rho)= \tr(\mathcal{F}_{2\text{p}}({P_{a_j,b_k}})\rho).\] Sin embargo para obtener el estado posterior es la medición es necesario descomponer estos efectos, definiendo un conjunto de operadores de Kraus de manera que se cumpla que \[\mathcal{F}_{2\text{p}}(P_{a_j,b_k})=K_{a_j,b_k}^\dagger K_{a_j,b_k}.\] Por tanto se pueden considerar los siguientes operadores \[K_{a_j,b_k}=\sqrt{\mathcal{F}_{2\text{p}}(P_{a_j,b_k}\rho)}.\] Esta descomposición no es única y hay infinitas soluciones para la descomposición de los efectos.



\subsection{Instrumento cuántico I}

El primer instrumento consiste en considerar que debido a un error en el
sistema cuántico, las partículas posiblemente experimenten un intercambio. Al
medir un observable ${A \tensor B}$ en el sistema representado por el estado
$\rho$, se obtiene un resultado  en el sistema clásico que puede ser cualquiera
de los valores propios de este observable. Luego, con cierta probabilidad el
estado posterior será la proyección del estado inicial del sistema al estado
propio correspondiente a la salida proporcionada. Sin embargo también es
posible que el estado inicial sufra una transformación, y sea este cambio del
sistema el que se proyecte en el espacio propio correspondiente a la salida del
observable $A\tensor B$.

\subsection{Instrumento cuántico II}

Este escenario involucra un error en la interpretación de los resultados. El segundo instrumento implica que, debido a un fallo en el sistema clásico, la lectura de las salidas sea incorrecta. Al medir un observable ${A \tensor B}$ en el sistema, con cierta probabilidad el estado posterior será la proyección del estado inicial del sistema al estado propio correspondiente a las salida proporcionada. No obstante, existe la posibilidad que el estado inicial sea proyectado a algún estado propio del observable pero al leer las salidas de la medición estas correspondan a salidas del observable $B\otimes A$.

\subsection{Instrumento cuántico III}
 En este instrumento se representa la posibilidad que al medirse el observable $A\otimes B$ se realice con cierta probabilidad una medición proyectiva ideal. Pero, también es verosímil que se realice una medición en la que no sea posible saber en que espacio se haya proyectado el sistema. Es igual de posible que se haya proyectado sobre un espacio propio del observable $A\otimes B$ como es posible que se haya proyectado en un espacio propio de $B\otimes A$,  condicionados ambos espacios por las salidas clásicas. 


\subsection{Equivalencia}

Dado el enfoque con el que fueron concebidos los tres instrumentos, se anticipaba que el valor esperado de cada uno sería el apropiado. Sin embargo, los instrumentos propuestos no son equivalentes y vale la pena preguntarse ¿bajo qué condiciones son equivalentes?
\begin{proposition}\label{prop:Equivalencia-instruments1-2}
    Para todo estado inicial $\rho$, los valores esperados de las alternativas
de los instrumentos cuánticos I y II son equivalentes si y solo si \begin{equation}\label{eq:Condicion-equivalencia1-2}
    \la a_j
b_k|B\otimes A|a_{j'}b_{k'}\ra=0, \forall j,k\ne j',k'.
\end{equation}
\end{proposition}

\begin{proposition}
    Si se satisface la condición  \[\la a_j b_k|B\otimes A|a_{j'}b_{k'}\ra=0, \forall j,k\ne j',k', \]  entonces $[A\otimes B,B \otimes A]=0$  
\end{proposition}


\begin{proposition}\label{prop:Equivalencia-instruments-1-3}
    Para todo estado inicial $\rho$, los valores esperados de las alternativas
de instrumentos cuánticos I y III son equivalentes si y solo si
$p=\dfrac{1+q}{2}$.
\end{proposition}


\subsection{Generalización}
La clave para lograr los efectos de manera correcta es también emplear el valor
esperado de una medición difusa. A pesar de ello, la diferencia de la
generalización reside en el hecho que los operadores de permutación para $N>3$
no son hermíticos, por lo que la aplicación del operador difuso debe llevarse a
cabo con precaución.
los efectos pueden escribirse como
\begin{equation*}
    {\{E_{\lambda_i}\}}_{\lambda_i \in \Lambda}={\left\{\sum_{\Pi_j \in S} p_j \permutdagger{j}{P_{\lambda_i}}\right\}}_{\lambda_i \in \Lambda},
\end{equation*}  

Por lo tanto para este sistema se propone utilizar
la raíz cuadrada de los efectos para los respectivos operadores de Kraus 
\begin{equation}
   K_{\lambda_i}=\sqrt{\sum_{\Pi_j \in S} p_j \permutdagger{j}{P_{\lambda_i} }},
\end{equation} 
esto se puede realizar debido a la positividad de los efectos.

\subsection{Generalización Instrumentos}
La generalización de los primeros dos instrumentos es fácilmente realizable.
Con cierta probabilidad la medición ocurre de manera ideal, y
el estado posterior a la medición será la proyección del estado inicial al
espacio propio correspondiente a la salida brinda por el sistema clásico. Sin
embargo, también es verosímil que el estado posterior sea la proyección al
espacio propio de la salida pero del estado inicial transformado, de forma que
represente un sistema en el que las partículas se cambian
\begin{equation*}
    \mathcal{I}_1(\rho)=\sum_{\lambda_i \in \Lambda }P_{\lambda_i}\otimes P_{\lambda_i}\fuzzy{\rho}P_{\lambda_i}.
\end{equation*} 


Este instrumento representa una equivocación en el sistema clásico,
y por esto al medir el observable $\mathcal{O}$, con alguna probabilidad el
estado inicial se proyectará en el espacio propio de la salida correspondiente.
Pero es probable que el estado inicial se proyecte de igual forma a un espacio
propio del observable pero la lectura de los resultados de la medición sean las
salidas de un observable distinto, el cual puede ser una transformación dada
por el operador de permutación $\Pi_j\mathcal{O}\Pi_j$ para algún $\Pi_j\in
\mathcal{S}$, lo que indica que las salidas corresponden a una medición en cada
partícula, diferente a las que proporcionaría una medición ideal

\begin{equation*}
    \mathcal{I}_2(\rho)= \sum_{\lambda_i \in \Lambda } \fuzzy{P_{\lambda_i}}\tensor P_{\lambda_i}\rho P_{\lambda_i}.
\end{equation*} 


\begin{proposition}
    Para todo estado inicial $\rho$, los valores esperados de estos dos instrumentos
cuánticos son equivalentes si y solo si \[\left \langle \lambda_j \left|\Pi_l^\dagger
\mathcal{O} \Pi_l\right|\lambda_k\right\rangle=0,\forall j\ne k \text{ y }
\forall \Pi_l \in \mathcal{S}.\]
\end{proposition} 



\section{Conclusiones}

Se llevó a cabo una discusión sobre ciertos conceptos y herramientas apropiadas para lograr una descripción completa de mediciones difusas en sistemas
de dos o más partículas. En concreto, en esta tesis se emplearon dos enfoques
principales que vale la pena enfatizar.


En primer lugar, las medidas POVM junto con los operadores de Kraus, los cuales
son la primera forma de aproximarse al problema. Los efectos de las medidas
POVM  analizadas brindan una distribución de probabilidad de acuerdo a cada una
de las posibles salidas que pueden ocurrir en una medición difusa. Estos
efectos fueron originados a partir del valor esperado de la medición para
asegurar su idoneidad.  Asimismo, los efectos también pueden descomponerse
para dar origen a los operadores de Kraus, los cuales proporcionan el estado
posterior a la medición.

Como segundo enfoque, se han examinado tres diferentes instrumentos cuánticos.Estos instrumentos se utilizaron con el fin de analizar las mediciones difusas con distintas interpretaciones y para describir las mediciones de una forma sucinta, de modo que vinculan tanto las salidas clásicas como las salidas cuánticas de la medición. A pesar de que se esperaba que los instrumentos propuestos de manera intuitiva, modelaran correctamente la medición difusa, no resultó ser así. Solo uno de los tres instrumentos estudiados resultó brindar una especificación concisa y general de la medición difusa.

\end{document}
