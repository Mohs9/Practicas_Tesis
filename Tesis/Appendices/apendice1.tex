\section{Teoremas y Lemas}
\begin{lemma}\label{lemma_traza_cero} Si $\forall\rho$, $\tr(A\rho)=0$, entonces $A=\mathbf{0}$.\end{lemma}\begin{proof} Supongamos por contradicción que $A\ne \mathbf{0}$. Entonces, sea $\rho=|a\rala a|$, con $|a\rangle$ un vector propio ortonormal, de $A$. Luego, al calcular $A|a\rala a|$, tenemos $A|a\rala a|=a|a\rala a|$ entonces, la traza  de $A\rho$ es exactamente $a$, que es distinto de cero.
Entonces, hemos encontrado una matriz $A$ distinta de cero tal que la traza de $A\rho$ sea distinta de cero. De ello se deduce que $A = \mathbf{0}$ si no existe un $\rho$ distinto de cero.\end{proof}
