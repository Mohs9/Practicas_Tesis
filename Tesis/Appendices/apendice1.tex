\section{Lemas útiles}
\begin{lemma}\label{lemma_traza_cero} Si $\forall\rho$, $\tr(A\rho)=0$, entonces $A=\mathbf{0}$.\end{lemma}\begin{proof} Supongamos por contradicción que $A\ne \mathbf{0}$. Entonces, sea $\rho=|a\rala a|$, con $|a\rangle$ un vector propio ortonormal, de $A$. Luego, al calcular $A|a\rala a|$, tenemos $A|a\rala a|=a|a\rala a|$ entonces, la traza  de $A\rho$ es exactamente $a$, que es distinto de cero.
Entonces, hemos encontrado una matriz $A$ distinta de cero tal que la traza de $A\rho$ sea distinta de cero. De ello se deduce que $A = \mathbf{0}$ si no existe un $\rho$ distinto de cero.\end{proof}
 
\begin{comment}
\section{Prueba de las características de los efectos propuestos}

\begin{enumerate}
    \item Son operadores hermíticos\[{\left(\sum_{\Pi \in S} p_\Pi \Pi(P_{\lambda_i})\right)}^\dagger =\sum_{\Pi \in S} p_\Pi {\left(\Pi(P_{\lambda_i})\right)}^\dagger = \sum_{\Pi \in S} p_\Pi \Pi(P_{\lambda_i}).\]
    \item Cumplen con la ecuación de completitud \[\sum_{\lambda_i \in \Lambda}\sum_{\Pi \in S} p_\Pi \Pi(P_{\lambda_i})=\]
\end{enumerate}
\end{comment}

\begin{lemma}\label{lemma:op-difuso-hermiticidad} El operador difuso $\mathcal{F}(\rho)$ es hermítico.
\end{lemma}

\begin{proof}
 \[\mathcal{F}^\dagger{(\rho)}={\left(\sum_{\Pi_j \in S} p_j \permut{j}{\rho}\right)}^\dagger =\sum_{\Pi_j \in S} p_j{\left(\permut{j}{\rho}\right)}^\dagger=\sum_{\Pi_j \in S} p_j {(\Pi_{j}^\dagger)}^\dagger \rho\Pi_{j}^\dagger= \fuzzy{\rho}.\]
\end{proof}

 