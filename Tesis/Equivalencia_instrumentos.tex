\documentclass[12pt,oneside]{book}\raggedbottom{}
%%%%%%%%%%%%%%%%%%%%%%%%%%%%%%%% Algunos Paquetes Necesarios 
% Importing document settings from our file "packages.sty"
\usepackage{packages}
\usepackage{comment}


\begin{document}

\newcommand{\myName}{Condiciones para que los instrumentos cu\'anticos sean equivalentes}
\newcommand{\myDate}{Primer semestre 2021}
\newcommand{\myCourse}{}
\newcommand{\degre}{\ensuremath{^\circ}}
\newcommand{\R}{\mathbb{R}}
\newcommand{\vi}{\mathbf{\hat{\i}}}
\newcommand{\vj}{\mathbf{\hat{\j}}}
\newcommand{\vk}{\mathbf{\hat{\k}}}
%%%%%%%%%%%%%%%%%%%%%%%%%%%%%%%%%%%%%%%Fancyboxes


\newcounter{problem}[section]\setcounter{problem}{0}
\renewcommand{\theproblem}{%\arabic{section}.
	\arabic{problem}}

\newenvironment{problem}[1][]{%
	\refstepcounter{problem}
	
	\ifstrempty{#1}%
	% if condition (without title)
	{\mdfsetup{%
			frametitle={%
				\tikz[baseline= (current bounding box.east), outer sep=0pt]
				\node[anchor=east,rectangle,fill=purple!20]
				{\strut{} Problema~\theproblem};}
		}%
		% else condition (with title)
	}{\mdfsetup{%
			frametitle={%
				\tikz[baseline= (current bounding box.east), outer sep=0pt]
				\node[anchor=east,rectangle,fill=purple!20]
				{\strut{} Problema~\theproblem:~#1};}%
		}%
	}%
	% Both conditions
	\mdfsetup{%
		innertopmargin=10pt,linecolor=pink!40,%
		linewidth=2pt,topline=true,%
		frametitleaboveskip=\dimexpr-\ht\strutbox\relax%
	}
\begin{mdframed}[]\relax%
	\label{#1}}{\end{mdframed}}


\newcounter{solution}[section]\setcounter{solution}{0}
\renewcommand{\thesolution}{%\arabic{section}.
	\arabic{solution}}
\newenvironment{solution}[1][]{%
	\refstepcounter{solution}%
	\ifstrempty{#1}%
	{\mdfsetup{%
			frametitle={%
				\tikz[baseline= (current bounding box.east), outer sep=0pt]
				\node[anchor=east,rectangle,fill=green!20]
				{\strut{} Soluci\'on~\thesolution};}}
	}%
	{\mdfsetup{%
			frametitle={%
				\tikz[baseline= (current bounding box.east), outer sep=0pt]
				\node[anchor=east,rectangle,fill=green!20]
				{\strut{} Soluci\'on~\thesolution:~#1};}}%
	}%
	\mdfsetup{innertopmargin=10pt,linecolor=green!20,%
		linewidth=2pt,topline=true,%
		frametitleaboveskip=\dimexpr-\ht\strutbox\relax
	}
	\begin{mdframed}[]\relax%
		\label{#1}}{\end{mdframed}}

%%%%%%%%%%%%%%%%%%%%%%%%%%%%%%%%%%% Tema - BEGIN
\newtheoremstyle{Tema}% name of the style to be used
  {5mm}% measure of space to leave above the theorem. E.g.: 3pt
  {3mm}% measure of space to leave below the theorem. E.g.: 3pt
  {}% name of font to use in the body of the theorem
  {}% measure of space to indent
  {\bfseries}% name of head font
  {\newline}% punctuation between head and body
  {20mm}% space after theorem head
  {}% Manually specify head

\theoremstyle{Tema} \newtheorem{Tema}{Tema} %%%%% Template para Temas
\theoremstyle{Tema} \newtheorem{serie}{Serie}              %%%%%  Template para Series de ejercicios
\theoremstyle{Tema} \newtheorem{ejercicio}{Ejercicio}    %%%%%  Template para Ejercicios
%%%%%%%%%%%%%%%%%%%%%%%%%%%%%%%%%%% Tema - END


%%%%%%%%%%%%%%%%%%%%%%%%%%%%%%%%%%% Encabezado - BEGIN %%%%%%%%%%
\fancypagestyle{firststyle}
{
\renewcommand{\headrulewidth}{1.5pt}
\fancyhead[R]{
			%\textbf{Universidad de San Carlos de Guatemala} \\
			%\textbf{Escuela de Ciencias Físicas y Matemáticas}\\
			%\textbf{\myDate{}} \\
			%\textbf{\myCourse{} }    %%%%%%%%%% Agregar nombre del curso 
			  %%%%%%%%%%%%%%%%%%%%%% Agregar fecha en formato: Enero 15, 2015
			}
\fancyhead[L]{ 
%	\includegraphics[height=1.6 cm, width=5.5 cm]{LogoColor}  
	}
}
%%%%%%%%%%%%%%%%%%%%%%%%%%%%%%%%%%% Encabezado - END %%%%%%%%%%

%%%%%%%%%%%%%%%%%%%%%%%%%%%%%%%%%%% Encabezado (pagina 2 en adelante) - BEGIN %%%
\fancypagestyle{allStyle}
{
\renewcommand{\headrulewidth}{0.8 pt}
\fancyhead[R]{
			\emph{\myName{} $-$ \myCourse{}} %%%% Modificar n\'umero de examen parcial y nombre del curso
			}
\fancyhead[L]{}  
\fancyfoot[C]{}
\fancyfoot[R]{\thepage}
}
%%%%%%%%%%%%%%%%%%%%%%%%%%%%%%%%%%% Encabezado (pagina 2 en adelante) - END %%%

\date{}
\setlength{\headheight}{0.5in} % fixes \headheight warning
 %%%%%%%%%%%%%%%%%%%%%%%% BEGIN%%%%%%%%%%%%%%

\pagestyle{allStyle}

\thispagestyle{firststyle}
%%%%%%%%%%%%%%%%%%%%%%%%%%%%%%%%% Titulo - BEGIN
\begin{center}
\LARGE
\textsc{\myName}\\% Modificar el N\'umero del examen parcial
\bigskip
\hrule height 1.5pt
\end{center}


El valor esperado dado por la primera ecuación del paper original es 
\begin{equation}\label{ecuacion1paper}\begin{split}
    p\tr(\rho A\otimes B)+(1-p)\tr(\rho B\otimes A)=    \la A\otimes B\ra
\end{split}\end{equation}		

esta ecuación debe ser el punto de partida.



\subsection*{Aproximaciones utilizando instrumento cuánticos}

\subsubsection{Primera alternativa}

 \begin{equation}
    \begin{split}
        \mathcal{I}_1(\rho)&=\sum_{j,k}P_{a_j,b_k}\otimes P_{a_j,b_k} \mathcal{F}(\rho) P_{a_j,b_k}\\
        &=\sum_{j,k}P_{a_j,b_k}\otimes[p P_{a_j,b_k}\rho P_{a_j,b_k}+(1-p)P_{a_j,b_k}S\rho S^\dagger P_{a_j,b_k}],
\end{split}
\end{equation}

donde $j,k=0,1$ y $P_{a_j,b_k}=|a_j b_k\rala a_j b_k|$ con $a_j$ y $b_k$ son los valores propios de los operadores $A$ y $B$ respectivamente.


Notar que si el sistema clásico indica que la salida de le medición es $a_j,b_k$, el estado de salida estará en el subespacio que indica el sistema clásico.

De este instrumento se calculó el  valor esperado de la medición, de la siguiente manera:\begin{equation}\label{ExpectedValueI1}\begin{split}
	\la A\otimes B \ra_{\mathcal{I}_1}&=\sum_{jk} a_{j}b_k[p\la a_j b_k|\rho |a_j b_k\ra + (1-p)\la b_k a_j|\rho | b_k a_j\ra] \\
   & = p\tr(\rho A\otimes B)+(1-p)\tr(\rho B\otimes A)\\
\end{split}\end{equation}

El valor esperado es claramente el mismo que en la ecuación {\ref{ecuacion1paper}}
\subsubsection{Segunda alternativa}

\begin{equation}
    \begin{split}
        \mathcal{I}_2(\rho)&=\sum_{j,k} pP_{a_j,b_k} \otimes P_{a_j,b_k} \rho P_{a_j,b_k}+(1-p)SP_{a_j,b_k}S^\dagger\otimes P_{a_j,b_k} \rho P_{a_j,b_k},\\
    \end{split}
\end{equation} 


%En esta alternativa al aplicarle el operador difuso al sistema clásico se puede notar que aunque el apuntador indique que el resultado es $|b_k a_j\ra$ es posible que con una probabilidad $(1-p)$, el resultado real sea $|a_j b_k\ra$.
También se calculó el valor esperado del observable:
\[\begin{split}
	\la A\otimes B \ra_{\mathcal{I}_2}&=\tr\{[(A\otimes B) \otimes \mathds{1}]\mathcal{I}_2\} \\
	&=\sum_{j,k}[p a_{j}b_k +(1-p)\tr( (B\otimes A)P_{a_j,b_k}) ]\tr(P_{a_j,b_k} \rho)\\
	&=\sum_{j,k}[p a_{j}b_k +{\color{blue}(1-p)\tr( (B\otimes A)P_{a_j,b_k})} ]\la a_j b_k|\rho |a_j b_k\ra \\
\end{split}\]

		Notemos que el término remarcado en azul difumina la parte de los resultados. Y en general esta ecuación no es igual a la ecuación {\ref{ecuacion1paper}}. Por ejemplo si $A\otimes B= \sigma_z\otimes \sigma_x$, sin importar cuál sea el estado inicial $\rho$ la parte en azul es siempre 0 lo que difiere con el valor esperado para el instrumento 1. 


Entonces la pregunta que ahora se deb responder es ¿cuándo los dos instrumentos anteriores son equivalentes? Es decir, ¿en qué condiciones los dos instrumentos arrojan los mismo resultados?

Tenemos que al simplificar el valor esperado del segundo instrumento:
\begin{equation}\label{ExpectedValueI2}\begin{split}\la A\otimes B \ra_{\mathcal{I}_2}&=\sum_{j,k} p Tr(a_{j}b_k P_{a_j b_k}\rho)+{\color{blue}(1-p)\tr( (B\otimes A)P_{a_j,b_k})} \tr(P_{a_j b_k}\rho) \\
    &=p\tr(A\otimes B \rho)+(1-p)\sum_{j,k}\tr((B\otimes A )P_{a_j b_k} )\tr(P_{a_j b_k}\rho)
\end{split}\end{equation}


Claim: Los instrumentos tienen los mismos valores esperados, en el caso de observables no degenerados, si y solo si $A\otimes B$ y $B\otimes A$ son compatibles o bien si $\rho $ es compatible con $A\otimes B$.

\begin{proof}
	$(\Rightarrow)$
Notamos que el primer término de {\ref{ExpectedValueI1}} y el primer término de {\ref{ExpectedValueI2}} son iguales, por lo tanto se igualan los segundos términos
 de estas ecuaciones \[\tr(B\otimes A \rho)=\sum_{j,k}\tr((B\otimes A) P_{a_j,b_k})\tr(P_{a_j b_k}\rho )\] Por la linealidad de la traza se puede reacomodar el lado izquierdo: \[\begin{split}\sum_{jk}\tr((B\otimes A) P_{a_j,b_k})\tr(P_{a_j b_k}\rho )&= \sum_{jk}\tr[\tr((B\otimes A) P_{a_j,b_k})(P_{a_j b_k}\rho )]\\&=\tr\left[\sum_{j,k}\tr((B\otimes A) P_{a_j,b_k})P_{a_j b_k}\rho \right]\end{split}.\]

De ello  \[B\otimes A=\sum_{j,k} \tr((B\otimes A)P_{a_j, b_k})P_{a_j, b_k}=\sum_{j,k}d_{j,k}P_{a_j,b_k},\] lo cual está indicando que el operador \(B\otimes A\), en el caso no degenerado debe conmutar con el operador \(A\otimes B\), o más específicamente la base de operadores de proyección de $B\otimes A$ debe ser la misma que $A\otimes B$ sin importar cuales sean sus valores propios.

$(\Leftarrow)$
Ahora si suponemos que $B\otimes A=\sum_{i,l}d_{i,l}P_{a_j, b_k}$

\[\begin{split}\sum_{j,k}\tr( \sum_{i,l}d_{i,l}P_{a_j, b_k} P_{a_j,b_k})\tr(P_{a_j b_k}\rho )&=\sum_{j,k}\sum_{i,l}d_{i,l}\delta_{j,k}^{i,l}\tr(P_{a_j b_k}\rho )\\&=\sum_{j,k}d_{j,k}\tr(P_{a_j b_k}\rho )=\tr(B\otimes A\rho)\end{split}.\]


\end{proof}





%En particular, realizando la misma idea se ve que si el operador de densidad conmuta con $A\otimes B$, es decir que \[\rho=\sum_{j,k} (\tr(\rho P_{a_j, b_k})) P_{a_j, b_k},\] de nuevo sin importar cuales sean sus valores propios, puede h.

Existen otras condiciones en las mucho más específicas y particulares que cumplen la equivalencia. Una de las que me parece más interesantes e intuitivas después de ver algunos ejemplos numéricos es que $A$ y $B$ sean dos operadores no degenerados y que conmuten puesto que conservarían cierta simetría, puesto que $A\otimes B$ y $B\otimes A$ compartirían el mismo espectro y el mismo conjunto de valores propios con la diferencia que los valores propios no serían asignados de igual forma a los vectores propios.


Bajo esta idea supongamos que: \begin{align}
    A&=\sum_i a_i |\psi_i\rala \psi_i|,& B&=\sum_i b_i |\psi_i\rala \psi_i|\\
    A\otimes B&= \sum_{j,k}a_j b_k |\psi_j\psi_k\rala\psi_j\psi_k|,&  B\otimes A&= \sum_{j,k}a_j b_k |\psi_k\psi_j\rala\psi_k\psi_j|.
\end{align}



Luego al sustituir en la ecuación de valor esperado del primer instrumento  {\ref{ExpectedValueI1}}:\begin{equation}\label{ExpectedValueI1.2}\la A\otimes B\ra_{\mathcal{I}_1}=\sum_{j,k}a_j b_k[p\la \psi_j \psi_k|\rho |\psi_j \psi_k\ra + (1-p)\la \psi_k \psi_j|\rho | \psi_k \psi_j\ra]. \end{equation}

Al sustituir en la ecuación de valor esperado del segundo instrumento {\ref{ExpectedValueI2}}:

\[\begin{split}
	\la A\otimes B \ra_{\mathcal{I}_2}&=\sum_{j,k}p a_{j}b_k  \la_j b_k|\rho |a_j b_k\ra+\sum_{j,k}{(1-p)\tr( (B\otimes A)P_{a_j,b_k})} \la a_j b_k|\rho |a_j b_k\ra \\
    &=\sum_{j,k}p a_{j}b_k  \la_j b_k|\rho |a_j b_k\ra+\sum_{j,k}{(1-p)\tr\left( \sum_{i,l}a_i b_l P_{b_l,a_i}P_{a_j,b_k}\right)} \la a_j b_k|\rho |a_j b_k\ra \\
    &=\sum_{j,k}p a_{j}b_k  \la\psi_j \psi_k|\rho |\psi_j \psi_k\ra+\sum_{j,k}(1-p) a_i b_l \tr(P_{\psi_l,\psi_i}P_{\psi_j,\psi_k})\la \psi_j \psi_k|\rho |\psi_j \psi_k\ra \\
    &=\sum_{j,k}p a_{j}b_k  \la\psi_j \psi_k|\rho |\psi_j \psi_k\ra+\sum_{j,k}(1-p) a_k b_j \la \psi_j \psi_k|\rho |\psi_j \psi_k\ra \\
    &=\sum_{j,k} a_{j} b_k[p  \la\psi_j \psi_k|\rho |\psi_j \psi_k\ra+(1-p) \la \psi_k\psi_j|\rho |\psi_k \psi_j\ra] \\
\end{split}\]
 donde la última expresión es igual a la expresión  {\ref{ExpectedValueI1.2}}.






















\begin{comment}


\subsubsection{Tercera alternativa}
Para la tercera alternativa se considera un tipo medición difusa diferente a las dos anteriores. En esta alternativa se asume que con una probabilidad $q$ se realiza una medida proyectiva ideal, pero con una probabilidad $(1-q)$ se realiza una medición en la que no es posible distinguir entre los posibles estados.


Sea $S$ el conjunto de pares ordenados $(j,k)$ de índices de  valores propios tales que el primer índice indica un valor propio del operador $A$ y el segundo índice indica algún valor propio del operador $B$. Sus respectivos operadores de proyección tienen la característica que  $|a_j b_k \rala a_j b_k|\ne|b_k a_j \rala b_k a_j |$. Los valores que toman $j$ y $k$ pueden ser $0$ o $1$. Entonces se introduce un nuevo operador \[P^{s}_{a_j,b_k}=|a_j b_k\rala a_j b_k|+|b_k a_j\rala b_k a_j|,\] %además los estados $|a_j b_k\rala a_j b_k|$ son ortogonales con $|b_k a_j\rala b_k a_j|$. 
El operador $P^s_{a_j,b_k}$ se interpreta como la falta de conocimiento si el estado se encuentra en $|a_j b_k\ra$ o en $|b_k a_j\ra$. 

De igual forma, sea $R$ el conjunto de pares ordenados $(i,l)$ de los índices de valores propios tales que su primer índice indica un valor propio del operador $A$ y el segundo índice indica algún valor propio del operador $B$. Sus respectivos operadores de proyección tienen la característica que  $|a_i b_l\rala a_i b_l|=|b_l a_i \rala b_l a_i |$, Los valores que toman $j$ y $k$ pueden ser $0$ o $1$. El operador de proyección asociado a los valores propios indexados por los elementos del conjunto $R$ es \[P^r_{a_i,b_l}=|a_i b_l\rala a_i b_l|=|b_l a_i \rala b_l a_i |.\] Luego, con una probabilidad $q$ el instrumento cuántico es \[\sum_{m,n} P_{a_m,b_n}\otimes P_{a_m,b_n}\rho P_{a_m,b_n},\] con $m,n=1,0$ y con una probabilidad $(1-q)$ es \[\sum_{(i,l)\in R} P_{a_i, b_l}\otimes  P^{r}_{a_i,b_l}\rho P^r_{a_i,n_l}+\sum_{(j,k)\in S}P_{a_j b_k} \otimes  \dfrac{1}{2}P^{s}_{a_j,b_k}\rho P^s_{a_j,b_k}.\]

Si el estado inicial se encuentra en el subespacio de $P^{s}_{a_j,b_k}$, el factor $(1/2)$ en la segunda sumatoria es la probabilidad con la que el sistema clásico indica  que el estado está en $|a_j b_k\ra $ o en $|b_k a_j\ra$. El instrumento es la suma convexa de estos dos posibilidades \begin{equation}
    \begin{split}
        \mathcal{I}_3(\rho)&=q\sum_{m,n}  P_{a_m,b_n}\otimes P_{a_m,b_n}\rho P_{a_m,b_n} +(1-q)\sum_{(i,l)\in R} P_{a_i, b_l}\otimes  P^{r}_{a_i,b_l}\rho P^r_{a_i,b_l}\\
        &+(1-q)\sum_{(j,k)\in S}P_{a_j,b_k} \otimes  \dfrac{1}{2}P^{s}_{a_j,b_k}\rho P^s_{a_j,b_k}.\\
    \end{split}
\end{equation}  


Su valor esperado es:
\begin{equation}
	\begin{split}
		\la A\otimes B\ra_{\mathcal{I}_3}&=\tr[((A\otimes B)\otimes \mathds{1} )\mathcal{I}_3]\\
		=&\sum_{m,n}q\tr\{A\otimes B P_{a_m,b_n}\}\tr\{P_{a_m,b_n}\rho P_{a_m,b_n}\}+\sum_{(i,l)\in R}(1-q)\tr\{A\otimes B P_{a_i,b_l}\}\tr\{P_{a_i,b_l}\rho P_{a_i,b_l}\}\\
		+&\sum_{(j,k)\in S} \dfrac{1-q}{2}\tr\{A\otimes B P_{a_i,b_l}\}\tr\{P_{a_i,b_l}^s\rho P_{a_i,b_l}^s\}\\
		=&\sum_{i,l \in R}a_i b_l \tr\{P_{a_m,b_n}\}\tr\{P_{a_i,b_l}\rho\}+\frac{(1+q)}{2}\sum_{(j,k)\in S}a_j b_k \tr\{P_{a_j,b_k}\} \tr\{P_{a_j,b_k}\rho P_{a_j,b_k}\}\\
		+&\frac{(1-q)}{2}\sum_{(j,k)\in S}a_j b_k \tr\{P_{a_j,b_k}\} \tr\{P_{a_j,b_k}\rho SP_{a_j,b_k}S\}\\+ &\frac{(1-q)}{2}\sum_{(j,k)\in S}a_j b_k \tr\{P_{a_j,b_k}\} \tr\{SP_{a_j,b_k}S\rho P_{a_j,b_k}\}\\
		+ &\frac{(1-q)}{2}\sum_{(j,k)\in S}a_j b_k \tr\{P_{a_j,b_k}\} \tr\{SP_{a_j,b_k}S\rho SP_{a_j,b_k}S\}
	\end{split}
\end{equation}


\begin{equation}
	\begin{split}
		=&\sum_{i,l \in R}a_i b_l \tr\{P_{a_i,b_l}\rho\}+\frac{(1+q)}{2}\sum_{(j,k)\in S}a_j b_k  \tr\{P_{a_j,b_k}\rho P_{a_j,b_k}\}\\
		+&\frac{(1-q)}{2}\sum_{(j,k)\in S}a_j b_k \tr\{P_{a_j,b_k}\rho SP_{a_j,b_k}S\}\\+ &\frac{(1-q)}{2}\sum_{(j,k)\in S}a_j b_k  \tr\{SP_{a_j,b_k}S\rho P_{a_j,b_k}\}\\
		+ &\frac{(1-q)}{2}\sum_{(j,k)\in S}a_j b_k  \tr\{SP_{a_j,b_k}S\rho SP_{a_j,b_k}S\}
	\end{split}
\end{equation}


\end{comment}

\end{document}