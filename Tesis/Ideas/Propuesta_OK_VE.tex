\documentclass[12pt,oneside]{book}\raggedbottom{}
%%%%%%%%%%%%%%%%%%%%%%%%%%%%%%%% Algunos Paquetes Necesarios 
% Importing document settings from our file "packages.sty"
\usepackage{packages}



\begin{document}

\newcommand{\myName}{Instrumentos cu\'anticos}
\newcommand{\myDate}{Primer semestre 2021}
\newcommand{\myCourse}{}
\newcommand{\degre}{\ensuremath{^\circ}}
\newcommand{\R}{\mathbb{R}}
\newcommand{\vi}{\mathbf{\hat{\i}}}
\newcommand{\vj}{\mathbf{\hat{\j}}}
\newcommand{\vk}{\mathbf{\hat{\k}}}
\newcommand{\I}{\mathcal{I}}
%%%%%%%%%%%%%%%%%%%%%%%%%%%%%%%%%%%%%%%Fancyboxes


\newcounter{problem}[section]\setcounter{problem}{0}
\renewcommand{\theproblem}{%\arabic{section}.
	\arabic{problem}}

\newenvironment{problem}[1][]{%
	\refstepcounter{problem}
	
	\ifstrempty{#1}%
	% if condition (without title)
	{\mdfsetup{%
			frametitle={%
				\tikz[baseline= (current bounding box.east), outer sep=0pt]
				\node[anchor=east,rectangle,fill=purple!20]
				{\strut{} Problema~\theproblem};}
		}%
		% else condition (with title)
	}{\mdfsetup{%
			frametitle={%
				\tikz[baseline= (current bounding box.east), outer sep=0pt]
				\node[anchor=east,rectangle,fill=purple!20]
				{\strut{} Problema~\theproblem:~#1};}%
		}%
	}%
	% Both conditions
	\mdfsetup{%
		innertopmargin=10pt,linecolor=pink!40,%
		linewidth=2pt,topline=true,%
		frametitleaboveskip=\dimexpr-\ht\strutbox\relax%
	}
\begin{mdframed}[]\relax%
	\label{#1}}{\end{mdframed}}


\newcounter{solution}[section]\setcounter{solution}{0}
\renewcommand{\thesolution}{%\arabic{section}.
	\arabic{solution}}
\newenvironment{solution}[1][]{%
	\refstepcounter{solution}%
	\ifstrempty{#1}%
	{\mdfsetup{%
			frametitle={%
				\tikz[baseline= (current bounding box.east), outer sep=0pt]
				\node[anchor=east,rectangle,fill=green!20]
				{\strut{} Soluci\'on~\thesolution};}}
	}%
	{\mdfsetup{%
			frametitle={%
				\tikz[baseline= (current bounding box.east), outer sep=0pt]
				\node[anchor=east,rectangle,fill=green!20]
				{\strut{} Soluci\'on~\thesolution:~#1};}}%
	}%
	\mdfsetup{innertopmargin=10pt,linecolor=green!20,%
		linewidth=2pt,topline=true,%
		frametitleaboveskip=\dimexpr-\ht\strutbox\relax
	}
	\begin{mdframed}[]\relax%
		\label{#1}}{\end{mdframed}}

%%%%%%%%%%%%%%%%%%%%%%%%%%%%%%%%%%% Tema - BEGIN
\newtheoremstyle{Tema}% name of the style to be used
  {5mm}% measure of space to leave above the theorem. E.g.: 3pt
  {3mm}% measure of space to leave below the theorem. E.g.: 3pt
  {}% name of font to use in the body of the theorem
  {}% measure of space to indent
  {\bfseries}% name of head font
  {\newline}% punctuation between head and body
  {20mm}% space after theorem head
  {}% Manually specify head

\theoremstyle{Tema} \newtheorem{Tema}{Tema} %%%%% Template para Temas
\theoremstyle{Tema} \newtheorem{serie}{Serie}              %%%%%  Template para Series de ejercicios
\theoremstyle{Tema} \newtheorem{ejercicio}{Ejercicio}    %%%%%  Template para Ejercicios
%%%%%%%%%%%%%%%%%%%%%%%%%%%%%%%%%%% Tema - END


%%%%%%%%%%%%%%%%%%%%%%%%%%%%%%%%%%% Encabezado - BEGIN %%%%%%%%%%
\fancypagestyle{firststyle}
{
\renewcommand{\headrulewidth}{1.5pt}
\fancyhead[R]{
			%\textbf{Universidad de San Carlos de Guatemala} \\
			%\textbf{Escuela de Ciencias Físicas y Matemáticas}\\
			%\textbf{\myDate{}} \\
			%\textbf{\myCourse{} }    %%%%%%%%%% Agregar nombre del curso 
			  %%%%%%%%%%%%%%%%%%%%%% Agregar fecha en formato: Enero 15, 2015
			}
\fancyhead[L]{ 
%	\includegraphics[height=1.6 cm, width=5.5 cm]{LogoColor}  
	}
}
%%%%%%%%%%%%%%%%%%%%%%%%%%%%%%%%%%% Encabezado - END %%%%%%%%%%

%%%%%%%%%%%%%%%%%%%%%%%%%%%%%%%%%%% Encabezado (pagina 2 en adelante) - BEGIN %%%
\fancypagestyle{allStyle}
{
\renewcommand{\headrulewidth}{0.8 pt}
\fancyhead[R]{
			\emph{\myName{} $-$ \myCourse{}} %%%% Modificar n\'umero de examen parcial y nombre del curso
			}
\fancyhead[L]{}  
\fancyfoot[C]{}
\fancyfoot[R]{\thepage}
}
%%%%%%%%%%%%%%%%%%%%%%%%%%%%%%%%%%% Encabezado (pagina 2 en adelante) - END %%%

\date{}
\setlength{\headheight}{0.5in} % fixes \headheight warning
 %%%%%%%%%%%%%%%%%%%%%%%% BEGIN%%%%%%%%%%%%%%

\pagestyle{allStyle}

\thispagestyle{firststyle}
%%%%%%%%%%%%%%%%%%%%%%%%%%%%%%%%% Titulo - BEGIN
\begin{center}
\LARGE
\textsc{\myName}\\\normalsize Nueva propuesta de operadores Kraus utilizando una operación SWAP % Modificar el N\'umero del examen parcial
\medskip
\hrule height 1.5pt
\end{center}

El primer instrumento que da un valor esperado correcto es \begin{equation}\label{primerInstrumento}
	\mathcal{I}_1(\rho)=\sum_{a_j,b_k} P_{a_j b_k}\otimes p P_{a_j b_k}\rho  P_{a_j b_k}+ (1-p) P_{a_j b_k}S\rho S  P_{a_j b_k},
\end{equation} los operadores de Kraus para este instrumento son:\[K_{a_j b_k}= P_{a_j b_k}\otimes  P_{a_j b_k}\sqrt{p} \] \[K_{a_j b_k}^S= P_{a_j b_k}\otimes  P_{a_j b_k}S\sqrt{1-p}, \]
ahora reescribiendo {\ref{primerInstrumento}} con los operadores de Kraus:\[\I_1(\rho)=\sum_{a_j,b_k}K_{a_j b_k}\rho K_{a_j b_k}^\dagger+K_{a_j b_k}^S\rho K_{a_j b_k}^{S\dagger}\]
la propuesta para obtener operadores de Kraus distintos tomando como base estos operadores es aplicar una operación SWAP al primer instrumento.

Definamos el siguiente operador:\begin{equation}
	\hat{S}=S_{13}\otimes S_{24},
\end{equation} donde $S_{ij}$ es la matriz Swap que intercambia los estados en las posiciones $i$ y $j$. El instrumento {\ref{primerInstrumento}} se transforma ahora en:\[\mathcal{I}_1\to \mathcal{I}'_1= \hat{S}\mathcal{I}_1\hat{S}\]
\[\begin{split}
	\mathcal{I}'_1=&\hat{S}\left(\sum_{a_j,b_k} P_{a_j b_k}\otimes p P_{a_j b_k}\rho  P_{a_j b_k}+ (1-p) P_{a_j b_k}S\rho S  P_{a_j b_k}\right)\hat{S}\\
	=&\sum_{a_j b_k}\hat{S} (|a_j b_k\rala a_j b_k| )\otimes p (|a_j b_k\rala a_j b_k|\rho |a_j b_k \rala a_j b_k|)\hat{S}\\
	+&\sum_{a_j b_k}\hat{S} (|a_j b_k\rala a_j b_k| )\otimes (1-p) (|a_j b_k\rala a_j b_k| S_{12}\rho S_{12}  |a_j b_k\rala a_j b_k| )\hat{S}\\
	=&p\langle a_j b_k|\rho |a_j b_k\rangle \sum_{a_j b_k}\hat{S} |a_j b_k\rala a_j b_k| \otimes  |a_j b_k\rala a_j b_k|\hat{S}\\
	+&(1-p)\langle a_j b_k| S_{12}\rho S_{12}  |a_j b_k\rangle \sum_{a_j b_k}\hat{S} |a_j b_k\rala a_j b_k|\otimes |a_j b_k\rala a_j b_k|\hat{S}\\
	=&p\langle a_j b_k|\rho |a_j b_k\rangle \sum_{a_j b_k}|a_j b_k\rala a_j b_k| \otimes  |a_j b_k\rala a_j b_k|\\
	+&(1-p)\langle a_j b_k| S_{12}\rho S_{12}  |a_j b_k\rangle \sum_{a_j b_k} |a_j b_k\rala a_j b_k|\otimes |a_j b_k\rala a_j b_k|\\
	=&\I_1\\
\end{split}\]

Después de esta álgebra noto que $\I_1'=\I_1$ por lo que el valor esperado sí será el mismo pero los operadores de Kraus siguen siendo los mismos.

Ejemplo: Supongamos que $A\otimes B=\sigma_x\otimes \sigma_z$ y $\rho=|+0\rala +0|$

\[\begin{split}
	\mathcal{I}'_1=&\hat{S}\left(|+0\rala+0| \otimes p |+0\rala+0| +0\rala+0|+0\rala+0|\right)\hat{S}\\
	+&\hat{S}\left(|+0\rala+0| \otimes (1-p) |+0\rala+0|S|+0\rala+0| S  |+0\rala+0|\right)\hat{S}\\
	=&\hat{S}\left(|+0\rala+0| \otimes p |+0\rala+0| +0\rala+0|+0\rala+0|\right)\hat{S}\\
	+&\hat{S}\left(|+0\rala+0| \otimes (1-p) |+0\rala+0|0+\rala0+|+0\rala+0|\right)\hat{S}\\
	=&\hat{S}\left(|+0\rala+0| \otimes p |+0\rala+0|\right)\hat{S}+\hat{S}\left(|+0\rala+0| \otimes (1-p) \frac{1}{4}|+0\rala+0|\right)\hat{S}\\	
	=&p\hat{S}\left(|+0\rala+0| \otimes |+0\rala+0|\right)\hat{S} +(1-p) \frac{1}{4}\hat{S}\left(|+0\rala+0| \otimes |+0\rala+0|\right)\hat{S}\\	
	=&p\left(|+0\rala+0| \otimes |+0\rala+0|\right)+(1-p) \frac{1}{4}\left(|+0\rala+0| \otimes |+0\rala+0|\right)\\	
	=&\I_1.\\
\end{split}\]




Notar que cambiar los operadores de Kraus a la siguiente forma (para intercambiar el sistema clásico con el cuántico):\[K_{a_j b_k}\to \sqrt{p}P_{a_j b_k}\otimes P_{a_j b_k}\]
\[K^S_{a_j b_k}\to \sqrt{1-p}P_{a_j b_k}S\otimes P_{a_j b_k},\] esto es aplicarle una operación $\hat{S}$ por la izquierda y por la derecha a los operadores de Kraus \[K_{a_j b_k}\to K'_{a_j b_k}=\hat{S}K_{a_j b_k}\hat{S}= \sqrt{p}P_{a_j b_k}\otimes P_{a_j b_k}\]
\[K^S_{a_j b_k}\to K'^S_{a_j b_k}=\hat{S}K_{a_j b_k}^S\hat{S}=\sqrt{1-p}P_{a_j b_k}S\otimes P_{a_j b_k}.\]

Con estos nuevos operadores el instrumento se transforma en:\[\I_1\to \sum_{a_j, b_k}K'_{a_j b_k}\rho K'^\dagger_{a_j b_k}+K'^S_{a_j b_k}\rho K'^{S\dagger}_{a_j b_k} \]

\[\begin{split} \I_1'&= \sum_{a_j, b_k}(\sqrt{p}P_{a_j b_k}\otimes P_{a_j b_k})(\mathds{1}\otimes \rho)(\sqrt{p}P_{a_j b_k}\otimes P_{a_j b_k})\\
	+&(\sqrt{1-p}P_{a_j b_k}S\otimes P_{a_j b_k})(\mathds{1}\otimes \rho) (\sqrt{1-p}SP_{a_j b_k} \otimes P_{a_j b_k})\\
	&=\sum_{a_j b_k} pP_{a_j b_k}\otimes P_{a_j b_k}\rho P_{a_j b_k}+ (1-p) P_{a_j b_k}\otimes P_{a_j b_k}\rho P_{a_j b_k}\\
	&=\sum_{a_j b_k} P_{a_j b_k}\otimes P_{a_j b_k}\rho P_{a_j b_k}
\end{split}\] el cual es un instrumento de una medida proyectiva ideal y por tanto no tendrán el mismo valor esperado.







%%%%%%%%%%%%%%%%%%%%%%%%%%%%%%%%% Instrucciones - BEGIN

\vspace{0.1 in}

\end{document} %%%%%%%%%%%%%%%%%%%%%%%% BEGIN%%%%%%%%%%%%%%