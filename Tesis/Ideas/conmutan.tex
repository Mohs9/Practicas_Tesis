\documentclass[12pt,oneside]{book}\raggedbottom{}
%%%%%%%%%%%%%%%%%%%%%%%%%%%%%%%% Algunos Paquetes Necesarios 
% Importing document settings from our file "packages.sty"
\usepackage{packages}
\usepackage{comment}


\begin{document}

\newcommand{\myName}{Si se cumple la condición no conmutan}
\newcommand{\myDate}{Primer semestre 2021}
\newcommand{\myCourse}{}
\newcommand{\degre}{\ensuremath{^\circ}}
\newcommand{\R}{\mathbb{R}}
\newcommand{\vi}{\mathbf{\hat{\i}}}
\newcommand{\vj}{\mathbf{\hat{\j}}}
\newcommand{\vk}{\mathbf{\hat{\k}}}
%%%%%%%%%%%%%%%%%%%%%%%%%%%%%%%%%%%%%%%Fancyboxes


\newcounter{problem}[section]\setcounter{problem}{0}
\renewcommand{\theproblem}{%\arabic{section}.
	\arabic{problem}}

\newenvironment{problem}[1][]{%
	\refstepcounter{problem}
	
	\ifstrempty{#1}%
	% if condition (without title)
	{\mdfsetup{%
			frametitle={%
				\tikz[baseline= (current bounding box.east), outer sep=0pt]
				\node[anchor=east,rectangle,fill=purple!20]
				{\strut{} Problema~\theproblem};}
		}%
		% else condition (with title)
	}{\mdfsetup{%
			frametitle={%
				\tikz[baseline= (current bounding box.east), outer sep=0pt]
				\node[anchor=east,rectangle,fill=purple!20]
				{\strut{} Problema~\theproblem:~#1};}%
		}%
	}%
	% Both conditions
	\mdfsetup{%
		innertopmargin=10pt,linecolor=pink!40,%
		linewidth=2pt,topline=true,%
		frametitleaboveskip=\dimexpr-\ht\strutbox\relax%
	}
\begin{mdframed}[]\relax%
	\label{#1}}{\end{mdframed}}


\newcounter{solution}[section]\setcounter{solution}{0}
\renewcommand{\thesolution}{%\arabic{section}.
	\arabic{solution}}
\newenvironment{solution}[1][]{%
	\refstepcounter{solution}%
	\ifstrempty{#1}%
	{\mdfsetup{%
			frametitle={%
				\tikz[baseline= (current bounding box.east), outer sep=0pt]
				\node[anchor=east,rectangle,fill=green!20]
				{\strut{} Soluci\'on~\thesolution};}}
	}%
	{\mdfsetup{%
			frametitle={%
				\tikz[baseline= (current bounding box.east), outer sep=0pt]
				\node[anchor=east,rectangle,fill=green!20]
				{\strut{} Soluci\'on~\thesolution:~#1};}}%
	}%
	\mdfsetup{innertopmargin=10pt,linecolor=green!20,%
		linewidth=2pt,topline=true,%
		frametitleaboveskip=\dimexpr-\ht\strutbox\relax
	}
	\begin{mdframed}[]\relax%
		\label{#1}}{\end{mdframed}}

%%%%%%%%%%%%%%%%%%%%%%%%%%%%%%%%%%% Tema - BEGIN
\newtheoremstyle{Tema}% name of the style to be used
  {5mm}% measure of space to leave above the theorem. E.g.: 3pt
  {3mm}% measure of space to leave below the theorem. E.g.: 3pt
  {}% name of font to use in the body of the theorem
  {}% measure of space to indent
  {\bfseries}% name of head font
  {\newline}% punctuation between head and body
  {20mm}% space after theorem head
  {}% Manually specify head

\theoremstyle{Tema} \newtheorem{Tema}{Tema} %%%%% Template para Temas
\theoremstyle{Tema} \newtheorem{serie}{Serie}              %%%%%  Template para Series de ejercicios
\theoremstyle{Tema} \newtheorem{ejercicio}{Ejercicio}    %%%%%  Template para Ejercicios
%%%%%%%%%%%%%%%%%%%%%%%%%%%%%%%%%%% Tema - END


%%%%%%%%%%%%%%%%%%%%%%%%%%%%%%%%%%% Encabezado - BEGIN %%%%%%%%%%
\fancypagestyle{firststyle}
{
\renewcommand{\headrulewidth}{1.5pt}
\fancyhead[R]{
			%\textbf{Universidad de San Carlos de Guatemala} \\
			%\textbf{Escuela de Ciencias Físicas y Matemáticas}\\
			%\textbf{\myDate{}} \\
			%\textbf{\myCourse{} }    %%%%%%%%%% Agregar nombre del curso 
			  %%%%%%%%%%%%%%%%%%%%%% Agregar fecha en formato: Enero 15, 2015
			}
\fancyhead[L]{ 
%	\includegraphics[height=1.6 cm, width=5.5 cm]{LogoColor}  
	}
}
%%%%%%%%%%%%%%%%%%%%%%%%%%%%%%%%%%% Encabezado - END %%%%%%%%%%

%%%%%%%%%%%%%%%%%%%%%%%%%%%%%%%%%%% Encabezado (pagina 2 en adelante) - BEGIN %%%
\fancypagestyle{allStyle}
{
\renewcommand{\headrulewidth}{0.8 pt}
\fancyhead[R]{
			\emph{\myName{} $-$ \myCourse{}} %%%% Modificar n\'umero de examen parcial y nombre del curso
			}
\fancyhead[L]{}  
\fancyfoot[C]{}
\fancyfoot[R]{\thepage}
}
%%%%%%%%%%%%%%%%%%%%%%%%%%%%%%%%%%% Encabezado (pagina 2 en adelante) - END %%%

\date{}
\setlength{\headheight}{0.5in} % fixes \headheight warning
 %%%%%%%%%%%%%%%%%%%%%%%% BEGIN%%%%%%%%%%%%%%

\pagestyle{allStyle}

\thispagestyle{firststyle}
%%%%%%%%%%%%%%%%%%%%%%%%%%%%%%%%% Titulo - BEGIN
\begin{center}
\LARGE
\textsc{\myName}\\% Modificar el N\'umero del examen parcial
\bigskip
\hrule height 1.5pt
\end{center}
\setcounter{chapter}{1}

\begin{proposition}
    Si la condición \[\la a_j b_k|B\otimes A|a_{j'}b_{k'}\ra=0, \forall j,k\ne j',k'\] se cumple entonces $[A\otimes B,B \otimes A]=0$  
\end{proposition}
\begin{proof}
    Por simplicidad se escribe la condición de forma matricial \[V^\dagger B\otimes A V=\text{Diag}(c_1,c_2,c_3,c_4)=D_{B\otimes A},\] donde $V$ es la yuxtaposición de los vectores propios de $A\otimes B$. De esta forma es fácil ver que $V^\dagger V= \mathds{1}$, puesto que suponemos que son vectores ortonormales. Luego $A\otimes B $ se puede expresar como \[A\otimes B=VD_{A\otimes B} V^\dagger,\] siendo $D_{A\otimes B}$ la matriz diagonal de los valores propios del observable. Luego \[\begin{split}
        [A\otimes B,B \otimes A]&= (A\otimes B) (B\otimes A)-(B\otimes A) (A\otimes B)\\
        &=(VD_{A\otimes B} V^\dagger)(VD_{B\otimes A} V^\dagger)-(VD_{B\otimes A} V^\dagger)(VD_{A\otimes B} V^\dagger)\\
        &= VD_{A\otimes B} D_{B\otimes A} V^\dagger -VD_{B\otimes A} D_{A\otimes B} V^\dagger\\
        &=V(D_{A\otimes B} D_{B\otimes A} -D_{B\otimes A} D_{A\otimes B})V^\dagger\\
        &=V\cancelto{0}{[D_{B\otimes A}, D_{A\otimes B}]}V^\dagger=0
    \end{split}\] donde la última igual de se cumple porque dos matrices diagonales reales siempre conmutan. 






\end{proof}








\end{document}