
\begin{figure}[H]
    \centering
    
  \begin{subfigure}{1\textwidth}
    \centering
    \resizebox{6.5cm}{!}{\begin{quantikz}[font=\small, row sep=0.3cm,column sep=0.4cm]%
    \lstick{$\rho_1$}& \qw& \qw& \qw& \gate[wires=3,style={
    starburst,fill=yellow,draw=red,line width=2pt,inner
    xsep=-2pt,inner ysep=-5pt},label style=blue]{\text{
    noise}} & \qw& \qw& \qw& \meter[style={draw=blue}]{$\sigma_z$} \vcwdouble{1-9}{5-9}{1}&\qw&\qw&\qw&\\
    &&&&&&&&&&&&\rstick{}\\
    \lstick{$\rho_2$} & \qw&\qw&\qw& \qw& \qw& \qw& \qw& \qw&\qw&\meter[style={draw=blue}]{$\sigma_x$}\vcwdouble{3-11}{5-11}{-1}\qw&\qw&\\
    &&&&&&&&&&&&\\
    \lstick{cl}& \cw& \cw& \cw& \cw& \cw& \cw& \cw&\cw&\cw&\cw&\cw&\\
  \end{quantikz}}
\hfill
\caption{}
\end{subfigure}
\hfill

\begin{subfigure}{0.4\textwidth}
  \begin{flushleft}
    \resizebox{7cm}{!}{
      \begin{quantikz}[font=\small,row sep=0.3cm,column sep=0.4cm]%
      \lstick{$\rho_{1}$}&   \qw& \qw& \meter[style={draw=blue}]{$\sigma_z$} \vcwdouble{1-4}{4-4}{1}&\qw&\qw&\qw\rstick[wires=4]{$p \tr(\sigma_z \tensor \sigma_x \rho)$}&\\[0.1cm]
      \lstick{$\rho_2$} &  \qw& \qw& \qw& \qw&\meter[style={draw=blue}]{$\sigma_x$}\vcwdouble{2-6}{4-6}{-1}&\qw& \\[0.1cm]
      &&&&&&&\\
      \lstick{cl}& \cw&  \cw&\cw&\cw&\cw&\cw&\\
    \end{quantikz}}
    \hfill
  \caption{}
  \end{flushleft}
   
\end{subfigure}


%  \begin{subfigure}[a]{0.5\textwidth}
 % \centering
  %\resizebox{6cm}{!}{
   % \begin{quantikz}[scale=1, row sep=0.3cm,column sep=0.4cm]%
    %\lstick{$\rho_{1}$}&  \qw& \qw& \qw& \qw& \qw&\qw\rstick[wires=3]{}\\ \\
    %&&&&&&& |[meter]| \vcwdouble{2-8}{5-8}{a_jb_k}\rstick{$A \tensor B$} \\
    %\lstick{$\rho_2$} & \qw& \qw& \qw& \qw& \qw&\qw&\\
    %\meter[style={draw=blue}]{B}\vcwdouble{2-11}{4-11}{b_k} \\
  %  & & && &&&\\
   % \lstick{cl}&  \cw& \cw& \cw& \cw&\cw&\cw&\cw\\
  %\end{quantikz}}
  %\hfill
%\caption{}
%\end{subfigure}

\begin{subfigure}{0.4\textwidth}
  \centering
\resizebox{8cm}{!}{
\begin{quantikz}[font=\small,scale=1, row sep=0.3cm,column sep=0.4cm]%
  \lstick{$\rho_1$}&\qw& \gate[swap]{} & \qw&  \meter[style={draw=blue}]{$\sigma_z$} \vcwdouble{1-5}{4-5}{ 1}&\qw&\qw&\qw\rstick[wires=4]{$(1-p)\tr(\sigma_x \tensor \sigma_x\rho)$}& \\
  \lstick{$\rho_2$} & \qw& \qw& \qw& \qw& \qw&\meter[style={draw=blue}]{$\sigma_x$}\vcwdouble{2-7}{4-7}{-1}&\qw& \\
  &&&&&&&&\\
  \lstick{cl}& \cw& \cw& \cw&\cw&\cw&\cw&\cw&\\
\end{quantikz}}
\hfill
\caption{}
\end{subfigure}
\caption{\textbf{(a)} En la primera imagen se ilustran un estado inicial $\rho= \rho_1\otimes \rho_2$, sin embargo el entorno no es ideal y es posible que ocurra una identificación errónea, en las que las salidas clásicas registradas para los observables $\sigma_z$ y $\sigma_x$ fueron $1$ y $-1$ respectivamente.\textbf{ (b)} En esta imagen se ilustra que con una probabilidad $p$ la identificación del sistema sea correcta y se realice la medición del observable $\sigma_z\tensor \sigma_x$.\textbf{ (c)} La imagen muestra que debido al ruido del entorno, las partículas experimentan un intercambio y con una probabilidad $(1-p)$ se realiza la medición del observable $\sigma_x\otimes \sigma_z$ }\label{diagrama-cajas}\source{elaboración propia}
\end{figure}
