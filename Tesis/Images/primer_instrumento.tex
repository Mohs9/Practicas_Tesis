
\begin{figure}[H]
  \centering
\begin{subfigure}[b]{0.4\textwidth}
\centering
   \resizebox{8cm}{!}{
      \begin{quantikz}[font=\normalsize,row sep=0.3cm,column sep=0.4cm]%
      &\lstick[wires=2]{$\rho_{1,2}$}& \qw& \qw& \meter[style={draw=blue}]{$A$} \vcwdouble{1-5}{4-5}{}&\qw&\qw&\qw&\qw\rstick{\small $P_{a_j}${\color{blue}{$\rho_1$}}$P_{a_j}$}&&&&&\rstick[wires=5]{$p$}\\[0.1cm]
      &&\qw& \qw& \qw& \qw&\meter[style={draw=red}]{$B$}\vcwdouble{2-7}{4-7}{}&\qw&\qw\rstick{\small $P_{b_k}${\color{red}{$\rho_2$}}$P_{b_k}$}&&&&&& \\[0.1cm]
      &&&&&&&&&&&&&\\
      &\lstick{cl}& \cw&  \cw&\cw&\cw&\cw&\cw&\cw&&&&&\\
      &&&&\rstick{$a_j$}&&\rstick{$b_k$}&&&&&&&\\
    \end{quantikz}
    }
    \hfill
  \caption{}\label{fig:1-instrumento-ideal}
\end{subfigure}

\hfill

\begin{subfigure}[c]{0.4\textwidth}
\centering
  \resizebox{10cm}{!}{
\begin{quantikz}[font=\normalsize,scale=1, row sep=0.3cm,column sep=0.4cm]%
  &\lstick[wires=2]{$\rho_{1,2}$}&\qw& \gate[swap]{} & \qw&  \meter[style={draw=blue}]{$A$} \vcwdouble{1-6}{4-6}{}&\qw&\qw&\qw&\qw\rstick{\small $P_{a_j}${\color{red}{$\rho_2$}}$P_{a_j}$}&&&&&\rstick[wires=5]{$1-p$}\\
  && \qw& \qw& \qw& \qw& \qw&\meter[style={draw=red}]{$B$}\vcwdouble{2-8}{4-8}{}&\qw&\qw \rstick{\small $P_{b_k}${\color{blue}{$\rho_1$}}$P_{b_k}$}&&&&&\\
  &&&&&&&&&&&&&&&\\
  &\lstick{cl}&\cw& \cw& \cw&\cw&\cw&\cw&\cw&\cw&&&&&&\\
  &&&&&\rstick{$a_j$}&&\rstick{$b_k$}&&&&&&&&\\
\end{quantikz}
}
\hfill
\caption{}\label{fig:1-instrumento-non-ideal}
\end{subfigure}
\caption{\textbf{(a)} Con una probabilidad $p$ se realiza una medición ideal del observable $A\tensor B$, con un estado posterior dado por $P_{a_j,b_k}\rho_{1,2} P_{a_j,b_k}$, cuando la salida indicada es $a_j b_k$. \textbf{ (b)} Con probabilidad $1-p$ ocurre un intercambio de las partículas y estado posterior correspondiente a la salida $a_j b_k$ será $P_{a_j,b_k}\rho_{1,2} P_{a_j,b_k}$.}\label{diagrama-cajas-primer-instrumento}\source{elaboración propia}
\end{figure}
