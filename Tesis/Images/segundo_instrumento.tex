
\begin{figure}[H]
    \centering 
\begin{subfigure}{0.4\textwidth}
  \begin{center}
   \resizebox{7.5cm}{!}{
      \begin{quantikz}[font=\large,row sep=0.3cm,column sep=0.4cm]%
      \lstick[wires=2]{$\rho_{1,2}$}&\qw& \qw&\meter[style={draw=blue}]{$A$} \vcwdouble{1-4}{4-4}{}&\qw&\qw&\qw&\qw\rstick{\small $P_{a_j}${\color{blue}{$\rho_1$}}$P_{a_j}$}&&&&&\rstick[wires=5]{$p$}\\[0.1cm]
       &  \qw& \qw& \qw& \qw&\meter[style={draw=red}]{$B$}\vcwdouble{2-6}{4-6}{}&\qw&\qw\rstick{\small $P_{b_k}${\color{red}{$\rho_2$}}$P_{b_k}$} &&&&\\[0.1cm]
      &&&&&&&&&&&&\\
      \lstick{cl}& \cw&  \cw&\cw&\cw&\cw&\cw&\cw&&&&\\
      &&&\rstick{$a_j$}&&\rstick{$b_k$}&&&&&&&\\
    \end{quantikz}
    }
    \hfill
  \caption{}
  \end{center}
\end{subfigure}

\hfill

\begin{subfigure}{0.4\textwidth}
    \begin{flushleft}
     \resizebox{8.5cm}{!}{
        \begin{quantikz}[font=\large,row sep=0.3cm,column sep=0.4cm]%
          \lstick[wires=2]{\color{black}{$\rho_{1,2}$}}&   \qw& \qw& \meter[style={draw=blue}]{$A$} \vcwdouble{1-4}{4-4}{}&\qw&\qw&\qw&\qw\rstick{\small $P_{a_j}${\color{blue}{$\rho_1$}}$P_{a_j}$}&&&&&\rstick[wires=5]{$1-p$}\\[0.1cm]
         &  \qw& \qw& \qw& \qw&\meter[style={draw=red}]{$B$}\vcwdouble{2-6}{4-6}{}&\qw&\qw\rstick{\small$P_{b_k}${\color{red}{$\rho_2$}}$P_{b_k}$}&&&& \\[0.1cm]
        &&\gate[wires=2,style={
          starburst,fill=yellow,draw=red,line width=2pt,inner
          xsep=1pt,inner ysep=-7pt},label style=blue]{\text{\small
          noise}}&&&&&&&&&&\\
        \lstick{cl}& \cw&\cw&\cw&\cw&\cw&\cw&\cw&&&&&\\
        &&&\rstick{$b_k$}&&\rstick{$a_j$}&&&&&&&\\
      \end{quantikz}
      }
      \hfill
    \caption{}
    \end{flushleft}
  \end{subfigure}
\caption{\textbf{(a)} Con una probabilidad $p$, se mide el observable $A\otimes B$ y el resultado es una medición ideal, con un estado posterior $P_{a_j, b_k}\rho_{1,2}P_{a_j,b_k}$ con la salida clásica indicándolo correctamente.\textbf{ (b)} Con una probabilidad $1-p$, ocurre un error en los resultados clásicos de la medición y en su lugar, la salidas se intercambian aunque el resultado en el sistema cuántico indique ser el correcto $P_{a_j, b_k}\rho_{1,2}P_{a_j,b_k}$.}\label{fig; diagrama-cajas-segundo-instrumento}\source{elaboración propia}
\end{figure}
