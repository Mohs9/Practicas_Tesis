
\begin{figure}[H]
  \centering
\begin{subfigure}[b]{0.4\textwidth}
\centering
   \resizebox{7cm}{!}{
      \begin{quantikz}[font=\normalsize,row sep=0.3cm,column sep=0.4cm]%
      \lstick[wires=2]{{$\rho_{1,2}$}}& \qw& \qw& \meter[style={draw=blue}]{$A$} \vcwdouble{1-4}{4-4}{}&\qw&\qw&\qw&\qw\rstick{$P_{a_j}${\color{blue}{$\rho_1$}}$P_{a_j}$}&&&&&\rstick[wires=5]{$q$}\\[0.1cm]
       &  \qw& \qw& \qw& \qw&\meter[style={draw=red}]{$B$}\vcwdouble{2-6}{4-6}{}&\qw&\qw\rstick{$P_{b_k}${\color{red}{$\rho_2$}}$P_{b_k}$}&&&& \\[0.1cm]
      &&&&&&&&&&&&&\\
      \lstick{cl}& \cw&  \cw&\cw&\cw&\cw&\cw&\cw&&&&&\\
      &&&\rstick{$a_j$}&&\rstick{$b_k$}&&&&&&&\\
    \end{quantikz}
    }
    \hfill
  \caption{}\label{fig:3-instrumento-ideal}
\end{subfigure}

\hfill

\begin{subfigure}[a]{0.5\textwidth}
\begin{flushleft}
    \resizebox{11cm}{!}{
        \begin{quantikz}[font=\normalsize,scale=1, row sep=0.3cm,column sep=0.4cm]%
          \lstick[wires=2]{{$\rho_{1,2}$}}&\qw&\gate[wires=2]{?}& \qw&\meter[style={draw=blue}]{$A$} \vcwdouble{1-5}{4-5}{}&\qw&\qw&\qw&\qw\rstick[wires=2]{\tiny $\dfrac{P_{a_j,b_k}+P_{b_k,a_j}}{2}${{$\rho_{1,2}$}}$\dfrac{P_{a_j,b_k}+P_{b_k,a_j}}{2}$}&&&&&&&&&&&&\rstick[wires=5]{$1-q$}\\
          & \qw& \qw& \qw& \qw& \qw&\meter[style={draw=red}]{$B$}\vcwdouble{2-7}{4-7}{}&\qw&\qw&&&&&&&&&&&&\\
          &&&&&&&&&&&&&&&&&&&&&\\
          \lstick{cl}& \cw& \cw& \cw&\cw&\cw&\cw&\cw&\cw&&&&&&&&&&&&&\\
          &&&&\rstick{$a_j$}&&\rstick{$b_k$}&&&&&&&&&&&&&&&\\
        \end{quantikz}
        }
        \hfill
        \caption{}\label{fig:3-instrumento-non-ideal}
\end{flushleft}
\end{subfigure}
\caption{\textbf{(a)}Con una probabilidad $q$ se realiza una medición ideal del observable $A\tensor B$, con un estado posterior dado por $P_{a_j,b_k}\rho_{1,2}P_{a_j,b_k}$, cuando la salida indicada es $a_j b_k$. \textbf{ (b)}Con probabilidad $1-q$ ocurre un error y no se conoce con certeza el estado posterior correspondiente a la salida $a_j b_k$, estará dado por una superposición de dos estados $\dfrac{P_{a_j,b_k}+P_{b_k,a_j}}{2}\rho_{1,2}\dfrac{P_{a_j,b_k}+P_{b_k,a_j}}{2}$.}\label{diagrama-cajas-tercer-instrumento}\source{elaboración propia}
\end{figure}
