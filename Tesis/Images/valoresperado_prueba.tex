
\begin{figure}[H]
  \centering
\begin{subfigure}[a]{0.48\textwidth}
  \centering
  \begin{tikzpicture}[scale=1.2]
    %Draw axis
    \coordinate (y) at (0,5);
    \coordinate (x) at (5,0);
    \draw[axis] (y) -- (0,0) --  (x);
    %Important coordinates. These are used in both figures and can be
    %moved to a seperate settings files
    %% These coordinates deside where boxes start on the y axis
    \coordinate (alphaas) at ($0.8*(y)$);
    \coordinate (alphabs) at ($0.2*(y)$);
    %% These coordinates deside where boxes end on the x axis
    \coordinate (cfas) at ($.2*(x)$);
    \coordinate (cfbs) at ($.9*(x)$);
    %These sets the interest rate lines 
    \coordinate (rl) at ($(cfas)+.15*(x)$);
    \coordinate (rla) at ($(rl)-(x)$);
    \coordinate (rlb) at ($(rl)+(x)$);
    %%%%%%%%%%%%%%%%%%%%%%
    %We makes some boxes and connect some coordinates
    %%%%%%%%%%%%%%%%%%%%%%
    %First, let us draw a line connecting alpha^\A_s og NV^\A_s
    \draw[important line] let \p1=(alphaas), \p2=(cfas) in 
    (\p1) node[left] {$\frac{5}{6}$} (\x2, \y1) -- (\p2)
    node[below] {$-1$};
    %Second, let us connect alpha^\B_s og NV^B_s
    \draw[] let \p1=(alphabs), \p2=(cfbs) in 
    (\p1) node[left] {$\frac{1}{6}$}  (\x2, \y1) -| (\p2)
    node[below] {$1$};
    %A line seperating the boxes.
    \draw[help lines,color=blue] let \p1=(rl), \p2=(y) in
    (\p1) node[below] {$-\frac{2}{3}$} -- (\x1, \y2);
    %%%%%%%%%%%%%%%%%%%%%%
    %The small boxes will be assinged letter
    %%%%%%%%%%%%%%%%%%%%%%
    %%C
    %\draw let \p1=($(alphaas)-(alphabs)$), \p2=(rl), \p3=(alphabs) in
    %($(.5*\x2, .5*\y1+\y3)$) node {$C$};
    %%D
    %\draw let \p1=($(alphaas)-(alphabs)$), \p2=($(cfas)-(rl)$),
    %\p3=(alphabs), \p4=(rl) in
    %($(.5*\x2+\x4, .5*\y1+\y3)$) node {$D$};
    %%E
    %\draw let \p1=(alphabs), \p2=(rl) in
    %($(.5*\x2, .5*\y1)$) node {$E$};
    %%F
    %\draw let \p1=(alphabs), \p2=($(cfas)-(rl)$), \p3=(rl) in
    %($(.5*\x2+\x3, .5*\y1)$) node {$F$};
    %%G
    %\draw let \p1=(alphabs), \p2=($(cfbs)-(cfas)$), \p3=(cfas) in
    %($(.5*\x2+\x3, .5*\y1)$) node {$G$};
    %%%%%%%%%%%%%%%%%%%%%%% Name of axis
    \draw[] let \p1=(y), \p2=(x) in
    (\p1) node[left]{$P(\lambda)$};
    \draw[] let \p1=(x), \p2=(x) in
    (\p1) node[above]{$\lambda$};
    \draw [fill=black] (0.2*5,0.8*5) circle (2pt);
    \draw [fill=black] (0.9*5,0.2*5) circle (2pt);
  \end{tikzpicture}
  \hfill
  \caption{}
\end{subfigure}
\hfill
\begin{subfigure}[a]{0.48\textwidth}
  \centering
  \begin{tikzpicture}[scale=1.2]
    %Draw axis
    \coordinate (y) at (0,5);
    \coordinate (x) at (5,0);
    \draw[axis] (y) -- (0,0) --  (x);
    %Important coordinates. These are used in both figures and can be
    %moved to a seperate settings files
    %% These coordinates deside where boxes start on the y axis
    \coordinate (alphaas) at ($0.35*(y)$);
    \coordinate (alphabs) at ($0.65*(y)$);
    %% These coordinates deside where boxes end on the x axis
    \coordinate (cfas) at ($.2*(x)$);
    \coordinate (cfbs) at ($.9*(x)$);
    %These sets the interest rate lines 
    \coordinate (rl) at ($(cfas)+.45*(x)$);
    \coordinate (rla) at ($(rl)-(x)$);
    \coordinate (rlb) at ($(rl)+(x)$);
    %%%%%%%%%%%%%%%%%%%%%%
    %We makes some boxes and connect some coordinates
    %%%%%%%%%%%%%%%%%%%%%%
    %First, let us draw a line connecting alpha^\A_s og NV^\A_s
    \draw[important line] let \p1=(alphaas), \p2=(cfas) in 
    (\p1) node[left] {$\frac{1}{3}$} (\x2, \y1) -- (\p2)
    node[below] {$-1$};
    %Second, let us connect alpha^\B_s og NV^B_s
    \draw[] let \p1=(alphabs), \p2=(cfbs) in 
    (\p1) node[left] {$\frac{2}{3}$}  (\x2, \y1) -| (\p2)
    node[below] {$1$};
    %A line seperating the boxes.
    \draw[help lines,color=red] let \p1=(rl), \p2=(y) in
    (\p1) node[below] {$\frac{1}{3}$} -- (\x1, \y2);
    %%%%%%%%%%%%%%%%%%%%%%
    %The small boxes will be assinged letter
    %%%%%%%%%%%%%%%%%%%%%%
    %%C
    %\draw let \p1=($(alphaas)-(alphabs)$), \p2=(rl), \p3=(alphabs) in
    %($(.5*\x2, .5*\y1+\y3)$) node {$C$};
    %%D
    %\draw let \p1=($(alphaas)-(alphabs)$), \p2=($(cfas)-(rl)$),
    %\p3=(alphabs), \p4=(rl) in
    %($(.5*\x2+\x4, .5*\y1+\y3)$) node {$D$};
    %%E
    %\draw let \p1=(alphabs), \p2=(rl) in
    %($(.5*\x2, .5*\y1)$) node {$E$};
    %%F
    %\draw let \p1=(alphabs), \p2=($(cfas)-(rl)$), \p3=(rl) in
    %($(.5*\x2+\x3, .5*\y1)$) node {$F$};
    %%G
    %\draw let \p1=(alphabs), \p2=($(cfbs)-(cfas)$), \p3=(cfas) in
    %($(.5*\x2+\x3, .5*\y1)$) node {$G$};
    %%%%%%%%%%%%%%%%%%%%%%% Name of axis
    \draw[] let \p1=(y), \p2=(x) in
    (\p1) node[left]{$P(\lambda)$};
    \draw[] let \p1=(x), \p2=(x) in
    (\p1) node[above]{$\lambda$};
    \draw [fill=black] (0.2*5,0.35*5) circle (2pt);
    \draw [fill=black] (0.9*5,0.65*5) circle (2pt);
  \end{tikzpicture}
  \hfill
  \caption{ }
\end{subfigure}
\caption{Representación gráfica del mapeo de resultados al medir el observable $\sigma_z\otimes \sigma_x$, con valores propios $\lambda=\pm 1$.\textbf{ (a)} La imagen indica la probabilidad de obtener cada una de las salidas del estado inicial $\rho$. La línea azul punteada, indica el valor esperado del observable.\textbf{ (b)} La gráfica indica la probabilidad de obtener cada una de las salidas en el estado $\mathcal{F}_{2\text{p}}(\rho)$. La línea roja punteada, indica el valor esperado del observable. En esta imagen se toma el valor de la probabilidad  $p$  de que las partículas se intercambien como $\frac{1}{4}$.}\label{valor-esperado-imagen}\source{elaboración propia.}
\end{figure}


  
  