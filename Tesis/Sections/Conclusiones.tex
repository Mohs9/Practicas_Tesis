\chapter{CONCLUSIONES}

% \rrnote{ En los capitulos previos se abordaron ideas y conceptos para
% alcanzar el objetivos general, que es describir completamente una medición
% difusa. Recalcar que se usaron dos principales vías para lograrlo.}

A lo largo de los capítulos anteriores se ha llevado a cabo una discusión sobre
ciertos conceptos y herramientas apropiadas para lograr una descripción
completa de un tipo de medición imperfecta, las mediciones difusas en sistemas
de dos o más partículas. En concreto, esta tesis se emplearon dos enfoques
principales que vale la pena enfatizar.

%En el segundo
%capítulo y al inicio del tercer capítulo, los ejemplos se han restringido a
%versiones relativamente simples de ese marco para mantener el análisis
%manejable. Posteriormente,  al concluir el capítulo {\ref{cap3:resultados}}, se
%ha ampliado la discusión de los elementos que se utilizaron para proporcionar
%una descripción completa de las mediciones difusas para sistemas más grandes.
% \cpnote{Demasiadas palabras para decir algo simple}%A pesar de su simplicidad, los elementos analizados en esta tesis son adecuadas para transmitir las principales ideas para una completa descripción de las mediciones difusas. Algunas de esas ideas presentan ciertas diferencias entre ellas, las cuales proporcionan dos enfoques para el análisis de este tipo de medición imperfecta. En concreto, hay dos herramientas clave que vale la pena enfatizar.
% \cpnote{Tambien demasiadas palabras para decir algo simple.un poco redundante con el primer
% parrafo}\rrnote{Junté las ideas de los dos párrafos para simplificar y no redundar.}
%En ese sentido, se aborda de manera más completa el problema tomando como eje en común el valor esperado del observable que se desea medir. 



% \rrnote{(Ojetivo1) Explicar en que consiste, la primera vía utilizada que son
% los operadores de Kraus  y las medidas POVM para describir completamente las
% mediciones difusas para sistemas de dos partículas.}

En primer lugar, las medidas POVM junto con los operadores de Kraus, los cuales
son la primera forma de aproximarse al problema. Los efectos de las medidas
POVM  analizadas brindan una distribución de probabilidad de acuerdo a cada una
de las posibles salidas que pueden ocurrir en una medición difusa. Estos
efectos fueron originados a partir del valor esperado de la medición para
asegurar su idoneidad.  Asimismo, los efectos también pueden descomponerse
para dar origen a los operadores de Kraus, los cuales proporcionan el estado
posterior a la medición. Esta descomposición no es única, pero en este estudio
se examinaron las características de los efectos, con el fin de utilizar una
descomposición simple.


% \rrnote{(Objetivo 2) Explicar la finalidad de utilizar  la segunda vía que son
% los instrumentos cuánticos para describir mediciones difusas en un sistema de
% dos partículas. Por otra parte, explicar los resultados obtenidos al examinar
% exhaustivamente estos instrumentos.}

Como segundo enfoque, se han examinado tres diferentes instrumentos cuánticos
para sistemas de dos partículas. Estos instrumentos se utilizaron con el fin de
analizar las mediciones difusas con distintas interpretaciones.  Igualmente, se
usaron para describir las mediciones de una forma sucinta, de modo que vinculan
tanto las salidas clásicas como las salidas cuánticas de la medición. A pesar
de que se esperaba que los instrumentos propuestos de manera intuitiva,
modelaran correctamente la medición difusa, no resultó ser así. Solo uno de los
tres instrumentos estudiados resultó brindar una especificación concisa y
general de la medición difusa. En consecuencia, las condiciones en las que los
instrumentos resultaban ser similares fueron estudiadas de manera exhaustiva. 


% \rrnote{(Objetivo 3 y 4) Generalizar los operadores de Kraus que especifican
% las mediciones difusas en sistemas de $N$ partículas y explicar la diferencia
% principal con un sistema de dos particulas. Resumir la conclusión de las
% mediciones difusas con observables no factorizables en sistemas de $N$
% partículas.}


Las mediciones difusas también fueron analizadas en sistemas más grandes de $N$
partículas en los cuales se generalizaron los operadores de Kraus que describen
completamente la medición. Estos operadores proporcionan tanto el mapeo de
probabilidades para obtener las posibles salidas, así como el estado posterior
correspondiente a cada una de ellas. Los operadores de Kraus generales son
parecidos a los estudiados en sistemas de dos partículas, sin embargo no son
exactamente análogos. Esto se debe a la construcción que se utilizó para
obtenerlos tomando como base el valor esperado y la composición del operador
difuso en sistema de $N\geq3$ partículas. %\cpnote{mayor que 3 o mayor igual?}\rrnote{Lo corregí, es mayor o igual}.
Particularmente, el análisis en un sistema en el que se desea realizar una
medición de un observable no factorizable, es comparable al de un observable
que sí lo es.

% \rrnote{(Objetivo 5) Mencionar la creacion de un programa que exhiba el mapa de
% probabilidades y el estado resultante luego de la medición, a partir de un
% estado inicial y un observable dado.}

Por último, el análisis de las mediciones difusas para diferentes observables,
con estados iniciales dados, puede ser estudiado de forma más práctica mediante
el programa creado que exhibe el valor esperado de la medición, así como el
mapeo de probabilidad y el estado resultante.

