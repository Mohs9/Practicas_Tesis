\chapter[MEDICIONES DIFUSAS PARA SISTEMA DE DOS PARTÍCULAS]{2. MEDICIONES DIFUSAS PARA SISTEMAS DE DOS PARTÍCULAS}
% Intro {{{

\section{Introducción}
 En este capítulo se presentan las mediciones difusas de sistemas cuánticos de
dos partículas en las que se pueden identificar partículas individuales, sin embargo siempre hay una probabilidad de identificarlas erróneamente. Estas detecciones imperfectas se exponen en la referencia {\cite{Pineda_2021}}. También se utiliza el lenguaje de las operaciones cuánticas por medio de los operadores de Kraus para poder describir las mediciones en los sistemas cuánticos de dos partículas.

La estructura de este capítulo se presenta a continuación. 
En la siguiente sección se establece el problema de las mediciones difusas para un sistema de dos partículas. 
%Luego, en la tercera sección, se discutirán algunos conjuntos de operadores de Kraus que describan los efectos de las mediciones en el estado de entrada del sistema para dos partículas y que proporcionen la probabilidad de cada resultado posible en la medición para cualquier estado inicial.  Además se exponen dos instrumentos cuánticos, los cuales son una herramienta conveniente para describir completamente un medición difusa sintetizadamente. Finalmente, en la cuarta sección se presentan algunos ejemplos de las mediciones y su descripción utilizando los operadores en la tercera sección.

% }}}

\section{Mediciones no ideales}
\rrnote{En esta sección vamos a hacer una discusión de medciones imperfectas que llevará a la definición  de una medición difusa como tal.}

\subsection{Ruido y detectores imperfectos}

\rrnote{Tocar el tema que los detectores del mundo real no son perfectos e introducen sus propias incertidumbres y as ineficiencias del detector, el ruido y otras imperfecciones pueden afectar la precisión de las mediciones.}



\subsection{Ejemplo de medición ideal y no ideal}

\rrnote{ Podría describir un ejemplo/experimento de mediciones no ideales, incluir imágenes }


\subsection{Medición difusa en sistemas cuánticos}
\rrnote{En esta sección quiero definir como tal las mediciones difusas, y expandir/ilustrar un ejemplo concreto como el de la cadena de iones }



\section{Mediciones difusas para dos partículas} % {{{




\section{Operadores de Kraus para mediciones difusas} % {{{




% }}}
% }}}
\subsection{POVM para mediciones difusas}\label{Sec_POVM_para_mediciones_difusas} % {{{



\rrnote{Agregar más de una implentación de efectos en esta sección (?)}

\subsection{Instrumentos cuánticos}\label{sec:instrumentos-cuanticos} % {{{




\subsubsection{Primera alternativa de instrumento cuántico} % {{{

\subsubsection{Segunda alternativa de instrumento cuántico}

\subsection{Equivalencia de los instrumentos}
\rrnote{Idea principal: En esta sección se comenta que se esperaba que los instrumentos fueran equivalentes pero resultaron distintos y por ello vale la pena proponer en qué condiciones sí lo son.}


\section{Ejemplos sobre los efectos de una medición difusa} % {{{
