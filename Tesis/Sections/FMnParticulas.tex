\chapter{ MEDICIONES DIFUSAS EN SISTEMAS DE \texorpdfstring{\boldmath{$N$}}{N} PARTÍCULAS}
\section{ Introducción}

















\section{Generalización de operadores de Kraus en sistemas de \texorpdfstring{\boldmath{$N$}}{N} partículas}



En sistema de $n$ partículas en el que se realiza una medición de un observable de la forma $A_1\tensor A_2\tensor \hdots \tensor A_N$, y en el que ocurre un error en el aparato de medición y confunde los resultados. Con cierta probabilidad se realiza la medición de acuerdo a la permutación $\Pi$ del observable. 

El operador difuso definido en la ecuación ({\ref{eq:operador_difuso2p}}), ahora se pude redefinir para un sistema de $N$ partículas se puede de forma más general como\begin{equation}\label{eq:fm-nparticles}
   \fuzzy{\rho}=\sum_{\Pi\in S}p_{\Pi}\Pi(\rho)
\end{equation}donde $\mathcal{S}$ es un subconjunto del grupo simétrico de $n$ partículas y $\sum_{\Pi \in \mathcal{S}} p_\Pi=1$. El valor esperado de esta medición será {\cite{Pineda_2021}}\begin{equation}\label{eq:ExpectedValue-generalForm}\begin{split}
    \left \langle \prodtensor A_i\right \rangle_{\mathcal{F}(\rho)} &=\tr\left(\fuzzy{\rho}\prodtensor A_i\right)\\
    &=\sum_{j \in S}p_{j}\tr \left(\Pi_j(\rho) \prodtensor A_i\right).
\end{split}
\end{equation} 

\subsection{Medidas POVM en sistemas de varias partículas}
En el capítulo anterior se describió el mapeo que puede realizarse con las medidas POVM, el cual será conveniente para proporcionar la probabilidad de cada posible salida de la medición. En un sistema de dos partículas se propuso intuitivamente un conjunto de efectos cuya generalización para $N$ partículas puede ser escrita como \begin{equation}\label{eq:effectsSetNp}
    {\{E_{\lambda_i}\}}_{\lambda_i \in \Lambda}={\{\fuzzy{P_{\lambda_i}}\}}_{\lambda_i \in \Lambda}={\left\{\sum_{\Pi \in S} p_\Pi \Pi(P_{\lambda_i})\right\}}_{\lambda_i \in \Lambda},
\end{equation}  
donde $P_{\lambda_i}$ es el operador de proyección correspondiente a cada vector propio que tiene asociado el valor propio $\lambda_i\in \Lambda$. El conjunto de valores propios del observable es \begin{equation}\label{eq:lambdaeigenvalues}
    \Lambda=\{a_{11}\cdot a_{21}\cdot \hdots \cdot a_{N1},\hdots,a_{1J}\cdot a_{2K}\cdot \hdots \cdot a_{NM}\},
\end{equation} donde $a_{jk}$ es el $k$-ésimo valor propio correspondiente al observable $A_j$.  Es fácilmente comprobable que estos operadores $E_{\lambda_i}$ son hermíticos, cumplen con la propiedad de completitud y son positivos.


Para obtener el estado posterior a la medición se requiere utilizar la descomposición de los efectos en operadores de Kraus $\{K_{\lambda_i}\}$. Para mediciones difusas en sistema de $N$ partículas se propone utilizar \begin{equation}
    K_{\lambda_i}=\sqrt{\sum_{\Pi \in S} p_\Pi \Pi(P_{\lambda_i})},
\end{equation} esto se puede realizar debido a la positividad de los efectos. De nuevo, esta descomposición no es única y el estado posterior de la medición dependerá de como se implementen las medidas POVM en el laboratorio. 



\subsection{Instrumentos cuánticos en sistemas de \texorpdfstring{\boldmath{$N$}}{N} partículas}
En esta parte también se consideran las dos alternativas de instrumentos cuánticos de la sección ({\ref{sec:instrumentos-cuanticos}}). La primera alternativa es el instrumento en el que las partículas se intercambian y luego se aplica una medición proyectiva \begin{equation}\label{eq:1instrumentnp}
    \mathcal{I}_1(\rho)=\sum_{\lambda_i \in \Lambda }P_{\lambda_i}\otimes P_{\lambda_i}\fuzzy{\rho}P_{\lambda_i},
\end{equation} donde $P_{\lambda_i}$ son los operadores de proyección y $\lambda_i$ son valores propios del observable.

\section{Observables no factorizables en mediciones difusas}
\section{Observables degenerados en mediciones difusas}
