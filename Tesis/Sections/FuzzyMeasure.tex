\chapter[MEDICIONES DIFUSAS ]{2. MEDICIONES DIFUSAS}
\section{Introducción} % Intro {{{
 En este capítulo se presentan las mediciones difusas de sistemas cuánticos las que se pueden identificar partículas individuales, sin embargo siempre hay una probabilidad de identificarlas erróneamente. Estas detecciones imperfectas se exponen en la referencia {\cite{Pineda_2021}}. %También se utiliza el lenguaje de las operaciones cuánticas por medio de los operadores de Kraus para poder describir las mediciones en los sistemas cuánticos de dos partículas.

La estructura de este capítulo se presenta a continuación. En la primera sección se exponen las mediciones no ideales y un ejemplo. En la siguiente sección se establece el problema de las mediciones difusas para un sistema cuántico. 
%Luego, en la tercera sección, se discutirán algunos conjuntos de operadores de Kraus que describan los efectos de las mediciones en el estado de entrada del sistema para dos partículas y que proporcionen la probabilidad de cada resultado posible en la medición para cualquier estado inicial.  Además se exponen dos instrumentos cuánticos, los cuales son una herramienta conveniente para describir completamente un medición difusa sintetizadamente. Finalmente, en la cuarta sección se presentan algunos ejemplos de las mediciones y su descripción utilizando los operadores en la tercera sección.

% }}}
\section{Mediciones no ideales}\label{sec:cap2MedicionesNoIdeales} % {{{

Los sistemas cuánticos ideales son sistemas cerrados sujetos a la teoría
cuántica, que no interactúan con su entorno y por lo tanto, no están sometidos
a perturbaciones que pueden provocar pérdida de información en el sentido de ruido clásico {\cite{wilde2011classical}}. %\cpnote{Como disturbios?}, \cpnote{Comose pierde informacion?}\rrnote{agregué lo que estaba en la referencia}.
%Aunque estos sistemas sufren cierto tipo de ruido cuántico\cpnote{que es ruido cuantico? Evita hablar de cosas que no tienes totalmente claras}\rrnote{de acuerdo, quite esa primera frase}
%, como el del principio de incertidumbre, 
La medición de estos sistemas puede describirse con mediciones
proyectivas o ideales. %\cpnote{porque la restriccion para sistemas no degenerados?}\rrnote{lo había puesto porque estaba en una referencia simplificada pero no es necesario.}
Estos sistemas con mediciones ideales son una
simplificación de la realidad difícil de justificar en la práctica. Las
mediciones cuánticas proyectivas son inadecuadas por varias razones. Una de
ellas es que se pueden realizar mediciones de tal manera que dispositivos
imprecisos agreguen ruido clásico al resultado de la medición. Otra razón es
que 
%pueden producirse errores en la medición es debido a al acoplamiento con  otros grados de libertad ajenos al sistema que se está controlando. O bien, que
en sistemas abiertos, en los que se intercambia constantemente información con
su entorno de forma incontrolable, lo que resulta en una decoherencia del
estado en estudio. %\cpnote{en que se diferencia esto del acoplamiento?}\rrnote{Dejé solo la última frase}. 
En el caso de sistemas cuánticos no ideales puede utilizarse
una definición más general de medición {\cite{wilde2011classical,
jaeger2007quantum}}.%\cpnote{Esta informacion de donde la sacaste? Está padre} \rrnote{De este libro \href{https://arxiv.org/pdf/1106.1445.pdf}{de Wilde}}


El siguiente ejemplo permite visualizar el concepto de una medición no ideal de forma que clarifica la diferencia entre una medición ideal y una en la que se involucran imperfecciones en el sistema.
%\cpnote{Si no lo vamos a ver en el trabajo, para que lo mencionamos? Toca justificar!}\rrnote{Agregué una pequeña justificación} 
A. Peres {\cite{peres1997quantum}} ejemplifica una situación de una medición no
ideal a la que llama <<medición difusa>>, sin embargo, es importante subrayar
que no es el enfoque que se trabajará en el documento. En este ejemplo existe
una señal clásica extremadamente débil, por lo cual no es posible ignorar las propiedades cuánticas del detector y debe ser tratado como un sistema cuántico. %entre la naturaleza continua de una señal clásica y el espectro discreto de un dispositivo cuántico %\cpnote{Me da cosa que esto lo copiaste textual de algun lado. No lo entiendo.}\rrnote{Sí,lo habia tomado casi literal del libro de A. Peres, el ejemplo trata sobre señales débiles que causan un conflicto con el detector y se agrega un aparato que mide al detector}
% Además, el resultado final no puede ser registrado por un sistema cuántico, que puede estar en una superposición de estados propios \cpnote{Porque no?}. 
 El detector  tiene que <<medirse>> por un aparato adecuado al que se le llama <<el medidor>>, que luego produce otra señal clásica.%\cpnote{El detector no mide? como?}.


Considérese una partícula de espín-$\frac{1}{2}$ cuyo estado inicial está descrito por $|\phi\rangle=\begin{pmatrix}\alpha\\\beta\end{pmatrix}$. Se desea medir el observable $\sigma_z$, el cual tiene valores propios $\pm 1$. Además, se supone que el medidor tiene un solo grado de libertad y es adimensional, $q$, que indica el valor de la variable medida y su función de onda inicial está dada por $\phi(q)$.

Luego, el vector de estado se transforma de la siguiente manera\begin{equation}\label{ejemplo-fm-peres}
    \begin{pmatrix}\alpha\\\beta\end{pmatrix}\otimes \phi(q)\to \begin{pmatrix}\alpha\\0\end{pmatrix}\otimes \phi(q-1)+\begin{pmatrix}0\\\beta\end{pmatrix}\otimes \phi(q+1).
\end{equation}Suponiendo que antes de la medición es más probable encontrar el medidor cerca de $q = 0$. Entonces, existe una amplitud de probabilidad $\alpha\phi(q-1)$ para encontrar el medidor cerca de $q = 1$ y una amplitud de probabilidad $\beta \phi(q + 1)$ para encontrarlo cerca de $q = -1$.
%Lo que llamamos “el valor observado de $\sigma_z$” es la posición final del medidor.
Idealmente, $\phi(q\mp 1)$ debería tener un ancho cero y el resultado de la medición debería ser uno de los valores propios de $\sigma_z$.   Sin embargo, el valor realmente observado de $\sigma_z$ puede diferir del resultado ideal en una cantidad de orden $\Delta q>2$. Esta discrepancia no es una <<dificultad técnica trivial>>. La probabilidad de observar el medidor entre $q_0$ y $q_0 + d q_0$ es\[P(q_0)dq_0=[|\alpha|^2|\phi(q_0-1)|^2+|\beta|^2|\phi(q_0+1)|^2]dq_0,\]si el valor observado $q_0$ es tal que ambos términos del lado derecho contribuyen, el estado final de la partícula no es un estado propio de $\sigma_z$.%\cpnote{Mejora la redaccion y porfa tambien explicame porque el estado de la particula no es un estado propio de sigma z}\rrnote{Agregué algunas oraciones y quité otras}.
Sino que viene dado por el lado derecho de la ecuación {\eqref{ejemplo-fm-peres}}, evaluada en $q = q_0$. La ubicación inicial del medidor es incierta, de modo que a esta medición A. Peres la llama <<difusa>>.   


% }}}
\section{Medición difusa en sistemas cuánticos}\label{sec:Cap2MedicionesDifusas} % {{{
% Intro {{{
%\rrnote{En esta sección, se pretende discutir cómo se pueden entender las mediciones difusas, y expandir/ilustrar un ejemplo concreto como el de la cadena de iones.}
%\rrnote{Explicar el resultado experimental de una medición difusa. Ejemplificando el  experimento de una cadena iones en la que se  realiza una medición con un detector imperfecto, el cual realiza una señal, sin embargo  no es posible determinar exactamente de qué ion proviene.  Se puede plantear las condiciones en las que se puede cuantificar una medición difusa. }

En esta tesis, el concepto de <<medición difusa>>, es distinta al planteado en el ejemplo del apartado anterior.  Ahora, se procede a estudiar las ideas detrás de las mediciones difusas que se abordan en el resto del documento. 
Para esto se considera la siguiente situación presentada por Pineda, Davalos, Viviescas y Rosado {\cite{Pineda_2021}}. Supóngase que se tiene una cadena de iones y se realiza una medición a uno solo de ellos. Se hace brillar la cadena de iones y se obtiene una señal fluorescente. Dicha señal se recibe con un dispositivo detector imperfecto, en consecuencia no se puede saber con seguridad de dónde provino {\cite{Pineda_2021}}. No obstante, es posible cuantificar la información que se obtuvo incluso cuando ésta es borrosa.  Con base en lo anterior, una medición difusa puede definirse como sigue. 

%<<Una cadena de iones se hace brillar  y se obtiene una señal fluorescente. Sin embargo, debido a la imperfecciones del detector, no es posible determinar el origen exacto de la señal fluorescente. La información obtenida en este caso se vuelve borrosa, pero su cuantificación aún es posible>>.


\begin{definition}Una medición difusa es un proceso no ideal en el cual, debido a ruido del entorno o a fallos en el dispositivo de detección, se presenta la probabilidad de una identificación errónea de las partículas del sistema.
\end{definition}

%\rrnote{Realizar una explicacion de lo que sucede en una medición de dos partículas con observable factorizable. Mencionar a que nos referimos con un operador factorizable. Simplificar el sistema a dos partículas y mencionar qué se obtendría al querer medir un observable.}

Ahora, se considera un sistema cuántico solo de dos partículas, el cual tiene un espacio de Hilbert dado por $\mathcal{H}=\mathcal{H}_1\otimes \mathcal{H}_2$, y en el que se desea medir un cierto observable $A\tensor B$.  Por simplicidad, se analiza un observable factorizable, cuya representación tensorial se puede expresar en un solo término. Se realiza una medición difusa, si con cierta probabilidad $p$, el aparato mide al observable $A$ en el sistema de la partícula en el espacio $\mathcal{H}_1$ y al observable $B$ en el subsistema de la partícula en el espacio $\mathcal{H}_2$. Sin embargo, existe una probabilidad de $1-p$ de que el detector confunda las partículas y, en su lugar, se mida el observable $B$ en el primer subsistema y a $A$ en el segundo subsistema {\cite{Pineda_2021}}.


% }}}
\subsection{Operador difuso}\label{subsec:Cap2OperadorDifusas} % {{{

El propósito de este apartado es introducir y establecer de manera formal el operador difuso aplicable a cualquier sistema cuántico que se analizará en las siguientes secciones. Para ello es necesario abordar el formalismo del <<operador de intercambio>> para sistemas de dos o más partículas.


%\rrnote{Primero se presenta más detalladamente las operaciones de intercambio. Se definen la transformación SWAP, para un sistema de dos partículas, y sus propiedades como que es una matriz unitaria, hermítica.}

Se iniciará estudiando el caso de un sistema cuántico que consta de solo dos partículas. El operador intercambio u operador SWAP, con respecto a las partículas en los espacios $\mathcal{H}_1$ y $\mathcal{H}_2$, se define de la siguiente forma\begin{equation}\label{swap}
S_{12}: |\psi\rangle_{1}\otimes |\varphi \rangle_{2}=\sum_{i,j}\psi_i\varphi_j|i,j\rangle \longmapsto     \sum_{i,j}\varphi_i\psi_j|j,i\rangle= |\varphi\rangle_{1}\otimes |\psi \rangle_{2}.
\end{equation} 

Con esta definición es inmediato notar que el operador SWAP para dos partículas es igual a su adjunto y a su transformación inversa. Lo que significa que el operador cumple con las siguientes propiedades\begin{enumerate}
    \item Es \textit{involutivo}   $S_{12}^{-1}={S}$ y ${S}_{12}^{2}=\mathds{1}$, esto es,  aplicar el operador de intercambio a las partículas dos veces deja al sistema sin cambios.
   % \[S^{2}|\psi \varphi \rangle=S|\varphi \psi \rangle=|\psi \varphi \rangle\]
    \item Es \textit{hermítico} $ {S}_{12}^{\dagger}={S}$.
    %\[\begin{split}
     %  \text{Primero se toma el siguiente producto interno } \langle\psi \varphi|S|\psi\varphi \rangle=\langle\psi \varphi|\varphi\psi \rangle=\langle\psi|\varphi\rala \varphi| \psi\rangle\\
     %   \langle\psi \varphi|S^\dagger|\psi\varphi \rangle=\langle \varphi\psi |\psi\varphi \rangle=\langle \varphi| \psi\rangle\langle\psi|\varphi\rangle \\
    %\end{split}\]
    \item Es \textit{unitario} $ {S}_{12}^{\dagger}{S}_{12}=\mathds{1}$.
\end{enumerate} 
%\rrnote{Segundo se propone la operación difusa para un sistema de dos partículas y la intuición que existe detrás del operador. Me gustaría definir el operador difuso}

Continuando en el espacio de Hilbert  $\mathcal{H}=\mathcal{H}_1\otimes\mathcal{H}_2$ para el sistema de dos partículas. Se construye una operación cuántica que cumpla con el teorema {\ref{teorema:equivalencia_propiedadesaxiomaticas_opKraus}}. A dicha operación se le  llamará \textit{operación difusa} u \textit{operador difuso}\begin{equation}\label{eq:op_F2p}
    \mathcal{F}_{2\text{p}}(\rho):=p\rho + (1-p)S_{12}\rho S_{12}^{\dagger}.
\end{equation}
El primer término indica que con una probabilidad $p$  se aplica la identidad es decir que el estado se deja intacto.  Además, el segundo término del operador difuso indica que con una probabilidad $(1-p)$ se aplica el operador SWAP al estado $\rho$, lo que implica que las partículas se intercambian con esa probabilidad.

%\rrnote{Luego, quisiera continuar con la generalización de los operadores de intercambio, que serían los operadores de permutación. Presenta  la definición matemática y mostrar si sigue cumpliendo con algunas propiedas de las matrices para dos partículas. }

Lo anterior se puede generalizar para sistemas con mayor complejidad, en los que se tiene más de dos partículas.  Es relevante notar que, en sistemas con varias partículas es factible intercambiar solo dos elementos con un operador $S_{ij}$, (los subíndices denotan el intercambio respecto a las partículas $i$ y $j$) dejando a los demás invariantes.  Sin embargo, es viable que en el sistema se lleve a cabo el intercambio de más de dos partículas. Para describir estas acciones de intercambio, es necesario hablar de los operadores de permutación.

Los operadores de permutación, como el nombre lo sugiere, intercambian algunas partículas con otras. El proceso puede extenderse a permutaciones entre un número arbitrario de $N$ partículas pero, no todas las propiedades obtenidas para el caso de dos partículas se aplican en general. Será más fácil, primero  observar el caso $N = 3$ antes de generalizar los resultados. El operador de permutación $\Pi_{mnp}$ es por definición tal que transforma el vector de estado $|\psi\ra= |u_i\ra_1\tensor |u_j\ra_2 \tensor |u_k\ra_3$, de la siguiente forma {\cite{cohen1977quantum}}\[\Pi_{mnp} |u_i\ra_1\tensor |u_j\ra_2 \tensor |u_k\ra_3= |u_i\ra_m\tensor |u_j\ra_n \tensor |u_k\ra_p.\] Por ejemplo, el operador de permutación $\Pi_{231}$ actúa en el estado $|\psi\ra$ como  
\[\Pi_{231} |u_i\ra_1\tensor |u_j\ra_2 \tensor |u_k\ra_3= |u_i\ra_2\tensor |u_j\ra_3 \tensor |u_k\ra_1,\] lo que es equivalente a 
\[\Pi_{231} |u_i\ra_1\tensor |u_j\ra_2 \tensor |u_k\ra_3= |u_k\ra_1\tensor|u_i\ra_2\tensor  |u_j\ra_3.\]  Existen $N$! permutaciones, para este caso $N = 3$, existen 6 formas de permutar las partículas. Notar que realizar una permutación cíclica en un operador dado lo deja sin cambios. 



En general,todo operador de permutación puede escribirse como producto de operaciones $S_{ij}$, \begin{equation}\label{eq:permutation_operator}\Pi_{\alpha}=S_{\beta_1}S_{\beta_2}\cdots S_{\beta_K}, \end{equation} donde $\alpha$ denota una permutación y $\beta$ un intercambio de algún par de partículas. Si para escribir el operador de permutación se requiere un número par de operadores de intercambio $S_{ij}$, se dice que ese operador es par, y si se requiere un número impar, se dice que es impar. A esto se le llama la <<paridad del operador>> {\cite{cohen1977quantum}}. 


No todas las propiedades obtenidas para el operador de intercambio de dos partículas se mantienen. Las propiedades destacables de los operadores de permutación se enumeran a continuación 
\begin{enumerate}
    \item Son \textit{unitarios} $\Pi_\alpha\Pi_\alpha^{\dagger}=\mathds{1}$, esto es inmediato del hecho que son el producto de los operadores de intercambio de dos partículas $S_{\beta_k}$.
    \item \textit{No son hermíticos} para $N>2$. 
    \item Su operador adjunto tiene la misma paridad.

\end{enumerate}

Además, los operadores de permutación forman un grupo. Puesto que, existe un operador de permutación que es igual a la identidad, cada permutación admite una operación inversa y el producto de dos permutaciones produce otra permutación.

%\rrnote{Finalmente, escribir detalladamente el resultado generalizado el operador difuso, para más partículas  (ecuación 2 del paper). De nuevo me gustaria puntualizar su definición, explicar lo que realiza en un sistema más grande y verficar si sigue cumpliendo las propiedades del operador disfuso para dos partículas. }

Los operadores de permutación fueron presentados puesto que son necesarios para ampliar el alcance del operador difuso, definido en la ecuación {\eqref{eq:op_F2p}}, para un sistema de más partículas. De forma general en un sistema de $N$ partículas se define como sigue\begin{equation}\label{eq:fuzzy-op-nparticles}
    \fuzzy{\rho}=\sum_{\Pi_i\in S}p_{i}\permut{i}{\rho}
 \end{equation}donde $\mathcal{S}$ es un subconjunto del grupo simétrico de $N$ partículas y $\sum_{i=1}^{N!} p_i=1$.  La idea detrás de este operador es análoga a la anterior. Cada uno de los términos representa que con cierta probabilidad $p_i$ el estado $\rho$ sufre una permutación. En otras palabras, las partículas son intercambiadas en alguna disposición dictada por el operador $\Pi_{i}$, con dicha probabilidad.






% }}}
\subsection{Valores esperados en mediciones difusas}\label{subsec:Valores_esperados} % {{{


En esta sección se ahonda en la relevancia del valor esperado y su aplicación
en este estudio.  Se presenta el concepto del valor esperado y se detalla qué
sucede con él en una medición difusa y su relación con el operador difuso.
Además, se ejemplifica la diferencia entre los valores esperados en una
medición ideal y una medición difusa. Por último, se argumenta la utilidad de
esta medida y se expone para qué se usará en este proyecto. 

 
% \cpnote{Mejorar la intro a la luz de lo que discutimos.}\rrnote{Detalle más que se hará en esta sección }

% \rrnote{Esqueleto: Decir qué es el valor esperado en mecánica cuántica. Y hacer la aclaración entre salidas de la medicion y el valor esperado.}


El valor esperado es un término que se deriva de la teoría básica de la
probabilidad y representa el promedio ponderado de los resultados posibles que
se pueden obtener, cuando se dispone de muchos sistemas, todos en el mismo
estado inicial, y se realiza la misma medición en cada uno de ellos. La
ponderación que se le asigna a cada resultado se determina por la probabilidad
de obtener dicho resultado
{\cite{gomez2010introduccion,sakurai2017modern}}.
Es importante, discernir entre el valor esperado de un observable y las salidas
que pueden resultar de una medición del observable. Esto significa que los
resultados de la medición podrían diferir pero el promedio será lo presentado
en la ecuación {\eqref{expectationvalue_traza}}. 

% \rrnote{Introducir el valor esperado de una medición difusa, y explicar la idea
% detrás de esta ecuación (1) del paper para sistemas de dos partículas. Y
% extenderlo al sistema de N partículas.}



%Un sistema que experimenta cambios aleatorios tendrá un valor esperado diferente a un sistema que no experimenta tales transformaciones. Estos cambios en el estado inicial, que ocurren como resultado de algún proceso físico, pueden ser capturados por las operaciones cuánticas. Entonces, si $\rho$ es el estado inicial antes del proceso, y $\mathcal{E}(\rho)$ representa la transformación que ocurre, el valor esperado de este estado final se puede expresar como
%\begin{equation}\label{eq:Expected-Value-Op-Cuanticas}
 %   \la \mathcal{O}\ra_{\mathcal{E}(\rho)}=\tr(\mathcal{E}(\rho)\mathcal{O}).
%\end{equation} 
% \cpnote{No veo este parrafo muy util. Discutir con Rubi la utilidad de este. }
%  \rrnote{Quité el párrafo para arreglar el párrafo siguiente.}

% \cpnote{HAblar con Rubi del situiente parrafo. Lo tiene alrevez :P }
% \cpnote{En este parrafo no está la motiacion de la medicion difusa. porque es ese y no otro? Creo que eso es lo mas improtante de esta seccion y te falto. Corrige eso e itera 
% de acá en adelante. Para el 2.4 y 2.5 voy a querer esqueletos mas detallados. }%\rrnote{La motivacion de la medición difusa la habia puesto en una sección anterior entonces me parecio redundante. Aca solo me estaba centrando en el valor esperado al calculo del valor esperado y mecionando que el operador difuso registra el posible intercambio de particulas. Y de acuerdo, más detallado en las siguientes secciones.}


Ahora, se puede analizar una medición difusa en un sistema
de dos partículas. En este sistema se desea realizar la medición del observable factorizable $A\otimes B$. Sin embargo, la medición es imperfecta, y con cierta probabilidad se realizará, en su lugar, la medición del observable $B\otimes A$. Como resultado, el valor esperado de la medición difusa se obtendrá mediante una combinación convexa de los valores esperados. Este se calculará sumando el valor esperado del observable $A\otimes B$, al cual se le asigna la probabilidad que se realice la medición correctamente, más el valor esperado de $B\otimes A$ ponderado por la probabilidad de medir este otro observable {\cite{Pineda_2021}}\begin{equation}\label{eq:Expected-Value-FM-2p}
    \begin{split}
         p\tr(\rho A\tensor B)+(1-p)\tr(\rho B\otimes A)=&\la {A\otimes B}\ra_{\mathcal{F}_{2\text{p}}(\rho)}.\\
    \end{split}
\end{equation}

%
Es posible desarrollar la parte derecha de esta última ecuación con el fin de simplificar la expresión, aprovechando la linealidad y la propiedad cíclica de la traza \begin{equation}\label{eq:Expected-Value-FM-2p-desarrollo}
    \begin{split}
         p\tr(\rho A\tensor B)+(1-p)\tr(\rho B\otimes A)
        =&p\tr(\rho A\tensor B)+(1-p)\tr(S\rho S^\dagger A\otimes B)\\
        =&\tr(\mathcal{F}_{2\text{p}}(\rho)A\otimes B).\\
    \end{split}
\end{equation} Este desarrollo permite justificar la introducción del operador difuso explicado en la ecuación {\eqref{eq:op_F2p}}. En este escenario, el estado inicial sufre una transformación, donde  con cierta probabilidad las partículas se intercambian. El operador difuso $\mathcal{F}_{2\text{p}}$ registra esta evolución del estado inicial $\rho$. En consecuencia, si se tienen muchas copias del estado posterior $\rho'=\mathcal{F}_{2\text{p}}(\rho)$, entonces al realizar mediciones en cada una de ellas, el promedio de las salidas estará dado por esta ecuación.  La última expresión es análoga al lado derecho de la ecuación {\eqref{expectationvalue_traza}}, con la diferencia que no se calcula con $\rho$ sino con el operador $\mathcal{F}_{2\text{p}}(\rho)$.



El valor esperado de un observable en un sistema con más de dos partículas es análogo al anterior. En este sistema, la transformación es representada por el operador difuso general de la ecuación {\eqref{eq:fuzzy-op-nparticles}}. En consecuencia, el valor esperado para un observable $\mathcal{O}$ en una medición difusa es\begin{equation}\label{eq:expected-value-fm-general}
    \la \mathcal{O}\ra_{\fuzzy{\rho}}=\sum_{\Pi_i\in S}p_i\tr(\permut{i}{\rho}\mathcal{O}) =\tr(\fuzzy{\rho}\mathcal{O}).
\end{equation}



% \rrnote{Proponer un ejemplo sencillo de un valor esperado para un estado
% simple y el observable $\sigma_z\otimes \sigma_x$. Obtener el valor esperado
% de la medición difusa, del mismo observable y el mismo estado inicial y
% comparar ambos valores esperados. Incluir una imagen de la funcion de
% probabilidad de los valores propios del observable y los distintos valores
% esperados. }
 

Ahora se propone un ejemplo para ilustrar la diferencia entre un valor esperado de un observable en una medición ideal y una medición difusa. Considere un sistema de dos partículas con el siguiente estado inicial \[\rho= \left(\frac{1}{3}\ket{0}\bra{0}+\frac{2}{3}\ket{1}\bra{1}\right)\otimes\left(\ket{+}\bra{+}\right)=\begin{pmatrix}
    1/6&1/6&0&0\\
    1/6&1/6&0&0\\
    0&0&1/3&1/3\\
    0&0&1/3&1/3\\
\end{pmatrix},\] en el que se quiere realizar la medición del observable $\sigma_z\tensor\sigma_x$. En principio, el valor esperado para este observable se puede obtener de la siguiente forma:
\[\begin{split}
    \tr(\sigma_z\tensor\sigma_x \rho) =& \tr\left[ \begin{pmatrix}
    0&1&0&0\\
    1&0&0&0\\
    0&0&0&-1\\
    0&0&-1&0\\
\end{pmatrix}\begin{pmatrix}
    1/6&1/6&0&0\\
    1/6&1/6&0&0\\
    0&0&1/3&1/3\\
    0&0&1/3&1/3\\
\end{pmatrix}\right]\\
=& \tr\left[ \begin{pmatrix}
    1/6&1/6&0&0\\
    1/6&1/6&0&0\\
    0&0&-1/3&-1/3\\
    0&0&-1/3&-1/3\\
\end{pmatrix}\right]=-1/3. \end{split}\] Los valores propios del observable son $\pm 1$, como se ve las salidas de la medición no coinciden con el valor esperado. Puesto que, como ya se explico este es el promedio ponderado de las salidas de la medición.  En este estado $\rho$, la probabilidad de obtener  la salida $-1$ es $\tr(\rho P_{0-})+\tr(\rho P_{1+})=2/3$. Mientras que la probabilidad de obtener $+1$ es solamente $\tr(\rho P_{0+})+\tr(\rho P_{1-})=1/3$. Donde, $ P_{0-}, P_{1+}$ son los operadores de proyección correspondientes al valor propio $-1$ y $ P_{0+}, P_{1-}$, corresponden al valor propio $+1$. De lo anterior, es fácilmente comprobable el resultado obtenido con la traza calculada.

Sin embargo, en una medición difusa este valor esperado cambia puesto que el estado experimenta transformaciones aleatorias capturadas por el operador difuso. Para este caso el valor esperado estará dado por \[\begin{split}\tr(\mathcal{F}_{2\text{p}}(\rho)\sigma_z\tensor \sigma_x)=&p\tr(\rho\sigma_z\tensor \sigma_x )+(1-p)\tr(\rho\sigma_x\tensor \sigma_z )\\
    =& \frac{p}{3}\tr\left[ \begin{pmatrix}
        \tfrac{1}{2}&\tfrac{1}{2}&0&0\\
        \tfrac{1}{2}&\tfrac{1}{2}&0&0\\
        0&0&-1&-1\\
        0&0&-1&-1\\
    \end{pmatrix}\right]+ \frac{(1-p)}{3}\tr\left[ \begin{pmatrix}
        0&0&\tfrac{1}{2}&-\tfrac{1}{2}\\
        0&0&\tfrac{1}{2}&-\tfrac{1}{2}\\
        1&1&0&0\\
        1&1&0&0\\
    \end{pmatrix}\right]\\
    =&-\frac{p}{3}.
\end{split}\]Si $p<1$, se obtiene un valor esperado distinto. Esto se debe a
que las probabilidades de obtener cada una de las salidas son
$\tr(\mathcal{F}_{2p}(\rho)P_{0+})+\tr(\mathcal{F}_{2p}(\rho)P_{1-})=\frac{3-p}{6}$
y
$\tr(\mathcal{F}_{2p}(\rho)P_{0-})+\tr(\mathcal{F}_{2p}(\rho)P_{1+})=\frac{3+p}{6}$.
Estas probabilidades ahora varían en función de la probabilidad $p$. En la
siguiente figura puede apreciarse gráficamente estos resultados.





\begin{figure}[H]
  \centering
\begin{subfigure}[a]{0.48\textwidth}
  \centering
  \begin{tikzpicture}[scale=1.2]
    %Draw axis
    \coordinate (y) at (0,5);
    \coordinate (x) at (5,0);
    \draw[axis] (y) -- (0,0) --  (x);
    %Important coordinates. These are used in both figures and can be
    %moved to a seperate settings files
    %% These coordinates deside where boxes start on the y axis
    \coordinate (alphaas) at ($0.8*(y)$);
    \coordinate (alphabs) at ($0.2*(y)$);
    %% These coordinates deside where boxes end on the x axis
    \coordinate (cfas) at ($.2*(x)$);
    \coordinate (cfbs) at ($.9*(x)$);
    %These sets the interest rate lines 
    \coordinate (rl) at ($(cfas)+.15*(x)$);
    \coordinate (rla) at ($(rl)-(x)$);
    \coordinate (rlb) at ($(rl)+(x)$);
    %%%%%%%%%%%%%%%%%%%%%%
    %We makes some boxes and connect some coordinates
    %%%%%%%%%%%%%%%%%%%%%%
    %First, let us draw a line connecting alpha^\A_s og NV^\A_s
    \draw[important line] let \p1=(alphaas), \p2=(cfas) in 
    (\p1) node[left] {$\frac{5}{6}$} (\x2, \y1) -- (\p2)
    node[below] {$-1$};
    %Second, let us connect alpha^\B_s og NV^B_s
    \draw[] let \p1=(alphabs), \p2=(cfbs) in 
    (\p1) node[left] {$\frac{1}{6}$}  (\x2, \y1) -| (\p2)
    node[below] {$1$};
    %A line seperating the boxes.
    \draw[help lines,color=blue] let \p1=(rl), \p2=(y) in
    (\p1) node[below] {$-\frac{2}{3}$} -- (\x1, \y2);
    %%%%%%%%%%%%%%%%%%%%%%
    %The small boxes will be assinged letter
    %%%%%%%%%%%%%%%%%%%%%%
    %%C
    %\draw let \p1=($(alphaas)-(alphabs)$), \p2=(rl), \p3=(alphabs) in
    %($(.5*\x2, .5*\y1+\y3)$) node {$C$};
    %%D
    %\draw let \p1=($(alphaas)-(alphabs)$), \p2=($(cfas)-(rl)$),
    %\p3=(alphabs), \p4=(rl) in
    %($(.5*\x2+\x4, .5*\y1+\y3)$) node {$D$};
    %%E
    %\draw let \p1=(alphabs), \p2=(rl) in
    %($(.5*\x2, .5*\y1)$) node {$E$};
    %%F
    %\draw let \p1=(alphabs), \p2=($(cfas)-(rl)$), \p3=(rl) in
    %($(.5*\x2+\x3, .5*\y1)$) node {$F$};
    %%G
    %\draw let \p1=(alphabs), \p2=($(cfbs)-(cfas)$), \p3=(cfas) in
    %($(.5*\x2+\x3, .5*\y1)$) node {$G$};
    %%%%%%%%%%%%%%%%%%%%%%% Name of axis
    \draw[] let \p1=(y), \p2=(x) in
    (\p1) node[left]{$P(\lambda)$};
    \draw[] let \p1=(x), \p2=(x) in
    (\p1) node[above]{$\lambda$};
    \draw [fill=black] (0.2*5,0.8*5) circle (2pt);
    \draw [fill=black] (0.9*5,0.2*5) circle (2pt);
  \end{tikzpicture}
  \hfill
  \caption{}
\end{subfigure}
\hfill
\begin{subfigure}[a]{0.48\textwidth}
  \centering
  \begin{tikzpicture}[scale=1.2]
    %Draw axis
    \coordinate (y) at (0,5);
    \coordinate (x) at (5,0);
    \draw[axis] (y) -- (0,0) --  (x);
    %Important coordinates. These are used in both figures and can be
    %moved to a seperate settings files
    %% These coordinates deside where boxes start on the y axis
    \coordinate (alphaas) at ($0.35*(y)$);
    \coordinate (alphabs) at ($0.65*(y)$);
    %% These coordinates deside where boxes end on the x axis
    \coordinate (cfas) at ($.2*(x)$);
    \coordinate (cfbs) at ($.9*(x)$);
    %These sets the interest rate lines 
    \coordinate (rl) at ($(cfas)+.45*(x)$);
    \coordinate (rla) at ($(rl)-(x)$);
    \coordinate (rlb) at ($(rl)+(x)$);
    %%%%%%%%%%%%%%%%%%%%%%
    %We makes some boxes and connect some coordinates
    %%%%%%%%%%%%%%%%%%%%%%
    %First, let us draw a line connecting alpha^\A_s og NV^\A_s
    \draw[important line] let \p1=(alphaas), \p2=(cfas) in 
    (\p1) node[left] {$\frac{1}{3}$} (\x2, \y1) -- (\p2)
    node[below] {$-1$};
    %Second, let us connect alpha^\B_s og NV^B_s
    \draw[] let \p1=(alphabs), \p2=(cfbs) in 
    (\p1) node[left] {$\frac{2}{3}$}  (\x2, \y1) -| (\p2)
    node[below] {$1$};
    %A line seperating the boxes.
    \draw[help lines,color=red] let \p1=(rl), \p2=(y) in
    (\p1) node[below] {$\frac{1}{3}$} -- (\x1, \y2);
    %%%%%%%%%%%%%%%%%%%%%%
    %The small boxes will be assinged letter
    %%%%%%%%%%%%%%%%%%%%%%
    %%C
    %\draw let \p1=($(alphaas)-(alphabs)$), \p2=(rl), \p3=(alphabs) in
    %($(.5*\x2, .5*\y1+\y3)$) node {$C$};
    %%D
    %\draw let \p1=($(alphaas)-(alphabs)$), \p2=($(cfas)-(rl)$),
    %\p3=(alphabs), \p4=(rl) in
    %($(.5*\x2+\x4, .5*\y1+\y3)$) node {$D$};
    %%E
    %\draw let \p1=(alphabs), \p2=(rl) in
    %($(.5*\x2, .5*\y1)$) node {$E$};
    %%F
    %\draw let \p1=(alphabs), \p2=($(cfas)-(rl)$), \p3=(rl) in
    %($(.5*\x2+\x3, .5*\y1)$) node {$F$};
    %%G
    %\draw let \p1=(alphabs), \p2=($(cfbs)-(cfas)$), \p3=(cfas) in
    %($(.5*\x2+\x3, .5*\y1)$) node {$G$};
    %%%%%%%%%%%%%%%%%%%%%%% Name of axis
    \draw[] let \p1=(y), \p2=(x) in
    (\p1) node[left]{$P(\lambda)$};
    \draw[] let \p1=(x), \p2=(x) in
    (\p1) node[above]{$\lambda$};
    \draw [fill=black] (0.2*5,0.35*5) circle (2pt);
    \draw [fill=black] (0.9*5,0.65*5) circle (2pt);
  \end{tikzpicture}
  \hfill
  \caption{ }
\end{subfigure}
\caption{Representación gráfica del mapeo de resultados al medir el observable $\sigma_z\otimes \sigma_x$, con valores propios $\lambda=\pm 1$.\textbf{ (a)} La imagen indica la probabilidad de obtener cada una de las salidas del estado inicial $\rho$. La línea azul punteada, indica el valor esperado del observable.\textbf{ (b)} La gráfica indica la probabilidad de obtener cada una de las salidas en el estado $\mathcal{F}_{2\text{p}}(\rho)$. La línea roja punteada, indica el valor esperado del observable. En esta imagen se toma el valor de la probabilidad  $p$  de que las partículas se intercambien como $\frac{1}{4}$.}\label{valor-esperado-imagen}\source{elaboración propia.}
\end{figure}


  
  
% \rrnote{Justificar la inclusión del valor esperado en el trabajo. Me gustaría
% explicar por qué es útil  y para qué se usará.}





Es importante destacar que el valor esperado es una medida de tendencia central
que proporciona una característica del mapeo que asocia las salidas de la
medición, con la probabilidad de ocurrencia de cada una de ellas. En otras
palabras, el valor esperado permite caracterizar distribuciones de
probabilidad. El valor esperado es útil porque la información capturada por un
estado cuántico puede ser representada por una distribución de probabilidad.
Aunque, el valor esperado solo proporciona una idea aproximada de dicha
distribución, es una condición necesaria para identificarla. 

 En este trabajo se hará uso de distintas herramientas, en las que el valor
esperado es indispensable. En primer lugar, el valor esperado servirá como
punto de partida para implementar las medidas POVM, las cuales se detallarán en
la siguiente sección. Además, se empleará para corroborar que distintas
interpretaciones de los instrumentos cuánticos no describan mapeos de
resultados distintos.  En este contexto, el valor esperado, es una medida
necesaria que se empleará como primera instancia para comprobar que dichos
instrumentos sean equivalentes.

\cpnote{Aca voy}
% }}}
% }}}
\section{Medidas POVM y operadores de Kraus para mediciones difusas}\label{sec:Cap2POVM_Kraus_mediciones_difusas} % {{{

\rrnote{Esqueleto Intro: Los POVM y los operadores de Kraus se hablaron en el marco teórico y se verán en las implementciones. En esta parte se mencionará como utilizar los operadores POVM y los operadores de Kraus conjuntamente para describir mediciones imperfectas y específicamente en mediciones difusas.}

\rrnote{Esqueleto: En este parrafo se menciona que en una medicion ideal los
operadores que describen completamente  una medicion  son los operadores de
proyeccion, sin embargo en operaciones difusas no son suficientes. Dar la razon
por que los operadores de proyeccion son insuficientes para describir
completamente una medición difusa.}
\cpnote{Expande esto aca dentro del esuqeleto. Caules son las razones?}

\rrnote{Esqueleto:En este parrafo, describir de forma corta la utilidad de las medidas POVM\@.}
\cpnote{Cual es la utilidad? porque aca?}

\rrnote{Esqueleto: Explicar la relevancia de los operadores de Kraus y su
relación con las medidas POVM\@. }
\cpnote{Porque aca? Que son?}

\rrnote{Esqueleto: Finalmente, describir como se usará el operador difuso y el
valor esperado para obtener las medidas POVM y los  operadores de Kraus, en un
sistema de dos partículas para simplificar la explicación (EN el siguiente
capítulo se extiende a un sistema más complejo).}
\cpnote{Esto va a tomar mas que un parrafo. Expande mas, o dime mas
explicitamente que quieres decir}. 




% }}}
\section{Instrumentos cuánticos}\label{sec:cap2instrumentos-cuanticos} % {{{

\rrnote{Esqueleto: En esta sección se introduce el concepto de instrumento cuántico. Se explica la utilidad de ellos además de detallar en cómo se calcula el valor esperado del instrumento y la razón de utilizarlos. Además se especifican las dos alternativas que se usarán}

\rrnote{Esqueleto: En este párrafo simplemente se define y explica qué es un instrumento cuántico.}

\rrnote{Esqueleto: Exponer cuál es la  utilidad de los instrumentos, y como obtener los elementos que describen completamente una medición.}

\rrnote{Esqueleto: Con base en lo anterior, me gustaria decir por qué es util hablar del valor esperado de los instrumentos cuánticos.}


\rrnote{Esqueleto: Presentar los dos  instrumentos para mediciones difusas, explicando el sentido que hay detrás de ellos. En un primer párrafo explicar el primer instrumento, haciendo hincapié en la idea de aplicar el operador difuso en el sistema cuántico. Y en un segundo párrafo explicar el segundo instrumento y detallar por que usar el operador difuso en el sistema clásico.}
% }}}



