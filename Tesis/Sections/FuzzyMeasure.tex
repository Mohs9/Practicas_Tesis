
\chapter[MEDICIONES DIFUSAS ]{2. MEDICIONES DIFUSAS}
% Intro {{{

\section{Introducción}
 En este capítulo se presentan las mediciones difusas de sistemas cuánticos las que se pueden identificar partículas individuales, sin embargo siempre hay una probabilidad de identificarlas erróneamente. Estas detecciones imperfectas se exponen en la referencia {\cite{Pineda_2021}}. %También se utiliza el lenguaje de las operaciones cuánticas por medio de los operadores de Kraus para poder describir las mediciones en los sistemas cuánticos de dos partículas.

La estructura de este capítulo se presenta a continuación. 
En la siguiente sección se establece el problema de las mediciones difusas para un sistema cuántico. 
%Luego, en la tercera sección, se discutirán algunos conjuntos de operadores de Kraus que describan los efectos de las mediciones en el estado de entrada del sistema para dos partículas y que proporcionen la probabilidad de cada resultado posible en la medición para cualquier estado inicial.  Además se exponen dos instrumentos cuánticos, los cuales son una herramienta conveniente para describir completamente un medición difusa sintetizadamente. Finalmente, en la cuarta sección se presentan algunos ejemplos de las mediciones y su descripción utilizando los operadores en la tercera sección.

% }}}

\section{Mediciones no ideales}

\rrnote{ En esta sección me gustaria escribir dos parráfos. En el primero mencionar que se refiere una medición no ideal en sistemas cuánticos. En esl segundo describir un ejemplo de una medicón no ideal.}


\rrnote{ En los modelos téoricos se asumen condiciones idealizadas y simplificadas de las mediciones sin embargo en la realidad existen situaciones en las que el proceso de medición no cumple con estas simplificaciones. Explicar factores que pueden introducir imperfecciones e incertidumbre en la medición.Revisar literatura acerca de mediciones no ideales}

\rrnote{ A. Peres menciona un ejemplo de <<Fuzzy Measurements>> sin embargo no es el enfoque que se le dará en este trabajo.}

\section{Medición difusa en sistemas cuánticos}
\rrnote{En esta sección, se pretende discutir cómo se pueden entender las mediciones difusas, y expandir/ilustrar un ejemplo concreto como el de la cadena de iones.}

    



\rrnote{Explicar el resultado experimental de una medición difusa. Ejemplificando el  experimento de una cadena iones en la que se  realiza una medición con un detector imperfecto, el cual realiza una señal, sin embargo  no es posible determinar exactamente de qué ion proviene.  Se puede plantear las condiciones en las que se puede cuantificar una medición difusa. }



\rrnote{Realizar una explicacion de lo que sucede en una medición de dos partículas con observable factorizable. Mencionar a que nos referimos con un operador factorizable. Simplificar el sistema a dos partículas y mencionar que se obtendría al quere medir un observable.}


\subsection{Operador difuso}
\rrnote{Presentar el operador difuso, operadores swap.}


\rrnote{Escribir detalladamente el resultado generalizado de lo anterior y el operador difuso, para más partículas  (ecuación 2 del paper)}


\subsection{Valores esperado en sistemas cuánticos}

\rrnote{Explicar el resultado del valor esperado en la medición de dos partículas (ecuación 1 del paper).}


\rrnote{Incluir imágenes de algún ejemplo de una medición difusa en un sistema específico. Tomar en cuenta consideraciones experimentales. Por ejemplo imágenes que ilustren la medición del observable $\sigma_z\otimes \sigma_x$ en un sistema de dos partículas. Con algún diagrama de cajas en las que se realiza una medición del observable particular. Comparar  una medición ideal en un sistema de dos partículas. Luego, en una segunda imagen agregar un diagrama que represente la matriz swap y obtener salidas intercambiadas  con cierta probabilidad.}

\section{Medidas POVM y operadores de Kraus para mediciones difusas}

\rrnote{ El formalismo de las mediciones ideales debe modificarse para considerar mediciones que proporcionan información parcial sobre el sistema. Primero, se debe encontrar la probabilidad del resultado $m$ de la medición, y segundo, se debe formular la regla para el estado después de la medición.}

\rrnote{En esta parte se mencionará como utilizar los operadores POVM y los operadores de Kraus en mediciones imperfectas.}


\section{Instrumentos cuánticos}\label{sec:instrumentos-cuanticos} % {{{

\rrnote{Aclarar la utilidad de los instrumentos cuánticos para las mediciones cuánticas}




\rrnote{ Detallar cómo se realiza el cálculo del valor esperado de un instrumento cuántico}


\rrnote{Presentar los instrumentos para mediciones difusas, explicando el sentido que hay detrás de ellos.}