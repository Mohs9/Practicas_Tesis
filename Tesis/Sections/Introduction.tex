\chapter[INTRODUCCIÓN]{Introducción}

En el transcurso del último siglo, la medición ha destacado como uno de los campos de estudio más notables dentro del ámbito de la mecánica cuántica. La medición es un tipo de evolución que puede experimentar un sistema cuántico, y que permite recuperar información clásica de un estado cuántico, y es la forma en la cual la información puede ser interpretada {\cite{wilde2011classical}}. El esfuerzo de muchos investigadores por comprender el proceso de medición en sistemas cuánticos ha llevado al desarrollo de instrumentos y técnicas que se utilizan ampliamente en ramas de investigación aplicadas {\cite{de2019introduction}}. %En este trabajo se ofrece una introducción a algunas de las herramientas básicas utilizadas para analizar el proceso de medición de manera amplia y una discusión de un tipo de medición no ideal, la medición difusa.


 La mayoría de literatura introductoria a la mecánica cuántica suele enfocarse en las mediciones proyectivas en sistemas ideales. En este enfoque, es posible seleccionar una base del observable que se desea medir y al hacerlo se encontrará que el sistema se proyecta a uno de los estados de la misma. Aunque pudo haber estado en cualquier estado antes de la medición {\cite{jacobs2014quantum}}. Este estado es aleatorio y la probabilidad de que se encuentre en cualquiera de estos estados está dictada por la regla de Born. Naturalmente, en la práctica las mediciones que se realizan son imperfectas, es por ello que es necesario emplear un formalismo más extenso, tal como el de las operaciones cuánticas y las mediciones POVM o generalizadas. Dichas herramientas pueden proporcionar una descripción total de las mediciones no ideales. %la regla que describe el estado posterior del sistema y la regla para describir las estadísticas de la medición, respectivamente.



 Este trabajo se enfoca en \textit{mediciones difusas}, las cuales son un tipo concreto de mediciones imperfectas. Estas ocurren cuando se lleva a cabo el proceso de medición en un sistema de múltiples partículas y hay una posibilidad de error en la identificación de las partículas. Aun así, es posible cuantificar los resultados de esta medición utilizando las herramientas formales mencionadas previamente. Estas detecciones imperfectas se exponen en la referencia {\cite{Pineda_2021}}.% El problema de las mediciones difusas se estudiará inicialmente, en el sistema más simple posible, un sistema de dos partículas. Posteriormente se generalizan los resultados obtenidos para sistemas de varias partículas.


 En los siguientes  tres capítulos se expone un marco que facilita la comprensión del proceso de medición, diseñado específicamente para mediciones difusas. A continuación, se ofrece una breve descripción de la estructura de esta tesis y el contenido de los distintos capítulos. 

 En el capítulo {\ref{cap1:MarcoTeorico}} se aborda un marco conceptual que contiene las herramientas más básicas en la descripción de mediciones en sistemas cuánticos. En primer lugar, se exponen los operadores de densidad los cuales permiten expresar al estado de un sistema cuántico de una forma general. %que considera de manera unificada las incertidumbres cuánticas, aún cuando el estado se conoce por completo y las incertidumbres clásicas, derivadas de la falta de conocimiento {\cite{jacobs2014quantum}}. 
 Por otro lado, se presenta el formalismo de las mediciones POVM y se aborda el tema de las operaciones cuánticas, las cuales desempeñan un papel importante en la configuración de la evolución que experimenta un sistema cuántico debido al proceso de medición.
%Estas matrices fueron introducidas de forma independiente por John von Neumann y Lev Landau {\cite{de2019introduction}}


 En el capítulo {\ref{cap2:MedicionesDifusas}}, se presenta el problema de las mediciones difusas de manera formal.  En este capítulo se describe los elementos clave del problema que se utilizarán como marco de referencia en el resto de la tesis, los cuales son el valor esperado de las mediciones difusas y el operador difuso.  Asimismo se introduce los instrumentos cuánticos, los cuales constituyen una nueva herramienta, que describe enteramente las mediciones.


El capítulo {\ref{cap3:resultados}}, aborda los resultados obtenidos utilizando como base las herramientas desarrolladas en los capítulos {\ref{cap1:MarcoTeorico}} y {\ref{cap2:MedicionesDifusas}}, para sistemas simples que estén conformados por dos partículas. Posteriormente, se generalizan los resultados y se formulan en un sistema de $N$ partículas, tomando en cuenta observables no factorizables.

 %  





