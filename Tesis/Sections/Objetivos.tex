
\chapter[OBJETIVOS]{Objetivos}
%\cpnote{Esto los copiaste del documento anterior?}
\section*{Objetivo General}


Se le llama medición difusa al proceso en el cual al medir un sistema de varias
partículas, existe una posibilidad de identificar erróneamente a las
partículas. El objetivo general de este proyecto es describir completamente
dichas mediciones. Para realizar la descripción es necesario un mapeo que
proporcione la probabilidad obtener las posibles salidas de una medición
difusa, así como el estado posterior a la medición.   

%En un sistema de varias partículas, una medición difusa es aquella en la que existe una posibilidad de identificar erróneamente a las partículas. El objetivo general de este proyecto es establecer los operadores que describan completamente una medición difusa. Para describir completamente una medición es necesario contar con un mapeo de salidas que proporcione la probabilidad de las posibles salidas de una medición difusa, así como el estado posterior a la medición. 

% }}}
\section*{Objetivos Específicos} % {{{
\begin{enumerate}
%\item Comprender las medidas POVM y su descomposición como operadores de Kraus, así como las operaciones cuánticas como la representación de suma de operadores de Kraus. 


\item Estudiar los operadores de Kraus  y las medidas POVM que describan completamente las mediciones difusas para sistemas de dos partículas.

\item 	Examinar instrumentos cuánticos equivalentes que describan completamente mediciones difusas en un sistema de dos partículas.

%\item Ejemplificar instrumentos cuánticos que describan correctamente las mediciones difusas.

%\item Examinar las condiciones en las que ciertos instrumentos cuánticos describan completamente las mediciones cuánticas.

\item Generalizar los operadores de Kraus que describan completamente las mediciones difusas en sistemas de $n$ partículas.

\item Analizar mediciones difusas con observables no factorizables en sistemas de $n$ partículas.

\item Crear un programa que exhiba el mapa de probabilidades y el estado resultante luego de la medición, a partir de un estado inicial y un observable dado.
\end{enumerate}