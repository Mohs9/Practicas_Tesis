\chapter[OPERADORES DE DENSIDAD Y OPERACIONES CUÁNTICAS]{1. OPERADORES DE DENSIDAD Y OPERACIONES CUÁNTICAS}
\section{Introducción}\label{cap:IntroduccionMarcoteorico} % {{{

Con el propósito de lograr los objetivos de esta tesis primero será
indispensable examinar el formalismo de algunas herramientas tales como el
operador densidad, las medidas POVM y las operaciones cuánticas. El operador de
densidad también conocido como <<matriz de densidad>> describe el estado
cuántico de un sistema físico y generaliza los vectores de estado, puesto que
éstos pueden representar estados mixtos. Este operador contiene todas las
propiedades estadísticas de un sistema cuántico, incluso cuando no es posible
describir el sistema mediante un estado puro. Por otro lado, las medidas POVM,
también llamadas medidas generalizadas, son apropiadas debido a que consideran
la interacción del sistema con su entorno. Estas medidas son especialmente
útiles porque proporcionan las estadísticas de medición. Por último, el
formalismo de las operaciones cuánticas describe los efectos de la medición en
un sistema cuántico y representa la descripción de los cambios  de estados
discretos, esto es transformaciones entre el estado inicial $\rho$ y el estado
final $\rho'$, sin necesidad de referenciar al paso del tiempo
{\cite{nielsen_chuang_2010}}.  


En la sección {\ref{sec:Cap1:OpDensidad}}, se presentan la definición y las propiedades del
operador de densidad y se reformulan los postulados de la mecánica cuántica
utilizando este operador. Además se expone el operador de densidad reducido,
una de las aplicaciones más importantes del operador de densidad. En la sección {\ref{sec:Cap1:MedidaPOVM}}, se hablará de un caso particular de las mediciones, las mediciones proyectivas, así como se discutirá acerca de los operadores POVM\@. Finalmente, en la sección {\ref{sec:Cap1:OpCuanticas}} se aborda el formalismo de las operaciones cuánticas que permitirá familiarizarse con la teoría básica de las mismas e involucra a los operadores conocidos como operadores de Kraus. Asimismo se presenta un teorema importante sobre la libertad unitaria en la representación de suma de
operadores para una operación cuántica.
% }}}
\section{Operador de densidad}\label{sec:Cap1:OpDensidad}% {{{
% Intro {{{
El lenguaje de los operadores de densidad proporciona un  medio conveniente
para describir los sistemas cuánticos, tales que su estado no es completamente
conocido. Cuando un sistema cuántico está representado por un único vector de
estado $|\psi_j\rangle $, completamente conocido, se le llama estado puro. No
obstante, es posible que un sistema esté en un conjunto estadístico de
distintos vectores puros de estado, por esta razón se puede enunciar la
siguiente definición.

\begin{definition}[\textbf{Operador de densidad}] El operador de densidad
$\rho$ correspondiente al ensamble de estados puros $\{p_j,|\psi_j \rangle \}$
es definido como {\cite{wilde2011classical}}
  	\begin{equation}{\label{defdensityoperator}}
  		\rho=\sum_{j}p_j|\psi_j\rangle \langle \psi_j|.
  	\end{equation}
\end{definition}
A menudo se utiliza el término matriz de densidad para referirse al operador de
densidad, en la práctica estos dos términos se utilizan indistintamente. Esta
formulación, es equivalente a la aproximación del vector de estado, sin embargo
a veces es mucho más fácil acercarse a problemas desde este nuevo punto de
vista.
 
Considérese una medición en un ensamble, de algún observable $\mathcal{O}$. Una
pregunta importante sería, ¿cuál es el valor promedio de $\mathcal{O}$ cuando
se reproduce un número muy grande de mediciones? La respuesta está dada por el
promedio $\langle \mathcal{O} \rangle$ del ensamble, el cual se define por
{\cite{sakurai2017modern}} \begin{equation}
 	\label{expectationvalue}
 	 \begin{split}
 		\langle \mathcal{O} \rangle &= \sum_{i}p_i \langle\psi_i|\mathcal{O}|\psi_i\rangle\\
 		&=\sum_i\sum_{j}\sum_k p_i \langle\psi_i||\phi_j\rangle \langle \phi_j|\mathcal{O}|\phi_k\rangle \langle \phi_k||\psi_i\rangle\\
 		&=\sum_{j}\sum_k  \langle \phi_k| \left(\sum_{i} p_i |\psi_i\rangle  \langle\psi_i|\right)|\phi_j\rangle \langle \phi_j|\mathcal{O}|\phi_k\rangle, 
 	\end{split}
\end{equation} 
donde $\{|\phi_j\rangle \}$ es una base ortonormal 
del espacio de Hilbert del
sistema. La matriz que se encuentra dentro del paréntesis en última línea en la ecuación ({\ref{expectationvalue}}) es una motivación para  definir el operador de densidad  en la forma de la ecuación
({\ref{defdensityoperator}}). Notar que $\langle \phi_k| \rho|\phi_j\rangle$ son los elementos de la matriz $\rho$, escrita en la base $|\phi_j\rangle$. Por lo tanto, el valor promedio del observable es {\cite{sakurai2017modern}}\begin{equation}
	\label{expectationvalue_traza}
	\begin{split}
		\langle \mathcal{O} \rangle_{\rho} &=\sum_{j}\sum_k \langle \phi_k|\rho|\phi_j\rangle \langle \phi_j|\mathcal{O}|\phi_k\rangle\\
		&= \rm{tr}(\rho \mathcal{O}).
	\end{split}
\end{equation}

\begin{comment}

Además las mediciones se pueden describir fácilmente utilizando matrices de
densidad. Si se realiza una medición descrita por lo operadores $M_m$ y con un
estado inicial $|\psi_i\rangle$, luego la probabilidad de obtener el resultado
$m$, dado el estado $i$ es $p(m|i)=\langle
\psi_i|M_m^{\dagger}M_m|\psi_i\rangle = \tr(M_m^{\dagger} M_m |\psi_i\rangle
\langle\psi_i|)$. Por la ley de probabilidad total, la probabilidad de
obtener un resultado $m$ es {\cite{nielsen_chuang_2010}}
\begin{equation}
	\label{probabilidad_m}
	\begin{split}
	p(m) &= \sum_{i}p(m|i)p_i\\
	&=\sum_i p_i\tr(M_m^{\dagger} M_m |\psi_i\rangle \langle\psi_i|)\\
	&=\sum_{i}\tr(M_m^{\dagger} M_m \rho)
	\end{split}
\end{equation} y el estado después de la medición será $|\psi_i^m\rangle=\frac{M_m|\psi_i\rangle}{\sqrt{\langle\psi_j|M_m^{\dagger}M_m|\psi_i\rangle}}$, por lo que el operador de densidad correspondiente al ensamble de estados $|\psi_i^m\rangle$ con probabilidades $p(i|m)$ estará dado por \[\rho_m=\sum_i p(i|m)\frac{M_m|\psi_i\rangle \langle \psi_i|M^{\dagger}}{\langle\psi_j|M_m^{\dagger}M_m|\psi_i\rangle}.\] Utilizando el teorema de Bayes, para $p(i|m)=\frac{p(m|i)p_i}{p_m}$ se obtiene finalmente que el operador de densidad $\rho_m$ es {\cite{nielsen_chuang_2010}} \begin{equation}
	\rho_m=\sum_i p_i \dfrac{M_m|\psi_i\rangle \langle \psi_i|M_m^{\dagger}}{\tr(M^{\dagger}_m M_m \rho )}=\dfrac{M_m \rho M_m^{\dagger}}{\tr (M^{\dagger}_m M_m\rho )}.
\end{equation}


Ahora si por alguna razón, el registro de la medición se pierde. Se tendría un
sistema cuántico en el estado $\rho_m$ con probabilidad $p(m)$, pero ya no se
conocería el valor real de $m$. El estado del sistema cuántico estaría descrito
por el operador densidad siguiente \begin{equation}
	\rho'=\sum_m p(m)\rho_m=\sum_m\tr(M^{\dagger}_m M_m\rho )\dfrac{M_m\rho M_m^{\dagger}}{\tr (M^{\dagger}_m M_m \rho )}=\sum_{m} M_m\rho M_m^{\dagger}.
\end{equation}


Supóngase que la evolución de un sistema cerrado es descrito por el operador
$U$. Si el sistema inicial se encuentra en el estado $|\psi_i\rangle$ con
probabilidad $p_i$ luego de la evolución, el sistema se encontrará en el estado
$U|\psi_i\rangle$ con probabilidad $p_i$. Luego, la evolución del operador de
densidad está descrita por
{\cite{nielsen_chuang_2010}}\[\rho=\sum_{i}p_i|\psi_i\rangle \langle
\psi_i|\xrightarrow{U}\rho'=\sum_{i}Up_i|\psi_i\rangle \langle
\psi_i|U^{\dagger}=U\rho U^{\dagger}.\]


Hasta el momento se puede ver con estas motivaciones que los postulados básicos
de la mecánica cuántica relacionados con la medición y la evolución unitaria
pueden ser reformulados en el lenguaje de los operadores de densidad. En la
siguiente sección se profundiza esta reformulación de los postulados.

\end{comment}
% }}}
\subsection{Propiedades del operador de densidad}\label{subsec:postulates} % {{{

En esta sección se desarrolla algunas de las características y propiedades de
los operadores de densidad. Asimismo se presenta la formulación de los
postulados de la mecánica cuántica. 

Para iniciar se enuncia la siguiente proposición y el siguiente teorema. 

\begin{proposition}El operador de densidad es un operador hermítico.\end{proposition}


\begin{proof}
	\begin{equation}
		\begin{split}
			\rho^\dagger&={\left(\sum_{i} p_i|\psi_i\rangle \langle \psi_i|\right)}^{\dagger}\\
			&=\sum_{i} p_i {\left(|\psi_i\rangle \langle\psi_i|\right)}^{\dagger}\\
			&=\sum_{i} p_i |\psi_i\rangle \langle\psi_i|\\
			&=\rho.
		\end{split}
	\end{equation}
	
\end{proof}

De acuerdo a Nielsen y Chuang {\cite{nielsen_chuang_2010}}:
\begin{theorem}[\textbf{Caracterización del operador de densidad}] Un operador
$\rho$ es un operador de densidad, asociado a algún ensamble $\{p_i,
|\psi_i\rangle\}$si y solo si este satisface las siguientes condiciones:
\begin{enumerate}
	\item $\rho$ tiene una traza igual a uno.
	\item $\rho $ es un operador \textit{positivo semidefinido}.
\end{enumerate}	
\end{theorem}

Ahora se procede a demostrar formalmente este teorema.
\begin{proof}
	Primero se supondrá que $\rho $ es un operador de densidad por lo que debe cumplir con la definición ({\ref{defdensityoperator}}), al computar la traza se obtiene que \begin{equation*}
		\begin{split}
			{\rm tr(\rho)}&=\tr\left(\sum_{i} p_i|\psi_i\rangle \langle \psi_i|\right) \\
			&=	\sum_{i}p_i \tr(|\psi_i\rangle \langle \psi_i|)\\
			&=\sum_{i}p_i \sum_j\langle \psi_j|\psi_i\rangle \langle \psi_i|\psi_j\rangle\\
			&=\sum_{i}\sum_j p_i \delta_{ij} \delta_{ij}\\
			&=\sum_i p_i=1.\\
		\end{split}
	\end{equation*}

Ahora para la segunda condición, se toma un vector de estado arbitrario $|\varphi \rangle$ \begin{equation*}
	\begin{split}
	\langle \varphi |	\rho|\varphi \rangle&=\sum_{i}p_i\langle \varphi |\psi_i\rangle \langle \psi_i|\varphi \rangle \\
	&=\sum_{i}p_i| \langle \psi_i|\varphi \rangle|^2 \ge 0.\\
	\end{split}
\end{equation*}
La desigualdad se sigue porque cada $p_i$ es una probabilidad y por lo tanto no es negativa. Por la definición de operadores positivos semidefinidos, $\rho$ cumple la segunda condición.


Ahora se debe mostrar el converso del enunciado anterior. Suponiendo $\rho$ es un operador que cumple las dos condiciones, entonces se debe demostrar que es un operador de densidad.

Dado que $\rho$ es un operador positivo semidefinido entonces tiene una descomposición espectral dada por la siguiente ecuación {\cite{nielsen_chuang_2010}} \[\rho=\sum_i \lambda_i |i\rangle \langle i|,\] con $|i\rangle$ vectores ortogonales y $\lambda_i$ son valores propios reales no negativos, de la matriz $\rho$. Además satisface que la traza es uno 
\[\tr(\rho)=1=\sum_i\lambda_i.\]

Luego, un sistema estado $|i\rangle$ con una probabilidad $\lambda_i$ tendrá un
operador asociado $\rho$. Esto significa que, el ensamble $\{\lambda_i,
|i\rangle\}$  corresponde al operador $\rho$ que cumple con
({\ref{defdensityoperator}}).


\end{proof}



Este teorema  permite obtener una definición equivalente del operador de densidad y con ello es posible reformular los postulados de la mecánica cuántica utilizando este operador. Nielsen y Chuang {\cite{nielsen_chuang_2010}} presentan la siguiente reformulación.


\setlength{\leftskip}{1cm}

 \textbf{Postulado 1:} Cualquier sistema físico tiene asociado un espacio de Hilbert $\mathcal{H}$, conocido como el \textit{espacio de estado} del sistema. El sistema está completamente descrito por su \textit{operador de densidad}, el cual es un operador $\rho$ positivo semidefinido con traza uno, actuando en el espacio de estado del sistema. Si un sistema cuántico está en el estado $\rho_i$ con una probabilidad $p_i$, entonces el operador de densidad para el sistema es $\sum_{i}p_i\rho_i$.


\textbf{Postulado 2:} La evolución de un sistema cuántico cerrado está descrita por una \textit{transformación unitaria}. Esto es, el estado $\rho$ del sistema en el tiempo $t_1$ está relacionado con el estado $\rho'$ del sistema en el tiempo $t_2$ por un operador unitario $U$ que depende solamente de $t_1$ y $t_2$, \begin{equation}\label{postulado 2}
\rho'=U\rho U^{\dagger}.
\end{equation}


\textbf{Postulado 3:} Las mediciones cuánticas están descritas por una
colección $\{M_m\}$ de \textit{operadores de medición}. Estos operadores actúan
en el espacio de estado del sistema que se está midiendo. El índice $m$ se
refiere a las salidas de la medición que pueden ocurrir en el experimento. Si
el estado del sistema cuántico es $\rho$ inmediatamente antes de la medición,
luego la probabilidad de obtener el resultado $m$ está dada por
\begin{equation}\label{probaility3postulate}
	p(m)=\tr(M^\dagger M \rho)
\end{equation} y el estado del sistema después de la medición es  \begin{equation}\label{state3postulate}
	\rho_m'=\dfrac{M_m\rho M_m^\dagger}{\tr(M_m^\dagger M_m \rho)}.
\end{equation}

Los operadores de medición satisfacen la relación de completitud \begin{equation}\label{completitud3postulate}
	  	\sum_m M_m^\dagger M_m=\mathds{1}.
\end{equation}

\textbf{Postulado 4:} El espacio de Hilbert de un sistema físico compuesto es el producto tensorial de los espacios de Hilbert individuales de cada uno de los sistemas que componen al sistema total. Es decir, si el sistema total se compone de $N$ subsistemas, entonces \begin{equation}\label{Htotal4postulado}
	\mathcal{H}_{\text{total}}=\mathcal{H}_1\otimes \mathcal{H}_2\otimes \cdots \otimes \mathcal{H}_N.
\end{equation}
 Más aún, si el $i$-ésimo sistema, se encuentra en el
estado $\rho_i$, entonces el estado del sistema total será
\begin{equation}\label{rhototal4postulado}
	\rho_{\text{total}}=\rho_1\otimes \rho_2 \otimes \cdots \otimes \rho_N.
\end{equation}


\setlength{\leftskip}{0pt}

Con esta reformulación de los postulados de la mecánica cuántica se tiene la
ventaja que es más fácil trabajar en la descripción de sistemas cuánticos cuyos estados son mixtos y la descripción de subsistemas de un sistema cuántico compuesto. Ahora, es necesario discutir estos conceptos, nuevas definiciones y hechos sobre operadores de densidad. 

%Un sistema cuántico cuyo estado no es conocido exactamente, se dice que se encuentra un {estado mixto}\footnote{En algunas referencias, se  usa el término <<estado mixto>> para incluir ambos estados cuánticos puros y mixtos. El origen de este uso puede ser que no se asume necesariamente que un estado es puro. Además, el término <<estado puro>> se usa a menudo en referencia a un vector de estado $|\psi\rangle $, para distinguirlo de un operador de densidad	$\rho$.};  es una mezcla estadística de los diferentes estado puros en el ensamble de $\rho$ {\cite{nielsen_chuang_2010}}. 

Antes de pasar a la siguiente sección, es indispensable discutir qué clase de
ensambles pueden dar una matriz de densidad particular y cuándo dos conjuntos
de vectores $\{|\psi_i\ra\}$ y$ \{|\phi_j\ra \}$ generan el mismo operador de
densidad. La respuesta a estas interrogantes tiene muchas aplicaciones en
información y computación cuántica. Para ello se presenta el siguiente
teorema y su demostración que enuncian Nielsen y Chuang
{\cite{nielsen_chuang_2010}}. 
\begin{theorem}[\textbf{Libertad unitaria en el ensamble para matrices de densidad}]
Los conjuntos $\{|\psi_i\ra\}$ y $ \{|\phi_j\ra \}$, no necesariamente
normalizados, generan la misma matriz de densidad si y solo si
\begin{equation}\label{teorema2.4}
|\psi_i\ra= \sum_{j}u_{ij}|\phi_j\ra,
\end{equation} donde $u_{ij}$ es una matriz unitaria compleja, con índices $i$ y $j$. Se completa con vectores 0 adicionales en caso uno de los conjuntos tenga menos elementos.
\end{theorem}

\begin{proof}
	
	Primero, se supone que $|\psi_i\ra= \sum_{j}u_{ij}|\phi_j\ra$ para alguna matriz $u_{ij}$. Luego \begin{equation*}
		\begin{split}
			\sum_{i}|\psi_i\ra \la \psi_i|&=\sum_{ijk} u_{ij}u_{ik}^*|\phi_j\ra \la \phi_k|\\
			&=\sum_{jk}\left(\sum_i  u_{ki}^\dagger u_{ij}\right) |\phi_j\ra \la \phi_k|\\
			&=\sum_{jk}\delta_{jk}|\phi_j\ra \la \phi_k|\\
			&=\sum_{j} |\phi_j\ra \la \phi_j|,\\
	\end{split}
	\end{equation*}	
	con lo que se prueba que $\{|\psi_i\ra\}$ y$ \{|\phi_j\ra \}$ generan el mismo operador. Notar que para estado normalizados $\{|\tilde{\psi}_i\ra\}$ y$ \{|\tilde{\phi}_j\ra \}$ y distribuciones de probabilidad $p_i$ y $q_j$, si $\sqrt{p_i}|\tilde{\psi}_i\ra=\sum_{j} u_{ij} \sqrt{q_j}|\tilde{\phi}_j\ra$ se obtendrá el mismo operador de densidad $\rho=\sum_i p_i |\tilde{\phi}_i\rala\tilde{\phi}_i|= \sum_j q_j|\tilde{\phi}_j \rala\tilde{\phi}_j| $.
	
Por otro lado, se supone que
ambos conjuntos generan el mismo operador
\[Q=\sum_{i}|\psi_i\ra \la \psi_i|=\sum_{j} |\phi_j\ra \la \phi_j|.\]
	
Luego la  descomposición espectral del operador es $Q = \sum_k \lambda_k |k \rala k|$ tal que los vectores $ |k\ra$ son ortonormales, y $\lambda_k$ son estrictamente  positivos. Sea $|\psi\ra$  un vector ortonormal al espacio generado por $|\tilde{k}\ra=\sqrt{\lambda_k}|k\ra$, de ello $\la \psi|\tilde{k}\ra \la \tilde{k}|\psi\ra=0$ para todo $ k$, y luego se ve que 	\[0=\la \psi |Q|\psi\ra=\sum_{i}\la\psi|\psi_i\rala \psi_i|\psi \ra =\sum_i |\la\psi|\psi_i\ra|^2 \] y $\la\psi|\psi_i\ra=0$ para todo $i$ y todo $|\psi \ra$ ortonormal al espacio generado  por el $|\tilde{k}\ra$. Esto se sigue que cada $|\psi_i\ra $ puede expresarse como una combinación lineal de los vectores  $|\tilde{k}\ra$, $|\psi_i\ra=\sum_k c_{ik} |\tilde{k}\ra$. 
	
Como  $Q=\sum_k  |\tilde{k}\rala \tilde{k}|=\sum_{i}|\psi_i\rala\psi_i|$, se ve
que \[\sum_k |\tilde{k}\rala \tilde{k}|=\sum_{kl} \left(\sum_i
c_{ik}c_{il}^{*}\right)|\tilde{k}\rala \tilde{l}|,\] los operadores $|\tilde{k}
\rala \tilde{l}|$ es fácil ver  que son linealmente independientes y debe ser
que $\sum_i c_{ik}c_{il}^{*}=\delta_{kl}$. Esto asegura que se puedan agregar
columnas extra a $c$ para obtener una matriz $v$ tal que
$|\psi_i\ra=\sum_{k}v_{ik}|\tilde{k}\ra  $, donde se han agregado  vectores cero a la
lista de $|\tilde{k}\ra$. Similarmente, se puede encontrar una matriz unitaria
$w$ tal que $|\phi_j\ra=\sum_{k}w_{jk}|\tilde{k}\ra$. Luego $|\psi_i\ra=\sum_j
u_{ij}|\phi_j\ra$, donde  $u=vw^\dagger$ es unitaria. De manera similar se concluye que para ensambles que generan el mismo operador de densidad $\rho=\sum_i p_i |\tilde{\phi}_i\rala\tilde{\phi}_i|= \sum_j q_j|\tilde{\phi}_j \rala\tilde{\phi}_j| $ entonces se tiene una matriz unitaria $u$ tal que  $\sqrt{p_i}|\tilde{\psi}_i\ra=\sum_{j} u_{ij} \sqrt{q_j}|\tilde{\phi}_j\ra$.
\end{proof}

% }}}
\subsection{El operador de densidad reducido} % {{{

Una de las aplicaciones más importantes del operador de densidad es el operador
de densidad reducido. Este operador es una herramienta descriptiva para subsistemas de un
sistema cuántico y su discusión se llevará a cabo a continuación. 


Supóngase que se tiene sistemas $A$ y $B$, cuyos estados están descritos por el
operador de densidad $\rho_{AB}$. El operador de densidad reducido o local para
el sistema $A$ es definido por  
\begin{equation}
	\rho_A=\tr_B(\rho_{AB}),
\end{equation} 
donde $\tr_B$ es un mapeo de operadores conocidos como la traza parcial sobre
el sistema $B$. La traza parcial tiene la siguiente definición, presentada
por Wilde {\cite{wilde2011classical}}.

\begin{definition}[\textbf{Traza parcial}]
Sea  $ \rho_{AB}$ un operador cuadrado actuando en un producto tensorial del
espacio de Hilbert $\mathcal{H}_A \otimes \mathcal{ H}_B$ y sea $\{|l\ra_b\}$
una base ortonormal para el espacio $\mathcal{H}_B$. Luego la traza parcial
sobre el sistema $\mathcal{H}_B$ está definida como sigue: 
\begin{equation}
		\tr_B(\rho_{AB})\equiv\sum_{l} (\mathds{1}\otimes\la l|_B)\rho_{AB}(\mathds{1}\otimes|l\ra_B).	
\end{equation} 
Por simplicidad, no se suele escribir los operadores identidad y se escribe de la siguiente forma:
\begin{equation}
	\tr_B(\rho_{AB})\equiv\sum_{l} \la l|_B\rho_{AB}|l\ra_B.
\end{equation}
\end{definition}

Por la misma razón que la definición de la traza es invariante bajo la elección
de una base ortonormal, lo mismo es cierto para la operación de traza parcial.
También se observa, a partir de la definición anterior, que la traza parcial es
una operación lineal.

En conclusión, dado un operador $\rho_{AB}$ que describe un estado conjunto de
los sistemas $A$ y $B$, es posible calcular un operador de densidad local
$\rho_A$, que describe el estado local de $A$ si el sistema $B$ es inaccesible
para $A$.


Según Wilde {\cite{wilde2011classical}}, existe una forma alternativa de
describir la traza parcial, la cual es útil estar consciente. Para un estado
simple de la forma  
\begin{equation}
	|x_1\rala x_2|_A\otimes|y_1\rala y_2|_B,
\end{equation} 
la acción de la traza partial es como sigue: 
\begin{equation}
	\begin{split}
	\tr_B(	|x_1\rala x_2|_A\otimes|y_1\rala y_2|_B)&=|x_1\rala x_2|_A\tr(|y_1\rala y_2|_B)\\
	&=|x_1\rala x_2|_A \la y_2|y_1\ra,
	\end{split}
\end{equation}
donde se calcula la traza del segundo sistema para obtener el operador de
densidad local del primero.

Esto se puede generalizar para un operador de densidad arbitrario $\rho_{AB}$,
con una base ortonormal $\{|i \ra_A \otimes |j\ra_B \} _{i,j}$ para un estado
conformado por dos sistemas:
\begin{equation}
	\begin{split}
	\rho_{AB}&=\sum_{i,j,k,l}\lambda_{i,j;k,l}(|i\ra_A \otimes|j\ra_B)(\la k|_A\otimes \la l|_B)\\
	&=\sum_{i,j,k,l}\lambda_{i,j;k,l} |i\rala k|_A\otimes|j\rala l|_B,
	\end{split}
\end{equation}
los coeficientes $\lambda_{i,j;k,l}$  son los elementos de la matriz $\rho_{AB}$ en la base respectiva.


Luego al aplicar la traza parcial y usando la linealidad de la traza, se obtiene 
\begin{equation}
	\begin{split}
		\rho_A&=\tr_B\left(\sum_{i,j,k,l}\lambda_{i,j;k,l} |i\rala k|_A\otimes|j\rala l|_B\right)\\
		&=\sum_{i,j,k,l}\lambda_{i,j;k,l} \tr_B(|i\rala k|_A\otimes|j\rala l|_B)\\
		&=\sum_{i,j,k,l}\lambda_{i,j;k,l}|i\rala k|_A \tr(|j\rala l|_B)\\
		&=\sum_{i,j,k,l}\lambda_{i,j;k,l}|i\rala k|_A \la l|j\ra \\
		%&=\sum_{i,j,k}\lambda_{i,j;k,j}|i\rala k|_A\\
		&=\sum_{i,k}\left(\sum_j\lambda_{i,j;k,j}\right)|i\rala k|_A.
	\end{split}
\end{equation}

La razón principal para estudiar el operador de densidad reducido es que es la única operación que proporciona la descripción correcta de observables para subsistemas de un sistema compuesto. 

Nielsen y Chuang {\cite{nielsen_chuang_2010}} proponen la siguiente situación.
Supóngase que $\mathcal{M}$ es un observable en el sistema de Hilbert
$\mathcal{H}_A$ y se tiene algún dispositivo de medición, el cual es capaz de
realizar la medición de $\mathcal{M}$. Además, sea $\tilde{\mathcal{M}}$ un observable para la misma medición, realizada en el sistema compuesto $\mathcal{H}_A\otimes
\mathcal{H}_B$. Entonces, es necesario mostrar que $\tilde{\mathcal{M}}$ debe
ser igual a $\mathcal{M}\otimes \mathds{1}_B$. Notar que si el sistema
$\mathcal{H}_A\otimes \mathcal{H}_B$ está preparado en el estado $|m \ra
\otimes|\psi\ra $, donde $|m\ra $ es un vector propio de $\mathcal{M}$ con
valor propio $m$ y $|\psi\ra$ es algún estado de $\mathcal{H}_B$, luego el
dispositivo de medición debe proporcionar  el resultado $m$ con una
probabilidad igual a uno.  En consecuencia, si $P_m$ es el operador de proyección al espacio propio de $m$ del observable $\mathcal{M}$, entonces el operador de proyector correspondiente para $\tilde{\mathcal{M}}$ es $P_m\otimes
\mathds{1}_B$. Luego, se tiene que \[\mathcal{\tilde{M}}=\sum_m mP_m\otimes
\mathds{1}_B=\mathcal{M}\otimes \mathds{1}_B.\]

El paso siguiente es mostrar que la traza parcial da como resultado las estadísticas de medición correctas para
las observaciones en una parte del sistema. Para probarlo, suponga que se realiza
una medición en el sistema $\mathcal{H}_A$ descrito por el observable
$\mathcal{M}$. La consistencia física requiere que el \textit{estado} $\rho_A$ que se asocia al sistema $\mathcal{H}_A$, debe tener la propiedad que los valores esperados de la medición sean los mismos ya sea que se calculen a través de $\rho_A$ o $\rho_{AB}$ {\cite{nielsen_chuang_2010}}
\begin{equation}\label{promedioDeMedicion}\tr(\mathcal{M}\rho_A)=\tr(\mathcal{\tilde{M}}\rho_{AB})=\tr((\mathcal{{M}}\otimes
\mathds{1}_B)\rho_{AB}).\end{equation} 

De acuerdo a Nielsen y Chuang {\cite{nielsen_chuang_2010}} la ecuación
({\ref{promedioDeMedicion}}) es satisfecha si se escoge $\rho_A\equiv \tr_B(\rho
_{AB})$. De hecho, la traza parcial debe ser la única función que cumpla esta
propiedad. Para comprobar la propiedad de unicidad, se toma  $f(\cdot)$ como
algún mapeo de operadores de densidad en $\mathcal{H}_A\otimes \mathcal{H}_B$ a
los operadores de densidad en $\mathcal{H}_A$ tal que \[\tr(\mathcal{M}
f(\rho_{AB} )) = \tr((\mathcal{M}\otimes \mathds{1}_B )\rho_{AB} ),\] para
todos los observables $\mathcal{M}$. 

Sea $\mathcal{M}_i$ una base ortonormal de operadores para el espacio de
operadores hermíticos con respecto al producto interno $(X,Y) \equiv \tr(XY)$. Luego, la expansión $f (\rho_{AB} )$ en esta base resulta como sigue
\[\begin{split}
	f(\rho_{AB})=&\sum_{i}\mathcal{M}_i\tr(\mathcal{M}_i f(\rho_{AB}))\\
	=&\sum_{i}\mathcal{M}_i \tr((\mathcal{M}_i\otimes \mathds{1}_B)\rho_{AB}),\\
\end{split}\] 
de ello se sigue que $f$ está únicamente determinada por la ecuación
({\ref{promedioDeMedicion}}). Además, la traza parcial satisface
({\ref{promedioDeMedicion}}), entonces es la única función que tiene esta
propiedad {\cite{nielsen_chuang_2010}}.

% }}}
% }}}
\section{Medidas POVM}\label{sec:Cap1:MedidaPOVM} % {{{
% Intro {{{
Hasta ahora, se ha descrito el espacio de estados cuánticos. Ahora se abordará
el tema del proceso de medición cuántica utilizando herramientas como las
medidas proyectivas y las medidas POVM\@. Aunque el proceso de medición  sigue siendo algo enigmático,
aquí simplemente se toma como cierto el postulado relacionado al colapso de la
función de onda, mencionado en el capítulo anterior (sección
{\ref{subsec:postulates}}). Según Nielsen y Chuang
{\cite{nielsen_chuang_2010}}, el postulado de la medición cuántica, involucra
dos elementos. El primero, proporciona una regla que describe las estadísticas
de medición, es decir, las probabilidades respectivas de los diferentes
resultados de medición posibles. En segundo lugar, da una regla que describe el
estado posterior a la medición del sistema.

%A diferencia de las medidas PVM o proyectivas, las cuales son un caso particular de las medidas aquí tratadas, y son válidas solo para sistemas cerrados, las medidas POVM, también conocidas como medidas generalizadas, tienen en cuenta la interacción del sistema con el entorno.


%El postulado de la medición tiene la virtud de la generalidad, mas no de la
%precisión, sin embargo presenta algunas ventajas como una estructura
%matemáticamente más simple y además que muchos problemas en información
%cuántica implican medidas generales
% }}}
\subsection{Medidas proyectivas PVM}\label{subsec:Cap1:medidasproyectivasPVM} % {{{
La medición cuántica realizada en un estado $\rho$ produce  la $m$-ésima 
salida con una probabilidad $p(m)$ dada por la ecuación
(\ref{probaility3postulate}) y transforma a $\rho $ en $\rho_m$ descrito por la
ecuación (\ref{state3postulate}). En tales mediciones, se registran resultados
concretos etiquetados por el subíndice $m$ y son llamadas \textit{selectivas}.
Si no se realiza una selección basada en la salida de la medición, el estado
inicial es transformado en una combinación convexa de todas las salidas
posibles, dado por 
\begin{equation}\label{eq: non-selective-measure}
	\rho'=\sum_m M_m\rho M_m^\dagger.
\end{equation}

En una medida proyectiva PVM (por sus siglas en inglés, projection-valued
measure) los operadores de medición son operadores proyectores ortogonales
tales que $M_{m}=M_{m}^{\dagger}=P_{m}$, y $P_{m}P_{n}=\delta_{mn}P_{m}$. Una
medición proyectiva es descrita por un observable $\mathcal{O}$ y las salidas
de la medición están etiquetadas por los valores propios de $\mathcal{O}$. Dado
que el $\mathcal{O}$ es operador hermítico, se puede utilizar la descomposición
espectral para representarlo como \begin{equation}
	\mathcal{O}=\sum_m \lambda_m P_m,
\end{equation} donde los operadores de medición del conjunto ortogonal $\{P_m\}$ satisfacen la relación de completitud {\cite{2007geometry}}. 

En una medición proyectiva selectiva la salida etiquetada por $\lambda_m$
ocurre con probabilidad $p(m)=\tr(P_m\rho P_m)$, y por la propiedad cíclica de
la traza,  $p(m)=\tr(P_m\rho)$; el estado inicial es transformado como 
\begin{equation}\label{stateProjective}
	\rho\to	\rho_m=\dfrac{P_m\rho P_m}{\tr(P_m \rho)}.
\end{equation}

 En una medición proyectiva no selectiva, el estado inicial es transformado así
\begin{equation}
	\rho \to \rho'=\sum_m P_m \rho P_m.
\end{equation} 
% }}}
\subsection{Operadores POVM}\label{subsec:operadoresPOVM} % {{{

%Para algunas aplicaciones, el estado posterior a la medición del sistema es de poco interés, siendo el principal elemento de interés las probabilidades de los respectivos resultados de medición. En tales casos, 
En esta sección se discutirá sobre el formalismo POVM (el acrónimo POVM significa <<medida valorada por el operador positivo>>), el cual es bastante elegante y ampliamente utilizado {\cite{nielsen_chuang_2010}}. Se procede a dar una definición de las medidas POVM\@. 

\begin{definition}[\textbf{POVM}] Una medida POVM (positive operator-valued measure) es un conjunto $\{E_{m}\}$ de operadores llamados <<efectos>> que satisfacen las siguientes condiciones {\cite{2007geometry}}:
	\begin{enumerate}
		\item Positividad: $\la \psi |E_m|\psi \ra \ge 0 $ para cualquier vector $|\psi\ra$.
		\item Hermiticidad: $E_m=E_{m}^\dagger$.
		\item  Completitud: $\sum_m E_m =\mathds{1}$.
	\end{enumerate}
\end{definition}

Una medida POVM aplicada al estado $\rho$ produce la $m$-ésima salida, con una
probabilidad $p(m)=\tr(E_m \rho)$. Un ejemplo de medida POVM son las ya
discutidas medidas proyectivas descritas por los operadores de proyección,
tales que $P_m P_n=\delta_{mn}P_{m}$ y $\sum_m P_m = \mathds{1}$. Solo en este
caso todos los elementos POVM son los mismos que los propios operadores de
medición, $E_m=P_m P_m^\dagger=P_m$ {\cite{nielsen_chuang_2010}}.

Los POVM se ajustan en el marco general del postulado de la medición cuántica, ya que se puede elegir $E_m=M_m M_m^{\dagger}$. Sin embargo es importante notar que los POVM no determinan los operadores de
medición $M_m$ de manera única, a excepción del caso particular de las
mediciones proyectivas. Lo que sucede con el estado exactamente, cuando una
medición es realizada depende en como los POVM son implementados en el
laboratorio {\cite{2007geometry}}.   

% \cpnote{De nuevo, para quequieres decir esto?}\rrnote{quité este párrafo,estaba de más.}
%Según Bengtsson y Zyczkowski {\cite{2007geometry}}, un POVM es llamado \textit{puro} si cada operadores $E_m$ es de rango uno, tales que existe un estado puro $|\phi_m\ra$ tal que $E_m$ es proporcional a $|\phi_m\rala\phi_m|$. Un POVM que es impuro puede siempre volverse puro, reemplazando cada operador $E_m$ por su descomposición espectral. 


Según Nielsen y Chuang {\cite{nielsen_chuang_2010}}, en computación e información cuántica, el objetivo es lograr un buen nivel de control sobre las mediciones que se pueden realizar, por consiguiente utilizar un formalismo más completo para la descripción de las mediciones es de gran ayuda. Por supuesto, las medidas proyectivas son completamente equivalentes a las mediciones más generales cuando se toman en cuenta los demás axiomas.


Existen varias razones para utilizar el formalismo general de las mediciones. La primera es que matemáticamente las mediciones generales son menos restrictivas que las medidas proyectivas. Por ejemplo, estas mediciones generales no necesariamente exigen la condición de idempotencia $P^2=P$ como si lo hacen las medidas proyectivas. Otra razón para utilizarlas es que existen muchos problemas en información y computación cuántica tales como la forma óptima de distinguir un conjunto de estados cuánticos cuya respuesta implica una medición general, en lugar de una medición proyectiva {\cite{nielsen_chuang_2010}}.


Una tercera razón para utilizar las mediciones generales esta relacionada con la propiedad de repetibilidad de las mediciones proyectivas. Las medidas proyectivas son repetibles en el sentido que si se realiza una medición proyectiva una vez y se obtiene un resultado $m$, al repetir la medición el resultado será nuevamente $m$ y el estado también será el mismo. Este hecho indica que muchas medidas importantes en mecánica cuántica no son medidas proyectivas.  Para tales mediciones, debe emplearse el postulado general de medición. Y en este contexto las medidas POVM son un caso especial del formalismo de la medición general, que proporciona el medio más simple en el que se pueden estudiar las estadísticas generales de medición, sin necesidad de conocer el estado posterior a la medición. Son una conveniencia matemática que a veces brinda información adicional sobre las mediciones cuánticas {\cite{nielsen_chuang_2010}}.


% }}}
% }}}
\section{Operaciones cuánticas}\label{sec:Cap1:OpCuanticas} % {{{
% Intro {{{

Los sistemas reales sufren interacciones indeseadas con el mundo exterior, y
éstas aparecen como ruido en información cuántica. Las operaciones cuánticas
son un herramienta capaz de describir dicho ruido cuántico y el comportamiento
de sistemas cuánticos abiertos.  %Adicionalmente, .

% }}}
\subsection{Aproximación axiomática de las operaciones cuánticas} % {{{
En esta sección se abordan las operaciones cuánticas desde un punto de vista
axiomático el cual será motivado por requerimientos físicos. 
Nielsen y
Chuang {\cite{nielsen_chuang_2010}}
presentan la siguiente definición de operación cuántica
\begin{definition}[\textbf{Operación cuántica}]\label{DefE(rho)} Una operación
cuántica $\mathcal{E}$ es un mapeo de un conjunto de operadores en un espacio
de Hilbert $\mathcal{H}_A$ de entrada a otro conjunto de operadores en un
espacio de Hilbert $\mathcal{H}_B$ de salida, $\E: \mathcal{H}_A \rightarrow
\mathcal{H}_B$, con las siguientes propiedades axiomáticas:

    \begin{itemize}
        \item \textit{Axioma 1:} La traza $\tr (\mathcal{E}(\rho))$ es la probabilidad de que el proceso representado por $\mathcal{E}$ ocurra, cuando $\rho$ es el estado inicial. Consecuentemente, $0 \le \tr (\mathcal{E}(\rho)) \le 1$ para cualquier estado $\rho$.
	\item \textit{Axioma 2:} El mapeo $\mathcal{E}$ es lineal 
en el conjunto de matrices de densidad.
Esto es, para probabilidades
$\{p_i\}$, \[\mathcal{E}\left(\sum _i p_i \rho _i\right)=\sum_i p_i
\mathcal{E}(\rho_i).\]
	\item\textit{Axioma 3:} El mapeo $\E$ es completamente positivo. Esto
significa que $\E$ mapea  operadores en el espacio de Hilbert
$\mathcal{H}_{A}$  a operadores en el espacio de Hilbert $\mathcal{H}_{A'}$, luego $\E(\rho)$ debe ser positivo para cualquier operador positivo $\rho $. Más aún, si se introduce un espacio de Hilbert $ \mathcal{H}_{B}$ y se considera el mapeo extendido $(\E \otimes \mathds{1}_B)(\rho)$, éste debe ser positivo también para cualquier operador positivo $\rho$ en el sistema combinado $\mathcal{H}_A\otimes \mathcal{H}_B$. 
    \end{itemize}
\end{definition}

El primer axioma resulta conveniente en  el caso de las mediciones. Para verlo
mejor, suponga que se realiza una medición
proyectiva en la base computacional de un solo qubit. Entonces la operación
cuántica $\E$ describirá este proceso si se define el mapeo como
$\E_0\equiv|0\rala0|\rho |0\rala 0|$ y $\E_1\equiv|1\rala1|\rho |1\rala 1|$.
Las probabilidad de las salidas serán entonces $\tr (\E_0(\rho))$ y $\tr
(\E_1(\rho))$ respectivamente.  Con esta convención la normalización correcta
para el estado final será \[\dfrac{\E(\rho)}{\tr (\E (\rho))}.\]

En el caso que no se realice ninguna medición, esto se reduce al requisito de
que $\tr[\E(\rho)] = 1 = \tr(\rho)$, para todo $\rho$. En este caso la
operación cuántica $\E$ preserva la traza, ya que por sí sola proporciona una
descripción completa del proceso cuántico 
{\cite{nielsen_chuang_2010}}.



Asimismo, para ver una razón física para proponer el segundo axioma, supóngase que un estado inicial $\rho_i$ está preparado con un probabilidad $p_i$ y luego se realiza la medición. Si el estado es $\rho_i $ luego la salida de la medición  $\alpha$ ocurre con la probabilidad condicional $p(\alpha|i)$, y el estado de la medición  posterior es $\E_\alpha(\rho_i)/p(\alpha|i)$; luego el ensamble de estado después de la medición está descrita por el operador de densidad 
\begin{equation}\label{lineal_y_convexo}
    \rho'=\sum_i p(i|\alpha)\dfrac{\E_\alpha(\rho_i)}{p(\alpha|i)},
\end{equation}
donde $p(i|\alpha)$ es la probabilidad a posteriori que el estado $\rho_i$
fuera preparado, tomando en cuenta la información obtenida haciendo la medición
{\cite{preskill2020quantum}}. 

Por otro lado, aplicando la operación $\E_\alpha$  a la combinación convexa del
estado inicial $\{\rho_i\}$ \begin{equation}
    \rho'=\dfrac{\E_\alpha\left(\sum_i p_i \rho_i\right)}{p_\alpha},
\end{equation} tomando en cuenta la regla de Bayes  $p(i|\alpha)=\dfrac{p_i
p(\alpha|i)}{ p_\alpha} $ a la ecuación ({\ref{lineal_y_convexo}}),  se ve que
$\E_\alpha$ debe ser lineal {\cite{preskill2020quantum}} 
\begin{equation}
    \E_\alpha\left(\sum_i p_i \rho_i\right)=\sum_i p_i\E_\alpha(\rho_i). 
\end{equation}

La tercera propiedad también se origina por un requerimiento físico. Es
razonable exigir que un mapeo sea completamente positivo si se va a describir
la evolución temporal de un sistema cuántico. Aunque el mapeo actúa solo en una
parte del sistema, debe representar un estado inicial de todo el sistema hacia
un estado final del sistema entero. Si $\rho_{AB}$ es una matriz de densidad de
un sistema conjunto de $\mathcal{H}_A$ y $\mathcal{H}_{B}$  y $\E$ actúa
solamente sobre $\mathcal{H}_A$, entonces $\E(\rho_{AB})$  debe ser un operador
de densidad también. Formalmente, supongamos que introducimos un segundo
sistema $B$ (de dimensión finita). Sea $\mathds{1}_{B}$ el mapa identidad del
sistema $B$. Entonces el mapa $\E\otimes \mathds{1}_{B}$ debe llevar operadores
positivos a operadores positivos {\cite{nielsen_chuang_2010, preskill2020quantum}}. 


\cpnote{Bien}
% }}}
\subsection{Operadores de Kraus} % {{{
\cpnote{Creo qeu hay demasiado detalle. Yo dejaria solo unas 3 o 4 paginas. }
% Intro {{{

Una operación cuántica puede representarse en una forma elegante 
como  \textit{suma de operadores}.   La representación de suma de operadores de
una operación cuántica,
también conocida como representación de Kraus\footnote{ La representación de
Kraus fue introducida por el físico alemán Karl Kraus en 1971, basada en el
resultado del teorema de Stinespring {\cite{2007geometry}}.}, se puede escribir
como $\E(\rho)=\sum_i K_i\rho K_i^\dagger$, para algún conjunto de operadores
$\{K_i\}$ que satisfacen la condición $\sum_i K_i^\dagger K_i\le \mathds{1}_N$.
Los operadores
$\{K_i\}$ son conocidos como \textit{operadores de Kraus}. Así también, los
mapeos completamente positivos que preservan la traza se conocen por varios
nombres: \textit{operaciones cuánticas deterministas, canales cuánticos o mapas
estocásticos}  {\cite{2007geometry}}.

Nielsen y Chuang {\cite{nielsen_chuang_2010}} presentan un teorema y una prueba de que la representación de suma de operadores es equivalente a la definición de la sección anterior.

\begin{theorem}\label{teorema:equivalencia_propiedadesaxiomaticas_opKraus}
    El mapeo $\E$ satisface los axiomas de la definición ({\ref{DefE(rho)}}) si y solo si 
    \begin{equation}
        \E(\rho)=\sum_i K_i \rho K_i^\dagger,
    \end{equation}
    para algún conjunto de operadores $\{K_i\}$ el cual mapea el espacio de Hilbert de entrada al espacio de Hilbert de salida, y $\sum_i K_i^\dagger K_i\le \mathds{1}_N$.
\end{theorem}


\begin{proof}
Suponiendo que $\E$ satisface ({\ref{DefE(rho)}}). Se introduce un sistema
$\mathcal{H}_B$, con las mismas dimensiones del sistema cuántico original
$\mathcal{H}_A$. Sea $|k_A\ra$ y $|k_B\ra$  bases ortonormales para $\mathcal{H}_A$ y $\mathcal{H}_B$
respectivamente. Se define un estado conjunto del sistema $\mathcal{H}_A\otimes
\mathcal{H}_B$ como \[|\alpha\ra \equiv \sum_k |k_A\ra |k_B\ra.\] El estado
$|\alpha\ra $ es un estado máximamente entrelazado de los sistemas
$\mathcal{H}_A$ y $\mathcal{H}_B$. También se define un operador $\sigma$ en
espacio de estado de $\mathcal{H}_A\otimes \mathcal{H}_B$ dado por 
\begin{equation}\label{sigma1}
    \sigma \equiv (\E\otimes\mathds{1}_B)(|\alpha\rala \alpha|).
\end{equation}


Se puede pensar en esto como un resultado de aplicar la operación cuántica $\E$
a la mitad de un estado de máximo entrelazamiento del sistema
$\mathcal{H}_A\otimes \mathcal{H}_B$. A continuación se probará el hecho que el operador $\sigma$ especifica completamente la operación cuántica $\E$. Es decir para saber como actúa $\E$ en un estado arbitrario de $\mathcal{H}_A$, es suficiente con saber cómo este actúa en un solo estado máximamente entrelazado de $\mathcal{H}_A$ con otro sistema. 

La estrategia que permite recuperar a $\E$ de $\sigma$ es la siguiente. Sea $|\psi\ra=\sum_j \psi_j |j_A\ra$ algún estado del sistema $\mathcal{H}_A$ y $|\tilde{\psi}\ra = \sum_j \psi_j^*|j_B\ra$ un estado correspondiente del sistema $\mathcal{H}_B$. Notar que
\begin{equation}
    \begin{split}
        \la \tilde \psi |\sigma |\tilde{\psi}\ra&= \la \tilde \psi |\left(\sum_{kj} \E (|k_A \rala j_A|)\otimes |k_B\rala j_B|\right) |\tilde{\psi}\ra\\
        &=\sum_{kj} \psi_k \psi_j^* \E (|k_A \rala j_A|)\\
        &=\E(|\psi\rala\psi|).
    \end{split}
\end{equation}

Luego, sea $\sigma=\sum_i |s_i\rala s_i|$ una descomposición del operador $\sigma$, donde los vectores $|s_i\ra$ no necesitan ser normalizados. Adicionalmente, se define el siguiente mapeo \begin{equation}K_i(|\psi\ra) \equiv \la\tilde{\psi}|s_i\ra.\end{equation} 
Se puede notar que este mapeo es lineal, por consiguiente $K_i$ es un operador lineal en el espacio de estado $\mathcal{H}_A$. Más aún, se tiene que \begin{equation}
    \begin{split}
       \sum_i K_i |\psi \rala \psi |K_i^\dagger&=\sum_i  \la\tilde{\psi}|s_i\ra \la s_i|\tilde{\psi}\ra\\
        &=\la \tilde \psi |\sigma |\tilde{\psi}\ra\\
        &=\E(|\psi \rala\psi|).\\
    \end{split}
\end{equation}



De ello se obtiene la siguiente igualdad \[\E(|\psi\rala \psi|)=\sum_i K_i |\psi \rala \psi |K_i^\dagger,\] para todos los estados $|\psi\ra$ de $\mathcal{H}_A$. Por el segundo axioma, se sigue que \[\E(\rho)=\sum_i K_i \rho K_i^\dagger.\] La condición $\sum_i K_i^\dagger K_i\le \mathds{1}$ se obtiene del primer axioma, identificando la traza de $\E(\rho) $ con una probabilidad. Esto concluye la primera parte de la demostración.

Ahora se supone el recíproco de lo anterior, sea $\E(\rho)=\sum_i K_i \rho K_i^\dagger$, tal que $\sum_i K_i^\dagger K_i \le \mathds{1}$. Es fácil ver que la operación cuántica $\E$ es lineal, pero falta chequear la completa positividad.

Sea $\mathcal{O}$ un operador positivo actuando en el espacio de estado del sistema extendido, $\mathcal{H}_A\otimes \mathcal{H}_B$ y $|\psi \ra$ algún estado del sistema $\mathcal{H}_A\otimes \mathcal{H}_B$. De igual forma, se define el estado  $|\varphi_i\ra \equiv (K_i^\dagger \otimes \mathds{1}_B)|\psi\ra$, de ello se tiene que \[\la \psi |(K_i \otimes \mathds{1}_B)\mathcal{O}(K_i^\dagger \otimes \mathds{1}_B)|\psi\ra=\la \varphi_i |\mathcal{O}|\varphi_i\ra \ge 0,\] por la positividad de $\mathcal{O}$. Por la linealidad se sigue que \[\la \psi| (\E \otimes \mathds{1}_B)(\mathcal{O})|\psi\ra=\sum_i \la \varphi_i|\mathcal{O}|\varphi_i\ra \ge 0,\] esto se cumple para cualquier operador positivo $\mathcal{O}$. En consecuencia, el operador $(\E\otimes \mathds{1}_B)(\mathcal{O})$ es positivo como se requiere. Finalmente, la condición $\sum_i K_i^\dagger K_i \le \mathds{1}$ asegura que las probabilidades son menores o iguales a 1. 
\end{proof}

La representación de suma de operadores describe la dinámica del sistema principal sin tener que considerar explícitamente las propiedades del entorno; todo lo que se necesita saber está agrupado en los operadores $\{K_i\}$, que actúan solo en el sistema principal {\cite{nielsen_chuang_2010}}.


% }}}

\begin{comment}
\subsubsection{Mediciones y representación de Kraus}\label{subsubsec:Medicion_RepresentacionDeKraus} % {{{

Considere un sistema primario $\mathcal{H}_Q$, inicialmente en el estado
$\rho$,  el cual está en contacto con un sistema auxiliar $\mathcal{H}_A$
(este también es llamado <<ancilla>> y puede ser pensado como el ambiente o un
aparato de medición), inicialmente en el estado $\sigma=\sum_k \lambda_k
|e_k\rala e_k|$, donde esta representación es la descomposición espectral de
$\sigma$. El sistema auxiliar está sujeto a las mediciones de von Neumann,
descritas por los operadores de proyección $P_\alpha=\sum_j |f_{\alpha j}\rala
f_{\alpha j}|$,  donde $|f_{\alpha j}\ra$ forman una base ortonormal de
$\mathcal{H}_A$ y satisfacen la relación de completitud. Si el resultado de la
medición es $\alpha$, el sistema auxiliar es observado estando en el subespacio
$S_\alpha$. Los dos sistemas interactúan durante un tiempo, esta interacción es
descrita por un operador unitario $U$ que actúa sobre el sistema conjunto
$\mathcal{H}_Q\otimes \mathcal{H}_A$. El estado no normalizado del sistema
después de la medición se obtiene proyectando el estado conjunto, en el
subespacio $S_\alpha$ y luego aplicando la traza parcial sobre $\mathcal{H}_A$,

\begin{equation}\label{measurement_model}
    \tr_A(P_\alpha U \rho \otimes \sigma U^\dagger P_\alpha)= \tr_A(P_\alpha U \rho \otimes \sigma U^\dagger)\equiv \E_\alpha(\rho),
\end{equation} 
donde $\E_\alpha$ es un mapeo lineal en el sistema de los operadores de
densidad. Este método de definir un conjunto de operaciones cuánticas en
términos de interacción con un ancilla inicialmente no correlacionado, seguido
de la medición en el ancilla, se denomina modelo de medición {\cite{unm2014,
nielsen_chuang_2010}}.

Reescribiendo, la ecuación ({\ref{measurement_model}}), en términos de la
representación de $P_\alpha$ y de $\sigma$ se obtiene que,
\begin{equation}
    \E_\alpha(\rho)=\sum_{j,k}\sqrt{\lambda_k}\la f_{\alpha j}|U|e_k\ra \rho \la e_k|U|f_{\alpha j}\ra \sqrt{\lambda_k}=\sum_{j,k}M_{\alpha j k}\rho M_ {\alpha j k}^\dagger,
\end{equation} donde \begin{equation} \label{KrausOp1}
    M_{\alpha j k}\equiv \sqrt{\lambda_k}\la f_{\alpha j}|U|e_k\ra.
\end{equation}
Estos operadores preservan la traza, \[\sum_{\alpha, j,k}
M_{\alpha,j,k}^\dagger M_{\alpha,j,k}=\sum_{\alpha, j,k} \lambda_k \la
e_k|U|f_{\alpha j}\ra \la f_{\alpha j}|U|e_k\ra= \tr_A(U^\dagger U
\sigma)=\tr_A(\mathds{1}_Q\otimes\sigma)=\mathds{1}_Q.\] De esta forma se obtiene la representación de Kraus de la operación $\E_\alpha$ y se definen los operadores de Kraus con la ecuación ({\ref{KrausOp1}}). 

La probabilidad de obtener el resultado $\alpha$ en la medición del ancilla es {\cite{unm2014}} 
\begin{equation}
    p_\alpha=\tr(P_\alpha U\rho\otimes\sigma U^\dagger)=\tr(\E_\alpha(\rho))=\tr\left(\rho \sum _{j,k} M_{\alpha, j,k}^\dagger M_{\alpha,j,k}\right)=\tr(\rho M_\alpha).
\end{equation}

%Cualquier modelo de medición da lugar a una medida POVM que describe las estadísticas de medición.
Los operadores \[E_\alpha \equiv \sum _{j,k} M_{\alpha, j,k}^\dagger
M_{\alpha,j,k}=\sum_{j,k} \lambda_k \la e_k|U|f_{\alpha j}\ra \la f_{\alpha
j}|U|e_k\ra=\tr_A(U^\dagger P_\alpha U\sigma) \] son claramente positivos y
preservan la traza. El estado normalizado del sistema posterior a la medición,
condicionado al resultado $\alpha$, es {\cite{unm2014}} 
\begin{equation}
    \rho_\alpha =\dfrac{\E_\alpha (\rho)}{\tr(\E_\alpha(\rho))}=\dfrac{\E_\alpha(\rho)}{p_\alpha}=\dfrac{1}{p_\alpha}\sum_{j,k}M_{\alpha,j,k}\rho M_{\alpha,j,k}^\dagger.
\end{equation}

Si no se sabe el resultado de la medición en el sistema auxiliar, el estado
después de la medición es obtenido promediando sobre las posibles resultados de
las mediciones {\cite{unm2014}}
\begin{equation}
    \rho'=\sum_\alpha p_\alpha \rho_\alpha=\sum_\alpha \E_\alpha(\rho)=\sum_{\alpha,j,k} M_{\alpha,j,k}\rho M_{\alpha,j,k}^\dagger\equiv\E(\rho).
\end{equation}
\end{comment}

% }}}
\subsubsection{No unicidad en la representación de Kraus} % {{{
 La representación de Kraus provee una descripción bastante general de la
dinámica de los sistemas cuánticos abiertos sin embargo esta representación no
es única. Es decir que distintos conjuntos de operadores $\{K_i\}$ y $\{M_j\}$
pueden generar la misma operación cuántica. Esto es importante puesto que,
diferentes procesos físicos pueden dar lugar a la misma dinámica del sistema y
entender la libertad de la representación es crucial para una buena comprensión
de la corrección de errores cuánticos. Nielsen y Chuang
{\cite{nielsen_chuang_2010}} proponen formalmente el siguiente teorema.

\begin{theorem}[Libertad unitaria en la representación de Kraus]\label{teorema:Libertad_unitaria}
Sean \[\{K_1,\ldots,K_n\} \text{ y } \{M_1,\ldots , M_m\}\] conjuntos de
operadores que generan operaciones cuánticas $\E$ y $\mathcal{F}$,
respectivamente. Además, se agregan operadores cero a la lista más corta de operadores, para asegurar que $m=n$. Luego $\E=\mathcal{F}$
si y solo si existe números complejos $u_{ij}$ tales que $K_i=\sum_j
u_{ij}M_j$, y $u_{ij}$ es una matriz
unitaria de $m$ por $m$
\end{theorem}


\begin{comment}
\begin{proof}
Suponiendo que $\{K_i\}$ y $\{M_j\}$ son dos conjuntos de operadores para la misma operación cuántica, $\sum_i K_i \rho K_i^\dagger= \sum_j M_j \rho M_j^\dagger$ para todo $\rho$. Se definen los siguientes estados
\begin{equation}
    |e_i\ra\equiv \sum_k (K_i|k_A\ra)|k_B\ra
\end{equation}
\begin{equation}
    |f_j\ra\equiv \sum_k (M_j|k_A\ra)|k_B\ra.
\end{equation}


Además se toma el operador $\sigma$ definido en la ecuación ({\ref{sigma1}}), de la cual se obtiene que $\sigma=\sum_i|e_i \rala e_i|=\sum_j |f_j\rala f_j|$, y luego existe una matriz unitaria tal que $  |e_i\ra=\sum_{j}u_{ij}|f_j\ra$.

Para un estado arbitrario $|\psi\ra$ 
\begin{equation}
    \begin{split}
        K_i|\psi\ra&=\sum_k \psi_k K_i|k_A\ra\\
                   &= \sum_{lk}\psi_l(K_i|k_A\ra)\la l_B |k_B\ra\\
                   &=\sum_{l}\psi_l\la l_B| e_i\ra=\sum_{jl}\psi_l u_{ij}\la l_B|f_j\ra\\
                   &=\sum_{jkl}\psi_l u_{ij}(M_j|k_A\ra)\la l_B|k_B\ra\\
                   &=\sum_{jl}\psi_l u_{ij}M_j|l_A\ra=\sum_j u_{ij} M_j |\psi\ra,\\
    \end{split}
\end{equation} luego \[K_i=\sum_j u_{ij}M_j.\]Por otra parte, se supone que $K_i$ y $M_j$ están relacionados por la matriz unitaria $K_i=\sum_j u_{ij}M_j$. Entonces, se tiene que
\begin{equation}
    \begin{split}
        \E(\rho)&=\sum_i  K_i\rho K_i^\dagger\\
            &=\sum_{ij} u_{ij}u_{ij}^*M_j\rho M_j^\dagger\\
            &=\sum_j M_j\rho M_j^\dagger\\
            &=\mathcal{F}(\rho).
    \end{split}
\end{equation}

Con ello se muestra que el conjunto de operadores $\{K_i\}$ es la misma operación cuántica que con el conjunto $\{M_j\}$.
\end{proof}
\end{comment}

La prueba de este teorema se fundamenta en el teorema ({\ref{teorema2.4}}). Para ejemplificarlo, Nielsen y Chuang {\cite{nielsen_chuang_2010}} proponen las siguientes operaciones $\E(\rho)=\sum_i K_i\rho K_i^\dagger$ y $\mathcal{F}(\rho)=\sum_i M_i\rho M_i^\dagger$, donde los conjuntos $\{K_n\}$ y $\{M_m\}$ tienen elementos definidos por 
\begin{equation}
    \begin{array}{ccc}
        K_1=\dfrac{1}{\sqrt{2}}\begin{bmatrix}
            1&0\\
            0&1\\
        \end{bmatrix},&&K_2=\dfrac{1}{\sqrt{2}}\begin{bmatrix}
            1&0\\
            0&-1\\
        \end{bmatrix}.\\
    \end{array}
\end{equation}

Esta operación $\E$ representa $1/2$ de probabilidad de aplicar el operador unitario y $1/2$ de probabilidad de aplicarle $Z$ al sistema cuántico y
\begin{equation}
    \begin{array}{ccc}
        M_1=\begin{bmatrix}
            1&0\\
            0&0\\
        \end{bmatrix},&&M_2=\begin{bmatrix}
            0&0\\
            0&1\\
        \end{bmatrix},\\
    \end{array}
\end{equation}

la segunda operación $\mathcal{F}$ corresponde a realizar una medición proyectiva en la base $\{|0\ra , |1\ra\}$, con el resultado de la medida desconocida. 



Las operaciones $\E$ y $\mathcal{F}$ son, matemáticamente la misma operación cuántica. Para verlo, notar que existe la matriz unitaria \[U=\begin{bmatrix}u_{11}&u_{12} \\u_{21}&u_{22}\end{bmatrix}=\dfrac{1}{\sqrt{2}}\begin{bmatrix}1&1 \\1&-1\end{bmatrix}\]tal que \[M_1=\sum_{j} u_{1j}K_j=(K_1+K_2)/\sqrt{2}\] y \[M_2=\sum_j u_{2j}K_j=(K_1-K_2)/\sqrt{2}\] luego,
\begin{equation} 
    \begin{split}
        \mathcal{F}(\rho)&=\dfrac{(K_1+K_2)\rho(K_1^\dagger+K_2^\dagger)+(K_1-K_2)\rho(K_1^\dagger-K_2^\dagger)}{2}\\
        &=K_1\rho K_1^\dagger+K_2\rho K_2^\dagger\\
        &=\E(\rho).
    \end{split}
\end{equation}
% }}}
% }}}
% }}}
\begin{comment}
\section{Proceso de tomografía cuántica} % {{{

Antes de describir el proceso de tomografía cuántica, vale la pena presentar la tomografía de estado cuántico. La tomografía de estado cuántico es el procedimiento experimental para determinar un estado cuántico desconocido. Es posible estimar $\rho$ si se tiene un gran número de copias de $\rho$. Por ejemplo, si $\rho$ es el estado cuántico producido por algún experimento, se puede repetir el experimento muchas veces para producir muchas copias del estado $\rho$ {\cite{nielsen_chuang_2010}}. 

Suponiendo que se tiene varias copias de un operador de densidad de un qubit $\rho$. El conjunto $\mathds{1}/\sqrt{2}$, $X/\sqrt{2}$, $Y/\sqrt{2}$, $Z/\sqrt{2}$ forma una base ortonormal de matrices, tal que $\rho$  puede escribirse con \[\rho=\dfrac{\tr(\rho)\mathds{1}+\tr(X\rho)X+\tr(Y\rho)Y+\tr(Z\rho)Z}{2}.\]

Por ejemplo, para estimar $\tr(Z\rho)$ se mide el observable $Z$ muchas veces, $m$, obteniendo salidas como $z_1,z_2,\ldots,z_m$, todos iguales a $+1$ o $-1$. El promedio empírico de esas cantidades, $\sum_i z_i/m$, es un estimado para el valor real de $\tr(Z\rho)$. Usando el teorema del límite central para determinar qué tan bien se comporta esta estimación para $m$ grande, donde se vuelve aproximadamente gaussiana con media igual a $\tr(Z\rho)$ y con desviación estándar  $\Delta(Z)/ m$, donde $\Delta(Z)$ es la desviación estándar para una sola medición de $Z$, que tiene un límite superior de 1, por ende, la desviación estándar en la estimación $\sum_i z_i/m$ es como máximo $1/\sqrt{m}$ {\cite{nielsen_chuang_2010}}. 

Al generalizar el procedimiento para el caso de más de un qubit es similar,
\[\rho=\sum_{\vec{v}}\dfrac{\tr(\sigma_{v_1}\otimes\sigma_{v_2}\otimes \ldots \otimes \sigma_{v_n}\rho)\sigma_{v_1}\otimes\sigma_{v_2}\otimes \ldots \otimes \sigma_{v_n}}{2^n},\] donde la suma es sobre los vectores $\vec{v} = (v_1 ,\ldots, v_n )$ con entradas vi elegidas del conjunto $\{0, 1, 2, 3\}$. Al realizar mediciones de observables que son productos de matrices de Pauli, se puede estimar cada término en esta suma, y así obtener una estimación de $\rho$ {\cite{nielsen_chuang_2010}}.

Para el proceso de tomografía cuántica, el objetivo es tener una forma de determinar una representación útil de $\E$  de los datos experimentales disponibles. El objetivo es encontrar un conjunto de operadores de Kraus $\{K_i\}$ para $\E$.


Nielsen y Chuang {\cite{nielsen_chuang_2010}} presentan una forma de determinar los $K_i$  a partir de los parámetros medibles.

Primero, se toman en cuenta una base fija de operadores $\tilde{K_i}$, tales que  \begin{equation}\label{Basefija}
    K_i=\sum_m e_{im}\tilde{K}_m
\end{equation}
para algún conjunto de números complejos $e_{im}$. Luego, 
\begin{equation}\label{operacionEnTerminosDeChi}
    \E(\rho)=\sum_{mn}\tilde{K}_m\rho \tilde{K}_n^\dagger \chi_{mn},
\end{equation}
donde $\chi_{mn}\equiv \sum_i e_{im}e_{in}^*$ son las entradas de una matriz que es hermitiana por definición. Esta expresión, conocida como la \textit{representación de la matriz chi}. Esto muestra que la operación $\E$ puede ser completamente descrita por la matriz $\chi$.

En general, $\chi$ tendrá $d^4-d^2$ parámetros reales independientes, puesto que un mapeo lineal general de matrices de $d$ por $d$ a matrices de $d$ por $d$ está descrito por $d^4$ parámetros independientes, pero debido a que $\rho$ es hermitiana de traza uno, se tienen $d^2$ parámetros fijos.

Sea $\rho_j$, $1 \le j\le d^2$ una base fija para el espacio de matrices de $d\times d$; cualquier matriz de $d\times d$ puede ser escrita como una única combinación de $\rho_j$. Luego, es posible determinar $\E(\rho_j)$ por la tomografía de estado cuántico, para cada $\rho_j$.

Luego, cada $\E(\rho_j)$ puede ser expresada como una combinación lineal de estados de la base 
\begin{equation}\label{E_en_terminos_de_la_base}
    \E(\rho_j)=\sum_k \lambda _{jk}\rho_k.
\end{equation}
Ahora, $\lambda_{jk}$ puede ser determinado por algoritmos de álgebra lineal. Se puede escribir lo siguiente
\begin{equation}
    \tilde{K}_m\rho_j \tilde{K}_n^\dagger=\sum_k\beta_{jk}^{mn}\rho_k,
\end{equation}
donde $\beta_{jk}^{mn}$ son número complejos los cuales también pueden ser determinados por los algoritmos de álgebra lineal, dados los operadores $\tilde{K}_m$ y $\rho_j$.  Combinando las dos últimas expresiones con la ecuación ({\ref{operacionEnTerminosDeChi}})
\begin{equation}
\sum_k \sum_{mn}\chi_{mn}\beta_{jk}^{mn}\rho_k=\sum_k\lambda_{jk}\rho_k.
\end{equation}

Debido a la independencia lineal de $\rho_k$ para cada $k$, 
\begin{equation}
    \sum_{mn}\beta_{jk}^{mn}\chi_{mn}=\lambda_{jk}.
\end{equation}

Esta relación es una condición necesaria y suficiente para la matriz $\chi$ para dar la operación cuántica correcta $\E$. Luego $\chi$ y $\lambda$ pueden verse como vectores, y $\beta$ como una matriz de $d^4\times d^4$, con las columnas nombradas por ${mn}$ y las columnas por ${jk}$. Se necesita computar la matriz inversa de $\beta_{jk}^{mn}$. Sea $\kappa$ la matriz inversa generalizada de la matriz $\beta$, que satisface \[\sum_{jk}\kappa_{jk}^{pq}\beta_{jk}^{mn}=\delta_{pm}\delta{qn},\] los elementos de $\chi$ se leen como 
\begin{equation}
    \chi_{mn}=\sum_{jk}\kappa_{jk}^{mn}\lambda_{jk}.
\end{equation}

Ahora que se determinó $\chi$ se puede obtener inmediatamente la representación de Kraus para $\E$ de la siguiente manera. Sea $U^\dagger$ una matriz unitaria que diagonaliza $\chi$,
\begin{equation}
    \chi_{mn}=\sum_{xy}U_{mx}d_x\delta_{xy}U_{ny}^*
\end{equation}

y de ello se puede verificar que  \begin{equation}
    K_i=\sqrt{d_i} \sum_j U_{ji}\tilde{K}_j
\end{equation} son elementos para $\E$. El proceso se puede resumir en determinar $\lambda$ experimentalmente usando el proceso de tomografía de estado, y luego determinar $\chi$ con $\vec{\chi}=\kappa\lambda$, la cual da una descripción completa de $\E$, incluyendo un conjunto de operadores $\{K_i\}$.

% }}}
\end{comment}







