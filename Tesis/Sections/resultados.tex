\chapter[RESULTADOS PARA N PARTÍCULAS]{3. Resultados para \texorpdfstring{\boldmath{$N$}}{N} PARTÍCULAS}
\section{ Introducción}





\section{Sistemas de dos partículas} % {{{
% }}}
% }}}
\rrnote{ En esta sección exponer los resultados ya trabajados y explorar más opciones. Primero especificar condiciones del sistema de dos partículas y de los observables.}


\subsection{POVM para mediciones difusas para dos partículas}\label{Sec_POVM_para_mediciones_difusas} % {{{

\rrnote{Detallar algunas implementaciones}


\rrnote{Agregar más de una implentación de efectos en esta sección (?)}

\subsection{Instrumentos cuánticos para dos partículas}



\subsubsection{Primera alternativa de instrumento cuántico} % {{{
    \rrnote{Se expone la primera alternativa de los intrumentos cuánticos, en sistemas simples de dos partículas y con observables factorizables. }
    \rrnote{Se obtiene su valor esperado.}


    \subsubsection{Segunda alternativa de instrumento cuántico}

    \rrnote{ De igual forma se explica la segunda alternativa para sistema de dos partículas y se obtiene su valor esperado. }

\subsection{Equivalencia de los instrumentos}

\rrnote{ Se hace una comparación entre los instrumentos anteriores y se expone por qué resultaron no ser equivalentes.}
\rrnote{En esta sección se comenta que se esperaba que los instrumentos fueran equivalentes pero resultaron distintos y por ello vale la pena proponer en qué condiciones sí lo son.}


\subsection{Ejemplos sobre los efectos de una medición difusa} % {{{

\rrnote{Describir algunos ejemplos con estados iniciales y observables particulares de una medición difusa en un sistema de dos particulas. }

\section{Generalización de operadores de Kraus en sistemas de \texorpdfstring{\boldmath{$N$}}{N} partículas}

\rrnote{En esta sección generaliza la implementación los sistema de dos particulas y sus implementaciones en sistemas más grandes y los instrumentos presentados en la última sección del capitulo 2}

\subsection{Medidas POVM en sistemas de varias partículas}





\subsection{Instrumentos cuánticos en sistemas de varias partículas} 


\begin{comment}
En un sistema de $N$ partículas en el que se pretende medir el observable $A_1\otimes A_2\otimes \hdots \otimes A_N$, pero ocurre un error en  el aparato y confunde los resultados.

Se redefine la operación difusa, para un sistema de $N$ partículas utilizando los operadores de permutación y se mencionan algunas características de los operadores de permutación. 

Se calcula el valor esperado de realizar la medición del nuevo observable.
  %Se plantea la primera alternativa de instrumento cuántico utilizando la %nueva operación difusa y el nuevo observable, a su vez se calcula el valor %esperado. 
  
  %De igual forma, se plantea la segunda alternativa y su valor esperado. 
  
  %Finalmente se demuestra la proposición que explica en que condiciones los %instrumentos para un sistema de $N$ partículas son equivalentes.


En sistema de $N$ partículas en el que se realiza una medición de un observable de la forma $A_1\tensor A_2\tensor \hdots \tensor A_N$, y en el que ocurre un error en el aparato de medición y confunde los resultados. Con cierta probabilidad se realiza la medición de acuerdo a la permutación $\Pi$ del observable. 

La operación difusa definida en la ecuación ({\ref{eq:operador_difuso2p}}), ahora se pude redefinir para un sistema de $N$ partículas de forma más general como\begin{equation}\label{eq:fm-nparticles}
   \fuzzy{\rho}=\sum_{\Pi_i\in S}p_{i}\permut{i}{\rho}
\end{equation}donde $\mathcal{S}$ es un subconjunto del grupo simétrico de $n$ partículas y $\sum_{i=1}^{N!} p_i=1$.  Los elementos de este grupo simétrico $\mathcal{S}$ pueden escribirse como producto de operaciones de intercambio $S_{ij}$. Si se requiere un número par de operadores de intercambio $S_{ij}$, se dice que la permutación tiene paridad par, de otro modo, se dice que la permutación tiene paridad impar. Algunas de las propiedades destacables de los operadores de permutación es que son unitarios, el operador adjunto tiene la misma paridad y no son hermíticos.

Ahora que se generaliza el operador difuso, puede utilizarse para obtener el valor esperado de esta medición para el sistema de $N$ partículas {\cite{Pineda_2021}}\begin{equation}\label{eq:ExpectedValue-generalForm}\begin{split}
    \left \langle \prodtensor A_i\right \rangle_{\mathcal{F}(\rho)} &=\tr\left(\fuzzy{\rho}\prodtensor A_i\right)\\
    &=\sum_{\Pi_j \in S}p_{j}\tr \left(\permut{j}{\rho} \prodtensor A_i\right).
\end{split}
\end{equation} 

Aunque los operadores del grupo simétrico, en general no son hermíticos, la operación difusa sí lo es y por tanto usando la propiedad cíclica de la traza, el valor esperado de la medición difusa puede escribirse también como \begin{equation}\label{eq:ExpectedValue-generalForm-2}\begin{split}
    \left \langle \prodtensor A_i\right \rangle_{\mathcal{F}(\rho)} &=\sum_{\Pi_j \in S}\tr \left(\fuzzy{\prodtensor A_i}\rho\right).
\end{split}
\end{equation} 



\subsection{Medidas POVM en sistemas de varias partículas}
En el capítulo anterior se describió el mapeo que puede realizarse con las medidas POVM, el cual será conveniente para proporcionar la probabilidad de cada posible salida de la medición. En un sistema de dos partículas se propuso intuitivamente un conjunto de efectos cuya generalización para $N$ partículas puede ser escrita como \begin{equation}\label{eq:effectsSetNp}
    {\{E_{\lambda_i}\}}_{\lambda_i \in \Lambda}={\{\fuzzy{P_{\lambda_i}}\}}_{\lambda_i \in \Lambda}={\left\{\sum_{\Pi_j \in S} p_j \permut{j}{P_{\lambda_i}}\right\}}_{\lambda_i \in \Lambda},
\end{equation}  
donde $P_{\lambda_i}$ es el operador de proyección correspondiente a cada vector propio que tiene asociado el valor propio $\lambda_i\in \Lambda$. El conjunto de valores propios del observable es \begin{equation}\label{eq:lambdaeigenvalues}
    \Lambda=\{a_{11}\cdot a_{21}\cdot \hdots \cdot a_{N1},\hdots,a_{1J}\cdot a_{2K}\cdot \hdots \cdot a_{NM}\},
\end{equation} donde $a_{jk}$ es el $k$-ésimo valor propio correspondiente al observable $A_j$. Es fácilmente comprobable que estos operadores $E_{\lambda_i}$ son hermíticos, cumplen con la propiedad de completitud y son positivos \rrnote{Creo que deberia probarlo.}.

Para obtener el estado posterior a la medición se requiere utilizar la descomposición de los efectos en operadores de Kraus $\{K_{\lambda_i}\}$. Para mediciones difusas en sistema de $N$ partículas se propone utilizar \begin{equation}
   K_{\lambda_i}=\sqrt{\sum_{\Pi_j \in S} p_j \permut{j}{P_{\lambda_i}}},
\end{equation} esto se puede realizar debido a la positividad de los efectos. De nuevo, esta descomposición no es única y el estado posterior de la medición dependerá de como se implementen las medidas POVM en el laboratorio. 



\subsection{Instrumentos cuánticos en sistemas de \texorpdfstring{\boldmath{$N$}}{N} partículas}
En esta parte también se consideran las dos alternativas de instrumentos cuánticos de la sección ({\ref{sec:instrumentos-cuanticos}}). 


La primera alternativa es el instrumento en el que las partículas se intercambian y luego se aplica una medición proyectiva \begin{equation}\label{eq:1instrumentnp}
    \mathcal{I}_1(\rho)=\sum_{\lambda_i \in \Lambda }P_{\lambda_i}\otimes P_{\lambda_i}\fuzzy{\rho}P_{\lambda_i},
\end{equation} donde $P_{\lambda_i}$ son los operadores de proyección y $\lambda_i \in \Lambda$ son los valores propios del observable.  


El valor esperado del resultado de la medición modelado con este instrumento puede calcularse de la siguiente manera \begin{equation*}
    \begin{split}
        \left \la \prodtensor A_i \right \ra_{\mathcal{I}_1}&=\tr\left( \left[\left(\prodtensor A_i\right) \otimes \mathds{1}\right]\mathcal{I}_1\right) \\
        &=\tr\left(\left[ \left(\prodtensor A_i\right)\otimes \mathds{1}\right]\sum_{\lambda_j \in \Lambda}P_{\lambda_j}\otimes P_{\lambda_j}\fuzzy{\rho}P_{\lambda_j} \right)\\
        &=\sum_{\lambda_j\in \Lambda} \tr\left(\left(\prodtensor A_i\right) P_{\lambda_j}\right) \tr\left(P_{\lambda_j}\fuzzy{\rho} P_{\lambda_j}\right) \\
        &=\sum_{\lambda_j\in \Lambda} \tr\left(\sum_{{\lambda_j, \lambda_k \in \Lambda}}\lambda_k P_{\lambda_k} P_{\lambda_j}\right) \tr\left(P_{\lambda_j}\fuzzy{\rho} P_{\lambda_j}\right)  \\
        &=\sum_{\lambda_j \in\Lambda} \tr\left(\lambda_j P_{\lambda_j}\right) \tr\left(P_{\lambda_j}\fuzzy{\rho}P_{\lambda_j}\right) \\
        &=\sum_{\lambda_j \in \Lambda} \lambda_j \tr\left(P_{\lambda_j}\fuzzy{\rho}\right) \\
    \end{split}
\end{equation*} con lo que se puede concluir que el valor esperado correspondiente a este instrumento es \begin{equation}\label{eq:valor-esperado-1instrumentnp}
        \left \la \prodtensor A_i \right \ra_{\mathcal{I}_1}= \tr\left( \prodtensor A_i \fuzzy{\rho}\right),
\end{equation}el mismo que el valor esperado correcto ({\ref{eq:ExpectedValue-generalForm}}).

La segunda alternativa presentada en ({\ref{second-instrument}}) es igualmente generalizable para un sistema de $N$ partículas como \begin{equation}\label{eq:second-instrumentnp}
    \mathcal{I}_2(\rho)= \sum_{\lambda_i \in \Lambda } \fuzzy{P_{\lambda_i}}\tensor P_{\lambda_i}\rho P_{\lambda_i},
\end{equation}  en esta alternativa se interpreta una confusión en la lectura de los resultados.




Con esta alternativa el valor esperado se calcula como 
\begin{equation*}
    \begin{split}
        \left \la \prodtensor A_i \right \ra_{\mathcal{I}_2}&=\tr\left( \left[\left(\prodtensor A_i\right) \otimes \mathds{1}\right]\mathcal{I}_2\right) \\
        &=\tr\left(\left[ \left(\prodtensor A_i\right)\otimes \mathds{1}\right]\sum_{\lambda_j \in \Lambda}\fuzzy{P_{\lambda_j}}\otimes P_{\lambda_j}{\rho}P_{\lambda_j} \right)\\
        &=\sum_{\lambda_j\in \Lambda} \tr\left(\prodtensor A_i\fuzzy{P_{\lambda_j}}\right) \tr\left(P_{\lambda_j}\rho\right). \\
    \end{split}
\end{equation*} finalmente el valor esperado es \begin{equation}\label{eq:valor-esperado-2instrumentnp}
    \left \la \prodtensor A_i \right \ra_{\mathcal{I}_2}=\sum_{\lambda_j\in \Lambda} \tr\left(\prodtensor A_i\fuzzy{P_{\lambda_j}}\right) \tr\left(P_{\lambda_j}\rho\right).
\end{equation} Este valor esperado no corresponde a (\ref{eq:ExpectedValue-generalForm}) por lo que análogamente a (\ref{prop:Equivalencia-instruments}) se tiene una proposición más general para la equivalencia de estos instrumentos en sistema de $N$ partículas.

\begin{proposition}\label{prop:Equivalencia-instrumentos-np}
    Para todo estado inicial $\rho$, los instrumentos cuánticos {\ref{eq:1instrumentnp}} y {\ref{eq:second-instrumentnp}} son equivalentes si y solo si \[\left \langle \lambda_j \left|{\permut{l}{\prodtensor A_i}}\right|\lambda_k\right\rangle=0,\forall j\ne k \text{ y } \forall \Pi_l \in \mathcal{S}\]
\end{proposition}

\begin{proof}
    $(\Rightarrow)$ Suponiendo que para todo estado inicial $\rho$ los valores esperados de los instrumentos son iguales \[ \tr\left( \prodtensor A_i \fuzzy{\rho}\right)=\sum_{\lambda_j\in \Lambda} \tr\left(\prodtensor A_i\fuzzy{P_{\lambda_j}}\right) \tr\left(P_{\lambda_j}\rho\right)\]
o bien, \[ \tr\left( \fuzzy{\prodtensor A_i }\rho\right)=\sum_{\lambda_j\in \Lambda} \tr\left(\fuzzy{\prodtensor A_i}P_{\lambda_j}\right) \tr\left(P_{\lambda_j}\rho\right).\] Aplicando la linealidad de la traza, \[ \tr\left( \fuzzy{\prodtensor A_i }\rho\right)=\tr\left(\sum_{\lambda_j\in \Lambda} \tr\left(\fuzzy{\prodtensor A_i}P_{\lambda_j}\right) P_{\lambda_j}\rho\right),\]de ello  \[ \begin{split}\tr\left( \left(\fuzzy{\prodtensor A_i }-\sum_{\lambda_j\in \Lambda} \tr\left(\fuzzy{\prodtensor A_i}P_{\lambda_j}\right) P_{\lambda_j}\right)\rho\right)&=0\\,\end{split}\] para todo estado inicial $\rho$ entonces\[\begin{split}\fuzzy{\prodtensor A_i }-\sum_{\lambda_j\in \Lambda} \tr\left(\fuzzy{\prodtensor A_i}P_{\lambda_j}\right) P_{\lambda_j}&=0\\
    \sum_{\Pi_l\in \mathcal{S}}p_l\permut{l}{\prodtensor A_i }-\sum_{\lambda_l\in \Lambda} \tr\left(\sum_{\Pi_l\in \mathcal{S}}p_l\permut{l}{\prodtensor A_i}P_{\lambda_j}\right) P_{\lambda_j}&=0, 
\end{split}\] de esta última ecuación se deduce que para cada $j$ se puede ver que \[\begin{split}
    \permut{l}{\prodtensor A_i }-\sum_{\lambda_l\in \Lambda} \tr\left(\permut{l}{\prodtensor A_i}P_{\lambda_j}\right) P_{\lambda_j}&=0.
\end{split}\] 

Si el primer término se escribe en la base de los vectores propios del observable que se desea medir  \[\begin{split}
    \sum_{\lambda_j, \lambda_k\in \Lambda} \tr\left(\permut{l}{\prodtensor A_i}|\lambda_j\rala \lambda_k|\right) |\lambda_j\rala\lambda_k|-\sum_{\lambda_j\in \Lambda} \tr\left(\permut{l}{\prodtensor A_i}P_{\lambda_j}\right) P_{\lambda_j}&=0,\\
    \sum_{\lambda_j\neq\lambda_k\in \Lambda} \tr\left(\permut{l}{\prodtensor A_i}|\lambda_j\rala \lambda_k|\right) |\lambda_j\rala\lambda_k|&=0,
\end{split}\] finalmente por la independencia lineal de los operadores\[ \tr\left(\permut{l}{\prodtensor A_i}|\lambda_j\rala \lambda_k|\right)= \left\langle\lambda_j\left|\permut{l}{\prodtensor A_i}\right|\lambda_k\right\rangle=0, \forall j\neq k, \forall \Pi_l \in \mathcal{S}.\]



$(\Leftarrow)$ Suponiendo que se cumple que \[ \left\langle\lambda_j\left|\permut{l}{\prodtensor A_i}\right|\lambda_k\right\rangle=0, \forall j\neq k, \forall \Pi_l \in \mathcal{S},\] se puede  escribir a la permutación $l$ como una combinación lineal de los operadores de proyección\[\permut{l}{\prodtensor A_i}=\sum_{\lambda_k \in \Lambda}d_{lk}P_{k}.\]Entonces en el valor esperado ({\ref{eq:valor-esperado-2instrumentnp}})
 \[\begin{split}\left \la \prodtensor A_i \right \ra_{\mathcal{I}_2}&=\sum_{\lambda_j\in \Lambda} \tr\left(\sum_{l}p_l{\sum_{\lambda_k \in \Lambda}d_{lk}P_{\lambda_k}}P_{\lambda_j}\right) \tr\left(P_{\lambda_j}\rho\right)\\ &=\sum_{\lambda_j\in \Lambda} \sum_{l}p_l d_{lj}\tr\left(P_{\lambda_j}\right)\\ &=\sum_{l}p_l \tr\left( \sum_{\lambda_j\in \Lambda} d_{lj}P_{\lambda_j}\rho\right)\\ &=\sum_{l}p_l\tr\left(\permut{l}{\prodtensor A_i}\rho\right)\\ 
    &=\tr\left(\fuzzy{\prodtensor A_i}\rho\right)=\left \la \prodtensor A_i \right \ra_{\mathcal{I}_1}
\end{split}\]

\end{proof}

\end{comment}




\section{Observables degenerados en mediciones difusas}

\rrnote{Explorar las consecuencias de tener operadores degenerados en las implementaciones de las medidas POVM y de los instrumentos.}


\section{Observables no factorizables en mediciones difusas}

\rrnote{Explorar que implicaciones tiene realizar mediciones estos observables.}


\section{Casos particulares de mediciones difusas}
\rrnote{En esta sección se desea ilustrar ejemplos de sistemas de $N$ partículas, como una cadena o un anillo de iones,  en los que se realicen mediciones difusas y poderlos describir completamente.}


\rrnote{ Realizar alguna ilustracion o simulación sobre estos ejemplos.}

