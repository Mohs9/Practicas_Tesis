 \chapter[RESULTADOS]{3. Resultados}\label{cap3:resultados}
\section{Introducción} % {{{

    

En los capítulos anteriores se estableció un base conceptual para realiza un
análisis de las mediciones difusas. En este capítulo se utilizará este marco
para realizar la descripción completa de las mediciones difusas en sistemas
cuánticos. Asimismo se presentan los resultado más relevantes que se obtuvieron
al trabajar con las distintas herramientas presentadas anteriormente. 

La estructura de este capítulo es la siguiente. En la sección {\ref{sec:Cap3
dos particulas}} se inicia la discusión de los resultados principales que se
obtuvieron al analizar el sistema más simple, el sistema de dos partículas. En
esta sección se presentan tanto las medidas POVM como los operadores de Kraus
que proporcionan una correcta descripción de la medición difusa. De igual
manera,  se presentan de manera intuitiva y más detallada las interpretaciones
de los instrumentos cuánticos, así como las condiciones en las que éstos son
equivalentes. Finalmente se expone un ejemplo específico que recopila todos
estos elementos para ilustrar de mejor manera la base teórica trabajada. En la
sección {\ref{sec:cap3 generalizacion-}} se examina y aborda la generalización
de todas las herramientas trabajadas en la sección {\ref{sec:Cap3 dos
particulas}}, para sistemas de $N$ partículas.

% }}}
\section{Mediciones difusas en sistemas de dos partículas}\label{sec:Cap3 dos particulas} % {{{
% Intro {{{
Las mediciones difusas solo cobran sentido en un sistema compuesto por dos o
más partículas. Con fines explicativos en el capítulo anterior se comenzó a
analizar una medición difusa en el sistema más simple, el cual está conformado
únicamente por dos partículas. Antes de ampliar el análisis a sistemas más
complejos, con el propósito de acabar de clarificar e ilustrar de manera
intuitiva el concepto de medición difusa, en esta sección se presentan los
resultados principales y algunos ejemplos obtenidos al trabajar con un sistema
que tiene un estado inicial $\rho \in \mathcal{H}_1\otimes \mathcal{H}_2$ y un
observable de la forma $A\otimes B$.\par




En la siguiente imagen se ilustra un esquema de medición del observable
$\sigma_z\otimes \sigma_x$ en la que debido a ruido del entorno, existe una
posibilidad de realizar la medición errónea y en su lugar medir el observable
$\sigma_x\otimes \sigma_z$. La salida clásica de los observables es $(1)$  para
el observable $\sigma _z$ y $(-1)$ para el observable $\sigma_x$. Sin embargo
su valor esperado será la suma ponderada de los valores esperados de realizar
las dos mediciones que son verosímiles, es decir $p \tr(\sigma_z\otimes \sigma
\rho )+ (1-p)\tr(\sigma_x\otimes\sigma_z \rho)$.

\begin{figure}[H]
    \centering
    
  \begin{subfigure}[a]{1\textwidth}
    \centering
    \resizebox{12.5cm}{!}{\begin{quantikz}[font=\small, row sep=0.3cm,column sep=0.4cm]%
    \lstick[wires=3]{$\rho$}& \qw& \qw& \qw& \gate[wires=3,style={
    starburst,fill=yellow,draw=red,line width=2pt,inner
    xsep=-2pt,inner ysep=-5pt},label style=blue]{\text{
    noise}} & \qw& \qw& \qw& \meter[style={draw=blue}]{$\sigma_z$} \vcwdouble{1-9}{5-9}{1}&\qw&\qw&\qw&\rstick[wires=5]{$p\tr(\sigma_z\otimes \sigma_x\rho)+(1-p)\tr(\sigma_x\otimes\sigma_z\rho)$}\\
    &&&&&&&&&&&&\\
    & \qw&\qw&\qw& \qw& \qw& \qw& \qw& \qw&\qw&\meter[style={draw=blue}]{$\sigma_x$}\vcwdouble{3-11}{5-11}{-1}\qw&\qw&\\
    &&&&&&&&&&&&\\
    \lstick{cl}& \cw& \cw& \cw& \cw& \cw& \cw& \cw&\cw&\cw&\cw&\cw&\\
  \end{quantikz}}
\hfill
\caption{}
\end{subfigure}
\hfill

\begin{subfigure}{0.4\textwidth}
  \begin{flushleft}
    \resizebox{7cm}{!}{
      \begin{quantikz}[font=\small,row sep=0.3cm,column sep=0.4cm]%
      \lstick[wires=2]{$\rho$}&   \qw& \qw& \meter[style={draw=blue}]{$\sigma_z$} \vcwdouble{1-4}{4-4}{1}&\qw&\qw&\qw\rstick[wires=4]{$p \tr(\sigma_z \tensor \sigma_x \rho)$}&\\[0.1cm]
       &  \qw& \qw& \qw& \qw&\meter[style={draw=blue}]{$\sigma_x$}\vcwdouble{2-6}{4-6}{-1}&\qw& \\[0.1cm]
      &&&&&&&\\
      \lstick{cl}& \cw&  \cw&\cw&\cw&\cw&\cw&\\
    \end{quantikz}}
    \hfill
  \caption{}
  \end{flushleft}
   
\end{subfigure}


%  \begin{subfigure}[a]{0.5\textwidth}
 % \centering
  %\resizebox{6cm}{!}{
   % \begin{quantikz}[scale=1, row sep=0.3cm,column sep=0.4cm]%
    %\lstick{$\rho_{1}$}&  \qw& \qw& \qw& \qw& \qw&\qw\rstick[wires=3]{}\\ \\
    %&&&&&&& |[meter]| \vcwdouble{2-8}{5-8}{a_jb_k}\rstick{$A \tensor B$} \\
    %\lstick{$\rho_2$} & \qw& \qw& \qw& \qw& \qw&\qw&\\
    %\meter[style={draw=blue}]{B}\vcwdouble{2-11}{4-11}{b_k} \\
  %  & & && &&&\\
   % \lstick{cl}&  \cw& \cw& \cw& \cw&\cw&\cw&\cw\\
  %\end{quantikz}}
  %\hfill
%\caption{}
%\end{subfigure}

\begin{subfigure}{0.4\textwidth}
  \centering
\resizebox{8cm}{!}{
\begin{quantikz}[font=\small,scale=1, row sep=0.3cm,column sep=0.4cm]%
  \lstick[wires=2]{$\rho$}&\qw& \gate[swap]{} & \qw&  \meter[style={draw=blue}]{$\sigma_z$} \vcwdouble{1-5}{4-5}{ 1}&\qw&\qw&\qw\rstick[wires=4]{$(1-p)\tr(\sigma_x \tensor \sigma_x\rho)$}& \\
  & \qw& \qw& \qw& \qw& \qw&\meter[style={draw=blue}]{$\sigma_x$}\vcwdouble{2-7}{4-7}{-1}&\qw& \\
  &&&&&&&&\\
  \lstick{cl}& \cw& \cw& \cw&\cw&\cw&\cw&\cw&\\
\end{quantikz}}
\hfill
\caption{}
\end{subfigure}
\caption{\textbf{(a)} En la primera imagen se ilustran un estado inicial $\rho$, sin embargo el entorno no es ideal y es posible que ocurra una identificación errónea, en las que las salidas clásicas registradas para los observables $\sigma_z$ y $\sigma_x$ fueron $1$ y $-1$ respectivamente.\textbf{ (b)} En esta imagen se ilustra que con una probabilidad $p$ la identificación del sistema sea correcta y se realice la medición del observable $\sigma_z\tensor \sigma_x$.\textbf{ (c)} La imagen muestra que debido al ruido del entorno, las partículas experimentan un intercambio y con una probabilidad $(1-p)$ se realiza la medición del observable $\sigma_x\otimes \sigma_z$ }\label{diagrama-cajas}\source{elaboración propia}
\end{figure}


% }}}
\subsection{POVM y operadores de Kraus para mediciones difusas en sistemas de dos partículas}\label{Sec_POVM_para_mediciones_difusas} % {{{
 

A continuación se especifica la forma de utilizar las medidas generalizadas conjuntamente con los operadores de Kraus para aproximarse a describir enteramente estas mediciones. En la ecuación {\eqref{eq:ordenar-valoresperado-povm}} se encontró intuitivamente que los efectos para una medición difusa, son $\{\mathcal{F}_{2\text{p}}({P_{a_j,b_k}})\}$. Estos efectos proporcionan el mapeo de probabilidades\begin{equation}\label{eq:mapeo-probabilidades-2p}
    E(a_j b_k, \rho)= \tr(\mathcal{F}_{2\text{p}}({P_{a_j,b_k}})\rho).
    \end{equation} Sin embargo para obtener el estado posterior es a la medición es necesario descomponer estos efectos. Debido a que los operadores $P_{a_j,b_k }$ y $P_{b_k,a_j}$ son  positivos, entonces  $\mathcal{F}_{2\text{p}}({P_{a_j,b_k}})$ es positivo y por tanto se puede considerar los siguientes operadores\[K_{a_j,b_k}=\sqrt{\mathcal{F}_{2\text{p}}({P_{a_j,b_k}})}.\] Luego de introducir los operadores de Kraus el
    estado de la medición es posterior a la medición POVM será
    \begin{equation}\label{eq:estado-posterior-povm1-2p}\begin{split}\rho'_{a_j,b_k}&=\dfrac{K_{a_j,b_k} \rho
    K_{a_j,b_k}^\dagger}{\tr(K_{a_j,b_k}^\dagger K_{a_j,b_k} \rho)}\\
    &=\dfrac{\sqrt{\mathcal{F}_{2\text{p}}(P_{a_j,b_k})} \rho
    \sqrt{\mathcal{F}_{2\text{p}}({P_{a_j,b_k}})}}{\tr(\mathcal{F}_{2\text{p}}({P_{a_j,b_k}}) \rho)}\\
    &=\dfrac{\sqrt{p P_{a_j,b_k}+(1-p)SP_{a_j,b_k}S^\dagger}\rho\sqrt{p P_{a_j,b_k}+(1-p)SP_{a_j,b_k}S^\dagger}}{\tr((p P_{a_j,b_k}+(1-p)SP_{a_j,b_k}S^\dagger) \rho)}\\
    &=\dfrac{\sqrt{p P_{a_j,b_k}+(1-p)P_{b_k,a_j}}\rho\sqrt{p P_{a_j,b_k}+(1-p)P_{b_k,a_j}} }{\tr((p P_{a_j,b_k}+(1-p)P_{b_k,a_j}) \rho)}.
    \end{split}\end{equation}La descomposición en operadores de Kraus
    no es única. No obstante, el conjunto de operadores $\{K_{a_j,b_k}\}$ es adecuado para ofrecer una especificación detallada de las mediciones difusas, puesto que con ellos se puede obtener tanto el mapeo de probabilidades $\tr(K_{a_j,b_k}K_{a_j,b_k}^\dagger \rho)$, como el estado posterior a la medición $\rho_{a_j,b_k}'$.



% }}}
\subsection{Instrumentos cuánticos para dos partículas}\label{subsec:Instrumentos_cuanticos_2p_3cap} % {{{
% Intro subsection {{{

En este apartado se busca hacer un análisis sobre las formulaciones de las tres
alternativas de los instrumentos presentadas en la sección
{\ref{sec:cap2instrumentos-cuanticos}}. En éste, se obtiene el sistema clásico
y cuántico para comparar las descripciones que proporcionan. Para una plena
especificación de las mediciones difusas, las interpretaciones son
intuitivamente equivalentes. Sin embargo, es indispensable verificar que las
dos proporcionen el mismo mapeo de resultados. En principio, una condición
necesaria para corroborar que tengan la misma distribución de probabilidad es
que el valor esperado de los instrumentos genere el mismo resultado que el de
la ecuación {\eqref{eq:Expected-Value-FM-2p}}. 


% }}}
\subsubsection{Primera alternativa de instrumento cuántico} % {{{

El primer instrumento consiste en considerar que debido a un error en el
sistema cuántico, las partículas posiblemente experimenten un intercambio. Al
medir un observable ${A \tensor B}$ en el sistema representado por el estado
$\rho$, se obtiene un resultado  en el sistema clásico que puede ser cualquiera
de los valores propios de este observable. Luego, con cierta probabilidad el
estado posterior será la proyección del estado inicial del sistema al estado
propio correspondiente a la salida proporcionada. Sin embargo también es
posible que el estado inicial sufra una transformación, y sea este cambio del
sistema el que se proyecte en el espacio propio correspondiente a la salida del
observable $A\tensor B$.


En la imagen {\ref{diagrama-cajas-primer-instrumento}} se puede apreciar a lo
que esto se refiere gráficamente.  En la figura, con una probabilidad $p$ se
realiza una medición ideal en la que se mide el observable $A$ en la primera
partícula y el observable $B$ en la segunda partícula. En esta medición ideal
el estado posterior será la proyección de $\rho_1=\tr_2(\rho_{1,2})$ al espacio
propio correspondiente a la salida proporcionada en el sistema clásico $a_j$,
conjuntamente con la proyección de $\rho_2=\tr_1(\rho_{1,2})$ al espacio propio
de las salida $b_k$.  No obstante, la figura también muestra la parte borrosa
de la medición en la que con una probabilidad $1-p$, la medición del observable
$B$ se realiza en el primer sistema y la medición del observable $A$ se realiza
en el segundo sistema, en consecuencia $\rho_2$ se proyecta al espacio propio
que corresponde a la salida $a_j$ y $\rho_1$ se proyecta al espacio de la
salida $b_k$.

\begin{figure}[H]
  \centering
\begin{subfigure}[b]{0.4\textwidth}
\centering
   \resizebox{7cm}{!}{
      \begin{quantikz}[font=\normalsize,row sep=0.3cm,column sep=0.4cm]%
      \lstick[wires=4]{$p$}&&\lstick{\color{blue}{$\rho_{1}$}}& \qw& \qw& \meter[style={draw=blue}]{$A$} \vcwdouble{1-6}{4-6}{}&\qw&\qw&\qw&\qw\rstick{$P_{a_j}${\color{blue}{$\rho_1$}}$P_{a_j}$}&\\[0.1cm]
      &&\lstick{\color{red}{$\rho_2$}} &  \qw& \qw& \qw& \qw&\meter[style={draw=red}]{$B$}\vcwdouble{2-8}{4-8}{}&\qw&\qw\rstick{$P_{b_k}${\color{red}{$\rho_2$}}$P_{b_k}$} \\[0.1cm]
      &&&&&&&&&&&\\
      &&\lstick{cl}& \cw&  \cw&\cw&\cw&\cw&\cw&\cw&\\
      &&&&&\rstick{$a_j$}&&\rstick{$b_k$}&&&&\\
    \end{quantikz}
    }
    \hfill
  \caption{}\label{fig:1-instrumento-ideal}
\end{subfigure}

\hfill

\begin{subfigure}[c]{0.4\textwidth}
\centering
  \resizebox{9cm}{!}{
\begin{quantikz}[font=\normalsize,scale=1, row sep=0.3cm,column sep=0.4cm]%
  \lstick[wires=4]{$1-p$}&&\lstick{\color{blue}{$\rho_{1}$}}&\qw& \gate[swap]{} & \qw&  \meter[style={draw=blue}]{$A$} \vcwdouble{1-7}{4-7}{}&\qw&\qw&\qw&\qw\rstick{$P_{a_j}${\color{red}{$\rho_2$}}$P_{a_j}$}\\
  &&\lstick{\color{red}{$\rho_{2}$}} & \qw& \qw& \qw& \qw& \qw&\meter[style={draw=red}]{$B$}\vcwdouble{2-9}{4-9}{}&\qw&\qw \rstick{$P_{b_k}${\color{blue}{$\rho_1$}}$P_{b_k}$}\\
  &&&&&&&&&&&\\
  &&\lstick{cl}& \cw& \cw& \cw&\cw&\cw&\cw&\cw&\cw&\\
  &&&&&&\rstick{$a_j$}&&\rstick{$b_k$}&&&\\
\end{quantikz}
}
\hfill
\caption{}\label{fig:1-instrumento-non-ideal}
\end{subfigure}
\caption{\textbf{(a)}Con una probabilidad $p$ se realiza una medición ideal del observable $A\tensor B$, con un estado posterior dado por $P_{a_j,b_k}\rho_1\otimes \rho_2P_{a_j,b_k}$, cuando la salida indicada es $a_j b_k$. \textbf{ (b)}Con probabilidad $1-p$ ocurre un intercambio de las partículas y estado posterior correspondiente a la salida $a_j b_k$ será $P_{a_j,b_k}\rho_2\otimes \rho_1P_{a_j,b_k}$.}\label{diagrama-cajas-primer-instrumento}\source{elaboración propia}
\end{figure}


Para un sistema de dos partículas el instrumento dado en la ecuación {\eqref{eq:primer-instrumento-general}}, puede escribirse como\begin{equation}\label{eq:fisrtinstrument2p}
    \begin{split}
        \mathcal{I}_1{(\rho)}&=\sum_{j,k}P_{a_j,b_k}\otimes P_{a_j,b_k} \mathcal{F}_{2\text{p}}(\rho) P_{a_j,b_k}\\
        &=\sum_{j,k}P_{a_j,b_k}\otimes[p P_{a_j,b_k}\rho P_{a_j,b_k}+(1-p)P_{a_j,b_k}S\rho S^\dagger P_{a_j,b_k}]\\
        &=\sum_{j,k}P_{a_j,b_k}\otimes P_{a_j,b_k} [p \bra{a_j b_k}\rho \ket{a_j,b_k}+(1-p)\bra{b_k a_j }\rho \ket {b_k a_j}],
\end{split}
\end{equation}

Es importante notar que el sistema clásico  puede diferir del estado inicial,
pero  indica correctamente el estado posterior a la medición en todos los
casos. Además, para una medida no selectiva el mapeo de resultados se obtiene
simplemente aplicando una traza parcial sobre el sistema cuántico 
\begin{equation}\label{eq:traza_q_1instrument2p}
    \begin{split}
        \tr_{\text{qu}}(\mathcal{I}_1{(\rho)})&=\sum_{j,k}\tr_{\text{qu}}(P_{a_j,b_k}\otimes P_{a_j,b_k} \mathcal{F}_{2\text{p}}(\rho) P_{a_j,b_k})\\
        &=\sum_{j,k}P_{a_j,b_k}\tr(p P_{a_j,b_k}\rho P_{a_j,b_k}+(1-p)P_{a_j,b_k}S\rho S^\dagger P_{a_j,b_k})\\
        &=\sum_{j,k}P_{a_j,b_k}[p \bra{a_j b_k}\rho\ket{a_j b_k} +(1-p)\bra{b_k a_j}\rho \ket{b_k a_j}].
\end{split}
\end{equation} 


Para obtener el mapa de resultados para una medición selectiva simplemente se
multiplica el operador de proyección correspondiente a la salida obtenida al
resultado anterior, y se calcula la traza\[\begin{split}
    E(a_i,b_l,\rho)&=\tr\left(P_{a_i, b_l}\left[\sum_{j,k}\tr_{\text{qu}}(P_{a_i,b_l}\otimes P_{a_i,b_l} \mathcal{F}_{2\text{p}}(\rho) P_{a_j,b_k})\right]\right)\\
    &=\tr\left(\sum_{j,k} P_{a_i,b_l}P_{a_j,b_k}[p \bra{a_j b_k}\rho\ket{a_j b_k} +(1-p)\bra{b_k a_j}\rho \ket{b_k a_j}]\right)\\
    &=\sum_{j,k}\tr\left( P_{a_i,b_l}P_{a_i,b_l}\right)[p \bra{a_j b_k}\rho\ket{a_j b_k} +(1-p)\bra{b_k a_j}\rho \ket{b_k a_j}]\\
    &=\sum_{j,k}\delta^{i,l}_{j,k}[p \bra{a_j b_k}\rho\ket{a_j b_k} +(1-p)\bra{b_k a_j}\rho \ket{b_k a_j}]\\
    &=p \bra{a_i b_l}\rho\ket{a_i b_l} +(1-p)\bra{b_l a_i}\rho \ket{b_l a_i}.
\end{split}\]Es importante notar que este mapeo de probabilidades resultante es exactamente el mismo al de la ecuación {\eqref{eq:mapeo-probabilidades-2p}}

Por otro lado, para obtener el estado posterior luego de haber realizado la medición, se debe realizar la traza parcial sobre el sistema clásico de la siguiente manera
\begin{equation}\label{eq:traza_cl_1instrument2p}
    \begin{split}
      \tr_{\text{cl}}(\mathcal{I}_1{(\rho)})&=\sum_{j,k}\tr_{\text{cl}}(P_{a_j,b_k}\otimes P_{a_j,b_k} \mathcal{F}_{2\text{p}}(\rho) P_{a_j,b_k})\\
    &=\sum_{j,k}P_{a_j,b_k} \mathcal{F}_{2\text{p}}(\rho) P_{a_j,b_k}.
\end{split}
\end{equation}Para una medida selectiva el estado posterior estará dado por
\[\E_{a_j,b_k}(\rho)=\dfrac{P_{a_j,b_k} \mathcal{F}_{2\text{p}}(\rho) P_{a_j,b_k}}{\tr(P_{a_j,b_k} \mathcal{F}_{2\text{p}}(\rho))}.\]



Por otra parte, el valor esperado del primer instrumento se calcula de la siguiente forma \begin{equation*}
    \begin{split}
        \la A\otimes B \ra_{\mathcal{I}_1}&=\tr([(A\otimes B) \otimes \mathds{1}]\mathcal{I}_1) \\
        &=\tr\left([(A\otimes B) \otimes \mathds{1}]\sum_{jk}P_{a_j b_k}\otimes (pP_{a_j b_k}\rho P_{a_j b_k}+ (1-p)P_{a_j b_k}S\rho S P_{a_j b_k})\right)\\
        &=\sum_{jk} \tr((A\otimes B) P_{a_j b_k}) \tr\left((pP_{a_j b_k}\rho P_{a_j b_k}+ (1-p)P_{a_j b_k}S\rho S P_{a_j b_k})\right) \\
        &=\sum_{jk} \tr\left(\sum_{il}a_{il}P_{a_i b_l} P_{a_j b_k}\right) \tr((pP_{a_j b_k}\rho P_{a_j b_k}+ (1-p)P_{a_j b_k}S\rho S P_{a_j b_k})) \\
        &=\sum_{jk} p\tr(a_{j}b_k P_{a_j b_k}) \tr(P_{a_j b_k}\rho)+ (1-p)\tr(a_{j}b_k P_{a_j b_k})\tr(P_{a_j b_k}S\rho S P_{a_j b_k}) \\
        &=\sum_{jk}  (a_{j}b_k)p  \tr(P_{a_j b_k}\rho)+(a_{j}b_k) (1-p) \tr(P_{a_j b_k}S\rho S P_{a_j b_k}),\\
    \end{split}
\end{equation*}  finalmente el valor esperados del instrumento es \begin{equation}\label{eq:expectedValueFirstInstrument2p}
    \begin{split}
        \la A\otimes B \ra_{\mathcal{I}_1}&=p \tr(A\otimes B \rho)+ (1-p)\tr(B\tensor A \rho ). \\
    \end{split}
\end{equation} Claramente con este instrumento se obtiene el valor esperado correcto para describir una medición difusa.


% }}}
\subsubsection{Segunda alternativa de instrumento cuántico} % {{{

Tras haber analizado el primer instrumento, el siguiente escenario involucra un
error en la interpretación de los resultados. El segundo instrumento implica
que, debido a un fallo en el sistema clásico, la lectura de las salidas sea
incorrecta. Al medir un observable ${A \tensor B}$ en el sistema, con cierta
probabilidad el estado posterior será la proyección del estado inicial del
sistema al estado propio correspondiente a las salida proporcionada. No
obstante, existe la posibilidad que el estado inicial sea proyectado a algún
estado propio del observable pero al leer las salidas de la medición estas
correspondan a salidas del observable $B\otimes A$.

En la figura {\ref{fig; diagrama-cajas-segundo-instrumento}} se ilustra las dos
posibilidades mencionadas anteriormente. En la primera de ellas se realiza una
medición proyectiva ideal, donde se mide el observable $A$ en el primer
subsistema y el observable $B$ en el segundo, con probabilidad $p$. El estado
posterior a la medición será el estado $\rho_1=\tr_2(\rho_{1,2})$ proyectándose
al espacio propio de la salida $a_j$, junto con la proyección del estado
$\rho_2=\tr_1(\rho_{1,2})$ al espacio propio de $b_k$. Adicionalmente, la
figura muestra la parte imperfecta de la medición en la que con una
probabilidad $1-p$, se realiza la medición de $A$ en la primera partícula y la
medición de $B$ en la segunda, sin embargo la salida de la medición para el
primer subsistema es $b_k$,un resultado correspondiente al observable $B$, y la
salida que se ofrece en la medición del segundo subsistema es un valor propio
de $A$, $a_j$. Aunque los estados posteriores a la medición sean correctos, las
salidas clásicas están intercambiadas.




\begin{figure}[H]
    \centering
    
\begin{subfigure}{0.4\textwidth}
  \begin{flushleft}
   \resizebox{7.5cm}{!}{
      \begin{quantikz}[font=\large,row sep=0.3cm,column sep=0.4cm]%
      \lstick[wires=2]{$\rho_{1,2}$}&\qw& \qw&\meter[style={draw=blue}]{$A$} \vcwdouble{1-4}{4-4}{}&\qw&\qw&\qw&\qw\rstick{\small $P_{a_j}${\color{blue}{$\rho_1$}}$P_{a_j}$}&&&&&\rstick[wires=5]{$p$}\\[0.1cm]
       &  \qw& \qw& \qw& \qw&\meter[style={draw=red}]{$B$}\vcwdouble{2-6}{4-6}{}&\qw&\qw\rstick{\small $P_{b_k}${\color{red}{$\rho_2$}}$P_{b_k}$} &&&&\\[0.1cm]
      &&&&&&&&&&&&\\
      \lstick{cl}& \cw&  \cw&\cw&\cw&\cw&\cw&\cw&&&&\\
      &&&\rstick{$a_j$}&&\rstick{$b_k$}&&&&&&&\\
    \end{quantikz}
    }
    \hfill
  \caption{}
  \end{flushleft}
\end{subfigure}

\hfill

\begin{subfigure}{0.4\textwidth}
    \begin{flushleft}
     \resizebox{8.5cm}{!}{
        \begin{quantikz}[font=\large,row sep=0.3cm,column sep=0.4cm]%
          \lstick[wires=2]{\color{black}{$\rho_{1,2}$}}&   \qw& \qw& \meter[style={draw=blue}]{$A$} \vcwdouble{1-4}{4-4}{}&\qw&\qw&\qw&\qw\rstick{\small $P_{a_j}${\color{blue}{$\rho_1$}}$P_{a_j}$}&&&&&\rstick[wires=5]{$1-p$}\\[0.1cm]
         &  \qw& \qw& \qw& \qw&\meter[style={draw=red}]{$B$}\vcwdouble{2-6}{4-6}{}&\qw&\qw\rstick{\small$P_{b_k}${\color{red}{$\rho_2$}}$P_{b_k}$}&&&& \\[0.1cm]
        &&\gate[wires=2,style={
          starburst,fill=yellow,draw=red,line width=2pt,inner
          xsep=1pt,inner ysep=-7pt},label style=blue]{\text{\small
          noise}}&&&&&&&&&&\\
        \lstick{cl}& \cw&\cw&\cw&\cw&\cw&\cw&\cw&&&&&\\
        &&&\rstick{$b_k$}&&\rstick{$a_j$}&&&&&&&\\
      \end{quantikz}
      }
      \hfill
    \caption{}
    \end{flushleft}
  \end{subfigure}
\caption{\textbf{(a)}.\textbf{ (b)}}\label{diagrama-cajas-segundo-instrumento}\source{elaboración propia}
\end{figure}


El segundo instrumento para un sistema de dos partículas es 
\begin{equation}\label{eq:second-instrument-2p}
    \begin{split}
        \mathcal{I}_2(\rho)&=\sum_{j,k}\mathcal{F}_{2\text{p}}(P_{a_j,b_k})\otimes P_{a_j,b_k} \rho P_{a_j,b_k}\\
        &=\sum_{j,k}[pP_{a_j,b_k}+(1-p)SP_{a_j,b_k}S^\dagger]\otimes P_{a_j,b_k} \rho P_{a_j,b_k}\\
        &=\sum_{j,k} p |a_j b_k\rala a_j b_k| \otimes P_{a_j,b_k} \rho P_{a_j,b_k}+(1-p)|b_k a_j\rala b_k a_j|\otimes P_{a_j,b_k} \rho P_{a_j,b_k}.\\
    \end{split}
\end{equation} 



En este caso es posible notar que el sistema clásico no siempre coincide con el estado posterior. La salida de la medición puede indicar que es $\ket{b_k a_j}$, y el estado posterior no será éste. De la misma manera al instrumento anterior, se desarrolla la traza parcial sobre el sistema cuántico del instrumento para obtener el mapeo de resultado  para una medición no selectiva \begin{equation}\label{eq:tr_q_second-instrument-2p}
    \begin{split}
        \tr_{\text{qu}}(\mathcal{I}_2(\rho))&=\sum_{j,k}\tr_{\text{qu}}(\mathcal{F}_{2\text{p}}(P_{a_j,b_k})\otimes P_{a_j,b_k} \rho P_{a_j,b_k})\\
        &=\sum_{j,k} (pP_{a_j,b_k}+(1-p)SP_{a_j,b_k}S^\dagger) \bra{a_j,b_k} \rho \ket{a_j,b_k}.\\
    \end{split}
\end{equation} 



Adicionalmente, el valor esperado de este instrumento, se calcula de la siguiente manera\begin{equation*}
    \begin{split}
        \la A\otimes B \ra_{\mathcal{I}_2}&=\tr([(A\otimes B) \otimes \mathds{1}]\mathcal{I}_2)\\
        &=\tr\left([(A\otimes B) \otimes \mathds{1}]\sum_{j,k}\mathcal{F}(P_{a_j,b_k})\otimes P_{a_j,b_k} \rho P_{a_j,b_k}\right)\\
        &=\sum_{j,k} p\tr((A\otimes B)P_{a_j,b_k}) \tr (P_{a_j,b_k} \rho P_{a_j,b_k})\\&
        +(1-p )\tr((A\otimes B)SP_{a_j,b_k}S) \tr(P_{a_j,b_k} \rho P_{a_j,b_k})\\
        &=\sum_{j,k}[a_{j} b_k p\tr(P_{a_j,b_k})+(1-p)\tr( (B\otimes A)P_{a_j,b_k}) ]\tr(P_{a_j,b_k} \rho),\\
    \end{split}
\end{equation*}simplificando se obtiene que \begin{equation}\label{eq:expectationvalueSecondInstrument2p}
    \begin{split}
        \la A\otimes B \ra_{\mathcal{I}_2}&=p\tr(A\tensor B\rho)+\sum_{j,k}{\color{blue}(1-p)\tr( (B\otimes A)P_{a_j,b_k})}\tr(P_{a_j,b_k} \rho). \\
    \end{split}
\end{equation}
Nótese que el término remarcado en azul difumina la parte de los resultados. Sin embargo, en general esta ecuación no es igual a la ecuación {\eqref{eq:Expected-Value-FM-2p}}, lo cual se discutirá más adelante, en la sección {\ref{subsec:Equivalencia }}.



% }}}
\subsubsection{Tercera alternativa} % {{{
    Esta última interpretación de la medición difusa incluye la posibilidad de no saber en que estado se encuentra el sistema luego de realizar la medición. En este instrumento se representa la posibilidad que al medirse el observable $A\otimes B$ se realice con cierta probabilidad una medición proyectiva ideal. Pero, también es verosímil que se realice una medición en la que no sea posible saber en que espacio se haya proyectado el sistema. Es igual de posible que se haya proyectado sobre un espacio propio del observable $A\otimes B$ como es posible que se haya proyectado en un espacio propio de $B\otimes A$,  condicionados ambos espacios por las salidas clásicas. 

    En la figura {\ref{fig: diagrama-cajas-tercer-instrumento}} se muestra esta tercera interpretación. Primero, se ilustra de la posibilidad de realizar una medición proyectiva ideal del observable $A\otimes B$ con una probabilidad $q$, al igual que en los dos casos anteriores. Sin embargo con una probabilidad $1-q$, se realiza una medición en la que no es posible saber si la partículas se intercambian o no, por esta razón el estado será proyectado al espacio dado por la superposición de los operadores de proyección de $A\otimes B$ y $B\otimes A$ correspondientes a la salidas $a_j$ y $ b_k$, las cuales son indicadas por el sistema clásico, es decir los operadores  $P_{a_j,b_k}$ y $P_{b_k,a_j}$ respectivamente. 





\begin{figure}[H]
  \centering
\begin{subfigure}[b]{0.4\textwidth}
\centering
   \resizebox{7cm}{!}{
      \begin{quantikz}[font=\normalsize,row sep=0.3cm,column sep=0.4cm]%
      \lstick[wires=2]{{$\rho_{1,2}$}}& \qw& \qw& \meter[style={draw=blue}]{$A$} \vcwdouble{1-4}{4-4}{}&\qw&\qw&\qw&\qw\rstick{$P_{a_j}${\color{blue}{$\rho_1$}}$P_{a_j}$}&&&&&\rstick[wires=5]{$q$}\\[0.1cm]
       &  \qw& \qw& \qw& \qw&\meter[style={draw=red}]{$B$}\vcwdouble{2-6}{4-6}{}&\qw&\qw\rstick{$P_{b_k}${\color{red}{$\rho_2$}}$P_{b_k}$}&&&& \\[0.1cm]
      &&&&&&&&&&&&&\\
      \lstick{cl}& \cw&  \cw&\cw&\cw&\cw&\cw&\cw&&&&&\\
      &&&\rstick{$a_j$}&&\rstick{$b_k$}&&&&&&&\\
    \end{quantikz}
    }
    \hfill
  \caption{}\label{fig:3-instrumento-ideal}
\end{subfigure}

\hfill

\begin{subfigure}[a]{0.5\textwidth}
\begin{flushleft}
    \resizebox{11cm}{!}{
        \begin{quantikz}[font=\normalsize,scale=1, row sep=0.3cm,column sep=0.4cm]%
          \lstick[wires=2]{{$\rho_{1,2}$}}&\qw&\gate[wires=2]{?}& \qw&\meter[style={draw=blue}]{$A$} \vcwdouble{1-5}{4-5}{}&\qw&\qw&\qw&\qw\rstick[wires=2]{\tiny $\dfrac{P_{a_j,b_k}+P_{b_k,a_j}}{2}${{$\rho_{1,2}$}}$\dfrac{P_{a_j,b_k}+P_{b_k,a_j}}{2}$}&&&&&&&&&&&&\rstick[wires=5]{$1-q$}\\
          & \qw& \qw& \qw& \qw& \qw&\meter[style={draw=red}]{$B$}\vcwdouble{2-7}{4-7}{}&\qw&\qw&&&&&&&&&&&&\\
          &&&&&&&&&&&&&&&&&&&&&\\
          \lstick{cl}& \cw& \cw& \cw&\cw&\cw&\cw&\cw&\cw&&&&&&&&&&&&&\\
          &&&&\rstick{$a_j$}&&\rstick{$b_k$}&&&&&&&&&&&&&&&\\
        \end{quantikz}
        }
        \hfill
        \caption{}\label{fig:3-instrumento-non-ideal}
\end{flushleft}
\end{subfigure}
\caption{\textbf{(a)} Con una probabilidad $q$ se realiza una medición ideal del observable $A\tensor B$, con un estado posterior dado por $P_{a_j,b_k}\rho_{1,2}P_{a_j,b_k}$, cuando la salida indicada es $a_j b_k$.\textbf{ (b)} Con probabilidad $1-q$ ocurre un error y no se conoce con certeza el estado posterior correspondiente a la salida $a_j b_k$, estará dado por una superposición de dos estados $\dfrac{P_{a_j,b_k}+P_{b_k,a_j}}{2}\rho_{1,2}\dfrac{P_{a_j,b_k}+P_{b_k,a_j}}{2}$.}\label{fig: diagrama-cajas-tercer-instrumento}\source{elaboración propia}
\end{figure}



El tercer instrumento cuántico puede desarrollarse de la siguiente forma
\begin{equation}\label{eq:quantum-instrument-3-desarrollo}
    \begin{split}
        \mathcal{I}_3(\rho)&=q\sum_{m,n}  P_{a_m,b_n}\otimes P_{a_m,b_n}\rho P_{a_m,b_n}\\
        &+(1-q)\left[\sum_{(j,k)\in K}P_{a_j,b_k} \otimes P^{K}_{a_j,b_k}\rho P^{K}_{a_j,b_k}+\sum_{(i,l) \in L}P_{a_i,b_l} \otimes  \dfrac{1}{2}P^{L}_{a_i,b_l}\rho P^L_{a_i,b_l}\right]\\
        &=\sum_{(j,k)\in K}  P_{a_j,b_k}\otimes P_{a_j,b_k}\rho P_{a_j,b_k}+q\sum_{(i,l)\in L}P_{a_i,b_l} \otimes P_{a_i,b_l}\rho P_{a_i,b_l}\\
        &(1-q)\sum_{(i,l) \in L}P_{a_i,b_l} \otimes  \dfrac{1}{2}P^{L}_{a_i,b_l}\rho P^L_{a_i,b_l}\\
        &=\sum_{(j,k)\in K}  P_{a_j,b_k}\otimes P_{a_j,b_k}\rho P_{a_j,b_k}+q\sum_{(i,l)\in L}P_{a_i,b_l} \otimes P_{a_i,b_l}\rho P_{a_i,b_l}\\
        &(1-q)\sum_{(i,l) \in L}P_{a_i,b_l} \otimes  \dfrac{1}{2}\left[P_{a_i,b_l}+P_{b_l,a_i}\right]\rho \left[P_{a_i,b_l}+P_{b_l,a_i}\right].\\
    \end{split}
\end{equation}En este instrumento se asume que los operadores de proyección de los observables $A\otimes B$ y $B\otimes A$ tales que comparten el mismo valor propio, son ortogonales. En este caso, la salida clásica no difiere del estado inicial pero existe la posibilidad que se desconozca si la salida clásica coincida o no  con el estado posterior. 

Para este instrumento, se obtiene el mapeo de resultados, de igual forma que los dos anteriores, realizando la traza parcial sobre el sistema cuántico como sigue\begin{equation}\label{eq:traza_q_3instrument2p}
    \begin{split}
       \tr_{\text{qu}}\left( \mathcal{I}_3(\rho)\right)&=\sum_{(j,k)\in K}  P_{a_j,b_k}\tr(P_{a_j,b_k}\rho)+q\sum_{(i,l)\in L}P_{a_i,b_l} \tr(P_{a_i,b_l}\rho) \\
        &+\dfrac{(1-q)}{2}\sum_{(i,l) \in L}P_{a_i,b_l} \tr(\left[P_{a_i,b_l}+P_{b_l,a_i}\right]\rho)\\
        &=\sum_{(j,k)\in K}  P_{a_j,b_k}\la a_i b_l|\rho|a_i b_l\ra +q\sum_{(i,l)\in L}P_{a_i,b_l} \la a_i b_l|\rho|a_i b_l\ra \\
        &+\dfrac{(1-q)}{2}\sum_{(i,l) \in L}P_{a_i,b_l} \left[ \la a_i b_l|\rho|a_i b_l\ra + \la b_l a_i|\rho| b_l a_i\ra \right].\\
        &=\sum_{(j,k)\in K}  P_{a_j,b_k}\la a_i b_l|\rho|a_i b_l\ra \\
        &+\dfrac{(1+q)}{2}\sum_{(i,l)\in L}P_{a_i,b_l} \la a_i b_l|\rho|a_i b_l\ra +\dfrac{(1-q)}{2}\sum_{(i,l) \in L}P_{a_i,b_l} \la b_l a_i|\rho| b_l a_i\ra.\\
    \end{split}
\end{equation}Nótese que el último término de la primera igualdad  se puede simplificar de esta manera puesto que se supone que los operadores $P_{a_j,b_k}^L$ son de proyección. 

%Esta ecuación puede compararse con la ecuación {\eqref{eq:traza_q_1instrument2p}} y ciertamente puede realizarse el mismo análisis para obtener el mapeo de resultados.




Asimismo, es posible calcular el valor esperado de este nuevo instrumento \begin{equation*}
    \begin{split}
        \la A\otimes B \ra_{\mathcal{I}_3}&=\tr([(A\otimes B) \otimes \mathds{1}]\mathcal{I}_3)\\
        &=\sum_{(j,k)\in K} \tr((A\otimes B )P_{a_j,b_k})\tr(P_{a_j,b_k}\rho)+q\sum_{(i,l)\in L}\tr((A\otimes B)P_{a_i,b_l}) \tr(P_{a_i,b_l}\rho)\\
        &+\dfrac{(1-q)}{2}\sum_{(i,l) \in L}\tr((A\otimes B )P_{a_i,b_l}) \tr(\left[P_{a_i,b_l}+P_{b_l,a_i}\right]\rho)\\
        &=\sum_{(j,k)\in K} a_j b_k\tr(P_{a_j,b_k}\rho)+q\sum_{(i,l)\in L}a_i b_l \tr(P_{a_i,b_l}\rho)\\
        &+\dfrac{(1-q)}{2}\sum_{(i,l) \in L} a_i b_l\tr(\left[P_{a_i,b_l}+P_{b_l,a_i}\right]\rho)\\
        &=\sum_{(j,k)\in K} a_j b_k \tr(P_{a_j, b_k}\rho)\\
        &+\dfrac{(1+q)}{2}\sum_{(i,l)\in L}a_i b_l \tr(P_{a_i,b_l}\rho) +\dfrac{(1-q)}{2}\sum_{(i,l) \in L}a_i b_l \tr(P_{b_l, a_i}\rho).
    \end{split}
\end{equation*}En suma, el valor esperado para este instrumento es \begin{equation}\label{eq:expectedValueThirdInstrument2p}
    \begin{split}
        \la A\otimes B \ra_{\mathcal{I}_3}&=\sum_{(j,k)\in K} a_j b_k \tr(P_{a_j, b_k}\rho)+\sum_{(i,l)\in L}a_i b_l \left[\tfrac{(1+q)}{2} \tr(P_{a_i,b_l}\rho) +\tfrac{(1-q)}{2}\tr(P_{b_l, a_i}\rho)\right].
    \end{split}
\end{equation} Este valor esperado puede conllevar más álgebra debido a la interpretación que se le da al estado posterior a la medición, sin embargo el  este valor esperado equivale matemáticamente al del primer instrumento y esta equivalencia será discutida en la siguiente sección. 






% }}}
% }}}
\subsection{Equivalencia de los instrumentos}\label{subsec:Equivalencia } % {{{

En esta sección se pretende comparar los valores esperados de los diferentes
instrumentos cuánticos analizados previamente, con el fin de examinar sus
equivalencias.  Dado el enfoque con el que fueron concebidos los tres
instrumentos, se anticipaba que el valor esperado de cada uno sería el
apropiado. Como se observó en las secciones anteriores, los instrumentos
propuestos no son idénticos. Por lo tanto, es importante preguntarse, ¿bajo qué
condiciones estos instrumentos son equivalentes?



En primer lugar, se contrasta los valores esperados del primer y segundo
instrumento. Con anterioridad se pudo apreciar que el valor esperado del
primero corresponde exactamente al de una medición difusa, sin embargo al
analizar el valor esperado del segundo, se puede apreciar que, en términos
generales, la parte que distorsiona los resultados de la medición no se
corresponde de manera similar con la parte que difumina la medición en el
primero. Para verificarlo basta con analizar un caso particular, por ejemplo si
$A\otimes B= \sigma_z\otimes \sigma_x$, sin importar cuál sea el estado inicial
$\rho$, el segundo término de  la ecuación
{\eqref{eq:expectationvalueSecondInstrument2p}} es siempre cero. No obstante,
la parte análoga del valor esperado del primer instrumento, el segundo término
en la ecuación {\eqref{eq:expectedValueFirstInstrument2p}} no siempre es cero
sino que dependerá del estado inicial del sistema.  


Aunque no en todos los casos estos instrumentos son equivalentes, existen una
condición en la que sí lo son. A continuación se presenta la proposición que
detalla esta condición de equivalencia del valor esperado entre los dos
primeros instrumentos y su demostración.


\begin{proposition}\label{prop:Equivalencia-instruments1-2}
    Para todo estado inicial $\rho$, los valores esperados de las alternativas
de los instrumentos cuánticos {\eqref{eq:fisrtinstrument2p}} y
{\eqref{eq:second-instrument-2p}}, son equivalentes si y solo si \begin{equation}\label{eq:Condicion-equivalencia1-2}
    \la a_j
b_k|B\otimes A|a_{j'}b_{k'}\ra=0, \forall j,k\ne j',k'.
\end{equation}
\end{proposition}

\begin{proof}
    $(\Rightarrow)$ Supongamos que para todo estado inicial $\rho$ los valores
esperados de los instrumentos son iguales. El primer término del lado derecho
de la ecuación {\eqref{eq:expectedValueFirstInstrument2p}} y el primer término
de la derecha en {\eqref{eq:expectationvalueSecondInstrument2p}} son iguales,
por lo tanto se igualan los segundos términos de estas ecuaciones,
i.e.\begin{equation}\label{eq: equivalencia_terminos}\tr(B\otimes A
\rho)=\sum_{j,k}\tr((B\otimes A) P_{a_j,b_k})\tr(P_{a_j b_k}\rho
).\end{equation} Por la linealidad de la traza se puede reacomodar el lado
derecho de esta última ecuación: 
\[\begin{split}\sum_{jk}\tr((B\otimes A)
P_{a_j,b_k})\tr(P_{a_j b_k}\rho )&= \sum_{jk}\tr[\tr((B\otimes A)
P_{a_j,b_k})(P_{a_j b_k}\rho )]\\
        &=\tr\left[\left(\sum_{j,k}\tr((B\otimes A) P_{a_j,b_k})P_{a_j b_k}\right)\rho \right]\end{split}.\]  reescribiendo la ecuación {\ref{eq: equivalencia_terminos}} 
        \[\tr(B\otimes A \rho)=\tr\left[\left(\sum_{j,k}\tr((B\otimes A) P_{a_j,b_k})P_{a_j b_k}\right)\rho \right]\\ \]
    \begin{equation}\label{trazaCero}\begin{split}
        \tr\left[\left(B\otimes A-\left(\sum_{j,k}\tr((B\otimes A) P_{a_j,b_k})P_{a_j b_k}\right)\right)\rho \right]=0\\ \end{split}.\end{equation}
        Al expandir $B\otimes A$ en la base de vectores propios de $A\otimes B$ obtenemos que \[B\otimes A=\sum_{j,k,j',k'}\tr(B\otimes A |a_j b_k\rala a_{j'}b_{k'}|)|a_j b_k\rala a_{j'}b_{k'}|,\] desarrollando la ecuación {\ref{trazaCero}}   
    \[\begin{split}
        \tr\left[\left(B\otimes A-\left(\sum_{j,k}\tr((B\otimes A) P_{a_j,b_k})P_{a_j b_k}\right)\right)\rho\right]&=0\\
    \tr\left[\left(\sum_{j,k,j',k'}\tr(B\otimes A |a_j b_k\rala a_{j'}b_{k'}|)|a_j b_k\rala a_{j'}b_{k'}|-\left(\sum_{j,k}\tr((B\otimes A) P_{a_j,b_k})P_{a_j b_k}\right)\right)\rho\right]&=0\\
    \tr\left[\left(\sum_{j,k\ne j',k'}\tr(B\otimes A |a_j b_k\rala a_{j'}b_{k'}|)|a_j b_k\rala a_{j'}b_{k'}|\right)\rho\right]&=0\\
    \tr\left[\left(\sum_{j,k\ne j',k'}  \langle a_j b_k |B\otimes A|a_{j'}b_{k'}\rangle |a_j b_k\rala a_{j'}b_{k'}|\right)\rho\right]&=0.\\
    \end{split}\]
    De la última ecuación, para cualquier estado inicial $\rho$, se obtiene que\footnote[1]{ver lema {\ref{lemma_traza_cero}}.} 
    \[\sum_{j,k\ne j', k'} \langle a_j b_k |B\otimes A|a_{j'}b_{k'}\rangle |a_j b_k\rala a_{j'}b_{k'}|=0,\] luego por independencia lineal de la base de vectores propios  
     \[\begin{array}{cc}
        \langle a_j b_k |B\otimes A|a_{j'}b_{k'}\rangle=0& \forall j,k\ne j',k'.\end{array}\]

        $(\Leftarrow)$
        Ahora si suponemos que los coeficientes  $\langle a_j b_k |B\otimes A|a_{j'}b_{k'}\rangle=0$,  $\forall j,k\ne j',k'$. Entonces el operador $B\otimes A=\sum_{i,l}d_{i,l}P_{a_i, b_l}$ escrito en la base de vectores propios de $A\otimes B$.
        
        Finalmente, se obtiene la siguiente igualdad \[\begin{split}\sum_{j,k}\tr( \sum_{i,l}d_{i,l}P_{a_i, b_l} P_{a_j,b_k})\tr(P_{a_j b_k}\rho )&=\sum_{j,k}\sum_{i,l}d_{i,l}\delta_{j,k}^{i,l}\tr(P_{a_j b_k}\rho )\\&=\sum_{j,k}d_{j,k}\tr(P_{a_j b_k}\rho )=\tr(B\otimes A\rho)\end{split}.\]


\end{proof}

%\begin{corollary}
 %  Para todo estado inicial $\rho$ si los observables $A\tensor B$ y $B\tensor
%A$ conmutan y  no son degenerados entonces ambos instrumentos tienen valores
%esperados equivalentes.
%\end{corollary}

Según la proposición previa, se concluye que los dos primeros instrumentos tendrán el mismo valor esperado, si y solo si la matriz de $B\otimes A$ es diagonal en la base de vectores propios del observable $A\otimes B$. Esta conclusión podría sugerir que es lo mismo afirmar que los observables cumplen con esta condición si y solo si conmutan. No obstante, esta afirmación no es correcta puesto que no obedece la doble implicación. Es posible demostrar que si se cumple la condición {\eqref{eq:Condicion-equivalencia1-2}}, entonces los observables conmutan. Sin embargo, la afirmación recíproca no es cierta. A continuación se explora sobre estas aseveraciones.

Para iniciar se presenta de manera formal la primera implicación y su demostración.\begin{proposition}
    Si se satisface la condición  \[\la a_j b_k|B\otimes A|a_{j'}b_{k'}\ra=0, \forall j,k\ne j',k', \]  entonces $[A\otimes B,B \otimes A]=0$  
\end{proposition}

\begin{proof}
    Con base en la condición es posible escribir el operador $B\otimes A$  en términos de los operadores de proyección del observable $A\otimes B$ \[ B\otimes A = \sum_{i,l} c_{i,l} |a_{i}b_{l}\rala a_i b_l|,\] donde los coeficientes $c_{i,l}$ se encuentran en la diagonal del operador $B\otimes A$ escrito en la base de $A\otimes B$. Luego, $A\otimes B $ también se puede expresar como \[A\otimes B= \sum_{j,k} a_j b_k |a_{j}b_{k}\rala a_j b_k|.\] Ahora es posible conmutar los observables \[\begin{split}
        [A\otimes B,B \otimes A]&= (A\otimes B) (B\otimes A)-(B\otimes A) (A\otimes B)\\
        &=\left(\sum_{j,k} a_j b_k |a_{j}b_{k}\rala a_j b_k|\right)\left(\sum_{i,l} c_{i,l} |a_{i}b_{l}\rala a_i b_l|\right)\\
		&-\left( \sum_{i,l} c_{i,l} |a_{i}b_{l}\rala a_i b_l|\right)\left(\sum_{j,k} a_j b_k |a_{j}b_{k}\rala a_j b_k|\right)\\
		&=\left(\sum_{j,k} \sum_{i,l} a_j b_k  c_{i,l} \delta_{j,k}^{i,l}|a_{j}b_{k}\rala a_i b_l|\right)-\left( \sum_{i,l} \sum_{j,k} c_{i,l} a_j b_k \delta_{i,l}^{j,k}|a_{i}b_{l}\rala a_j b_k|\right)\\
		&=\left(\sum_{j,k}  a_j b_k  c_{j,k} |a_{j}b_{k}\rala a_j b_k|\right)-\left( \sum_{j,k} a_j b_k  c_{j,k} |a_{j}b_{k}\rala a_j b_k|\right)\\
		&=0
	\end{split}\]
\end{proof}

Luego, tal como se mencionó previamente, la condición necesaria para cumplir con {\eqref{eq:Condicion-equivalencia1-2}} es que los observables $A\otimes B$ y $B\otimes A$ conmuten, mas no es suficiente. Para probarlo basta con mostrar el siguiente contraejemplo. El observable $\sigma_z\tensor \sigma_x$ conmuta con $\sigma_x \tensor \sigma_z$,\[\begin{split}[\sigma_z\tensor \sigma_x,\sigma_x \tensor \sigma_z]&=\sigma_z \sigma_x \tensor \sigma_x \sigma_z- \sigma_x\sigma_z \tensor \sigma_z\sigma_x\\
&=\sigma_y\tensor \sigma_y -\sigma_y\tensor \sigma_y =0,
\end{split}\]sin embargo escribir al observable $\sigma_x\tensor \sigma_z $ en la base de vectores propios de $\sigma_z\tensor \sigma_x$, no es una matriz diagonal, basta con calcular lo siguiente \[\begin{split}
    \la 0+|\sigma_x\otimes \sigma_z|1-\ra&=\begin{pmatrix}
        1&1&0&0
    \end{pmatrix}\begin{pmatrix}
        0&0&1&0\\
        0&0&0&-1\\
        1&0&0&0\\
        0&-1&0&0\\
    \end{pmatrix}\begin{pmatrix}
         0\\
         0\\
         1\\
        -1\\
    \end{pmatrix}=2.\\
\end{split}\]


Por otro lado, debido a que el primer instrumento presenta el valor esperado
correcto y el segundo no en todos los casos, lo razonable es comparar el tercer
instrumento con el primero únicamente. Anteriormente se vio que estos
instrumentos tenían un valor esperado similar matemáticamente. Sin embargo, es
importante hacer la aclaración que el tercer instrumento solo tiene sentido, en
el caso particular en el que los operadores de proyección de los observables
$A\tensor B$ y $B\otimes A$ cumplen con
$P_{a_m,b_n}P_{b_n,a_m}=\delta_{b_n,a_m}^{a_m,b_n}P_{a_m,b_n}$. Por esta razón,
ahora se presenta la proposición que establece formalmente cuando estos
instrumentos son análogos.

\begin{proposition}\label{prop:Equivalencia-instruments-1-3}
    Para todo estado inicial $\rho$, los valores esperados de las alternativas
de instrumentos cuánticos {\eqref{eq:fisrtinstrument2p}} y
{\eqref{eq:quantum-instrument-3-desarrollo}} son equivalentes si y solo si
$p=\dfrac{1+q}{2}$.
\end{proposition}


\begin{proof}
El valor esperado del tercer instrumento está dado por la ecuación {\eqref{eq:expectedValueThirdInstrument2p}}, en el que la primera suma toma los operadores de proyección indexados por los elementos del conjunto $K$, es decir que $P_{a_j,b_k}=P_{a_k,b_j}$. Por lo tanto esta ecuación puede reescribirse de la siguiente forma \begin{equation*}
    \begin{split}
        \la A\otimes B \ra_{\mathcal{I}_3}&=\sum_{(j,k)\in K} a_j b_k \left[\tfrac{(1+q)}{2} \tr(P_{a_j, b_k}\rho)+ \tfrac{(1-q)}{2} \tr(P_{a_j, b_k}\rho)\right]\\
        &+\sum_{(i,l)\in L}a_i b_l \left[\tfrac{(1+q)}{2}\tr(P_{a_i,b_l}\rho) +\tfrac{(1-q)}{2} \tr(P_{b_l, a_i}\rho)\right]\\
        &=\sum_{(j,k)\in K} \tfrac{(1+q)}{2} a_j b_k\tr(P_{a_j, b_k}\rho) + \tfrac{(1-q)}{2} a_j b_k \tr(P_{ b_k,a_j}\rho)\\
        &+\sum_{(i,l)\in L}\tfrac{(1+q)}{2}a_i b_l \tr(P_{a_i,b_l}\rho) +\tfrac{(1-q)}{2}a_i b_l \tr(P_{b_l, a_i}\rho)\\
        &=\tfrac{(1+q)}{2}\left[\sum_{(j,k)\in K}  a_j b_k \tr(P_{a_j, b_k}\rho)+\sum_{(i,l)\in L}a_i b_l \tr(P_{a_i,b_l}\rho)  \right] \\
        &+\tfrac{(1-q)}{2}\left[\sum_{(j,k)\in K}a_j b_k \tr(P_{ b_k,a_j}\rho)+\sum_{(i,l) \in L}a_i b_l \tr(P_{b_l, a_i}\rho)\right]\\
        &=\tfrac{(1+q)}{2}\left[\tr(A\otimes B \rho)  \right] +\tfrac{(1-q)}{2}\left[\tr(B\otimes A \rho)\right].\\
    \end{split}
\end{equation*}De la última igualdad se puede observar que los valores esperados serán equivalentes cuando la probabilidad $p=\dfrac{1+q}{2}$ y $(1-p)=\dfrac{1-q}{2}$.
\end{proof}

Con base en esta proposición es posible analizar la equivalencia entre las interpretaciones de los instrumentos. Cuando la probabilidad %$(1-p)$
de obtener una medición donde las partículas experimentan un intercambio, sea igual a la mitad de la probabilidad %$(1-q)$
 de no poder diferenciar en qué estado se encuentra el sistema después de la medición, el valor esperado de los instrumentos  será exactamente el mismo.   






% }}}
\subsection{Ejemplos sobre una medición difusa utilizando distintas herramientas} % {{{
% Into sub {{{


En esta sección se realizarán algunos ejemplos que ilustran la descripción completa de la medición de un observable de la forma $A\otimes B$ en un estado inicial $\rho$ utilizando las distintas herramientas discutidas en los apartados anteriores, es decir, las medidas POVM conjunto a los operadores de Kraus, así como los distintos instrumentos cuánticos. Continuando con el ejemplo de la sección {\ref{subsec:Valores_esperados}} en el que se quiere medir el observable $\sigma_z\otimes \sigma_x$, con vectores propios $\{\ket{0+},\ket{0-},\ket{1+},\ket{1-}\}$. El estado inicial del sistema es  \begin{equation}\label{eq:estado-ejemplos-2p}\rho= \left(\frac{1}{3}\ket{0}\bra{0}+\frac{2}{3}\ket{1}\bra{1}\right)\otimes\left(\ket{+}\bra{+}\right)=\begin{pmatrix}
    1/6&1/6&0&0\\
    1/6&1/6&0&0\\
    0&0&1/3&1/3\\
    0&0&1/3&1/3\\
\end{pmatrix}.\end{equation}


% }}}
\subsubsection{Ejemplo utilizando las mediciones POVM y los operadores de Kraus} % {{{


Previamente se graficó el mapeo de probabilidades de obtener los valores propios del observables, los cuales son $1$ y $-1$. Este mapeo puede realizarse con los efectos discutidos como \[\begin{split}E(1,\rho)=&\tr(\mathcal{F}_{2\text{p}}(P_{0+})\rho) +\tr(\mathcal{F}_{2\text{p}}(P_{1-})\rho)=\tr(\mathcal{F}_{2\text{p}}(P_{0+}+P_{1-})\rho)\\ 
    =&\frac{1}{3}\tr\left[\left[p\begin{pmatrix}
        1&1&0&0\\
        1&1&0&0\\
        0&0&1&-1\\
        0&0&-1&1\\
    \end{pmatrix}+ (1-p)\begin{pmatrix}
        1&0&1&0\\ 
        0&1&0&-1\\
        1&0&1&0\\
        0&-1&0&1\\
    \end{pmatrix}\right]\begin{pmatrix}
        \tfrac{1}{2}&\tfrac{1}{2}&0&0\\
        \tfrac{1}{2}&\tfrac{1}{2}&0&0\\
        0&0&1&1\\
        0&0&1&1\\
    \end{pmatrix}\right]\\
    =&\frac{3-p}{6},
    \end{split}\]  análogamente para el otro valor propio
    \[E(-1,\rho)=\tr(\mathcal{F}_{2\text{p}}(P_{0-})\rho) +\tr(\mathcal{F}_{2\text{p}}(P_{1+})\rho)=\tr(\mathcal{F}_{2\text{p}}(P_{0-}+P_{1+})\rho)=\frac{3+p}{6}.\]
    



Por ejemplo, se obtiene la salida  $a_1=1$ y  $b_{1}=1$, la probabilidad de obtener dicha salida será solamente \(\tr(\mathcal{F}_{2\text{p}}(P_{0+})\rho)\), con una probabilidad $p=0.25$ de intercambio de las partículas,  el efecto es \[\mathcal{F}_{2\text{p}}(P_{0+})= \begin{pmatrix}
    1/2 &  1/8 &  3/8 &  0\\
    1/8 &  1/8 &  0 &  0\\
    3/8 &  0 &  3/8 &  0\\
    0&  0&  0&  0
  \end{pmatrix}.\]La operación cuántica que representa la medición difusa será \[\E_{a_1 ,b_1} (\rho) = \dfrac{K_{a_1 ,b_1} \rho K^\dagger_{a_1 ,b_1}}{\tr(\mathcal{F}_{2\text{p}}(P_{0+})\rho)}. \]  El canal representará precisamente que con una probabilidad $p$ el estado estará en el espacio propio de $\ket{0+}$ y con una probabilidad $(1 - p)$, en el espacio propio $\ket{+0}$
\[\rho'_{0+}=\dfrac{\sqrt{p P_{0+}+(1-p)P_{+0}}\rho\sqrt{p P_{0+}+(1-p)P_{+0} }}{\tr(\mathcal{F}_{2\text{p}}(P_{0+})\rho)}.\]  Para obtener estos operadores de Kraus se debe obtener primero los operadores de proyección del efecto, es decir \[\mathcal{F}_{2\text{p}}(P_{0+})= \sum_i \lambda_i \ket{\lambda_i}\bra{\lambda_i},\] El efecto tiene rango tres y sus valores propios son $\dfrac{4+\sqrt{7}}{8}$, $\dfrac{4-\sqrt{7}}{8}$ y $0$ con degeneración 2,  los vectores propios de los primeros dos valores son\[\frac{1}{\sqrt{28 - 2 \sqrt{7} }}\begin{pmatrix}1 + \sqrt{7}\\-2 + \sqrt{7}\\ 3\\0 \end{pmatrix} \text{ y } \frac{1}{\sqrt{28 + 2 \sqrt{7} }}\begin{pmatrix}1 - \sqrt{7}\\-2 - \sqrt{7}\\ 3\\0 \end{pmatrix},\] respectivamente. Para obtener la raíz cuadrada del operador se puede realizar el siguiente cálculo\footnote[2]{En general, los operadores de Kraus no son sencillos de calcular con lápiz y papel. Es por esta razón que se puede utilizar un programa, el cual se encuentra en el repositorio {\cite{enlacepropio}}.} \[\sqrt{\mathcal{F}_{2\text{p}}(P_{0+})}=\sum_i \sqrt{\lambda_i} \ket{\lambda_i}\bra{\lambda_i},\] es decir \[\begin{split}\sqrt{\mathcal{F}_{2\text{p}}(P_{0+})}&=\sqrt{\frac{4+\sqrt{7}}{8}} \left[\frac{1}{28-2\sqrt{7}} \begin{pmatrix}
    2(4+\sqrt{7}) & (5-\sqrt{7}) &  3(1+\sqrt{7}) &  0\\
    (5-\sqrt{7}) &  (11-4\sqrt{7}) &  3(-2+\sqrt{7}) &  0\\
    3(1+\sqrt{7})  &  3(-2+\sqrt{7})  &  9 &  0\\
    0&  0&  0&  0
  \end{pmatrix}\right]\\&+ \sqrt{\frac{4-\sqrt{7}}{8}}\left[ \frac{1}{28+2\sqrt{7}} \begin{pmatrix}
    2(4-\sqrt{7}) & (5+\sqrt{7}) &  3(1-\sqrt{7}) &  0\\
    (5+\sqrt{7}) &  (11+4\sqrt{7}) &  3(-2-\sqrt{7}) &  0\\
    3(1+\sqrt{7})  &  3(-2-\sqrt{7})  &  9 &  0\\
    0&  0&  0&  0
  \end{pmatrix}\right]. \end{split}\]Finalmente el operador de Kraus estará dado por las siguiente matriz \[K_{0+}=\begin{pmatrix}
    0.566947& 0.188982& 0.377964& 0\\
    0.188982 &  0.283473& -0.094491 &  0\\
    0.377964 & -0.094491  &  0.472456 &  0\\
    0 &  0&  0 &  0
  \end{pmatrix}.\] 

  En virtud de lo anterior, el estado posterior a la medición resulta ser 
\[\rho_{0+}= \dfrac{K_{0 ,+} \rho K^\dagger_{0 ,+}}{\tr(\mathcal{F}_{2\text{p}}(P_{0+})\rho)}=\begin{pmatrix}
    0.5275 &  0.1758 &  0.3516 &  0\\
    0.1758 &  0.1483 &  0.0275 &  0\\
    0.3516 &  0.0275 &  0.3242 &  0\\
    0 &  0 &  0&  0
  \end{pmatrix}.\] En este escenario, el estado inicial se proyecta en el espacio dado por la superposición de dos de los estados propios del efecto correspondiente a la salida $a_1 b_1=(1)(1)=1$. El estado posterior a la medición, resulta ser un nuevo estado mixto, con el mismo rango del efecto. Entonces, $\rho_{0+}$ se encuentra en el subespacio generado por los espacios propios del efecto, escalados por coeficientes que están asociados con la probabilidad que se encuentre en cada uno de los espacios propios. 

% }}}
\subsubsection{Ejemplo utilizando los instrumentos cuánticos} % {{{

En la sección {\ref{subsec:Instrumentos_cuanticos_2p_3cap}} se ha analizado tres interpretaciones de las mediciones difusas utilizando instrumentos cuánticos, ahora es conveniente analizar un ejemplo para un estado específico para completar la comprensión de estos instrumentos. De igual manera se utilizará el estado propuesto en la ecuación {\eqref{eq:estado-ejemplos-2p}} y el observable $\sigma_z\otimes\sigma_x$.


Para el primer instrumento  los operadores que integran el sistema clásico son
$P_{0+},P_{0-},P_{1+},P_{1-}$, y lo operadores que conforman el sistema
cuántico condicionados a sus respectivas salidas serán las proyecciones de
estos operadores sobre el operador difuso aplicado a $\rho$. El primer
instrumento para este caso es 
\[\begin{split}
\mathcal{I}_1(\rho)&=P_{0+}\otimes P_{0+}\mathcal{F}_{2\text{p}}(\rho)P_{0+}+P_{0-}\otimes P_{0-}\mathcal{F}_{2\text{p}}(\rho)P_{0-}\\
    &+ P_{1+}\otimes P_{1+}\mathcal{F}_{2\text{p}}(\rho)P_{1+}+P_{1-}\otimes P_{1-}\mathcal{F}_{2\text{p}}(\rho)P_{1-}, \\
\end{split} \]
donde 
\[\mathcal{F}_{2\text{p}}(\rho)= p\begin{pmatrix}
        1/6&1/6&0&0\\
        1/6&1/6&0&0\\
        0&0&1/3&1/3\\
        0&0&1/3&1/3\\
    \end{pmatrix}
+ (1-p)\begin{pmatrix} 1/6&0&1/6&0\\ 0&1/3&0&1/3\\ 1/6&0&1/6&0\\ 0&1/3&0&1/3\\ \end{pmatrix}.
\] Aplicarle el operador difuso al estado inicial representa que al realizarse
la medición con probabilidad $p$ se experimenta un intercambio de partículas y
con probabilidad $(1-p)$ no lo hace. Luego, este nuevo estado se proyecta en
uno de los espacios propios de la salida obtenida.
La descripción de esta medición queda
condensada en estos cuatro términos, con lo que puede obtenerse su mapa de
resultados y el estado posterior. 
   
   
El mapa de resultados se obtiene utilizando la traza parcial sobre el sistema cuántico 
\[\begin{split}\tr_{\text{qu}}(\mathcal{I}_1(\rho))&=P_{0+}\tr(P_{0+}\mathcal{F}_{2\text{p}}(\rho))+P_{0-}\tr(P_{0-}\mathcal{F}_{2\text{p}}(\rho))\\
        &+ P_{1+}\tr(P_{1+}\mathcal{F}_{2\text{p}}(\rho))+P_{1-}\tr( P_{1-}\mathcal{F}_{2\text{p}}(\rho)). \\
\end{split}\]
Si se desea la probabilidad de cada una de las posibles salidas puede
proyectarse en el espacio de cada una de ellas y obtener su traza, esto es
análogo a utilizar las medidas POVM para los mapeos de cada una de las cuatro
posibles salidas. La probabilidad de obtener la salida correspondiente al
estado $|0+\ra$, será
\[\begin{split}\tr(P_{0+}\tr_{\text{qu}}(\mathcal{I}_1(\rho)))&=\tr(P_{0+1})\tr(P_{0+}\mathcal{F}_{2\text{p}}(\rho))=\tr(\mathcal{F}_{2\text{p}}(P_{0+})\rho)\\=&\frac{1}{6}\tr\left[\left[p\begin{pmatrix}
            1&1&0&0\\
            1&1&0&0\\
            0&0&0&0\\
            0&0&0&0\\
        \end{pmatrix}+ (1-p)\begin{pmatrix}
            1&0&1&0\\
            0&0&0&0\\
            1&0&1&0\\
            0&0&0&0\\
        \end{pmatrix}\right]\begin{pmatrix}
            \tfrac{1}{2}&\tfrac{1}{2}&0&0\\
            \tfrac{1}{2}&\tfrac{1}{2}&0&0\\
            0&0&1&1\\
            0&0&1&1\\
        \end{pmatrix}\right]\\
        &=\frac{3+p}{12}.\end{split}\] 
Finalmente, si la salida que se obtiene es $a_1 b_1$ correspondiente al estado
propio  $|0+\rangle$, el estado posterior será
simplemente\[\begin{split}\dfrac{P_{a_1,b_1}(p\rho+(1-p)S\rho S^\dagger)
P_{a_1,b_1}}{\tr(P_{a_1,b_1} \mathcal{F}_{2\text{p}}(\rho))}&=\dfrac{1}{\frac{3.25}{12}}\begin{pmatrix}
        0.1354 &  0.1354  &  0 &  0\\
        0.1354 &  0.1354 &  0 &  0\\
        0 &  0 &  0&  0\\
        0 &  0 &  0 &  0
      \end{pmatrix}\\
      &=\begin{pmatrix}
        0.5 &  0.5 &  0 &  0\\
        0.5 &  0.5 &  0 &  0\\
        0 &  0 &  0&  0\\
        0 &  0 &  0 &  0
      \end{pmatrix}.\\
    \end{split} \] 
En esta situación, el estado que sufre un intercambio de partículas es
proyectado al espacio del estado propio $|0+\rangle$. El estado resultante debe
ser el operador $P_{0+}$, el cual es un estado puro, en contraste con el estado
inicial.
Por otro lado, el valor esperado del instrumentos es exactamente el mismo que el obtenido en la ecuación {\eqref{eq:valor-esperado-ejemplo-cap2}}
\begin{equation}\label{eq:1instrument-ejemplo}\begin{split}\la A\tensor B\ra_{\mathcal{I}_1(\rho)}=&p\cancelto{\frac{-1}{3}}{\tr(\rho\sigma_z\tensor \sigma_x )}+(1-p)\cancelto{0}{\tr(\rho\sigma_x\tensor \sigma_z )}=-\frac{p}{3}.\\
    \end{split}\end{equation} El segundo término se anula debido a que la traza del producto de cada uno de los operadores de proyección del observable $\sigma_z\otimes\sigma_x$, con el observable $\sigma_x\otimes\sigma_z$ es cero, mientras que el estado inicial es una superposición de dos de estos operadores de proyección.



Para la segunda alternativa, el error ocurre al realizar la lectura de los resultados por lo tanto el operador difuso se aplica a los operadores de proyección en el sistema clásico \[\begin{split}\mathcal{I}_2(\rho)&=\mathcal{F}_{2\text{p}}(P_{0+})\otimes P_{0+}\rho P_{0+}+\mathcal{F}_{2\text{p}}(P_{0-})\otimes P_{0-}\rho P_{0-}\\
    &+ \mathcal{F}_{2\text{p}}({P_{1+}})\otimes P_{1+}\rho P_{1+}+\mathcal{F}_{2\text{p}}(P_{1-})\otimes P_{1-}\rho P_{1-}, \\\end{split}\] en este caso el operador difuso captura la confusión de los resultados. Por ejemplo, con una probabilidad $p$ indicaría que el estado es $|0+\ra$ y con una probabilidad $1-p$ se obtiene una salida correspondiente al estado $|+0\ra$ sin embargo el estado posterior a la medición será la proyección en espacio propio del operador $P_{0+}$. 
    
    El valor esperado de este instrumento es \begin{equation}\begin{split}
        \la \sigma_z\otimes \sigma_x \ra_{\mathcal{I}_2(\rho)}&=p\tr(\sigma_z\tensor \sigma_x\rho)\\
       &+{(1-p)\cancelto{0}{\tr( (\sigma_x\otimes \sigma_z)P_{0+})}}\tr(P_{0+} \rho)+(1-p)\cancelto{0}{\tr( (\sigma_x\otimes \sigma_z)P_{0-})}\tr(P_{0-} \rho)\\
        &+(1-p)\cancelto{0}{\tr( (\sigma_x\otimes \sigma_z)P_{1+})}\tr(P_{1+} \rho)+(1-p)\cancelto{0}{\tr( (\sigma_x\otimes \sigma_z)P_{1-})}\tr(P_{1-} \rho)\\
        &=-\dfrac{p}{3}.
    \end{split}\end{equation} En este caso en específico, el valor esperado del segundo instrumento es el mismo al de la ecuación {\eqref{eq:1instrument-ejemplo}}. Esto se debe a que los últimos términos también se cancelan, de la misma forma que el último término del valor esperado del primer instrumento. Sin embargo, el observable no ejemplifica la proposición {\ref{prop:Equivalencia-instruments1-2}}, puesto que la condición {\eqref{eq:Condicion-equivalencia1-2}} no se cumple y por esta razón no serán equivalentes para todo estado inicial.

   Por otro lado, para determinar el mapeo de probabilidades, primero es necesario encontrar el operador de densidad reducido de la siguiente forma 
\begin{equation}\label{eq:operador-reducido-instrumento2}\begin{split}\tr_{\text{qu}}(\mathcal{I}_2(\rho))&=\mathcal{F}_{2\text{p}}(P_{0+})\tr(P_{0+}\rho)+\mathcal{F}_{2\text{p}}(P_{0-})\tr(P_{0-}\rho )\\&+ \mathcal{F}_{2\text{p}}({P_{1+}})\tr( P_{1+}\rho)+\mathcal{F}_{2\text{p}}(P_{1-})\tr(P_{1-}\rho). \\\end{split}\end{equation} Particularmente, si se desea conocer la probabilidad de obtener la salida que corresponde al estado $|0+\ra$, es necesario emplear el operador de proyección $P_{0+}$, junto con el cálculo del operador reducido {\eqref{eq:operador-reducido-instrumento2}}, como se muestra a continuación\[\begin{split}
        \tr(P_{0+}\tr_{\text{qu}}(\mathcal{I}_2(\rho)))&=\tr(P_{0+}\mathcal{F}_{2\text{p}}(P_{0+}))\tr(P_{0+}\rho)+\tr(P_{0+}\mathcal{F}_{2\text{p}}(P_{0-}))\tr(P_{0-}\rho )\\
        &+\tr(P_{0+} \mathcal{F}_{2\text{p}}({P_{1+}}))\tr( P_{1+}\rho)+\tr(P_{0+}\mathcal{F}_{2\text{p}}(P_{1-}))\tr(P_{1-}\rho)\\
        &=\tr(pP_{0+}+(1-p)P_{0+}P_{+0})\tr(P_{0+}\rho)+\tr((1-p)P_{0+}P_{-0})\tr(P_{0-}\rho )\\
        &+\tr((1-p)P_{0+} P_{+1})\tr( P_{1+}\rho)+\tr((1-p)P_{0+}P_{-1})\tr(P_{1-}\rho)\\
        &=p\tr(P_{0+}\rho)+\frac{(1-p)}{4}\tr((P_{+0}+P_{-0}+P_{+1}+P_{-1})\rho)=\frac{3+5p}{12}.
    \end{split}\]



Para el tercer instrumento, el observable $\sigma_z\otimes \sigma_x$ no cumple
con una de las condiciones para que esta última alternativa tenga sentido,
puesto que los operadores de proyección del observable no son ortogonales con lo operadores de proyección del observable $\sigma_x\otimes \sigma_z$. %$P_{0+}P_{+0}\ne 0$,  $P_{0-}P_{-0}\ne 0$, $P_{1+}P_{+1}\ne 0$ y
%$P_{1-}P_{-1}\ne 0$. 
Es por ello que para ejemplificar este instrumento se
utilizará el observable $\sigma_z\otimes \sigma_z$ con los operadores de
proyección $P_{00},P_{01},P_{10},P_{11}$ y un estado inicial $\rho$. El
instrumento es entonces 
\[\begin{split}\mathcal{I}_3(\rho)=&P_{00}\otimes P_{00}\rho P_{00}+P_{11}\otimes P_{11}\rho P_{11}+q P_{01}\otimes P_{01}\rho P_{01}+q P_{10}\otimes P_{10}\rho P_{10}\\
&\frac{1-q}{2}\left[P_{01}\otimes (P_{01}+P_{10})\rho(P_{01}+P_{10})+ P_{10}\otimes(P_{10}+P_{01})\rho (P_{10}+P_{01}) \right]\\
&=P_{00}\otimes P_{00}\rho P_{00}+P_{11}\otimes P_{11}\rho P_{11}\\
&+P_{01}\otimes \left[q P_{01}\rho P_{01}+\frac{1-q}{2}(P_{01}+P_{10})\rho(P_{01}+P_{10})\right]\\
&+P_{10}\otimes \left[q P_{10}\rho P_{10}+\frac{1-q}{2}(P_{10}+P_{01})\rho (P_{10}+P_{01}) \right].\\
 \end{split}\] 
En esta última ecuación, el sistema clásico indica que para las salidas correspondientes a los operadores $ P_{01}$ y $P_{10}$, con una probabilidad $q$ al sistema se le aplicará una medición ideal, pero con una probabilidad $(1-q)$ no es posible distinguir entre los estados $|01\ra$ y $|10\ra$. Su mapa de resultado se puede obtener realizando la traza parcial sobre el sistema cuántico \[\begin{split}\tr_{\text{qu}}(\mathcal{I}_3(\rho)) &=P_{00} \tr(P_{00}\rho)+P_{11}\tr(P_{11}\rho)+P_{01}\left[q \tr(P_{01}\rho)+\frac{1-q}{2}\tr((P_{01}+P_{10})\rho)\right]\\
    &+P_{10} \left[q \tr(P_{10}\rho)+\frac{1-q}{2}\tr((P_{10}+P_{01})\rho)\right]\\
    &=P_{00} \tr(P_{00}\rho)+P_{11}\tr(P_{11}\rho)+P_{01}\left[\frac{1+q}{2} \tr(P_{01}\rho)+\frac{1-q}{2}\tr(P_{10}\rho)\right]\\
    &+P_{10} \left[\frac{1+q}{2} \tr(P_{10}\rho)+\frac{1-q}{2}\tr(P_{01}\rho)\right].\\
\end{split}\] 
De esta última igualdad es posible notar fácilmente que para este observable el
mapa de resultados coincide con el que se podría plantear en el primer
instrumento cuando se cumpla que $p=\frac{1+q}{2}$. Además es importante notar
que aunque el observable es bastante sencillo, los resultados pueden verse
difuminados para dos de los estados, cambiando la probabilidad de obtener las
salidas correspondientes a los operadores $P_{01}$ y $P_{10}$.


% }}}
% }}}
% }}}
\section{Generalización de operadores de Kraus en sistemas de \texorpdfstring{\boldmath{$N$}}{N} partículas}\label{sec:cap3 generalizacion-} % {{{

% Intro {{{
El estudio del sistema que se analizó en las secciones anteriores y sus
implicaciones para la especificación completa de las mediciones difusas es útil
desde un punto de vista ilustrativo, aunque no es general. En esta sección se
pretende discutir el problema desde sistemas más generales, con más de dos
partículas y en los que se tomen en cuenta observables que no sean
factorizables.



Al generalizar las ideas anteriores para sistemas de $N>2$ partículas,
naturalmente se piensa realizar una medición de un observable de la forma
$A_1\tensor A_2\tensor \hdots \tensor A_N$, en la cual, por ruido del entorno
se produce un  error en el aparato de medición y confunde los resultados.
Ahora, existe una probabilidad de una identificación errónea.  Por ejemplo
puede realizarse la medición del observable $A_2\tensor A_1\tensor \hdots
\tensor A_N$ con cierta probabilidad $p$. Sin embargo, es posible que la
partículas experimenten un intercambio de cierta forma tal que la medición que
se produzca sea una de las $N{!}$ permutaciones en las que se puede configurar
el operador original. En este caso el valor esperado de la medición será el
promedio de los valores esperados  ponderado por probabilidad de que se
produzca alguna medición del observable incorrecto. 

Para obtener el valor esperado de esta medición en el sistema de $N$
partículas, es necesario el operador difuso presentado en la ecuación
{\eqref{eq:fuzzy-op-nparticles}}, el cuál utiliza los operadores de permutación
previamente explicados. El valor esperado de medir al observable $\prodtensor
A_i$, en un sistema inicial $\rho$,  es el siguiente
\begin{equation*}\label{eq:ExpectedValue-generalForm}\begin{split}
    \left \langle \prodtensor A_i\right \rangle_{\mathcal{F}(\rho)} &=\tr\left(\fuzzy{\rho}\prodtensor A_i\right)=\sum_{\Pi_j \in S}p_{j}\tr \left(\permut{j}{\rho} \prodtensor A_i\right).\\
\end{split}
\end{equation*}  

Aunque lo anterior es válido no es la forma más general en la que una medición
difusa puede ser estudiada. Estas mediciones pueden analizarse utilizando
cualquier tipo de observables, incluyendo los observables no factorizables los
cuales no pueden escribirse como un producto tensorial de $N$ observables, pero
pueden ser expresados como la superposición de todos los productos posibles de 
los operadores de una base del espacio de Hilbert $\mathcal{H}$. Por lo tanto, al llevar a cabo una medición difusa de un observable  $\mathcal{O} \in \mathcal{H}^{\otimes N}$, los operadores de cada término de la superposición que conforma el observable original se pueden intercambiar de alguna manera. En consecuencia, es posible  utilizar la linealidad del operador difuso para representar el valor esperado de cualquier observable, tal y como se presenta en el ecuación
{\eqref{eq:expected-value-fm-general}}.

%Entonces, al llevar a cabo una medición difusa de un observable cualquiera $\mathcal{O}$, se puede encontrar que hay una probabilidad $p_1$ de realizar una medición proyectiva ideal, sin embargo, también es factible que la medición efectuada sea la de un observable distinto, las partículas experimentarán un cambio, difuminando los resultados de la medición. 

% }}}
\subsection{Medidas POVM y operadores de Kraus en sistemas de varias partículas} % {{{

Anteriormente se describió el mapeo que puede realizarse con las medidas POVM
en un sistema de dos partículas. Se propuso intuitivamente un conjunto de
efectos, en los cuales a los operadores proyección se les aplicaba el operador
difuso y con base en estos efectos era posible obtener el conjunto de
operadores de Kraus que describen la medición. Para sistemas formados por más
de dos partículas los efectos son ligeramente diferentes y en consecuencia los
operadores de Kraus también.

La clave para lograr los efectos de manera correcta es también emplear el valor
esperado de una medición difusa. A pesar de ello, la diferencia de la
generalización reside en el hecho que los operadores de permutación para $N>3$
no son hermíticos, por lo que la aplicación del operador difuso debe llevarse a
cabo con precaución. Utilizando la propiedad cíclica de la traza, el valor
esperado de la medición difusa puede escribirse también como
\begin{equation}\label{eq:ExpectedValue-generalForm-2}\begin{split}
    \left \langle \mathcal{O}\right \rangle_{\mathcal{F}(\rho)} &=\sum_{\Pi_j \in S}p_j\tr \left(\Pi_j^\dagger{\mathcal{O}}\Pi_j\rho\right).\\
\end{split}
\end{equation} 
De esta expresión es fácil interpretar  que se
está aplicando un operador al observable,
sin embargo no es el mismo operador difuso que el de la ecuación
{\eqref{eq:fuzzy-op-nparticles}} debido a que en cada término de la suma el
operador aplicado por la izquierda es el adjunto del operador de permutación y
por la derecha se aplica el operador de permutación. En este contexto, aplicar un operador por la
izquierda es distinto a aplicarlo por la derecha, puesto que los
operadores de permutación no son hermíticos. También
es importante aclarar
que tampoco se le está aplicando el adjunto del operador difuso puesto que este
es hermítico\footnote[3]{ver lema {\ref{lemma:op-difuso-hermiticidad} }}. 


Por lo tanto los efectos ya no pueden ser simplemente la aplicación del
operador difuso a los operadores de proyección puesto que esto resultaría en un
valor esperado distinto a {\eqref{eq:expected-value-fm-general}}, puesto que
las permutaciones estarían ponderadas con diferentes probabilidades. En su
lugar, los efectos pueden escribirse como
\begin{equation}\label{eq:effectsSetNp}
    {\{E_{\lambda_i}\}}_{\lambda_i \in \Lambda}={\left\{\sum_{\Pi_j \in S} p_j \permutdagger{j}{P_{\lambda_i}}\right\}}_{\lambda_i \in \Lambda},
\end{equation}  
donde $P_{\lambda_i}$ es el operador de proyección correspondiente a cada
vector propio que tiene asociado el valor propio $\lambda_i\in \Lambda$.  Para un observable factorizable de $N$ partículas,  el valor propio
$\lambda_i$ es el producto de los $a_{jk}$ valores propios, donde los
subíndices indican que es $k-$ésimo valor propio correspondiente al observable
que se desea medir en la $j-$ésima partícula.\cpnote{Que pasa si los observables
no son degenerados, pero el producto es el mismo? Platicar con Rubi}
Adicionalmente, puede suceder que ninguno de los $N$ observables sea degenerado y que algunos productos de los valores propios de los observables sean iguales, de manera que $\lambda_i=\lambda_l$, con $i\ne l$. En este escenario, es probable que los operadores de proyección se modifiquen y por lo tanto, los efectos sean distintos a los de la ecuación {\eqref{eq:effectsSetNp}}. No obstante, se requiere realizar más investigación para corroborar si existe algún inconveniente en estas circunstancias. \rrnote{Agregué estas últimas frases.}

 Luego, para obtener el estado posterior a la medición se requiere utilizar la
descomposición de los efectos en operadores de Kraus $\{K_{\lambda_i}\}$. De
manera análoga al sistema de dos partículas es fácilmente comprobable que estos
operadores $E_{\lambda_i}$ son hermíticos, cumplen con la propiedad de
completitud y son positivos. Por lo tanto para este sistema se propone utilizar
la raíz cuadrada de los efectos 
\begin{equation}
   K_{\lambda_i}=\sqrt{\sum_{\Pi_j \in S} p_j \permutdagger{j}{P_{\lambda_i} }},
\end{equation} 
esto se puede realizar debido a la positividad de los efectos.

Finalmente, con el fin de mostrar el mapa de probabilidades y el estado resultante luego de la medición en un sistema de varias partículas, de forma práctica, se ha creado un programa. Éste utiliza como base los efectos y operadores Kraus descritos en esta sección  y está disponible en el siguiente repositorio {\cite{enlacepropio}}
.


% }}}
\subsection{Instrumento cuántico en sistemas de varias partículas}  % {{{

En esta sección se generalizan la idea detrás de los dos primeros instrumentos
cuánticos en sistemas más complejos que involucran varias partículas, así como
los primeros resultados obtenidos en la sección {\ref{subsec:Equivalencia }},
en la cual se comprobó que los instrumentos no generan las mismas
distribuciones de probabilidad en todos los casos.

La primera alternativa es el instrumento en el que por ruido en el sistema
ocurre una equivocación y al medir un observable $\mathcal{O}$ es probable que
las partículas experimenten un intercambio. Este instrumento representa que la
salida clásica corresponderá a los valores propios del observable.
Adicionalmente, con cierta probabilidad la medición ocurre de manera ideal, y
el estado posterior a la medición será la proyección del estado inicial al
espacio propio correspondiente a la salida brinda por el sistema clásico. Sin
embargo, también es verosímil que el estado posterior sea la proyección al
espacio propio de la salida pero del estado inicial transformado, de forma que
represente un sistema en el que las partículas se cambian
\begin{equation}\label{eq:1instrumentnp}
    \mathcal{I}_1(\rho)=\sum_{\lambda_i \in \Lambda }P_{\lambda_i}\otimes P_{\lambda_i}\fuzzy{\rho}P_{\lambda_i}.
\end{equation} 
Por otra parte el valor esperado del resultado de la medición modelado con este
instrumento puede calcularse de la siguiente manera 
\begin{equation*}
    \begin{split}
        \left \la \mathcal{O} \right \ra_{\mathcal{I}_1}&=\tr\left( \left[\left(\mathcal{O}\right) \otimes \mathds{1}\right]\mathcal{I}_1\right) \\
        &=\tr\left(\left[ \left(\mathcal{O}\right)\otimes \mathds{1}\right]\sum_{\lambda_j \in \Lambda}P_{\lambda_j}\otimes P_{\lambda_j}\fuzzy{\rho}P_{\lambda_j} \right)\\
        &=\sum_{\lambda_j\in \Lambda} \tr\left(\left(\mathcal{O}\right) P_{\lambda_j}\right) \tr\left(P_{\lambda_j}\fuzzy{\rho} P_{\lambda_j}\right) \\
        &=\sum_{\lambda_j\in \Lambda} \tr\left(\sum_{{\lambda_j, \lambda_k \in \Lambda}}\lambda_k P_{\lambda_k} P_{\lambda_j}\right) \tr\left(P_{\lambda_j}\fuzzy{\rho} P_{\lambda_j}\right)  \\
        &=\sum_{\lambda_j \in\Lambda} \tr\left(\lambda_j P_{\lambda_j}\right) \tr\left(P_{\lambda_j}\fuzzy{\rho}P_{\lambda_j}\right) \\
        &=\sum_{\lambda_j \in \Lambda} \lambda_j \tr\left(P_{\lambda_j}\fuzzy{\rho}\right) \\
    \end{split}
\end{equation*} 
con lo que se puede concluir que el valor esperado correspondiente a este instrumento es 
\begin{equation}\label{eq:valor-esperado-1instrumentnp}
        \left \la\mathcal{O} \right \ra_{\mathcal{I}_1}= \tr\left( \mathcal{O} \fuzzy{\rho}\right),
\end{equation}
el mismo que el valor esperado correcto {\eqref{eq:expected-value-fm-general}}.

La segunda alternativa es igualmente generalizable para un sistema de $N$
partículas. Este instrumento representa una equivocación en el sistema clásico,
y por esto al medir el observable $\mathcal{O}$, con alguna probabilidad el
estado inicial se proyectará en el espacio propio de la salida correspondiente.
Pero es probable que el estado inicial se proyecte de igual forma a un espacio
propio del observable pero la lectura de los resultados de la medición sean las
salidas de un observable distinto, el cual puede ser una transformación dada
por el operador de permutación $\Pi_j\mathcal{O}\Pi_j$ para algún $\Pi_j\in
\mathcal{S}$, lo que indica que las salidas corresponden a una medición en cada
partícula, diferente a las que proporcionaría una medición ideal
\begin{equation}\label{eq:second-instrumentnp}
    \mathcal{I}_2(\rho)= \sum_{\lambda_i \in \Lambda } \fuzzy{P_{\lambda_i}}\tensor P_{\lambda_i}\rho P_{\lambda_i}.
\end{equation} 
Con esta alternativa el valor esperado se calcula como 
\begin{equation*}
    \begin{split}
        \left \la \mathcal{O} \right \ra_{\mathcal{I}_2}&=\tr\left( \left[\left(\mathcal{O}\right) \otimes \mathds{1}\right]\mathcal{I}_2\right) \\
        &=\tr\left(\left[ \left(\mathcal{O}\right)\otimes \mathds{1}\right]\sum_{\lambda_j \in \Lambda}\fuzzy{P_{\lambda_j}}\otimes P_{\lambda_j}{\rho}P_{\lambda_j} \right)\\
        &=\sum_{\lambda_j\in \Lambda} \tr\left(\mathcal{O}\fuzzy{P_{\lambda_j}}\right) \tr\left(P_{\lambda_j}\rho\right). \\
    \end{split}
\end{equation*} 
finalmente el valor esperado es \begin{equation}\label{eq:valor-esperado-2instrumentnp}
    \left \la \mathcal{O}\right \ra_{\mathcal{I}_2}=\sum_{\lambda_j\in \Lambda} \tr\left(\mathcal{O}\fuzzy{P_{\lambda_j}}\right) \tr\left(P_{\lambda_j}\rho\right).
\end{equation} 
Este valor esperado no corresponde a {\eqref{eq:expected-value-fm-general}} por
lo que análogamente se tiene una proposición más general para la equivalencia
de estos instrumentos en sistema de $N$ partículas.

\begin{proposition}\label{prop:Equivalencia-instrumentos-np}
    Para todo estado inicial $\rho$, los valores esperados de los instrumentos
cuánticos {\ref{eq:1instrumentnp}} y {\ref{eq:second-instrumentnp}} son
equivalentes si y solo si \[\left \langle \lambda_j \left|\Pi_l^\dagger
\mathcal{O} \Pi_l\right|\lambda_k\right\rangle=0,\forall j\ne k \text{ y }
\forall \Pi_l \in \mathcal{S}.\]
\end{proposition} 
La demostración es análoga a la de la proposición {\ref{prop:Equivalencia-instruments1-2}}.


% }}}
% }}}


