\chapter[RESULTADOS]{3. Resultados}
\section{Introducción} % {{{

\rrnote{Esqueleto: Intro: En los capítulos anteriores se estableció un base conceptual para realiza un análisis de las mediciones difusas. En este capítulo se utilizará este marco para realizar la descripción completa de las mediciones difusas en sistemas cuánticos.}

% }}}
\section{Mediciones difusas en sistemas de dos partículas} % {{{
% Intro {{{
% \rrnote{Esqueleto: En esta sección exponer los resultados ya trabajados y 
% explorar más opciones. Primero especificar condiciones del sistema de dos 
% partículas y de los observables. Inicialmente, el estado del sistema es $\rho  
% \in  \mathcal{H}_1\otimes \mathcal{H}_2$,  en las que se quiere realizar la 
% medición de un observable $A\otimes B$, donde $A$ y $B$ son ambos de dimensión 
% $d$.}


%Con fines ilustrativos, en este apartado se presentan los resultados
%principales obtenidos al trabajar con las siguientes condiciones.
%Inicialmente, el estado del sistema es $\rho  \in \mathcal{H}$ ,  en él  se quiere realizar la medición de un observable
%factorizable \[A\otimes B=\sum_{j,k}a_j b_k P_{a_j, b_k}=\sum_{j,k}a_j b_k
%P_{a_j}\otimes P_{b_k},\] siendo $P_{a_j}$ y $P_{b_k}$ los operadores de
%proyección que corresponden al valor propio $a_j$ del observable $A$ y el
%valor propio $b_k$ de $B$ respectivamente, de dimensión $d\times d$. Sin
%embargo, la medición realizada es difusa y siguiendo con la definición
%{\ref{def:medicion-difusa}}, la probabilidad que se realice la medición del
%observable $B\otimes A$ no es cero, sino $1-p$ y la posibilidad de realizar
%una medición ideal en el sistema  es $p$. 
Las mediciones difusas solo cobran sentido en un sistema compuesto por dos o
más partículas. Con fines explicativos en el capítulo anterior se comenzó a
analizar una medición difusa en el sistema más simple, el cual está conformado
únicamente por dos partículas. Antes de ampliar el análisis a sistemas más
complejos, con el propósito de acabar de clarificar e ilustrar de manera
intuitiva el concepto de medición difusa, en esta sección se presentan los
resultados principales y algunos ejemplos obtenidos al trabajar con un sistema
que tiene un estado inicial $\rho \in \mathcal{H}_1\otimes \mathcal{H}_2$ y un
observable de la forma $A\otimes B$.\par



% \rrnote{Esqueleto:Incluir imágenes de algún ejemplo de una medición difusa en 
% un sistema.  Con algún diagrama de cajas en las que se realiza una medición del 
% observable particular. Comparar  una medición ideal en un sistema de dos 
% partículas. Luego, en una segunda imagen agregar un diagrama que represente la 
% matriz swap y obtener salidas intercambiadas  con cierta probabilidad.}

En la siguiente imagen se ilustra un esquema de medición del observable
$\sigma_z\otimes \sigma_x$ en la que debido a ruido del entorno, existe una
posibilidad de realizar la medición errónea y en su lugar medir el observable
$\sigma_x\otimes \sigma_z$. La salida clásica de los observables es $(1)$  para
el observable $\sigma _z$ y $(-1)$ para el observable $\sigma_x$. Sin embargo
su valor esperado será la suma ponderada de los valores esperados de realizar
las dos mediciones que son verosímiles, es decir $p \tr(\sigma_z\otimes \sigma
\rho )+ (1-p)\tr(\sigma_x\otimes\sigma_z \rho)$.

\begin{figure}[H]
    \centering
    
  \begin{subfigure}[a]{1\textwidth}
    \centering
    \resizebox{12.5cm}{!}{\begin{quantikz}[font=\small, row sep=0.3cm,column sep=0.4cm]%
    \lstick[wires=3]{$\rho$}& \qw& \qw& \qw& \gate[wires=3,style={
    starburst,fill=yellow,draw=red,line width=2pt,inner
    xsep=-2pt,inner ysep=-5pt},label style=blue]{\text{
    noise}} & \qw& \qw& \qw& \meter[style={draw=blue}]{$\sigma_z$} \vcwdouble{1-9}{5-9}{1}&\qw&\qw&\qw&\rstick[wires=5]{$p\tr(\sigma_z\otimes \sigma_x\rho)+(1-p)\tr(\sigma_x\otimes\sigma_z\rho)$}\\
    &&&&&&&&&&&&\\
    & \qw&\qw&\qw& \qw& \qw& \qw& \qw& \qw&\qw&\meter[style={draw=blue}]{$\sigma_x$}\vcwdouble{3-11}{5-11}{-1}\qw&\qw&\\
    &&&&&&&&&&&&\\
    \lstick{cl}& \cw& \cw& \cw& \cw& \cw& \cw& \cw&\cw&\cw&\cw&\cw&\\
  \end{quantikz}}
\hfill
\caption{}
\end{subfigure}
\hfill

\begin{subfigure}{0.4\textwidth}
  \begin{flushleft}
    \resizebox{7cm}{!}{
      \begin{quantikz}[font=\small,row sep=0.3cm,column sep=0.4cm]%
      \lstick[wires=2]{$\rho$}&   \qw& \qw& \meter[style={draw=blue}]{$\sigma_z$} \vcwdouble{1-4}{4-4}{1}&\qw&\qw&\qw\rstick[wires=4]{$p \tr(\sigma_z \tensor \sigma_x \rho)$}&\\[0.1cm]
       &  \qw& \qw& \qw& \qw&\meter[style={draw=blue}]{$\sigma_x$}\vcwdouble{2-6}{4-6}{-1}&\qw& \\[0.1cm]
      &&&&&&&\\
      \lstick{cl}& \cw&  \cw&\cw&\cw&\cw&\cw&\\
    \end{quantikz}}
    \hfill
  \caption{}
  \end{flushleft}
   
\end{subfigure}


%  \begin{subfigure}[a]{0.5\textwidth}
 % \centering
  %\resizebox{6cm}{!}{
   % \begin{quantikz}[scale=1, row sep=0.3cm,column sep=0.4cm]%
    %\lstick{$\rho_{1}$}&  \qw& \qw& \qw& \qw& \qw&\qw\rstick[wires=3]{}\\ \\
    %&&&&&&& |[meter]| \vcwdouble{2-8}{5-8}{a_jb_k}\rstick{$A \tensor B$} \\
    %\lstick{$\rho_2$} & \qw& \qw& \qw& \qw& \qw&\qw&\\
    %\meter[style={draw=blue}]{B}\vcwdouble{2-11}{4-11}{b_k} \\
  %  & & && &&&\\
   % \lstick{cl}&  \cw& \cw& \cw& \cw&\cw&\cw&\cw\\
  %\end{quantikz}}
  %\hfill
%\caption{}
%\end{subfigure}

\begin{subfigure}{0.4\textwidth}
  \centering
\resizebox{8cm}{!}{
\begin{quantikz}[font=\small,scale=1, row sep=0.3cm,column sep=0.4cm]%
  \lstick[wires=2]{$\rho$}&\qw& \gate[swap]{} & \qw&  \meter[style={draw=blue}]{$\sigma_z$} \vcwdouble{1-5}{4-5}{ 1}&\qw&\qw&\qw\rstick[wires=4]{$(1-p)\tr(\sigma_x \tensor \sigma_x\rho)$}& \\
  & \qw& \qw& \qw& \qw& \qw&\meter[style={draw=blue}]{$\sigma_x$}\vcwdouble{2-7}{4-7}{-1}&\qw& \\
  &&&&&&&&\\
  \lstick{cl}& \cw& \cw& \cw&\cw&\cw&\cw&\cw&\\
\end{quantikz}}
\hfill
\caption{}
\end{subfigure}
\caption{\textbf{(a)} En la primera imagen se ilustran un estado inicial $\rho$, sin embargo el entorno no es ideal y es posible que ocurra una identificación errónea, en las que las salidas clásicas registradas para los observables $\sigma_z$ y $\sigma_x$ fueron $1$ y $-1$ respectivamente.\textbf{ (b)} En esta imagen se ilustra que con una probabilidad $p$ la identificación del sistema sea correcta y se realice la medición del observable $\sigma_z\tensor \sigma_x$.\textbf{ (c)} La imagen muestra que debido al ruido del entorno, las partículas experimentan un intercambio y con una probabilidad $(1-p)$ se realiza la medición del observable $\sigma_x\otimes \sigma_z$ }\label{diagrama-cajas}\source{elaboración propia}
\end{figure}


% }}}
\subsection{POVM y operadores de Kraus para mediciones difusas en sistemas de dos partículas}\label{Sec_POVM_para_mediciones_difusas} % {{{
 

A continuación se especifica la forma de utilizar las medidas generalizadas conjuntamente con los operadores de Kraus para aproximarse a describir enteramente estas mediciones. En la ecuación {\eqref{eq:ordenar-valoresperado-povm}} se encontró intuitivamente que los efectos para una medición difusa, son $\{\mathcal{F}_{2\text{p}}({P_{a_j,b_k}})\}$. Estos efectos proporcionan el mapeo de probabilidades\begin{equation}\label{eq:mapeo-probabilidades-2p}
    E(a_j b_k, \rho)= \tr(\mathcal{F}_{2\text{p}}({P_{a_j,b_k}})\rho).
    \end{equation} Sin embargo para obtener el estado posterior es a la medición es necesario descomponer estos efectos. Debido a que los operadores $P_{a_j,b_k }$ y $P_{b_k,a_j}$ son  positivos, entonces  $\mathcal{F}_{2\text{p}}({P_{a_j,b_k}})$ es positivo y por tanto se puede considerar los siguientes operadores\[K_{a_j,b_k}=\sqrt{\mathcal{F}_{2\text{p}}({P_{a_j,b_k}})}.\] Luego de introducir los operadores de Kraus el
    estado de la medición es posterior a la medición POVM será
    \begin{equation}\label{eq:estado-posterior-povm1-2p}\begin{split}\rho'_{a_j,b_k}&=\dfrac{K_{a_j,b_k} \rho
    K_{a_j,b_k}^\dagger}{\tr(K_{a_j,b_k}^\dagger K_{a_j,b_k} \rho)}\\
    &=\dfrac{\sqrt{\mathcal{F}_{2\text{p}}(P_{a_j,b_k})} \rho
    \sqrt{\mathcal{F}_{2\text{p}}({P_{a_j,b_k}})}}{\tr(\mathcal{F}_{2\text{p}}({P_{a_j,b_k}}) \rho)}\\
    &=\dfrac{\sqrt{p P_{a_j,b_k}+(1-p)SP_{a_j,b_k}S^\dagger}\rho\sqrt{p P_{a_j,b_k}+(1-p)SP_{a_j,b_k}S^\dagger}}{\tr((p P_{a_j,b_k}+(1-p)SP_{a_j,b_k}S^\dagger) \rho)}\\
    &=\dfrac{\sqrt{p P_{a_j,b_k}+(1-p)P_{b_k,a_j}}\rho\sqrt{p P_{a_j,b_k}+(1-p)P_{b_k,a_j}} }{\tr((p P_{a_j,b_k}+(1-p)P_{b_k,a_j}) \rho)}.
    \end{split}\end{equation}La descomposición en operadores de Kraus
    no es única. No obstante, el conjunto de operadores $\{K_{a_j,b_k}\}$ es adecuado para ofrecer una especificación detallada de las mediciones difusas, puesto que con ellos se puede obtener tanto el mapeo de probabilidades $\tr(K_{a_j,b_k}K_{a_j,b_k}^\dagger \rho)$, como el estado posterior a la medición $\rho_{a_j,b_k}'$.



% }}}
\subsection{Instrumentos cuánticos para dos partículas} % {{{
% Intro subsection {{{
% \rrnote{Esqueleto:Intro: En este apartado se busca hacer un análisis sobre
% las formulaciones de las alternativas de los instrumentos presentadas en la
% sección {\ref{sec:cap2instrumentos-cuanticos}}. Se obtiene el sistema clásico
% y cuántico para comparar las descripciones que proporcionan. Se verificar que
% las dos primeras alternativas cumplan con la condición necesaria para
% verificar que tengan  la misma distribución de probabilidad, con el valor
% esperado. Además de analizar el tercer instrumento y compararlo con el
% primero}

En este apartado se busca hacer un análisis sobre las formulaciones de las tres alternativas de los instrumentos presentadas en la sección {\ref{sec:cap2instrumentos-cuanticos}}. En éste, se obtiene el sistema clásico y cuántico para comparar las descripciones que proporcionan. Para una plena especificación de las mediciones difusas, las interpretaciones son intuitivamente equivalentes. Sin embargo, es indispensable verificar que las dos proporcionen el mismo mapeo de resultados. En principio, una condición necesaria para corroborar que tengan la misma distribución de probabilidad es que el valor esperado de los instrumentos genere el mismo resultado que el de la ecuación {\eqref{eq:Expected-Value-FM-2p}}. 


% }}}
\subsubsection{Primera alternativa de instrumento cuántico} % {{{

% \rrnote{Esqueleto: Se revisita la primera alternativa de los intrumentos
% cuánticos, en sistemas simples de dos partículas y con observables
% factorizables. En esta parte se  aplica la traza parcial sobre el sistema
% cuántico y el clásico para realizar el análisis del instrumento. Se calcula su
% valor esperado para saber si cumple con la ecuación del valor esperado}


El primer instrumento consiste en considerar que debido a un error en el
sistema cuántico, las partículas posiblemente experimenten un intercambio. Al
medir un observable ${A \tensor B}$ en el sistema representado por el estado $\rho$, se obtiene un resultado  en
el sistema clásico que puede ser cualquiera de los valores propios de este
observable. Luego, con cierta probabilidad el estado posterior será la
proyección del estado inicial del sistema al estado propio correspondiente a
las salida proporcionada. Sin embargo también es posible que el estado inicial
sufra una transformación, y sea este cambio del sistema el que se proyecte en
el espacio propio correspondiente a la salida del observable $A\tensor B$.


En la imagen {\ref{diagrama-cajas-primer-instrumento}} se puede apreciar a lo
que esto se refiere gráficamente.  En la figura, con una probabilidad $p$ se
realiza una medición ideal en la que se mide el observable $A$ en la primera
partícula y el observable $B$ en la segunda partícula. En esta medición ideal
el estado posterior será la proyección de $\rho_1=\tr_2(\rho_{1,2})$
al espacio propio
correspondiente a la salida proporcionada en el sistema clásico $a_j$,
conjuntamente con la proyección de $\rho_2=\tr_1(\rho_{1,2})$ al espacio propio de las
salida $b_k$. 
No obstante, la figura también muestra la parte borrosa de la medición en la que con una probabilidad $1-p$,
la medición del observable $B$ se realiza en el primer sistema y la medición
del observable $A$ se realiza en el segundo sistema, en consecuencia $\rho_2$ se proyecta al espacio propio que corresponde a la salida
$a_j$ y $\rho_1$ se proyecta al espacio de la salida $b_k$.

\begin{figure}[H]
  \centering
\begin{subfigure}[b]{0.4\textwidth}
\centering
   \resizebox{7cm}{!}{
      \begin{quantikz}[font=\normalsize,row sep=0.3cm,column sep=0.4cm]%
      \lstick[wires=4]{$p$}&&\lstick{\color{blue}{$\rho_{1}$}}& \qw& \qw& \meter[style={draw=blue}]{$A$} \vcwdouble{1-6}{4-6}{}&\qw&\qw&\qw&\qw\rstick{$P_{a_j}${\color{blue}{$\rho_1$}}$P_{a_j}$}&\\[0.1cm]
      &&\lstick{\color{red}{$\rho_2$}} &  \qw& \qw& \qw& \qw&\meter[style={draw=red}]{$B$}\vcwdouble{2-8}{4-8}{}&\qw&\qw\rstick{$P_{b_k}${\color{red}{$\rho_2$}}$P_{b_k}$} \\[0.1cm]
      &&&&&&&&&&&\\
      &&\lstick{cl}& \cw&  \cw&\cw&\cw&\cw&\cw&\cw&\\
      &&&&&\rstick{$a_j$}&&\rstick{$b_k$}&&&&\\
    \end{quantikz}
    }
    \hfill
  \caption{}\label{fig:1-instrumento-ideal}
\end{subfigure}

\hfill

\begin{subfigure}[c]{0.4\textwidth}
\centering
  \resizebox{9cm}{!}{
\begin{quantikz}[font=\normalsize,scale=1, row sep=0.3cm,column sep=0.4cm]%
  \lstick[wires=4]{$1-p$}&&\lstick{\color{blue}{$\rho_{1}$}}&\qw& \gate[swap]{} & \qw&  \meter[style={draw=blue}]{$A$} \vcwdouble{1-7}{4-7}{}&\qw&\qw&\qw&\qw\rstick{$P_{a_j}${\color{red}{$\rho_2$}}$P_{a_j}$}\\
  &&\lstick{\color{red}{$\rho_{2}$}} & \qw& \qw& \qw& \qw& \qw&\meter[style={draw=red}]{$B$}\vcwdouble{2-9}{4-9}{}&\qw&\qw \rstick{$P_{b_k}${\color{blue}{$\rho_1$}}$P_{b_k}$}\\
  &&&&&&&&&&&\\
  &&\lstick{cl}& \cw& \cw& \cw&\cw&\cw&\cw&\cw&\cw&\\
  &&&&&&\rstick{$a_j$}&&\rstick{$b_k$}&&&\\
\end{quantikz}
}
\hfill
\caption{}\label{fig:1-instrumento-non-ideal}
\end{subfigure}
\caption{\textbf{(a)}Con una probabilidad $p$ se realiza una medición ideal del observable $A\tensor B$, con un estado posterior dado por $P_{a_j,b_k}\rho_1\otimes \rho_2P_{a_j,b_k}$, cuando la salida indicada es $a_j b_k$. \textbf{ (b)}Con probabilidad $1-p$ ocurre un intercambio de las partículas y estado posterior correspondiente a la salida $a_j b_k$ será $P_{a_j,b_k}\rho_2\otimes \rho_1P_{a_j,b_k}$.}\label{diagrama-cajas-primer-instrumento}\source{elaboración propia}
\end{figure}


Para un sistema de dos partículas el instrumento dado en la ecuación {\eqref{eq:primer-instrumento-general}}, puede escribirse como\begin{equation}\label{eq:fisrtinstrument2p}
    \begin{split}
        \mathcal{I}_1{(\rho)}&=\sum_{j,k}P_{a_j,b_k}\otimes P_{a_j,b_k} \mathcal{F}_{2\text{p}}(\rho) P_{a_j,b_k}\\
        &=\sum_{j,k}P_{a_j,b_k}\otimes[p P_{a_j,b_k}\rho P_{a_j,b_k}+(1-p)P_{a_j,b_k}S\rho S^\dagger P_{a_j,b_k}]\\
        &=\sum_{j,k}P_{a_j,b_k}\otimes P_{a_j,b_k} [p \bra{a_j b_k}\rho \ket{a_j,b_k}+(1-p)\bra{b_k a_j }\rho \ket {b_k a_j}],
\end{split}
\end{equation}




% \rrnote{Esqueleto:Para una medida no selectiva el mapea de resultados se
% obtiene simplemente aplicando una traza parcial sobre el sistema cuántico. Se
% calcula esta traza parcial. Además el se calcula la traza parcial del sistema
% clásico para obtener un canal cuántico para una medición selectiva. }
% 

Es importante notar que el sistema clásico  puede diferir del estado inicial, pero  indica correctamente el estado posterior a la medición en todos los casos. Además, para una medida no selectiva el mapeo de resultados se obtiene simplemente aplicando una traza parcial sobre el sistema cuántico 
\begin{equation}\label{eq:traza_q_1instrument2p}
    \begin{split}
        \tr_{\text{qu}}(\mathcal{I}_1{(\rho)})&=\sum_{j,k}\tr_{\text{qu}}(P_{a_j,b_k}\otimes P_{a_j,b_k} \mathcal{F}_{2\text{p}}(\rho) P_{a_j,b_k})\\
        &=\sum_{j,k}P_{a_j,b_k}\tr(p P_{a_j,b_k}\rho P_{a_j,b_k}+(1-p)P_{a_j,b_k}S\rho S^\dagger P_{a_j,b_k})\\
        &=\sum_{j,k}P_{a_j,b_k}[p \bra{a_j b_k}\rho\ket{a_j b_k} +(1-p)\bra{b_k a_j}\rho \ket{b_k a_j}].
\end{split}
\end{equation} 


Para obtener el mapa de resultados para una medición selectiva simplemente se
multiplica el operador de proyección correspondiente a la salida obtenida al
resultado anterior, y se calcula la traza\[\begin{split}
    E(a_i,b_l,\rho)&=\tr\left(P_{a_i, b_l}\left[\sum_{j,k}\tr_{\text{qu}}(P_{a_i,b_l}\otimes P_{a_i,b_l} \mathcal{F}_{2\text{p}}(\rho) P_{a_j,b_k})\right]\right)\\
    &=\tr\left(\sum_{j,k} P_{a_i,b_l}P_{a_j,b_k}[p \bra{a_j b_k}\rho\ket{a_j b_k} +(1-p)\bra{b_k a_j}\rho \ket{b_k a_j}]\right)\\
    &=\sum_{j,k}\tr\left( P_{a_i,b_l}P_{a_i,b_l}\right)[p \bra{a_j b_k}\rho\ket{a_j b_k} +(1-p)\bra{b_k a_j}\rho \ket{b_k a_j}]\\
    &=\sum_{j,k}\delta^{i,l}_{j,k}[p \bra{a_j b_k}\rho\ket{a_j b_k} +(1-p)\bra{b_k a_j}\rho \ket{b_k a_j}]\\
    &=p \bra{a_i b_l}\rho\ket{a_i b_l} +(1-p)\bra{b_l a_i}\rho \ket{b_l a_i}.
\end{split}\]Es importante notar que este mapeo de probabilidades resultante es exactamente el mismo al de la ecuación {\eqref{eq:mapeo-probabilidades-2p}}

Por otro lado, para obtener el estado posterior luego de haber realizado la medición, se debe realizar la traza parcial sobre el sistema clásico de la siguiente manera\cpnote{También esta frase está como volando} \rrnote{quité la frase anterior.}
\begin{equation}\label{eq:traza_cl_1instrument2p}
    \begin{split}
      \tr_{\text{cl}}(\mathcal{I}_1{(\rho)})&=\sum_{j,k}\tr_{\text{cl}}(P_{a_j,b_k}\otimes P_{a_j,b_k} \mathcal{F}_{2\text{p}}(\rho) P_{a_j,b_k})\\
    &=\sum_{j,k}P_{a_j,b_k} \mathcal{F}_{2\text{p}}(\rho) P_{a_j,b_k}.
\end{split}
\end{equation}Para una medida selectiva el estado posterior estará dado por
\[\E_{a_j,b_k}(\rho)=P_{a_j,b_k} \mathcal{F}_{2\text{p}}(\rho) P_{a_j,b_k}.\]



\rrnote{Esqueleto: Por otro lado, se calcula el valor esperado del primer instrumento y se compara con el valor esperado original de la medición difusa  }
  

Por otra parte, el valor esperado del primer instrumento se calcula de la siguiente forma \begin{equation*}
    \begin{split}
        \la A\otimes B \ra_{\mathcal{I}_1}&=\tr([(A\otimes B) \otimes \mathds{1}]\mathcal{I}_1) \\
        &=\tr\left([(A\otimes B) \otimes \mathds{1}]\sum_{jk}P_{a_j b_k}\otimes (pP_{a_j b_k}\rho P_{a_j b_k}+ (1-p)P_{a_j b_k}S\rho S P_{a_j b_k})\right)\\
        &=\sum_{jk} \tr((A\otimes B) P_{a_j b_k}) \tr\left((pP_{a_j b_k}\rho P_{a_j b_k}+ (1-p)P_{a_j b_k}S\rho S P_{a_j b_k})\right) \\
        &=\sum_{jk} \tr\left(\sum_{il}a_{il}P_{a_i b_l} P_{a_j b_k}\right) \tr((pP_{a_j b_k}\rho P_{a_j b_k}+ (1-p)P_{a_j b_k}S\rho S P_{a_j b_k})) \\
        &=\sum_{jk} p\tr(a_{j}b_k P_{a_j b_k}) \tr(P_{a_j b_k}\rho)+ (1-p)\tr(a_{j}b_k P_{a_j b_k})\tr(P_{a_j b_k}S\rho S P_{a_j b_k}) \\
        &=\sum_{jk}  (a_{j}b_k)p  \tr(P_{a_j b_k}\rho)+(a_{j}b_k) (1-p) \tr(P_{a_j b_k}S\rho S P_{a_j b_k}),\\
    \end{split}
\end{equation*}  finalmente el valor esperados del instrumento es \begin{equation}\label{eq:expectedValueFirstInstrument2p}
    \begin{split}
        \la A\otimes B \ra_{\mathcal{I}_1}&=p \tr(A\otimes B \rho)+ (1-p)\tr(B\tensor A \rho ). \\
    \end{split}
\end{equation} Claramente con este instrumento se obtiene el valor esperado correcto para describir una medición difusa.


% }}}
\subsubsection{Segunda alternativa de instrumento cuántico} % {{{

\rrnote{Esqueleto:Después de haber hecho el cálculo para el valor esperado del primer instrumento, el siguiente es el caso en el que se supone que el error ocurre en la lectura de los resultados y se desarrolla el segundo instrumento para un sistema de dos partículas.}


Tras haber analizado el primer instrumento, el siguiente escenario involucra un error en la interpretación de los resultados. El segundo instrumento implica que, debido a un fallo en el sistema clásico, la lectura de las salidas sea incorrecta. Al medir un observable ${A \tensor B}$ en el sistema, con cierta probabilidad el estado posterior será la proyección del estado inicial del sistema al estado propio correspondiente a las salida proporcionada. No obstante, existe la posibilidad que el estado inicial sea proyectado a algún estado propio del observable pero al leer las salidas de la medición estas correspondan a salidas del observable $B\otimes A$.

En la figura {\ref{fig; diagrama-cajas-segundo-instrumento}} se ilustra las dos posibilidades mencionadas anteriormente. En la primera de ellas se realiza una medición proyectiva ideal, donde se mide el observable $A$ en el primer subsistema y el observable $B$ en el segundo, con probabilidad $p$. El estado posterior a la medición será el estado $\rho_1=\tr_2(\rho_{1,2})$ proyectándose al espacio propio de la salida $a_j$, junto con la proyección del estado $\rho_2=\tr_1(\rho_{1,2})$ al espacio propio de $b_k$. Adicionalmente, la figura muestra la parte imperfecta de la medición en la que con una probabilidad $1-p$, se realiza la medición de $A$ en la primera partícula y la medición de $B$ en la segunda, sin embargo la salida de la medición para el primer subsistema es $b_k$,un resultado correspondiente al observable $B$, y la salida que se ofrece en la medición del segundo subsistema es un valor propio de $A$, $a_j$. Aunque los estados posteriores a la medición sean correctos, las salidas clásicas están intercambiadas.




\begin{figure}[H]
    \centering
    
\begin{subfigure}{0.4\textwidth}
  \begin{flushleft}
   \resizebox{7.5cm}{!}{
      \begin{quantikz}[font=\large,row sep=0.3cm,column sep=0.4cm]%
      \lstick[wires=2]{$\rho_{1,2}$}&\qw& \qw&\meter[style={draw=blue}]{$A$} \vcwdouble{1-4}{4-4}{}&\qw&\qw&\qw&\qw\rstick{\small $P_{a_j}${\color{blue}{$\rho_1$}}$P_{a_j}$}&&&&&\rstick[wires=5]{$p$}\\[0.1cm]
       &  \qw& \qw& \qw& \qw&\meter[style={draw=red}]{$B$}\vcwdouble{2-6}{4-6}{}&\qw&\qw\rstick{\small $P_{b_k}${\color{red}{$\rho_2$}}$P_{b_k}$} &&&&\\[0.1cm]
      &&&&&&&&&&&&\\
      \lstick{cl}& \cw&  \cw&\cw&\cw&\cw&\cw&\cw&&&&\\
      &&&\rstick{$a_j$}&&\rstick{$b_k$}&&&&&&&\\
    \end{quantikz}
    }
    \hfill
  \caption{}
  \end{flushleft}
\end{subfigure}

\hfill

\begin{subfigure}{0.4\textwidth}
    \begin{flushleft}
     \resizebox{8.5cm}{!}{
        \begin{quantikz}[font=\large,row sep=0.3cm,column sep=0.4cm]%
          \lstick[wires=2]{\color{black}{$\rho_{1,2}$}}&   \qw& \qw& \meter[style={draw=blue}]{$A$} \vcwdouble{1-4}{4-4}{}&\qw&\qw&\qw&\qw\rstick{\small $P_{a_j}${\color{blue}{$\rho_1$}}$P_{a_j}$}&&&&&\rstick[wires=5]{$1-p$}\\[0.1cm]
         &  \qw& \qw& \qw& \qw&\meter[style={draw=red}]{$B$}\vcwdouble{2-6}{4-6}{}&\qw&\qw\rstick{\small$P_{b_k}${\color{red}{$\rho_2$}}$P_{b_k}$}&&&& \\[0.1cm]
        &&\gate[wires=2,style={
          starburst,fill=yellow,draw=red,line width=2pt,inner
          xsep=1pt,inner ysep=-7pt},label style=blue]{\text{\small
          noise}}&&&&&&&&&&\\
        \lstick{cl}& \cw&\cw&\cw&\cw&\cw&\cw&\cw&&&&&\\
        &&&\rstick{$b_k$}&&\rstick{$a_j$}&&&&&&&\\
      \end{quantikz}
      }
      \hfill
    \caption{}
    \end{flushleft}
  \end{subfigure}
\caption{\textbf{(a)}.\textbf{ (b)}}\label{diagrama-cajas-segundo-instrumento}\source{elaboración propia}
\end{figure}


El segundo instrumento para un sistema de dos partículas es \begin{equation}\label{eq:second-instrument-2p}
    \begin{split}
        \mathcal{I}_2(\rho)&=\sum_{j,k}\mathcal{F}_{2\text{p}}(P_{a_j,b_k})\otimes P_{a_j,b_k} \rho P_{a_j,b_k}\\
        &=\sum_{j,k}[pP_{a_j,b_k}+(1-p)SP_{a_j,b_k}S^\dagger]\otimes P_{a_j,b_k} \rho P_{a_j,b_k}\\
        &=\sum_{j,k} p |a_j b_k\rala a_j b_k| \otimes P_{a_j,b_k} \rho P_{a_j,b_k}+(1-p)|b_k a_j\rala b_k a_j|\otimes P_{a_j,b_k} \rho P_{a_j,b_k}.\\
    \end{split}
\end{equation} 

\rrnote{Esqueleto: Se interpretan los resultados comparando con los obtenidos con el primer instruemento. Se desarrolla la traza parcial sobre el sistema cuántico del instrumento para obtener el mapeo de resultado  para una medición no selectiva es el siguiente. Además el se calcula la traza parcial del sistema clásico para obtener un canal cuántico para una medición selectiva.}


En este caso es posible notar que el sistema clásico no siempre coincide con el estado posterior. La salida de la medición puede indicar que es $\ket{b_k a_j}$, y el estado posterior no será éste. De la misma manera al instrumento anterior, se desarrolla la traza parcial sobre el sistema cuántico del instrumento para obtener el mapeo de resultado  para una medición no selectiva \begin{equation}\label{eq:tr_q_second-instrument-2p}
    \begin{split}
        \tr_{\text{qu}}(\mathcal{I}_2(\rho))&=\sum_{j,k}\tr_{\text{qu}}(\mathcal{F}_{2\text{p}}(P_{a_j,b_k})\otimes P_{a_j,b_k} \rho P_{a_j,b_k})\\
        &=\sum_{j,k} (pP_{a_j,b_k}+(1-p)SP_{a_j,b_k}S^\dagger) \bra{a_j,b_k} \rho \ket{a_j,b_k}.\\
    \end{split}
\end{equation} 


\rrnote{Esqueleto: En el contexto de esta interpretación, se calcula el valor esperado de esta alternativa.}

Adicionalmente, el valor esperado de este instrumento, se calcula de la siguiente manera\begin{equation*}
    \begin{split}
        \la A\otimes B \ra_{\mathcal{I}_2}&=\tr([(A\otimes B) \otimes \mathds{1}]\mathcal{I}_2)\\
        &=\tr\left([(A\otimes B) \otimes \mathds{1}]\sum_{j,k}\mathcal{F}(P_{a_j,b_k})\otimes P_{a_j,b_k} \rho P_{a_j,b_k}\right)\\
        &=\sum_{j,k} p\tr((A\otimes B)P_{a_j,b_k}) \tr (P_{a_j,b_k} \rho P_{a_j,b_k})\\&
        +(1-p )\tr((A\otimes B)SP_{a_j,b_k}S) \tr(P_{a_j,b_k} \rho P_{a_j,b_k})\\
        &=\sum_{j,k}[a_{j} b_k p\tr(P_{a_j,b_k})+(1-p)\tr( (B\otimes A)P_{a_j,b_k}) ]\tr(P_{a_j,b_k} \rho),\\
    \end{split}
\end{equation*}simplificando se obtiene que \begin{equation}\label{eq:expectationvalueSecondInstrument2p}
    \begin{split}
        \la A\otimes B \ra_{\mathcal{I}_2}&=p\tr(A\tensor B\rho)+\sum_{j,k}{\color{blue}(1-p)\tr( (B\otimes A)P_{a_j,b_k})}\tr(P_{a_j,b_k} \rho). \\
    \end{split}
\end{equation}
Nótese que el término remarcado en azul difumina la parte de los resultados. Sin embargo, en general esta ecuación no es igual a la ecuación {\eqref{eq:Expected-Value-FM-2p}}, lo cual se discutirá más adelante.




% }}}
\subsubsection{Tercera alternativa} % {{{
    Esta última interpretación de la medición difusa incluye la posibilidad de no saber en que estado se encuentra el sistema luego de realizar la medición. En este instrumento se representa la posibilidad que al medirse el observable $A\otimes B$ se realice con cierta probabilidad una medición proyectiva ideal. Pero, también es verosímil que se realice una medición en la que no sea posible saber en que espacio se haya proyectado el sistema. Es igual de posible que se haya proyectado sobre un espacio propio del observable $A\otimes B$ como es posible que se haya proyectado en un espacio propio de $B\otimes A$,  condicionados ambos espacios por las salidas clásicas. 

    En la figura {\ref{fig: diagrama-cajas-tercer-instrumento}} se muestra esta tercera interpretación. Primero, se ilustra de la posibilidad de realizar una medición proyectiva ideal del observable $A\otimes B$ con una probabilidad $q$, al igual que en los dos casos anteriores. Sin embargo con una probabilidad $1-q$, se realiza una medición en la que no es posible saber si la partículas se intercambian o no, por esta razón el estado será proyectado al espacio dado por la superposición de los operadores de proyección de $A\otimes B$ y $B\otimes A$ correspondientes a la salidas $a_j$ y $ b_k$, las cuales son indicadas por el sistema clásico, es decir los operadores  $P_{a_j,b_k}$ y $P_{b_k,a_j}$ respectivamente. 





\begin{figure}[H]
  \centering
\begin{subfigure}[b]{0.4\textwidth}
\centering
   \resizebox{7cm}{!}{
      \begin{quantikz}[font=\normalsize,row sep=0.3cm,column sep=0.4cm]%
      \lstick[wires=2]{{$\rho_{1,2}$}}& \qw& \qw& \meter[style={draw=blue}]{$A$} \vcwdouble{1-4}{4-4}{}&\qw&\qw&\qw&\qw\rstick{$P_{a_j}${\color{blue}{$\rho_1$}}$P_{a_j}$}&&&&&\rstick[wires=5]{$q$}\\[0.1cm]
       &  \qw& \qw& \qw& \qw&\meter[style={draw=red}]{$B$}\vcwdouble{2-6}{4-6}{}&\qw&\qw\rstick{$P_{b_k}${\color{red}{$\rho_2$}}$P_{b_k}$}&&&& \\[0.1cm]
      &&&&&&&&&&&&&\\
      \lstick{cl}& \cw&  \cw&\cw&\cw&\cw&\cw&\cw&&&&&\\
      &&&\rstick{$a_j$}&&\rstick{$b_k$}&&&&&&&\\
    \end{quantikz}
    }
    \hfill
  \caption{}\label{fig:3-instrumento-ideal}
\end{subfigure}

\hfill

\begin{subfigure}[a]{0.5\textwidth}
\begin{flushleft}
    \resizebox{11cm}{!}{
        \begin{quantikz}[font=\normalsize,scale=1, row sep=0.3cm,column sep=0.4cm]%
          \lstick[wires=2]{{$\rho_{1,2}$}}&\qw&\gate[wires=2]{?}& \qw&\meter[style={draw=blue}]{$A$} \vcwdouble{1-5}{4-5}{}&\qw&\qw&\qw&\qw\rstick[wires=2]{\tiny $\dfrac{P_{a_j,b_k}+P_{b_k,a_j}}{2}${{$\rho_{1,2}$}}$\dfrac{P_{a_j,b_k}+P_{b_k,a_j}}{2}$}&&&&&&&&&&&&\rstick[wires=5]{$1-q$}\\
          & \qw& \qw& \qw& \qw& \qw&\meter[style={draw=red}]{$B$}\vcwdouble{2-7}{4-7}{}&\qw&\qw&&&&&&&&&&&&\\
          &&&&&&&&&&&&&&&&&&&&&\\
          \lstick{cl}& \cw& \cw& \cw&\cw&\cw&\cw&\cw&\cw&&&&&&&&&&&&&\\
          &&&&\rstick{$a_j$}&&\rstick{$b_k$}&&&&&&&&&&&&&&&\\
        \end{quantikz}
        }
        \hfill
        \caption{}\label{fig:3-instrumento-non-ideal}
\end{flushleft}
\end{subfigure}
\caption{\textbf{(a)} Con una probabilidad $q$ se realiza una medición ideal del observable $A\tensor B$, con un estado posterior dado por $P_{a_j,b_k}\rho_{1,2}P_{a_j,b_k}$, cuando la salida indicada es $a_j b_k$.\textbf{ (b)} Con probabilidad $1-q$ ocurre un error y no se conoce con certeza el estado posterior correspondiente a la salida $a_j b_k$, estará dado por una superposición de dos estados $\dfrac{P_{a_j,b_k}+P_{b_k,a_j}}{2}\rho_{1,2}\dfrac{P_{a_j,b_k}+P_{b_k,a_j}}{2}$.}\label{fig: diagrama-cajas-tercer-instrumento}\source{elaboración propia}
\end{figure}



El tercer instrumento cuántico puede desarrollarse de la siguiente forma
\begin{equation}\label{eq:quantum-instrument-3-desarrollo}
    \begin{split}
        \mathcal{I}_3(\rho)&=q\sum_{m,n}  P_{a_m,b_n}\otimes P_{a_m,b_n}\rho P_{a_m,b_n}\\
        &+(1-q)\left[\sum_{(j,k)\in K}P_{a_j,b_k} \otimes P^{K}_{a_j,b_k}\rho P^{K}_{a_j,b_k}+\sum_{(i,l) \in L}P_{a_i,b_l} \otimes  \dfrac{1}{2}P^{L}_{a_i,b_l}\rho P^L_{a_i,b_l}\right]\\
        &=\sum_{(j,k)\in K}  P_{a_j,b_k}\otimes P_{a_j,b_k}\rho P_{a_j,b_k}+q\sum_{(i,l)\in L}P_{a_i,b_l} \otimes P_{a_i,b_l}\rho P_{a_i,b_l}\\
        &(1-q)\sum_{(i,l) \in L}P_{a_i,b_l} \otimes  \dfrac{1}{2}P^{L}_{a_i,b_l}\rho P^L_{a_i,b_l}\\
        &=\sum_{(j,k)\in K}  P_{a_j,b_k}\otimes P_{a_j,b_k}\rho P_{a_j,b_k}+q\sum_{(i,l)\in L}P_{a_i,b_l} \otimes P_{a_i,b_l}\rho P_{a_i,b_l}\\
        &(1-q)\sum_{(i,l) \in L}P_{a_i,b_l} \otimes  \dfrac{1}{2}\left[P_{a_i,b_l}+P_{b_l,a_i}\right]\rho \left[P_{a_i,b_l}+P_{b_l,a_i}\right].\\
    \end{split}
\end{equation}En este instrumento se asume que los operadores de proyección de los observables $A\otimes B$ y $B\otimes A$ tales que comparten el mismo valor propio, son ortogonales. En este caso, la salida clásica no difiere del estado inicial pero existe la posibilidad que se desconozca si la salida clásica coincida o no  con el estado posterior. 

\rrnote{Esqueleto:De igual forma que en los otros instruementos se desarrolla el calculo de la traza parcial para obtener el mapeo de resultados y el operador de densidad para el sistema cuántico.  }

Para este instrumento, se obtiene el mapeo de resultados, de igual forma que los dos anteriores, realizando la traza parcial sobre el sistema cuántico como sigue\begin{equation}\label{eq:traza_q_3instrument2p}
    \begin{split}
       \tr_{\text{qu}}\left( \mathcal{I}_3(\rho)\right)&=\sum_{(j,k)\in K}  P_{a_j,b_k}\tr(P_{a_j,b_k}\rho)+q\sum_{(i,l)\in L}P_{a_i,b_l} \tr(P_{a_i,b_l}\rho) \\
        &+\dfrac{(1-q)}{2}\sum_{(i,l) \in L}P_{a_i,b_l} \tr(\left[P_{a_i,b_l}+P_{b_l,a_i}\right]\rho)\\
        &=\sum_{(j,k)\in K}  P_{a_j,b_k}\la a_i b_l|\rho|a_i b_l\ra +q\sum_{(i,l)\in L}P_{a_i,b_l} \la a_i b_l|\rho|a_i b_l\ra \\
        &+\dfrac{(1-q)}{2}\sum_{(i,l) \in L}P_{a_i,b_l} \left[ \la a_i b_l|\rho|a_i b_l\ra + \la b_l a_i|\rho| b_l a_i\ra \right].\\
        &=\sum_{(j,k)\in K}  P_{a_j,b_k}\la a_i b_l|\rho|a_i b_l\ra \\
        &+\dfrac{(1+q)}{2}\sum_{(i,l)\in L}P_{a_i,b_l} \la a_i b_l|\rho|a_i b_l\ra +\dfrac{(1-q)}{2}\sum_{(i,l) \in L}P_{a_i,b_l} \la b_l a_i|\rho| b_l a_i\ra.\\
    \end{split}
\end{equation}Nótese que el último término de la primera igualdad  se puede simplificar de esta manera puesto que se supone que los operadores $P_{a_j,b_k}^L$ son de proyección. 

%Esta ecuación puede compararse con la ecuación {\eqref{eq:traza_q_1instrument2p}} y ciertamente puede realizarse el mismo análisis para obtener el mapeo de resultados.




\rrnote{Esqueleto:Finalmente, se calcula el valor esperado de este último instrumento cuántico.}

Asimismo, es posible calcular el valor esperado de este nuevo instrumento \begin{equation*}
    \begin{split}
        \la A\otimes B \ra_{\mathcal{I}_3}&=\tr([(A\otimes B) \otimes \mathds{1}]\mathcal{I}_3)\\
        &=\sum_{(j,k)\in K} \tr((A\otimes B )P_{a_j,b_k})\tr(P_{a_j,b_k}\rho)+q\sum_{(i,l)\in L}\tr((A\otimes B)P_{a_i,b_l}) \tr(P_{a_i,b_l}\rho)\\
        &+\dfrac{(1-q)}{2}\sum_{(i,l) \in L}\tr((A\otimes B )P_{a_i,b_l}) \tr(\left[P_{a_i,b_l}+P_{b_l,a_i}\right]\rho)\\
        &=\sum_{(j,k)\in K} a_j b_k\tr(P_{a_j,b_k}\rho)+q\sum_{(i,l)\in L}a_i b_l \tr(P_{a_i,b_l}\rho)\\
        &+\dfrac{(1-q)}{2}\sum_{(i,l) \in L} a_i b_l\tr(\left[P_{a_i,b_l}+P_{b_l,a_i}\right]\rho)\\
        &=\sum_{(j,k)\in K} a_j b_k \tr(P_{a_j, b_k}\rho)\\
        &+\dfrac{(1+q)}{2}\sum_{(i,l)\in L}a_i b_l \tr(P_{a_i,b_l}\rho) +\dfrac{(1-q)}{2}\sum_{(i,l) \in L}a_i b_l \tr(P_{b_l, a_i}\rho).
    \end{split}
\end{equation*}En suma, el valor esperado para este instrumento es \begin{equation}\label{eq:expectedValueThirdInstrument2p}
    \begin{split}
        \la A\otimes B \ra_{\mathcal{I}_3}&=\sum_{(j,k)\in K} a_j b_k \tr(P_{a_j, b_k}\rho)+\sum_{(i,l)\in L}a_i b_l \left[\tfrac{(1+q)}{2} \tr(P_{a_i,b_l}\rho) +\tfrac{(1-q)}{2}\tr(P_{b_l, a_i}\rho)\right].
    \end{split}
\end{equation} Este valor esperado puede conllevar más álgebra debido a la interpretación que se le da al estado posterior a la medición, sin embargo el valor esperado en este caso es equivalente al del primer instrumento y esta equivalencia será discutida en la siguiente sección. 






% }}}
% }}}
\subsection{Equivalencia de los instrumentos} % {{{
\rrnote{Esqueleto: Se hace una comparación entre los instrumentos anteriores y se expone por qué resultaron no ser equivalentes. En esta sección se comenta que se esperaba que los instrumentos fueran equivalentes pero resultaron distintos y por ello vale la pena proponer en qué condiciones sí lo son.}

Debido a la manera en que se diseñaron los tres instrumentos, se esperaba que el valor esperado de cada uno fuera el adecuado. Como se observó en las secciones anteriores, los instrumentos propuestos no son iguales. Lamentablemente, el valor esperado del segundo instrumento  no es el correcto. Por lo tanto, es importante preguntarse, ¿bajo qué condiciones estos instrumentos son equivalentes?




\rrnote{Esqueleto: Se presenta la proposición que detalla las condiciones de equivalencia del valor esperado entre los dos instrumentos y su demostración.}


A continuación se presenta la proposición que detalla las condiciones de equivalencia del valor esperado entre los dos primeros instrumentos y su demostración.


\begin{proposition}\label{prop:Equivalencia-instruments1-2}
    Para todo estado inicial $\rho$, los valores esperados de las alternativas de los instrumentos cuánticos {\eqref{eq:fisrtinstrument2p}} y {\eqref{eq:second-instrument-2p}}, son equivalentes si y solo si \[\la a_j b_k|B\otimes A|a_{j'}b_{k'}\ra=0, \forall j,k\ne j',k'.\] 
\end{proposition}

\begin{proof}
    $(\Rightarrow)$ Supongamos que para todo estado inicial $\rho$ los valores esperados de los instrumentos son iguales. El primer término del lado derecho de la ecuación {\eqref{eq:expectedValueFirstInstrument2p}} y el primer término de la derecha en {\eqref{eq:expectationvalueSecondInstrument2p}} son iguales, por lo tanto se igualan los segundos términos de estas ecuaciones, i.e.\begin{equation}\label{equivalencia_terminos}\tr(B\otimes A \rho)=\sum_{j,k}\tr((B\otimes A) P_{a_j,b_k})\tr(P_{a_j b_k}\rho ).\end{equation} Por la linealidad de la traza se puede reacomodar el lado derecho de esta última ecuación: \[\begin{split}\sum_{jk}\tr((B\otimes A) P_{a_j,b_k})\tr(P_{a_j b_k}\rho )&= \sum_{jk}\tr[\tr((B\otimes A) P_{a_j,b_k})(P_{a_j b_k}\rho )]\\
        &=\tr\left[\left(\sum_{j,k}\tr((B\otimes A) P_{a_j,b_k})P_{a_j b_k}\right)\rho \right]\end{split}.\]  reescribiendo la ecuación {\ref{equivalencia_terminos}} 
        \[\tr(B\otimes A \rho)=\tr\left[\left(\sum_{j,k}\tr((B\otimes A) P_{a_j,b_k})P_{a_j b_k}\right)\rho \right]\\ \]
    \begin{equation}\label{trazaCero}\begin{split}
        \tr\left[\left(B\otimes A-\left(\sum_{j,k}\tr((B\otimes A) P_{a_j,b_k})P_{a_j b_k}\right)\right)\rho \right]=0\\ \end{split}.\end{equation}
        Al expandir $B\otimes A$ en la base de vectores propios de $A\otimes B$ obtenemos que \[B\otimes A=\sum_{j,k,j',k'}\tr(B\otimes A |a_j b_k\rala a_{j'}b_{k'}|)|a_j b_k\rala a_{j'}b_{k'}|,\] desarrollando la ecuación {\ref{trazaCero}}   
    \[\begin{split}
        \tr\left[\left(B\otimes A-\left(\sum_{j,k}\tr((B\otimes A) P_{a_j,b_k})P_{a_j b_k}\right)\right)\rho\right]&=0\\
    \tr\left[\left(\sum_{j,k,j',k'}\tr(B\otimes A |a_j b_k\rala a_{j'}b_{k'}|)|a_j b_k\rala a_{j'}b_{k'}|-\left(\sum_{j,k}\tr((B\otimes A) P_{a_j,b_k})P_{a_j b_k}\right)\right)\rho\right]&=0\\
    \tr\left[\left(\sum_{j,k\ne j',k'}\tr(B\otimes A |a_j b_k\rala a_{j'}b_{k'}|)|a_j b_k\rala a_{j'}b_{k'}|\right)\rho\right]&=0\\
    \tr\left[\left(\sum_{j,k\ne j',k'}  \langle a_j b_k |B\otimes A|a_{j'}b_{k'}\rangle |a_j b_k\rala a_{j'}b_{k'}|\right)\rho\right]&=0.\\
    \end{split}\]
    De la última ecuación, para cualquier estado inicial $\rho$, se obtiene que\footnote[1]{ver lema {\ref{lemma_traza_cero}}.} 
    \[\sum_{j,k\ne j', k'} \langle a_j b_k |B\otimes A|a_{j'}b_{k'}\rangle |a_j b_k\rala a_{j'}b_{k'}|=\mathbf{0},\] luego por independencia lineal de la base de vectores propios  
     \[\begin{array}{cc}
        \langle a_j b_k |B\otimes A|a_{j'}b_{k'}\rangle=0& \forall j,k\ne j',k'.\end{array}\]

        $(\Leftarrow)$
        Ahora si suponemos que los coeficientes  $\langle a_j b_k |B\otimes A|a_{j'}b_{k'}\rangle=0$,  $\forall j,k\ne j',k'$. Entonces el operador $B\otimes A=\sum_{i,l}d_{i,l}P_{a_i, b_l}$ escrito en la base de vectores propios de $A\otimes B$.
        
        Finalmente, se obtiene la siguiente igualdad \[\begin{split}\sum_{j,k}\tr( \sum_{i,l}d_{i,l}P_{a_i, b_l} P_{a_j,b_k})\tr(P_{a_j b_k}\rho )&=\sum_{j,k}\sum_{i,l}d_{i,l}\delta_{j,k}^{i,l}\tr(P_{a_j b_k}\rho )\\&=\sum_{j,k}d_{j,k}\tr(P_{a_j b_k}\rho )=\tr(B\otimes A\rho)\end{split}.\]


\end{proof}

\begin{corollary}
   Para todo estado inicial $\rho$ si los observables $A\tensor B$ y $B\tensor A$ conmutan y  no son degenerados entonces ambos instrumentos tienen valores esperados equivalentes.
\end{corollary}



Debido a que el primer instrumento presenta el valor esperado correcto y el segundo no en todos los casos, lo razonable es comparar el tercer instrumento con el primero únicamente. Anteriormente se vio que estos instrumentos tenían un valor esperado muy similar. Sin embargo, es importante hacer la aclaración que el tercer instrumento solo tiene sentido, en el caso particular en el que los operadores de proyección de los observables $A\tensor B$ y $B\otimes A$ cumplen con $P_{a_m,b_n}P_{b_n,a_m}=\delta_{b_n,a_m}^{a_m,b_n}P_{a_m,b_n}$.











\rrnote{Esqueleto: Se hace una comparación con los valores esperados del primer y tercer instrumento para precisar en que condiciones estos son equivalentes. }
% }}}
\subsection{Ejemplos sobre una medición difusa utilizando distintas herramientas} % {{{
% Into sub {{{
\rrnote{Esqueleto: Describir algunos ejemplos con estados iniciales y observables particulares de una medición difusa en un sistema de dos particulas. Continuando con el ejemplo de la sección {\ref{subsec:Valores_esperados}} en el que se quiere medir el observable $\sigma_z\otimes \sigma_x$, con vectores propios $\{\ket{0+},\ket{0-},\ket{1+},\ket{1-}\}$. El estado inicial del sistema es  \[\rho= \left(\frac{1}{3}\ket{0}\bra{0}+\frac{2}{3}\ket{1}\bra{1}\right)\otimes\left(\ket{+}\bra{+}\right)=\begin{pmatrix}
    1/6&1/6&0&0\\
    1/6&1/6&0&0\\
    0&0&1/3&1/3\\
    0&0&1/3&1/3\\
\end{pmatrix}.\] }
% }}}
\subsubsection{Ejemplo utilizando las mediciones POVM y los operadores de Kraus} % {{{

\rrnote{Esqueleto: Previamente se graficó el mapeo de probabilidades de obtener los valores propios del observables, los cuales son $1$ y $-1$. Realizar este mapeo con los efectos discutidos}

\rrnote{Esqueleto: ilustrar de sitinta forma las probabilidades del observable $\sigma_z\otimes \sigma_x$, cuando se realiza una medición difusa con una probabilidad $p=0.25$. En el eje $x$ se encuentran los valores propios del observable $\sigma_z$ y en el eje $y$ se ilustran los valores propios del observable $\sigma_x$ y en el eje $z$ las probabilidades de obtener las  salidas $a_1 b_1=(1)(1)$, $a_1 b_2=(1)(-1)$ $a_2 b_1=(-1)(1)$, $a_2 b_2=(-1)(-1)$}

\rrnote{Esqueleto: Ilustrar uno de los operadores de Kraus si se obtiene la salida  $a_1=1$ y  $b_{1}=1$, el canal que representa la medición difusa será \[\E_{a_1 ,b_1} (\rho) = \dfrac{K_{a_1 ,b_1} \rho K^\dagger_{a_1 ,b_1}}{\tr(\mathcal{F}_{2\text{p}}(P_{0+})\rho)}. \]Representará precisamente que con una probabilidad $p$ el estado estará en el espacio propio de $\ket{0+}$ y con una probabilidad $(1 - p)$, en el espacio propio $\ket{+0}$
\[\rho'_{0+}=\dfrac{\sqrt{p P_{0+}+(1-p)P_{+0}}\rho\sqrt{p P_{0+}+(1-p)P_{+0} }}{\tr(\mathcal{F}_{2\text{p}}(P_{0+})\rho)}.\] }

\rrnote{Esqueleto colocar como se vería el estado posterior matricialmente.}
% }}}
\subsubsection{Ejemplo utilizando los instrumentos cuánticos} % {{{

\rrnote{Esqueleto: Utilizando el mismo estado inicial del sistema y el mismo observable del ejemplo anterior se utiliza para describir de manera detallada el instrumento cuántico y cómo se utiliza para la medición difusa en este sistema. Además se obtiene el mapeo de resultados y el canal cuántico luego de la medición. } 


\rrnote{Esqueleto: Análogamente se realiza este procedimiento con el instrumento dos y tres. Calculo el mapeo de resultados y el canal cuántico luego de la medición con las trazas parciales.}



% }}}
% }}}
% }}}
\section{Generalización de operadores de Kraus en sistemas de \texorpdfstring{\boldmath{$N$}}{N} partículas} % {{{


\subsection{Medidas POVM y operadores de Kraus en sistemas de varias partículas}


\rrnote{\textit{¿Cómo deberían ser las medidas POVM\@? Ahora el operador de permutación no es hermítico, POVM y los operadores de Kraus cambian ligeramente.}}

\rrnote{Esqueleto: Antes se describió el mapeo que puede realizarse con las medidas POVM, el cual será conveniente para proporcionar la probabilidad de cada posible salida de la medición. En un sistema de dos partículas se propuso intuitivamente un conjunto de efectos, en los cuales a los operadores proyección se les aplicaba el operador difuso para dos partículas.}



\subsection{Instrumento cuántico en sistemas de varias partículas} 



% }}}
\section{Observables degenerados en mediciones difusas} % {{{



% }}}
\section{Casos particulares de mediciones difusas} % {{{
\rrnote{En esta sección se desea ilustrar ejemplos de sistemas de $N$ partículas en los que se realicen mediciones difusas y poderlos describir completamente. Un caso particular podría ser el de una cadena de iones en una recta o en una fila donde solo puedan intercambiarse con sus vecinos. O una cadena en una circunferencia.} 

\rrnote{\textit{¿Cómo cambia cuando tenemos una circunferencia o una fila de iones? ¿Cómo es el valor esperado? ¿Cuál es el operador difuso? ¿Cómo se describe estas mediciones? }}


% }}}


