\chapter[RESULTADOS]{3. Resultados}
\section{Introducción} % {{{

\rrnote{Esqueleto: Intro: En los capítulos anteriores se estableció un base conceptual para realiza un análisis de las mediciones difusas. En este capítulo se utilizará este marco para realizar la descripción completa de las mediciones difusas en sistemas cuánticos.}

% }}}
\section{Mediciones difusas en sistemas de dos partículas} % {{{
% Intro {{{
\rrnote{Esqueleto: En esta sección exponer los resultados ya trabajados y
explorar más opciones. Primero especificar condiciones del sistema de dos
partículas y de los observables. Inicialmente, el estado del sistema es
$\rho=\rho_1\otimes \rho_2$, donde $\rho_1 \in  \mathcal{H}_1$ y $\rho_2\in
\mathcal{H}_2$,  en las que se quiere realizar la medición de un observable
$A\otimes B=\sum_{j,k}a_j b_k P_{a_j, b_k}$, donde $A$ y $B$ son ambos de
dimensión $d$.}


Las mediciones difusas solo cobran sentido en un sistema compuesto por dos o
más partículas, razón por la cual el sistema más básico para comenzar a
analizar una medición difusa es aquel conformado únicamente por dos partículas.
Inicialmente, el estado del sistema es \[\rho=\rho_1\otimes \rho_2,\] donde
$\rho_1 \in  \mathcal{H}_1$ y $\rho_2\in \mathcal{H}_2$ \cpnote{porque un estado
factorizable?},  en ellos se quiere
realizar la medición de un observable factorizable 
\[A\otimes B=\sum_{j,k}a_j
b_k P_{a_j, b_k}=\sum_{j,k}a_j b_k P_{a_j}\otimes P_{b_k},\] siendo $P_{a_j}$ y
$P_{b_k}$ los operadores de proyección que corresponden al valor propio $a_j$
del observable $A$ y el valor propio $b_k$ de $B$ respectivamente, ambos de
dimensión $d$. Sin embargo, la medición realizada es difusa y siguiendo con la
definición {\ref{def:medicion-difusa}}, la probabilidad que se realice la
medición del observable $B\otimes A$ no es cero, sino $1-p$ y la posibilidad de
realizar una medición ideal en el sistema  es $p$. 

\rrnote{Esqueleto:Incluir imágenes de algún ejemplo de una medición difusa en
un sistema.  Con algún diagrama de cajas en las que se realiza una medición del
observable particular. Comparar  una medición ideal en un sistema de dos
partículas. Luego, en una segunda imagen agregar un diagrama que represente la
matriz swap y obtener salidas intercambiadas  con cierta probabilidad.}

En la siguiente imagen se ilustra un esquema de medición del observable
$\sigma_z\otimes \sigma_x$ en la que debido a ruido del entorno, existe una
posibilidad de realizar la medición errónea y en su lugar medir el observable
$\sigma_x\otimes \sigma_z$. La salida clásica de los observables es $(1)$  para
el observable $\sigma _z$ y $(-1)$ para el observable $\sigma_x$. Sin embargo
su valor esperado será la suma ponderada de los valores esperados de realizar
las dos mediciones que son verosímiles, es decir $p \tr(\sigma_z\otimes \sigma
\rho )+ (1-p)\tr(\sigma_x\otimes\sigma_z \rho
)$.
\begin{figure}[H]
    \centering
    
  \begin{subfigure}[a]{1\textwidth}
    \centering
    \resizebox{12.5cm}{!}{\begin{quantikz}[font=\small, row sep=0.3cm,column sep=0.4cm]%
    \lstick[wires=3]{$\rho$}& \qw& \qw& \qw& \gate[wires=3,style={
    starburst,fill=yellow,draw=red,line width=2pt,inner
    xsep=-2pt,inner ysep=-5pt},label style=blue]{\text{
    noise}} & \qw& \qw& \qw& \meter[style={draw=blue}]{$\sigma_z$} \vcwdouble{1-9}{5-9}{1}&\qw&\qw&\qw&\rstick[wires=5]{$p\tr(\sigma_z\otimes \sigma_x\rho)+(1-p)\tr(\sigma_x\otimes\sigma_z\rho)$}\\
    &&&&&&&&&&&&\\
    & \qw&\qw&\qw& \qw& \qw& \qw& \qw& \qw&\qw&\meter[style={draw=blue}]{$\sigma_x$}\vcwdouble{3-11}{5-11}{-1}\qw&\qw&\\
    &&&&&&&&&&&&\\
    \lstick{cl}& \cw& \cw& \cw& \cw& \cw& \cw& \cw&\cw&\cw&\cw&\cw&\\
  \end{quantikz}}
\hfill
\caption{}
\end{subfigure}
\hfill

\begin{subfigure}{0.4\textwidth}
  \begin{flushleft}
    \resizebox{7cm}{!}{
      \begin{quantikz}[font=\small,row sep=0.3cm,column sep=0.4cm]%
      \lstick[wires=2]{$\rho$}&   \qw& \qw& \meter[style={draw=blue}]{$\sigma_z$} \vcwdouble{1-4}{4-4}{1}&\qw&\qw&\qw\rstick[wires=4]{$p \tr(\sigma_z \tensor \sigma_x \rho)$}&\\[0.1cm]
       &  \qw& \qw& \qw& \qw&\meter[style={draw=blue}]{$\sigma_x$}\vcwdouble{2-6}{4-6}{-1}&\qw& \\[0.1cm]
      &&&&&&&\\
      \lstick{cl}& \cw&  \cw&\cw&\cw&\cw&\cw&\\
    \end{quantikz}}
    \hfill
  \caption{}
  \end{flushleft}
   
\end{subfigure}


%  \begin{subfigure}[a]{0.5\textwidth}
 % \centering
  %\resizebox{6cm}{!}{
   % \begin{quantikz}[scale=1, row sep=0.3cm,column sep=0.4cm]%
    %\lstick{$\rho_{1}$}&  \qw& \qw& \qw& \qw& \qw&\qw\rstick[wires=3]{}\\ \\
    %&&&&&&& |[meter]| \vcwdouble{2-8}{5-8}{a_jb_k}\rstick{$A \tensor B$} \\
    %\lstick{$\rho_2$} & \qw& \qw& \qw& \qw& \qw&\qw&\\
    %\meter[style={draw=blue}]{B}\vcwdouble{2-11}{4-11}{b_k} \\
  %  & & && &&&\\
   % \lstick{cl}&  \cw& \cw& \cw& \cw&\cw&\cw&\cw\\
  %\end{quantikz}}
  %\hfill
%\caption{}
%\end{subfigure}

\begin{subfigure}{0.4\textwidth}
  \centering
\resizebox{8cm}{!}{
\begin{quantikz}[font=\small,scale=1, row sep=0.3cm,column sep=0.4cm]%
  \lstick[wires=2]{$\rho$}&\qw& \gate[swap]{} & \qw&  \meter[style={draw=blue}]{$\sigma_z$} \vcwdouble{1-5}{4-5}{ 1}&\qw&\qw&\qw\rstick[wires=4]{$(1-p)\tr(\sigma_x \tensor \sigma_x\rho)$}& \\
  & \qw& \qw& \qw& \qw& \qw&\meter[style={draw=blue}]{$\sigma_x$}\vcwdouble{2-7}{4-7}{-1}&\qw& \\
  &&&&&&&&\\
  \lstick{cl}& \cw& \cw& \cw&\cw&\cw&\cw&\cw&\\
\end{quantikz}}
\hfill
\caption{}
\end{subfigure}
\caption{\textbf{(a)} En la primera imagen se ilustran un estado inicial $\rho$, sin embargo el entorno no es ideal y es posible que ocurra una identificación errónea, en las que las salidas clásicas registradas para los observables $\sigma_z$ y $\sigma_x$ fueron $1$ y $-1$ respectivamente.\textbf{ (b)} En esta imagen se ilustra que con una probabilidad $p$ la identificación del sistema sea correcta y se realice la medición del observable $\sigma_z\tensor \sigma_x$.\textbf{ (c)} La imagen muestra que debido al ruido del entorno, las partículas experimentan un intercambio y con una probabilidad $(1-p)$ se realiza la medición del observable $\sigma_x\otimes \sigma_z$ }\label{diagrama-cajas}\source{elaboración propia}
\end{figure}


\cpnote{No entiendo el objetivo de esta seccion. Se me hace un poco repetitiva. Pero 
quizá si hay un buen motivo para ponerla. Cuentame, antes de borrar!!}

% }}}
\subsection{POVM y operadores de Kraus para mediciones difusas en sistemas de dos partículas}\label{Sec_POVM_para_mediciones_difusas} % {{{
 


\rrnote{Partiendo del valor esperado para dos particulas, se realiza una implementación de los operadores de Kraus. Recordando la ecuación que se encontró intuitivamente que los efectos para una medición difusa, son $\{\mathcal{F}_{2\text{p}}({P_{a_j,b_k}})\}$. Estos efectos proporcionan el mapeo de probabilidades$E(a_j b_k, \rho)= \tr(\mathcal{F}_{2\text{p}}({P_{a_j,b_k}}))$. Sin embargo para obtener el estado posterior es a la medición es necesario descomponer estos efectos. Debido a que $\mathcal{F}_{2\text{p}}({P_{a_j,b_k}})$ es un operador positivo se puede asumir que $K_{a_j,b_k}=\sqrt{\mathcal{F}_{2\text{p}}({P_{a_j,b_k}})}$}


Como se mencionó en el capítulo anterior una forma de aproximarse a la describir enteramente estas mediciones es utilizando conjuntamente las medidas POVM y los operadores de Kraus. Recordando la ecuación {\eqref{eq:ordenar-valoresperado-povm}} se encontró intuitivamente que los efectos para una medición difusa, son $\{\mathcal{F}_{2\text{p}}({P_{a_j,b_k}})\}$, donde $P_{a_j,b_k}$ son los operadores de proyección del observable, correspondiente al valor propio ${a_j b_k}$. Estos efectos proporcionan el mapeo de probabilidades\begin{equation}\label{eq:mapeo-probabilidades-2p}
    E(a_j b_k, \rho)= \tr(\mathcal{F}_{2\text{p}}({P_{a_j,b_k}})).
    \end{equation} Sin embargo para obtener el estado posterior es a la medición es necesario descomponer estos efectos. Debido a que $\mathcal{F}_{2\text{p}}({P_{a_j,b_k}})$ es un operador positivo se puede asumir que
    $K_{a_j,b_k}=\sqrt{\mathcal{F}_{2\text{p}}({P_{a_j,b_k}})}$. Luego de introducir los operadores de Kraus el
    estado de la medición es posterior a la medición POVM será
    \begin{equation}\label{eq:estado-posterior-povm1-2p}\begin{split}\rho'_{a_j,b_k}&=\dfrac{K_{a_j,b_k} \rho
    K_{a_j,b_k}^\dagger}{\tr(K_{a_j,b_k}^\dagger K_{a_j,b_k} \rho)}\\
    &=\dfrac{\sqrt{\mathcal{F}_{2\text{p}}(P_{a_j,b_k})} \rho
    \sqrt{\mathcal{F}_{2\text{p}}({P_{a_j,b_k}})}}{\tr(\mathcal{F}_{2\text{p}}({P_{a_j,b_k}}) \rho)}\\
    &=\dfrac{\sqrt{p P_{a_j,b_k}+(1-p)SP_{a_j,b_k}S^\dagger}\rho\sqrt{p P_{a_j,b_k}+(1-p)SP_{a_j,b_k}S^\dagger}}{\tr((p P_{a_j,b_k}+(1-p)SP_{a_j,b_k}S^\dagger) \rho)}\\
    &=\dfrac{\sqrt{p P_{a_j,b_k}+(1-p)P_{b_k,a_j}}\rho\sqrt{p P_{a_j,b_k}+(1-p)P_{b_k,a_j}} }{\tr((p P_{a_j,b_k}+(1-p)P_{b_k,a_j}) \rho)}.
    \end{split}\end{equation}La descomposición en operadores de Kraus
    no es única. Como se mencionó anteriormente, cuando se realiza una medición, lo que pasa con el estado exactamente depende en como los POVM son implementados en el laboratorio.





% }}}
\subsection{Instrumentos cuánticos para dos partículas} % {{{
% Intro subsection {{{
\rrnote{Esqueleto:Intro: En este apartado se busca hacer un análisis sobre las formulaciones de las alternativas de los instrumentos presentadas en la sección {\ref{sec:cap2instrumentos-cuanticos}}. Se obtiene el sistema clásico y cuántico para comparar las descripciones que proporcionan. Se verificar que las dos primeras alternativas cumplan con la condición necesaria para verificar que tengan  la misma distribución de probabilidad, con el valor esperado. Además de analizar el tercer instrumento y compararlo con el primero}

En este apartado se busca hacer un análisis sobre las formulaciones de las tres alternativas de los instrumentos presentadas en la sección {\ref{sec:cap2instrumentos-cuanticos}}. En éste, se obtiene el sistema clásico y cuántico para comparar las descripciones que proporcionan. Para una plena especificación de las mediciones difusas, las interpretaciones son intuitivamente equivalentes. Sin embargo, es indispensable verificar que las dos proporcionen el mismo mapeo de resultados. En principio, una condición necesaria para corroborar que tengan la misma distribución de probabilidad es que el valor esperado de los instrumentos genere el mismo resultado que el de la ecuación {\eqref{eq:Expected-Value-FM-2p}}. 


% }}}
\subsubsection{Primera alternativa de instrumento cuántico} % {{{


\rrnote{Esqueleto: Se revisita la primera alternativa de los intrumentos cuánticos, en sistemas simples de dos partículas y con observables factorizables. En esta parte se  aplica la traza parcial sobre el sistema cuántico y el clásico para realizar el análisis del instrumento. Se calcula su valor esperado para saber si cumple con la ecuación del valor esperado}

\rrnote{Esqueleto:Para una medida no selectiva el mapea de resultados se obtiene simplemente aplicando una traza parcial sobre el sistema cuántico. Se calcula esta traza parcial. Además el se calcula la traza parcial del sistema clásico para obtener un canal cuántico para una medición selectiva. }


\rrnote{Esqueleto:  Mientras que el estado posterior se obtiene realizando la traza parcial sobre el sistema clásico. Se realiza el desarrollo }

\rrnote{Esqueleto: Por otro lado, se calcula el valor esperado del primer instrumento y se compara con el valor esperado original de la medición difusa  }

% }}}
\subsubsection{Segunda alternativa de instrumento cuántico} % {{{

\rrnote{Esqueleto:Después de haber hecho el cálculo para el valor esperado del primer instrumento, el siguiente es el caso en el que se supone que el error ocurre en la lectura de los resultados y se desarrolla el segundo instrumento para un sistema de dos partículas.}

\rrnote{Esqueleto: Se desarrolla la traza parcial sobre el sistema cuántico del instrumento para obtener el mapeo de resultado  para una medición no selectiva es el siguiente. Además el se calcula la traza parcial del sistema clásico para obtener un canal cuántico para una medición selectiva. Se interpretan los resultados comparando con los obtenidos con el primer instruemento.}

\rrnote{Esqueleto: En el contexto de esta interpretación, se calcula el valor esperado de esta alternativa.}

% }}}
\subsubsection{Tercera alternativa} % {{{

\rrnote{Esqueleto:De igual forma que en los otros instruementos se desarrolla el calculo de la traza parcial para obtener el mapeo de resultados y el operador de densidad para el sistema cuántico.  Se interpretan los resultados comparando con los obtenidos con el primer instruemento y segundo instrumento.}

\rrnote{Esqueleto:Finalmente, se calcula el valor esperado de este último instrumento cuántico. }


% }}}
% }}}
\subsection{Equivalencia de los instrumentos} % {{{
\rrnote{Esqueleto: Se hace una comparación entre los instrumentos anteriores y se expone por qué resultaron no ser equivalentes. En esta sección se comenta que se esperaba que los instrumentos fueran equivalentes pero resultaron distintos y por ello vale la pena proponer en qué condiciones sí lo son.}

\rrnote{Esqueleto: Se presenta la proposición que detalla las condiciones de equivalencia del valor esperado entre los dos instrumentos y su demostración.}



\rrnote{Esqueleto: Se hace una comparación con los valores esperados del primer y tercer instrumento para precisar en que condiciones estos son equivalentes. }
% }}}
\subsection{Ejemplos sobre una medición difusa utilizando distintas herramientas} % {{{
% Into sub {{{
\rrnote{Esqueleto: Describir algunos ejemplos con estados iniciales y observables particulares de una medición difusa en un sistema de dos particulas. Continuando con el ejemplo de la sección {\ref{subsec:Valores_esperados}} en el que se quiere medir el observable $\sigma_z\otimes \sigma_x$, con vectores propios $\{\ket{0+},\ket{0-},\ket{1+},\ket{1-}\}$. El estado inicial del sistema es  \[\rho= \left(\frac{1}{3}\ket{0}\bra{0}+\frac{2}{3}\ket{1}\bra{1}\right)\otimes\left(\ket{+}\bra{+}\right)=\begin{pmatrix}
    1/6&1/6&0&0\\
    1/6&1/6&0&0\\
    0&0&1/3&1/3\\
    0&0&1/3&1/3\\
\end{pmatrix}.\] }
% }}}
\subsubsection{Ejemplo utilizando las mediciones POVM y los operadores de Kraus} % {{{

\rrnote{Esqueleto: Previamente se graficó el mapeo de probabilidades de obtener los valores propios del observables, los cuales son $1$ y $-1$. Realizar este mapeo con los efectos discutidos}

\rrnote{Esqueleto: ilustrar de sitinta forma las probabilidades del observable $\sigma_z\otimes \sigma_x$, cuando se realiza una medición difusa con una probabilidad $p=0.25$. En el eje $x$ se encuentran los valores propios del observable $\sigma_z$ y en el eje $y$ se ilustran los valores propios del observable $\sigma_x$ y en el eje $z$ las probabilidades de obtener las  salidas $a_1 b_1=(1)(1)$, $a_1 b_2=(1)(-1)$ $a_2 b_1=(-1)(1)$, $a_2 b_2=(-1)(-1)$}

\rrnote{Esqueleto: Ilustrar uno de los operadores de Kraus si se obtiene la salida  $a_1=1$ y  $b_{1}=1$, el canal que representa la medición difusa será \[\E_{a_1 ,b_1} (\rho) = \dfrac{K_{a_1 ,b_1} \rho K^\dagger_{a_1 ,b_1}}{\tr(\mathcal{F}_{2\text{p}}(P_{0+})\rho)}. \]Representará precisamente que con una probabilidad $p$ el estado estará en el espacio propio de $\ket{0+}$ y con una probabilidad $(1 - p)$, en el espacio propio $\ket{+0}$
\[\rho'_{0+}=\dfrac{\sqrt{p P_{0+}+(1-p)P_{+0}}\rho\sqrt{p P_{0+}+(1-p)P_{+0} }}{\tr(\mathcal{F}_{2\text{p}}(P_{0+})\rho)}.\] }

\rrnote{Esqueleto colocar como se vería el estado posterior matricialmente.}
% }}}
\subsubsection{Ejemplo utilizando los instrumentos cuánticos} % {{{

\rrnote{Esqueleto: Utilizando el mismo estado inicial del sistema y el mismo observable del ejemplo anterior se utiliza para describir de manera detallada el instrumento cuántico y cómo se utiliza para la medición difusa en este sistema. Además se obtiene el mapeo de resultados y el canal cuántico luego de la medición. } 


\rrnote{Esqueleto: Análogamente se realiza este procedimiento con el instrumento dos y tres. Calculo el mapeo de resultados y el canal cuántico luego de la medición con las trazas parciales.}



% }}}
% }}}
% }}}
\section{Generalización de operadores de Kraus en sistemas de \texorpdfstring{\boldmath{$N$}}{N} partículas} % {{{


\subsection{Medidas POVM y operadores de Kraus en sistemas de varias partículas}


\rrnote{\textit{¿Cómo deberían ser las medidas POVM\@? Ahora el operador de permutación no es hermítico, POVM y los operadores de Kraus cambian ligeramente.}}

\rrnote{Esqueleto: Antes se describió el mapeo que puede realizarse con las medidas POVM, el cual será conveniente para proporcionar la probabilidad de cada posible salida de la medición. En un sistema de dos partículas se propuso intuitivamente un conjunto de efectos, en los cuales a los operadores proyección se les aplicaba el operador difuso para dos partículas.}



\subsection{Instrumento cuántico en sistemas de varias partículas} 



\begin{comment}
En un sistema de $N$ partículas en el que se pretende medir el observable $A_1\otimes A_2\otimes \hdots \otimes A_N$, pero ocurre un error en  el aparato y confunde los resultados.

Se redefine la operación difusa, para un sistema de $N$ partículas utilizando los operadores de permutación y se mencionan algunas características de los operadores de permutación. 

Se calcula el valor esperado de realizar la medición del nuevo observable.
  %Se plantea la primera alternativa de instrumento cuántico utilizando la %nueva operación difusa y el nuevo observable, a su vez se calcula el valor %esperado. 
  
  %De igual forma, se plantea la segunda alternativa y su valor esperado. 
  
  %Finalmente se demuestra la proposición que explica en que condiciones los %instrumentos para un sistema de $N$ partículas son equivalentes.


En sistema de $N$ partículas en el que se realiza una medición de un observable de la forma $A_1\tensor A_2\tensor \hdots \tensor A_N$, y en el que ocurre un error en el aparato de medición y confunde los resultados. Con cierta probabilidad se realiza la medición de acuerdo a la permutación $\Pi$ del observable. 

La operación difusa definida en la ecuación ({\ref{eq:operador_difuso2p}}), ahora se pude redefinir para un sistema de $N$ partículas de forma más general como\begin{equation}\label{eq:fm-nparticles}
   \fuzzy{\rho}=\sum_{\Pi_i\in S}p_{i}\permut{i}{\rho}
\end{equation}donde $\mathcal{S}$ es un subconjunto del grupo simétrico de $n$ partículas y $\sum_{i=1}^{N!} p_i=1$.  Los elementos de este grupo simétrico $\mathcal{S}$ pueden escribirse como producto de operaciones de intercambio $S_{ij}$. Si se requiere un número par de operadores de intercambio $S_{ij}$, se dice que la permutación tiene paridad par, de otro modo, se dice que la permutación tiene paridad impar. Algunas de las propiedades destacables de los operadores de permutación es que son unitarios, el operador adjunto tiene la misma paridad y no son hermíticos.

Ahora que se generaliza el operador difuso, puede utilizarse para obtener el valor esperado de esta medición para el sistema de $N$ partículas {\cite{Pineda_2021}}\begin{equation}\label{eq:ExpectedValue-generalForm}\begin{split}
    \left \langle \prodtensor A_i\right \rangle_{\mathcal{F}(\rho)} &=\tr\left(\fuzzy{\rho}\prodtensor A_i\right)\\
    &=\sum_{\Pi_j \in S}p_{j}\tr \left(\permut{j}{\rho} \prodtensor A_i\right).
\end{split}
\end{equation} 

Aunque los operadores del grupo simétrico, en general no son hermíticos, la operación difusa sí lo es y por tanto usando la propiedad cíclica de la traza, el valor esperado de la medición difusa puede escribirse también como \begin{equation}\label{eq:ExpectedValue-generalForm-2}\begin{split}
    \left \langle \prodtensor A_i\right \rangle_{\mathcal{F}(\rho)} &=\sum_{\Pi_j \in S}\tr \left(\fuzzy{\prodtensor A_i}\rho\right).
\end{split}
\end{equation} 



\subsection{Medidas POVM en sistemas de varias partículas}
En el capítulo anterior se describió el mapeo que puede realizarse con las medidas POVM, el cual será conveniente para proporcionar la probabilidad de cada posible salida de la medición. En un sistema de dos partículas se propuso intuitivamente un conjunto de efectos cuya generalización para $N$ partículas puede ser escrita como \begin{equation}\label{eq:effectsSetNp}
    {\{E_{\lambda_i}\}}_{\lambda_i \in \Lambda}={\{\fuzzy{P_{\lambda_i}}\}}_{\lambda_i \in \Lambda}={\left\{\sum_{\Pi_j \in S} p_j \permut{j}{P_{\lambda_i}}\right\}}_{\lambda_i \in \Lambda},
\end{equation}  
donde $P_{\lambda_i}$ es el operador de proyección correspondiente a cada vector propio que tiene asociado el valor propio $\lambda_i\in \Lambda$. El conjunto de valores propios del observable es \begin{equation}\label{eq:lambdaeigenvalues}
    \Lambda=\{a_{11}\cdot a_{21}\cdot \hdots \cdot a_{N1},\hdots,a_{1J}\cdot a_{2K}\cdot \hdots \cdot a_{NM}\},
\end{equation} donde $a_{jk}$ es el $k$-ésimo valor propio correspondiente al observable $A_j$. Es fácilmente comprobable que estos operadores $E_{\lambda_i}$ son hermíticos, cumplen con la propiedad de completitud y son positivos \rrnote{Creo que deberia probarlo.}.

Para obtener el estado posterior a la medición se requiere utilizar la descomposición de los efectos en operadores de Kraus $\{K_{\lambda_i}\}$. Para mediciones difusas en sistema de $N$ partículas se propone utilizar \begin{equation}
   K_{\lambda_i}=\sqrt{\sum_{\Pi_j \in S} p_j \permut{j}{P_{\lambda_i} }},
\end{equation} esto se puede realizar debido a la positividad de los efectos. De nuevo, esta descomposición no es única y el estado posterior de la medición dependerá de como se implementen las medidas POVM en el laboratorio. 



\subsection{Instrumentos cuánticos en sistemas de \texorpdfstring{\boldmath{$N$}}{N} partículas}
En esta parte también se consideran las dos alternativas de instrumentos cuánticos de la sección ({\ref{sec:instrumentos-cuanticos}}). 


La primera alternativa es el instrumento en el que las partículas se intercambian y luego se aplica una medición proyectiva \begin{equation}\label{eq:1instrumentnp}
    \mathcal{I}_1(\rho)=\sum_{\lambda_i \in \Lambda }P_{\lambda_i}\otimes P_{\lambda_i}\fuzzy{\rho}P_{\lambda_i},
\end{equation} donde $P_{\lambda_i}$ son los operadores de proyección y $\lambda_i \in \Lambda$ son los valores propios del observable.  


El valor esperado del resultado de la medición modelado con este instrumento puede calcularse de la siguiente manera \begin{equation*}
    \begin{split}
        \left \la \prodtensor A_i \right \ra_{\mathcal{I}_1}&=\tr\left( \left[\left(\prodtensor A_i\right) \otimes \mathds{1}\right]\mathcal{I}_1\right) \\
        &=\tr\left(\left[ \left(\prodtensor A_i\right)\otimes \mathds{1}\right]\sum_{\lambda_j \in \Lambda}P_{\lambda_j}\otimes P_{\lambda_j}\fuzzy{\rho}P_{\lambda_j} \right)\\
        &=\sum_{\lambda_j\in \Lambda} \tr\left(\left(\prodtensor A_i\right) P_{\lambda_j}\right) \tr\left(P_{\lambda_j}\fuzzy{\rho} P_{\lambda_j}\right) \\
        &=\sum_{\lambda_j\in \Lambda} \tr\left(\sum_{{\lambda_j, \lambda_k \in \Lambda}}\lambda_k P_{\lambda_k} P_{\lambda_j}\right) \tr\left(P_{\lambda_j}\fuzzy{\rho} P_{\lambda_j}\right)  \\
        &=\sum_{\lambda_j \in\Lambda} \tr\left(\lambda_j P_{\lambda_j}\right) \tr\left(P_{\lambda_j}\fuzzy{\rho}P_{\lambda_j}\right) \\
        &=\sum_{\lambda_j \in \Lambda} \lambda_j \tr\left(P_{\lambda_j}\fuzzy{\rho}\right) \\
    \end{split}
\end{equation*} con lo que se puede concluir que el valor esperado correspondiente a este instrumento es \begin{equation}\label{eq:valor-esperado-1instrumentnp}
        \left \la \prodtensor A_i \right \ra_{\mathcal{I}_1}= \tr\left( \prodtensor A_i \fuzzy{\rho}\right),
\end{equation}el mismo que el valor esperado correcto ({\ref{eq:ExpectedValue-generalForm}}).

La segunda alternativa presentada en ({\ref{second-instrument}}) es igualmente generalizable para un sistema de $N$ partículas como \begin{equation}\label{eq:second-instrumentnp}
    \mathcal{I}_2(\rho)= \sum_{\lambda_i \in \Lambda } \fuzzy{P_{\lambda_i}}\tensor P_{\lambda_i}\rho P_{\lambda_i},
\end{equation}  en esta alternativa se interpreta una confusión en la lectura de los resultados.




Con esta alternativa el valor esperado se calcula como 
\begin{equation*}
    \begin{split}
        \left \la \prodtensor A_i \right \ra_{\mathcal{I}_2}&=\tr\left( \left[\left(\prodtensor A_i\right) \otimes \mathds{1}\right]\mathcal{I}_2\right) \\
        &=\tr\left(\left[ \left(\prodtensor A_i\right)\otimes \mathds{1}\right]\sum_{\lambda_j \in \Lambda}\fuzzy{P_{\lambda_j}}\otimes P_{\lambda_j}{\rho}P_{\lambda_j} \right)\\
        &=\sum_{\lambda_j\in \Lambda} \tr\left(\prodtensor A_i\fuzzy{P_{\lambda_j}}\right) \tr\left(P_{\lambda_j}\rho\right). \\
    \end{split}
\end{equation*} finalmente el valor esperado es \begin{equation}\label{eq:valor-esperado-2instrumentnp}
    \left \la \prodtensor A_i \right \ra_{\mathcal{I}_2}=\sum_{\lambda_j\in \Lambda} \tr\left(\prodtensor A_i\fuzzy{P_{\lambda_j}}\right) \tr\left(P_{\lambda_j}\rho\right).
\end{equation} Este valor esperado no corresponde a (\ref{eq:ExpectedValue-generalForm}) por lo que análogamente a (\ref{prop:Equivalencia-instruments}) se tiene una proposición más general para la equivalencia de estos instrumentos en sistema de $N$ partículas.

\begin{proposition}\label{prop:Equivalencia-instrumentos-np}
    Para todo estado inicial $\rho$, los instrumentos cuánticos {\ref{eq:1instrumentnp}} y {\ref{eq:second-instrumentnp}} son equivalentes si y solo si \[\left \langle \lambda_j \left|{\permut{l}{\prodtensor A_i}}\right|\lambda_k\right\rangle=0,\forall j\ne k \text{ y } \forall \Pi_l \in \mathcal{S}\]
\end{proposition}

\begin{proof}
    $(\Rightarrow)$ Suponiendo que para todo estado inicial $\rho$ los valores esperados de los instrumentos son iguales \[ \tr\left( \prodtensor A_i \fuzzy{\rho}\right)=\sum_{\lambda_j\in \Lambda} \tr\left(\prodtensor A_i\fuzzy{P_{\lambda_j}}\right) \tr\left(P_{\lambda_j}\rho\right)\]
o bien, \[ \tr\left( \fuzzy{\prodtensor A_i }\rho\right)=\sum_{\lambda_j\in \Lambda} \tr\left(\fuzzy{\prodtensor A_i}P_{\lambda_j}\right) \tr\left(P_{\lambda_j}\rho\right).\] Aplicando la linealidad de la traza, \[ \tr\left( \fuzzy{\prodtensor A_i }\rho\right)=\tr\left(\sum_{\lambda_j\in \Lambda} \tr\left(\fuzzy{\prodtensor A_i}P_{\lambda_j}\right) P_{\lambda_j}\rho\right),\]de ello  \[ \begin{split}\tr\left( \left(\fuzzy{\prodtensor A_i }-\sum_{\lambda_j\in \Lambda} \tr\left(\fuzzy{\prodtensor A_i}P_{\lambda_j}\right) P_{\lambda_j}\right)\rho\right)&=0\\,\end{split}\] para todo estado inicial $\rho$ entonces\[\begin{split}\fuzzy{\prodtensor A_i }-\sum_{\lambda_j\in \Lambda} \tr\left(\fuzzy{\prodtensor A_i}P_{\lambda_j}\right) P_{\lambda_j}&=0\\
    \sum_{\Pi_l\in \mathcal{S}}p_l\permut{l}{\prodtensor A_i }-\sum_{\lambda_l\in \Lambda} \tr\left(\sum_{\Pi_l\in \mathcal{S}}p_l\permut{l}{\prodtensor A_i}P_{\lambda_j}\right) P_{\lambda_j}&=0, 
\end{split}\] de esta última ecuación se deduce que para cada $j$ se puede ver que \[\begin{split}
    \permut{l}{\prodtensor A_i }-\sum_{\lambda_l\in \Lambda} \tr\left(\permut{l}{\prodtensor A_i}P_{\lambda_j}\right) P_{\lambda_j}&=0.
\end{split}\] 

Si el primer término se escribe en la base de los vectores propios del observable que se desea medir  \[\begin{split}
    \sum_{\lambda_j, \lambda_k\in \Lambda} \tr\left(\permut{l}{\prodtensor A_i}|\lambda_j\rala \lambda_k|\right) |\lambda_j\rala\lambda_k|-\sum_{\lambda_j\in \Lambda} \tr\left(\permut{l}{\prodtensor A_i}P_{\lambda_j}\right) P_{\lambda_j}&=0,\\
    \sum_{\lambda_j\neq\lambda_k\in \Lambda} \tr\left(\permut{l}{\prodtensor A_i}|\lambda_j\rala \lambda_k|\right) |\lambda_j\rala\lambda_k|&=0,
\end{split}\] finalmente por la independencia lineal de los operadores\[ \tr\left(\permut{l}{\prodtensor A_i}|\lambda_j\rala \lambda_k|\right)= \left\langle\lambda_j\left|\permut{l}{\prodtensor A_i}\right|\lambda_k\right\rangle=0, \forall j\neq k, \forall \Pi_l \in \mathcal{S}.\]



$(\Leftarrow)$ Suponiendo que se cumple que \[ \left\langle\lambda_j\left|\permut{l}{\prodtensor A_i}\right|\lambda_k\right\rangle=0, \forall j\neq k, \forall \Pi_l \in \mathcal{S},\] se puede  escribir a la permutación $l$ como una combinación lineal de los operadores de proyección\[\permut{l}{\prodtensor A_i}=\sum_{\lambda_k \in \Lambda}d_{lk}P_{k}.\]Entonces en el valor esperado ({\ref{eq:valor-esperado-2instrumentnp}})
 \[\begin{split}\left \la \prodtensor A_i \right \ra_{\mathcal{I}_2}&=\sum_{\lambda_j\in \Lambda} \tr\left(\sum_{l}p_l{\sum_{\lambda_k \in \Lambda}d_{lk}P_{\lambda_k}}P_{\lambda_j}\right) \tr\left(P_{\lambda_j}\rho\right)\\ &=\sum_{\lambda_j\in \Lambda} \sum_{l}p_l d_{lj}\tr\left(P_{\lambda_j}\right)\\ &=\sum_{l}p_l \tr\left( \sum_{\lambda_j\in \Lambda} d_{lj}P_{\lambda_j}\rho\right)\\ &=\sum_{l}p_l\tr\left(\permut{l}{\prodtensor A_i}\rho\right)\\ 
    &=\tr\left(\fuzzy{\prodtensor A_i}\rho\right)=\left \la \prodtensor A_i \right \ra_{\mathcal{I}_1}
\end{split}\]

\end{proof}

\end{comment}


% }}}
\section{Observables degenerados en mediciones difusas} % {{{
\rrnote{\textit{¿Cómo se puede describir completamente una medición de un observable que no es factorizable? ¿ ¿Cómo son las salidas en el sistema clásico? ¿La descripción de las mediciones difusas es muy diferente a la de Observables factorizables, en qúe se diferencia? ¿Cómo se aplica el operador difuso?}}

%Para observables, de dimensión $d^N$ los cuales no pueden escribirse como un producto tensorial de $N$ observables, es decir que si $\mathcal{O}$ es un observable no factorizable no puede escribirse como $\prodtensor A_i$. Sin embargo siempre es posible escribirlo como una combinación lineal de los operadores de la base.  


% }}}
\section{Casos particulares de mediciones difusas} % {{{
\rrnote{En esta sección se desea ilustrar ejemplos de sistemas de $N$ partículas en los que se realicen mediciones difusas y poderlos describir completamente. Un caso particular podría ser el de una cadena de iones en una recta o en una fila donde solo puedan intercambiarse con sus vecinos. O una cadena en una circunferencia.} 

\rrnote{\textit{¿Cómo cambia cuando tenemos una circunferencia o una fila de iones? ¿Cómo es el valor esperado? ¿Cuál es el operador difuso? ¿Cómo se describe estas mediciones? }}


% }}}


