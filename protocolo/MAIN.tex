\documentclass[12pt,oneside]{book}\raggedbottom{} % {{{


% Importing document settings from our file "packages.sty"
\usepackage{packages}

\setlength{\headheight}{50pt}
\addtolength{\topmargin}{-25pt}
% }}}
\begin{document}
\begin{sloppypar}
% Inserting title page {{{


\pdfbookmark[0]{PORTADA}{prt} 
\thispagestyle{empty}

\begin{figure}[h]
\setbox0 \vbox{\begin{flushleft}
\vspace*{-18pt}\hspace{-10mm}\includegraphics[width = 2.5cm]{Images/escudo-eps-converted-to.jpg}\\
\end{flushleft}}
\wd0 = 0pt \ht0 = 0pt \box0
\end{figure} \hangindent= 18mm \hangafter=-3\vspace{-10mm}
\noindent Universidad de San Carlos de Guatemala\\[5pt]
Escuela de Ciencias Físicas y Matemáticas \\[5pt]
Departamento de Física

\vfill

\begin{center}
{\setlength{\baselineskip}{1.3\baselineskip} \large \textbf{\MakeUppercase{Protocolo de tesis: Operadores de Kraus para mediciones difusas en sistemas de \boldmath{$n$} partículas}}\par}

\vfill

\textbf{Rubí Esmeralda Ramírez Milián}\\[6pt]
Asesorado por: \\
\textit{Dr.\@ Carlos Francisco Pineda Zorrilla e Ing. Rodolfo Samayoa }  

\vfill

Guatemala, {noviembre 2023}
\end{center}

%%%%%%%%%%%%PAGINA EN BLANCO%%%%%%%%%%%%%%%%%%%%%%%%%%%%

\newpage

\textcolor[rgb]{1,1,1}{.} \thispagestyle{empty}
\begin{comment}
%%%%%%%%%%%%%% SEGUNDA PORTADA %%%%%%%%%%%%%%%%%%%%%%%%%%
\newpage

\pdfbookmark[1]{IDENTIFICACIÓN}{idn}
\thispagestyle{empty}

\begin{center}
UNIVERSIDAD DE SAN CARLOS DE GUATEMALA
\end{center}

\begin{figure}[h]
  \begin{center}
    \includegraphics[width=4.15cm]{escudo-eps-converted-to.pdf}\\
  \end{center}
\end{figure}

\begin{center}
\vspace{-0.5mm}ESCUELA DE CIENCIAS FÍSICAS Y MATEMÁTICAS

\vfill

\recuadro{\MakeUppercase{Operadores de Kraus para mediciones difusas en sistemas de $N$ partículas}}
\vfill

TRABAJO DE GRADUACIÓN \\[3pt] PRESENTADO A LA JEFATURA DEL \\[3pt]
DEPARTAMENTO DE FÍSICA\\[3pt] POR \\[1.15cm]
\textbf{\MakeUppercase{Rubí Esmeralda Ramírez Milián}} \\[6pt]
ASESORADO POR {\MakeUppercase{Carlos Francisco Pineda Zorrilla}} 

\vfill
AL CONFERÍRSELE EL TÍTULO DE \\[6pt]
\textbf{LICENCIADA EN FÍSICA APLICADA}

\vfill

GUATEMALA, {\MakeUppercase{2023}}

\end{center}


%%%%%%%%%PAGINA VACÍA%%%%%%%%%%%%%%%%%%%%%%%%%%%%%%%%%%%
\newpage
\textcolor[rgb]{1,1,1}{.} 

\thispagestyle{empty}




%%%%%%%%%%%%%%% PAGINA DEL PRIVADO%%%%%%%%%%%%%%%%%%%%%%%%%%%%%
\newpage

\pdfbookmark[1]{NÓMINA DEL CONSEJO DIRECTIVO}{ndj}
\thispagestyle{empty}

\begin{center}
UNIVERSIDAD DE SAN CARLOS DE GUATEMALA\\[6pt] ESCUELA DE CIENCIAS FÍSICAS Y MATEMÁTICAS
\end{center}

\begin{figure}[h]
  \begin{center}
    \includegraphics[width=4.15cm]{escudo-eps-converted-to.pdf}\\
  \end{center}
\end{figure}

\begin{center}
\textbf{CONSEJO DIRECTIVO}
\end{center}

{\doublespacing{}
\begin{tabbing}
  SECRETARIO ACADÉMICO xs \= xxxxxxxxxxxxxxxxxxxxxxxxxxxxxxxxxxxx  \kill
  DIRECTOR \> Marcelo Ixquiac \\
%  VOCAL I \>  vocal-1\\
%  VOCAL II \> vocal-2 \\
%  VOCAL III \> vocal-3 \\
%  VOCAL IV \> vocal-4 \\
%  VOCAL V \> vocal-5 \\
  SECRETARIO ACADÉMICO \> Edgar Cifuentes
\end{tabbing}

\vspace{24pt}

\begin{center}
\textbf{TRIBUNAL QUE PRACTICÓ EL EXAMEN GENERAL PRIVADO}
\end{center}

\begin{tabbing}
  SECRETARIO ACADÉMICOxs \= xxxxxxxxxxxxxxxxxxxxxxxxxxxxxxxxxxxx  \kill
%  DIRECTOR \> decano en funciones al hacer privado \\
  EXAMINADOR \> 1 \\
  EXAMINADOR \> 2 \\
  EXAMINADOR \> 3 \\
%  SECRETARIO ACADÉMICO \> secretario en funciones al hacer privado
\end{tabbing}
\par}

\newpage
\textcolor[rgb]{1,1,1}{.} 

\thispagestyle{empty}
\newpage









%%%%%%%%%%%%%%%%%%%%%%%%%%%%%%%%%%%%%%%%%%%%%%%%%%%%%%%%%%%%%%%%%%%%%%%%%%%%%%%%%%%%%%%%%%%%%%%
\begin{titlepage}
\vbox{ }
\vbox{ }
\begin{center}
	% Title
	\HRule{} \\[0.4cm]
	{ \huge \bfseries Operadores de Kraus en mediciones difusas para dos partículas}\\[0.4cm]
	\HRule{} \\[1.5cm]
	
	
	\Large
	%\emph{Autora}\\
	\textit{Rubí Esmeralda Ramírez Milián}\\[1.5cm]
% Upper part of the page
%\includegraphics[width=0.50\textwidth]{Images/logos.jpeg}\\
%\includegraphics[width=0.30\textwidth]{Images/logocolor.png}\\[1cm]
\textsc{\Large Universidad de San Carlos de Guatemala}\\[0.2cm]
\textsc{\Large Escuela de Ciencias Físicas y Matemáticas}\\[1.5cm]
\textsc{\large Informe final de año de prácticas}\\[1.5cm]
\vbox{ }


% Author
\large
Supervisado por:\\
\emph{Dr.\ Carlos Pineda (IF-UNAM) e Ing.\ Rodolfo Samayoa (ECFM-USAC)}\\
\vfill

% Bottom of the page
{\large\emph{18 de marzo 2023}}
\end{center}
\end{titlepage}
\end{comment}



\thispagestyle{empty}
\newpage
% }}}
\frontmatter{ \onehalfspacing{
\tableofcontents  
\newpage
\chapter*{Objetivos} \addcontentsline{toc}{chapter}{OBJETIVOS} % {{{

\section*{Objetivo General}

Estudiar los operadores de Kraus que describan completamente las mediciones
difusas en un sistema de  $n$ partículas. 
\cpnote{Me gustaría un objetivo más fisico. Para que estamos estudiando esto?
Cual es el objetivo de verdad? Me gustaría que lo pensaras y que si no lo 
sabes, que lo discutamos. }

\rrnote{Escribí un parráfo que explica qué es una medición difusa y para qué uilizamos los operadores de Kraus}

En un sistema de varias partículas, una medición difusa es aquella en la que existe una posibilidad de identificar erróneamente a las partículas. El objetivo general de este proyecto es establecer operadores que brinden un mapeo que proporcione la probabilidad obtener las posibles salidas de una medición difusa, así como el estado posterior a la medición.   

%En un sistema de varias partículas, una medición difusa es aquella en la que existe una posibilidad de identificar erróneamente a las partículas. El objetivo general de este proyecto es establecer los operadores que describan completamente una medición difusa. Para describir completamente una medición es necesario contar con un mapeo de salidas que proporcione la probabilidad de las posibles salidas de una medición difusa, así como el estado posterior a la medición. 


\section*{Objetivos Específicos}
\begin{enumerate}
\item Comprender las medidas POVM y su descomposición como operadores de Kraus, así como las operaciones cuánticas como la representación de suma de operadores de Kraus. 


\item Estudiar los operadores de Kraus que describan completamente las mediciones difusas para sistemas de dos partículas.

\item 	Examinar instrumentos cuánticos equivalentes que describan completamente mediciones difusas en un sistema de dos partículas.

%\item Ejemplificar instrumentos cuánticos que describan correctamente las mediciones difusas.

%\item Examinar las condiciones en las que ciertos instrumentos cuánticos describan completamente las mediciones cuánticas.

\item Generalizar los operadores de Kraus que describan completamente las mediciones difusas en sistemas de $n$ partículas.

\item Analizar mediciones difusas con observables no factorizables en sistemas de $n$ partículas.

\end{enumerate}
\newpage

% }}}
\chapter*{Introducción} \addcontentsline{toc}{chapter}{INTRODUCCIÓN} % {{{

El problema de la medición ha jugado un papel muy importante en la física cuántica desde su fundación. Ciertamente,  se ha logrado avanzar en este problema a través de los años, gracias al desarrollo y al entendimiento de los procesos de medición en mecánica cuántica {\cite{jacobs2014quantum}} \cpnote{Creo que no es una buena referencia 
para ese comentario. También creo qeu la frase anterior le faltan comas o le 
sobran palabras}.\rrnote{Quite algunas palabras y agregué algunas comas. Cambié la referencia a la que toman en el paper}


\cpnote{El salto que das acá está demasiado grande. No contextualizas lo que estaoms
discutiendo. Croe que para el siguiente documento toca primero plantear un esqueleto 
y luego ya escribir.}





%Debido a que en la naturaleza es posible obtener mediciones imperfectas
 En este trabajo se trata particularmente
con \textit{mediciones difusas}, en las que en un sistema de varias partículas
existe una posibilidad finita de identificar erróneamente las partículas. Una descripción completa de una medición debe proporcionar las probabilidades respectivas de los diferentes resultados posibles de la medición y el estado posterior a la medición del sistema.

\cpnote{Me gustaria un poco entender un poco mejor la estructura de esta intro. Creo 
que puedes ampliar un poco mas los dos primeros parrafos donde hablas un poco del 
problema y lo contextualizas y leuog dejar los ultimos 4 mas o menos como están}




Para poder abordar el problema es necesario estudiar un marco conceptual
previo, iniciando por la reformulación de los postulados de la mecánica
cuántica en el lenguaje de la matriz de densidad. Se exponen las medidas cuyos
valores son operadores positivos (POVM por sus siglas en inglés) las cuales son
una herramienta que generaliza las medidas proyectivas. Asimismo se recurre a
lenguaje de las operaciones cuánticas porque permiten representar la evolución
de los sistemas más allá de los ideales, de una manera valiosa como suma de
operadores, los cuales están relacionados con las medidas POVM\@. Además, la
teoría de las operaciones cuánticas propone un formalismo para describir la
evolución de los sistemas abiertos de manera discreta
{\cite{nielsen_chuang_2010}}.

En sistemas de varias partículas en los que se realiza una medición de un
observable, en la cual existe una posibilidad de identificar las partículas
equivocadamente, es posible describir completamente dicha medición utilizando
las herramientas mencionadas anteriormente. Estas detecciones imperfectas se
exponen en la referencia {\cite{Pineda_2021}}.\cpnote{Acá si esta bien puesta 
la cita. Por cierto, no necesitas dobles llaves para las citas.}\rrnote{Solo las pongo para silenciar las advertencias que me da chktex}


Se inicia la formulación del problema de las mediciones difusas en el sistema
más simple posible, un sistema de dos partículas. Primero, se exploran algunos
conjuntos de operadores de Kraus que describen las mediciones difusas y
ejemplos\cpnote{porque dices que numéricos?} \rrnote{quité esa palabra. Era porque en el documento anterior había usado un ejemplo con números específicos} de la aplicación de los operadores. En este trabajo se
introducen los instrumentos cuánticos, los cuales constituyen una herramienta
que describe completamente las mediciones. Se evalúa las condiciones en las que
algunos instrumentos cuánticos son equivalentes para caracterizar las
mediciones difusas en sistema de dos partículas.

En el último capítulo, se generaliza los resultados obtenidos para un sistema
sencillo, y se formulan los operadores de Kraus en un sistema de $n$
partículas, tomando en cuenta observables no factorizables. Asimismo, se
exploran las implicaciones de trabajar con observables degenerados en
mediciones difusas. Finalmente, se proponen algunos ejemplos de operadores de
Kraus que describen completamente sistemas específicos de varias partículas. 


}}
% }}}

\mainmatter{\onehalfspacing{ \linenumbers{} \setlength\linenumbersep{3pt}
\chapter{METODOLOGÍA}% \addcontentsline{toc}{chapter}{METODOLOGÍA} % {{{


El proyecto de graduación se iniciará estudiando \cpnote{No consiste en 
iniciar, sino que tu arrancaras estudiando los conectos fundamentales. Reformula 
esta frase.} \rrnote{Reformulé la frase anterior.}conceptos fundamentales para abordar el
problema de las mediciones difusas. Se planteará el problema y se explorarán
los operadores de Kraus para la completa descripción de las mediciones difusas
para sistemas de dos partículas. Finalmente, se propone establecer los
operadores de Kraus que describan las mediciones difusas para cualquier
sistema.

\cpnote{Veo este capitulo muy parecido a la introduccion. Que es lo que va en uno y 
que es lo que va en otro? Puedes pensarlo y platicamos? Dejo las correcciones de esto 
para cuando lo tengamos claro}


%En este proyecto, se estudiará la literatura especializada en las herramientas matemáticas necesarias para este trabajo, principalmente el operador de densidad, las medidas POVM y la teoría de las operaciones cuánticas. Se usará el informe final de prácticas como base inicial así como  otras fuentes especializadas para comprender y analizar algunos teoremas y temas puntuales.
%Se planteará el problema de mediciones difusas para sistemas de dos partículas. Se revistarán los operadores de Kraus y algunos ejemplos que fueron inspeccionados durante el proyecto de prácticas finales. 

 %Finalmente, se aplicará el procedimiento que se utilizó en la segunda parte buscando generalizarlo para sistemas con $n$ partículas. Además se trabajarán con la idea de observables que no son factorizables. Se explorará el concepto de observables degenerados y sus consecuencias en los operadores de Kraus y en los instrumentos cuánticos que describan completamente las mediciones difusas en sistemas de $n$ partículas.

% }}}
\chapter{DESCRIPCIÓN DE LOS CAPÍTULOS}  % {{{
\cpnote{Otra vez acá es la repeticion de lo anterior. Debemos platicar 
un poco a ver que es lo que se quiere hacer en cada parte de este documento. }
\rrnote{Sí, me gustaría trabajar esta parte simultáneamente junto con la tesis}
\section{OPERADORES DE DENSIDAD Y OPERACIONES CUÁNTICAS}
En el primer capítulo  se presenta la revisión y estudio bibliográfico que se
realizó sobre el formalismo de la matriz de densidad en mecánica cuántica, las
medidas POVM y las operaciones cuánticas. La matriz de densidad es una
herramienta utilizada para describir al estado de un sistema cuántico que será
útil en el desarrollo de todo el trabajo.  Se expone el estudio del formalismo
de las medidas POVM\@ las cuales son una herramienta que generaliza las medidas
proyectivas y que permiten un efecto más suave en el sistema medido.  Se
introducirá la teoría de las operaciones cuánticas  porque permiten representar
la evolución de los sistemas más allá de los ideales, de una manera valiosa
como suma de operadores.

\section{MEDICIONES DIFUSAS EN SISTEMAS DE DOS PARTÍCULAS}
Se formulará el enunciado del problema de las mediciones difusas en sistemas de
dos partículas en las que se pueden identificar partículas, pero existe una
probabilidad de identificarlas erróneamente. Con el lenguaje de las operaciones
cuánticas se exploran algunos conjuntos de operadores de Kraus que describen
las mediciones difusas y ejemplos numéricos de la aplicación de los operadores.
Se definirá una nueva herramienta llamada instrumento cuántico. Con esta nueva herramienta se propondrán al menos dos instrumentos cuánticos con distinta interpretación física. A estos instrumentos se les calcularán los valores esperados y se evaluará en que condiciones existen instrumentos cuánticos equivalentes que describan completamente mediciones difusas para sistemas de dos partículas.

%En este capítulo se presenta el concepto de instrumentos cuánticos y su utilidad en la descripción de mediciones en sistemas cuánticos. Y se indagarán las condiciones en la que instrumentos cuánticos que representan la idea detrás de mediciones difusas, sean equivalentes.


\section{OPERADORES DE KRAUS EN SISTEMAS DE \texorpdfstring{\boldmath{$n$}}{n} PARTÍCULAS}

En este capítulo se presentará un razonamiento análogo al del capítulo
anterior para sistemas de $n$ partículas. Se  desarrollarán los operadores de
Kraus, así como los instrumentos cuánticos que describan completamente las
mediciones difusas para sistemas de $n$ partículas. Se examinarán los
observables no factorizables. Se explorarán las consecuencias de trabajar con
observables degenerados en mediciones difusas.  Finalmente, se presentan
ejemplos de sistemas de varias partículas en mediciones difusas que ilustran la
generalización de los operadores de Kraus con cualquier observable.  

% }}}
\chapter{CONTENIDOS*} % {{{
\textit{*Nota: El contenido que se presenta a continuación es tentativo. Conforme se avance en la realización de la tesis es posible que se modifique.}


\section*{LISTA DE FIGURAS}

\section*{LISTA DE TABLAS}

\section*{LISTA DE SÍMBOLOS}

\section*{OBJETIVOS}

\section*{INTRODUCCIÓN}

\section*{1 OPERADOR DE DENSIDAD Y OPERACIONES CUÁNTICAS}
\begin{itemize}
\item[1.1] Introducción
\item[1.2] Operador de densidad
\item[1.3] Medidas POVM
\item[1.4] Operaciones cuánticas 
\end{itemize}

\section*{2 MEDICIONES DIFUSAS EN SISTEMAS DE DOS PARTÍCULAS}
\begin{itemize}
\item[2.1] Introducción
\item[2.2] Operadores de Kraus para mediciones difusas en sistemas de dos partículas
\item[2.4] Ejemplos sobre los efectos de una medición difusa
\item[2.5] Instrumentos cuánticos en mediciones difusas
\item[2.6] Condiciones para equivalencia de los instrumentos 
\end{itemize}
\section*{3 OPERADORES DE KRAUS EN SISTEMAS DE \texorpdfstring{\boldmath{$n$}}{n} PARTÍCULAS}
\begin{itemize}
\item[3.1] Introducción
\item[3.2] Generalización de operadores de Kraus en sistemas de $n$ partículas
\item[3.3] Observables no factorizables en mediciones difusas.
\item[3.4] Observables degenerados en mediciones difusas.  
\end{itemize}
\section*{CONCLUSIONES }
\section*{TRABAJO FUTURO}
% }}}

\nocite{Hall2013}
\nocite{sakurai2017modern}
\nocite{wilde2011classical}
\nocite{2007geometry}
}

\newpage
\nocite{gomez2010introduccion}

\bibliographystyle{ieeetr}
\bibliography{references}
}
\end{sloppypar}
\end{document}
