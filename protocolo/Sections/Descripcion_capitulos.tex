\chapter{DESCRIPCIÓN DE LOS CAPÍTULOS} 
\section{INTRODUCCIÓN A LOS OPERADORES DE DENSIDAD Y A LAS OPERACIONES CUÁNTICAS}
En el primer capítulo  se presenta la revisión y estudio bibliográfico que se realizó sobre el formalismo de la matriz de densidad en mecánica cuántica, las medidas POVM y las operaciones cuánticas. La matriz de densidad es una herramienta utilizada para describir al estado de un sistema cuántico que será útil en el desarrollo de todo el trabajo.  Se expone el estudio del formalismo de las medidas POVM\@ las cuales son una herramienta que generaliza las medidas proyectivas y que permiten un efecto más suave en el sistema medido.  Se introducirá la teoría de las operaciones cuánticas  porque permiten representar la evolución de los sistemas más allá de los ideales, de una manera valiosa como suma de operadores.

\section{MEDICIONES DIFUSAS EN SISTEMAS DE DOS PARTÍCULAS}
Se formulará el enunciado del problema de las mediciones difusas en sistemas de dos partículas en las que se pueden identificar partículas, pero existe una probabilidad de identificarlas erróneamente. Con el lenguaje de las operaciones cuánticas se exploran algunos conjuntos de operadores de Kraus que describen las mediciones difusas y ejemplos numéricos de la aplicación de los operadores


%En este capı́tulo se presentan las mediciones difusas de sistemas cuánticos de dos partı́culas (que puede ser generalizado para más de dos partı́culas) en las que se pueden identificar partı́culas individuales, sin embargo siempre hay una probabilidad de identificarlas erróneamente. Estas detecciones imperfectas se exponen en la referencia [1]. También se utiliza el lenguaje de las operaciones cuánticas por medio de los operadores de Kraus para poder describir las mediciones en los sistemas cuánticos de dos partı́culas.


\section{ INSTRUMENTOS CUÁNTICOS EN MEDICIONES DIFUSAS} 


\section{OPERADORES DE KRAUS EN SISTEMAS DE \texorpdfstring{\boldmath{$n$}}{n} PARTÍCULAS}

 \lipsum[1]