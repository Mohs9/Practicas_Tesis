\addcontentsline{toc}{chapter}{INTRODUCCIÓN}
\chapter*{Introducción}

El problema de la medición ha jugado un papel muy importante en la física cuántica desde su fundación. Ciertamente, se ha logrado avanzar en este problema, a través de los años gracias al desarrollo de las técnicas de medición y al entendimiento de los procesos de medición en mecánica cuántica {\cite{Pineda_2021}}. 

%Debido a que en la naturaleza es posible obtener mediciones imperfectas
 Una descripción completa de una medición debe proporcionar las probabilidades respectivas de los diferentes resultados posibles de la medición y el estado posterior a la medición del sistema. En este trabajo se trata particularmente con \textit{mediciones difusas}, en las que en un sistema de varias partículas existe una posibilidad finita de identificar erróneamente las partículas.


Para poder abordar el problema es necesario estudiar un marco conceptual previo, iniciando por la reformulación de los postulados de la mecánica cuántica en el lenguaje de la matriz de densidad. Se exponen las medidas cuyos valores son operadores positivos (POVM por sus siglas en inglés) las cuales son una herramienta que generaliza las medidas proyectivas. Asimismo se recurre a lenguaje de las operaciones cuánticas porque permiten representar la evolución de los sistemas más allá de los ideales, de una manera valiosa como suma de operadores, los cuales están relacionados con las medidas
POVM\@. Además, la teoría de las operaciones cuánticas propone un formalismo para describir la evolución de los sistemas abiertos de manera discreta {\cite{nielsen_chuang_2010}}.

En sistemas de varias partículas en los que se realiza una
medición de un observable, en la cual existe una posibilidad de identificar las partículas equivocadamente, es posible describir completamente dicha medición utilizando las herramientas mencionadas anteriormente. Estas detecciones imperfectas se exponen en la referencia {\cite{Pineda_2021}}.

Se inicia la formulación del problema de las mediciones difusas en el sistema más simple posible, un sistema de dos partículas. Primero, se exploran algunos conjuntos de operadores de Kraus que describen las mediciones difusas y ejemplos numéricos de la aplicación de los operadores. En este trabajo se introducen los instrumentos cuánticos, los cuales constituyen una herramienta que describe completamente las mediciones. Se evalúa las condiciones en las que algunos instrumentos cuánticos son equivalentes para caracterizar las mediciones difusas en sistema de dos partículas.

En el último capítulo, se generaliza los resultados obtenidos para un sistema sencillo, y se formulan los operadores de Kraus en un sistema de $n$ partículas, tomando en cuenta observables no factorizables. Asimismo, se exploran las implicaciones de trabajar con observables degenerados en mediciones difusas. Finalmente, se proponen algunos ejemplos de operadores de Kraus que describen completamente sistemas específicos de varias partículas. 



\begin{comment}


Durante el proyecto de prácticas se trabajó en la descripción completa de las mediciones difusas para el caso en el que el sistema solo cuenta con dos partículas.

La matriz de densidad es una herramienta utilizada para describir al estado de un sistema cuántico que será útil en el desarrollo de todo el trabajo. En el primer capítulo  se presenta la revisión y estudio bibliográfico que se realizó sobre el formalismo de la matriz de densidad en mecánica cuántica. 


 

 En el tercer capítulo se estudia el formalismo de
las operaciones cuánticas y su representación como suma de operadores.
%Adicionalmente se presenta el proceso de tomografía cuántica. 
% \cpnote{Ver si esta 
% ultima frase va o no va, ya cuando tengamos mas listo el tercer capitulo}\rrnote{Sí, quité la última frase}

En suma, En especial, la
representación como suma de operadores de las operaciones cuánticas y los operadores POVM serán de utilidad para la descripción de las mediciones difusas. 


%Asimismo se recurre al lenguaje de las operaciones cuánticas y su representación como  suma de operadores. Debido a que la teoría de las operaciones cuánticas propone un formalismo para describir la evolución de los sistemas abiertos de manera discreta {\cite{nielsen_chuang_2010}}. Las operaciones cuánticas son una herramienta que permiten describir la evolución de los sistemas más allá de los ideales. Más aún permite representarla de una manera valiosa como suma de operadores tales que están relacionados con las medidas POVM\@. 

% \cpnote{Estas pasando como por la estructura general del trabajo. Sin embargo no hablas del capitulo 4 sino dos frases en la parte de la estructura. Te propongo que armes un parrafo hablando del 4, y que quizá elimines el siguiente parrafo, indicando en los parrafos anteriores donde va, por ejemplo, lo de los POVMs.} \rrnote{Agregué un último párrafo y la parte donde indicaba la estructura del documento la fui colocando en cada uno de los párrafos. }



\end{comment}