\chapter{METODOLOGÍA}% \addcontentsline{toc}{chapter}{METODOLOGÍA}


La metodología del proyecto de graduación consiste en iniciar estudiando los conceptos fundamentales. Se estudiará un marco conceptual para abordar el problema de las mediciones difusas. Se planteará el problema y se explorarán los operadores de Kraus para la completa descripción de las mediciones difusas para sistemas de dos partículas. Finalmente,se propone establecer los operadores de Kraus que describan las mediciones difusas para cualquier sistema.



Para la primera parte se estudiará la literatura especializada en las herramientas matemáticas necesarias para este trabajo, principalmente el operador de densidad, las medidas POVM y la teoría de las operaciones cuánticas. Se usará el informe final de prácticas como base inicial así como  otras fuentes para comprender y analizar algunos teoremas y temas puntuales.




Para la segunda parte se utilizará los operadores de Kraus y algunos ejemplos que fueron inspeccionados durante el proyecto de prácticas finales. Primero se evaluará los sistemas de dos partículas con observables factorizables. Se usarán herramientas como instrumentos cuánticos y se evaluará en que condiciones existen instrumentos cuánticos equivalentes que describan completamente mediciones difusas para sistemas de dos partículas.

 Finalmente, en la última parte se aplicará el procedimiento   que se utilizó en la segunda parte buscando generalizarlo para sistemas con $n$ partículas. Además se trabajarán con la idea de observables que no son factorizables. Se explorará el concepto de observables degenerados y sus consecuencias en los operadores de Kraus y en los instrumentos cuánticos que describan completamente las mediciones difusas en sistemas de $n$ partículas.