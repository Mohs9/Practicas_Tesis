\chapter*{METODOLOGÍA} \addcontentsline{toc}{chapter}{METODOLOGÍA}

Este trabajo de graduación se puede desglosar en tres partes y estas tres partes son: Estudiar un marco conceptual para abordar el problema de las mediciones difusas. Plantear el problema y explorar los operadores de Kraus para la completa descripción de las mediciones difusas para sistemas de dos partículas. Finalmente, establecer los operadores de Kraus que describan las mediciones difusas para cualquier sistema.



Para la primera parte se estudiará la literatura especializada en las herramientas matemáticas necesarias para este trabajo, principalmente el operador de densidad, las medidas POVM y la teoría de las operaciones cuánticas. Se usará el informe final de prácticas como base inicial así como  otras fuentes para comprender y analizar algunos teoremas y temas puntuales.




Para la segunda parte se utilizará los operadores de Kraus y algunos instrumentos que fueron inspeccionados durante el proyecto de prácticas finales. Primero se evaluará los sistemas de dos partículas con observables factorizables.